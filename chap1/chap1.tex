
\chapter{Pré-Requisitos Conjuntistas}
Durante o texto, precisamos de algumas definições e resultados envolvendo noções básicas sobre conjuntos e funções.

Não é objetivo deste texto desenvolver a parte inicial da Teoria dos Conjuntos.
Também não é o objetivo desta seção explicar toda a notação de conjuntos utilizada.
Assumimos familiaridade do leitor com funções e com manipulação de conjuntos a nível básico. Apenas apresentaremos algumas definições, notações e resultados básicos que utilizaremos ao longo do texto.

\section{Famílias e produtos cartesianos}

Famílias são funções com notação especial.
Muitas vezes, ao pensar em funções, pensamos em um ``dispositivo de entrada/saída''.
Quando, ao invés disso, estamos pensando apenas em um  ``conjunto indexado de valores'', a notação de família pode ser mais conveniente.

No quadro abaixo, apresentamos uma comparação entre as duas notações.
Enfatizamos que, matemáticamente, funções e famílias podem ser vistas como o mesmo objeto.
\begin{table}[h]
    \centering
    \begin{tabular}{lllll}
        \hline
        \textbf{Conceito} & \textbf{Função} & \textbf{Família} \\ \hline
        Mapa & $u:I\rightarrow A$ & $(u_i)_{i \in I}=(u_i: i \in I)$ \\
        Valor & $u(i)$ & $u_i$ \\
        Imagem & $\ran u$ & $\{u_i: i \in I\}$\\
        Intuição & objeto dinâmico & objeto estático \\
        Inputs & domínio $I$ & conjunto de índices $I$ \\
        \hline
    \end{tabular}
    \caption{Comparativo de família e função}
\end{table}

Como exemplos, consideremos sequências infinitas e finitas:

\begin{exemplo}[Sequências]
    Uma sequência é uma família cujo conjunto de índices é $\mathbb N$.
    Compare a intuição que passa as notações:
    \begin{itemize}
    \item Considere a sequência $u=(\frac{1}{2^n}))_{n \in \mathbb N}$...
    \item Considere a função $u:\mathbb N\rightarrow \mathbb R$ dada por $u(n)=\frac{1}{2^n}$...
    \end{itemize}
    

\end{exemplo}

\begin{exemplo}[Sequências finitas]
    Se $n\geq 1$, identificamos $n=\{0, 1, \dots, n-1\}$.
    Assim:
    \begin{itemize}
    \item Uma família com $n$ elementos é uma família $(a_i)_{i<n}=(a_i)_{i \in n}=(a_0, \dots, a_{n-1})$.
    \end{itemize}

    Essa notação é bastante funcional no sentido de que dá significado como conjunto aos números naturais, e corresponde à construção usual dos números naturais na Teoria dos Conjuntos.
    Como desvantagem, seus contadores se iniciam no $0$, e não no $1$, o que pode ser pouco intuitivo e não coincidir com a notação da maioria dos textos de matemática, apesar de ser muito adotada em textos mais próximos de Teoria dos Conjuntos.
\end{exemplo}

Agora vamos seguir para a definição de produto cartesiano.
Primeiro, vamos lembrar a definição de produto cartesiano de dois conjuntos.

\begin{definition}[Produto cartesiano de dois conjuntos]
    Sejam $A, B$ conjuntos. Então $A\times B=\{(a, b): a\in A, b \in B\}$ é o \emph{produto cartesiano de $A$ e $B$}.
    Ou seja, o conjunto de todos os pares ordenados $(a, b)$ tais que $a\in A$ e $b\in B$.
\end{definition}

Pares ordenados são conjuntos especiais que carregam duas coordenadas de modo a permitem distinguir a ordem dos elementos. Sua propriedade principal é a de se $a, b, c, d$ são conjuntos,então $(a, b)=(c, d)$ se, e somente se $a=c$ e $b=d$. Uma construção usual, chamada de par de Kuratowski, para a qual não é difícil provar que vale essa propriedade, é dada por $(a, b)=\{\{a\}, \{a, b\}\}$. Porém, isso não será importante neste texto.


\begin{definition}[Produto cartesiano de conjuntos]
Seja $(A_i)_{i \in I}$ uma família de conjuntos. O produto cartesiano de conjuntos é o conjunto $\prod_{i \in I} A_i$ definido como o conjunto de todas as famílias $(a_i: i \in I)$ tais que para cada $i \in I$, $a_i \in A_i$.
$$\prod_{i \in I} A_i=\{(a_i)_{i \in I}: \forall i \in I\, a_i \in A_i\}.$$
\end{definition}


\begin{definition}[Exponenciação de conjuntos]
    Sejam $A, I$ conjuntos. O conjunto $A^I$ é o conjunto de todas as funções de $I$ em $A$. Ou seja, $A^I=\{f:I\rightarrow A\}$. Note que:

    $$A^I=\prod_{i \in I}A=\{(a_i)_{i \in I}: \forall i \in I\,  a_i\in A\}.$$
    \end{definition}

    Na notação anterior, se $n\geq 1$ $$A^n=\{(a_i)_{i<n}:\forall i<n\, a_i \in A\}=\{(a_0, \dots, a_{n-1}):a_0, \dots, a_{n-1}\in A\}\approx A\times \dots \times A \,(n \text{ vezes}).$$

    \section{Operações}

\begin{definition}[Operações $n$-árias]
    Se $X$ é um conjunto e $n \in \mathbb N$, uma operação $n$-ária em $X$ é uma função $f:X^n\rightarrow X$.
\end{definition}

Operações $2$-árias e $1$-árias são frequentemente chamadas de \emph{binárias} e \emph{unárias}, respectivamente.

Caso $*$ seja uma operação binária, a notação $x*y$ é frequentemente utilizada para denotar $x*y$.

Caso $*$ seja uma operação unária, a notação $*x$ é frequentemente utilizada para denotar $*(x)$.
