\chapter[A transcendentalidade de pi]{A transcendentalidade de $\pi$}

Neste apêndice, provaremos a transcendentalidade de $pi$.

Assumiremos as seguintes proposições:
\begin{lemma}
    Para todo $z \in \mathbb C$, temos:

    \begin{align*}
        e^z&=\sum_{n=0}^\infty \frac{z^n}{n!}\\
        \cos z&=\sum_{n=0}^\infty (-1)^n\frac{z^{2n}}{(2n)!}\\
        \sin z&=\sum_{n=0}^\infty (-1)^n\frac{z^{2n+1}}{(2n+1)!}
    \end{align*}

    Além disso, em $\mathbb R$:

    \begin{equation*}
        e=\sum_{n=0}^\infty \frac{1}{n!}
    \end{equation*}
\end{lemma}
Iniciaremos verificando que $e$ é irracional.
\begin{prop}
    O número $e$ é irracional.
\end{prop}