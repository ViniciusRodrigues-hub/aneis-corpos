\chapter{Extensão de corpos}
Neste capítulo, estudaremos o básico da teoria de extensões de corpos, dando como aplicação a resolução de problemas clássicos envolvendo construtibilidade com régua e compasso.
\section{Definições básicas}
\begin{definition}
Dizemos que um corpo $K$ estende um corpo $L$ se $L\subseteq K$ e $L$ é um subanel (subcorpo) de $K$. Escrevemos que $K/L$ é uma extensão de corpos.
\end{definition}

Lembremos que, da álgebra linear, se $K, L$ são corpos e $K$ estende $L$ então $K$ é naturalmente um $L$-espaço vetorial. Explicitamente:

\begin{prop}
    Sejam $K$ um corpo estendendo $L$. Então $K$ é, com a multiplicação usual, um $L$-espaço vetorial.
\end{prop}

\begin{proof}
    $(K, +)$ é claramente um grupo abeliano. Além disso, se $u, v \in L$ e $a, b \in K$, temos que $u(a+b)=ua+ub$, que $(u+v)a=ua+uv$, que $1a=a$ e que $(uv)a=u(va)$.
\end{proof}

\begin{definition}
Na notação anterior, a dimensão de $K$ como $L$-espaço vetorial é denotada por $[K:L]$. Tal dimensão é chamada de \emph{grau de $F/E$}.
\end{definition}

\begin{lemma}
Seja $E, F, K$ corpos com $K$ estendendo $F$ e $F$ estendendo $K$. Então $[K:E]$ é finito se, e somente se $[K:F]$ e $[F:E]$ são finitos, e, nesse caso, $[K:E]=[K:F][F:E]$.
\end{lemma}

\begin{proof}
    Primeiro, suponha que $[K:F]$ e $[F:E]$ são finitos. Seja $\mathcal B=(b_i: 1\leq i\leq n)$ uma base de $K$ como $F$-espaço vetorial e $\mathcal C=(c_j: 1\leq j\leq m)$ uma base de $F$ como $E$-espaço vetorial.

    Seja $\mathcal D=(b_ic_j: 1\leq i\leq n, 1\leq j\leq m)$. Afirmo que $\mathcal D$ é $E$-base de $K$.

    $\mathcal D$ é L.I.: com efeito, sejam $x_{ij}\in E$ tais que $\sum\left(x_{ij}b_ic_j: 1\leq i\leq n, 1\leq j\leq m\right)=0$. Então $\sum_{i=1}^n\left(\sum_{j=1}^mx_{ij}c_j\right)b_i=0$. Como $\mathcal B$ é L.I., cada $\sum_{j=1}^m x_{ij}c_j=0$. Como $\mathcal C$ é L.I., cada $x_{ij}=0$. Portanto, $\mathcal D$ é L.I.

    $\mathcal D$ é gerador: dado $k \in K$, existem $y_1, \dots, y_n$ tais que $k=\sum_{i=1}^n y_i b_i$. Como $\mathcal C$ é gerador, existem $x_{ij}\in E$ tais que $y_i=\sum_{j=1}^m x_{ij}c_j$ para cada $1\leq i\leq k$. Portanto, $k=\sum_{i=1}^n\left(\sum_{j=1}^mx_{ij}c_j\right)b_i=\sum_{i=1}^n\sum_{j=1}^mx_{ij}c_jb_i$, o que mostra que $\mathcal D$ é gerador.

    Note que $\mathcal D$ é uma família de tamanho $mn=[K:F][F:E]$.

    Reciprocamente, suponha que $[K:E]$ é finito. Como $F$ é um subespaço de $K$ como $E$-espaço vetorial, temos que $[F:E]\leq [K:E]$. Ademais, uma base de $K$ como $E$-espaço vetorial é um gerador de $K$ como $F$-espaço vetorial, e, assim, esta contém uma base (finita). Logo, $[K:F]\leq [K:E]$.
\end{proof}

\begin{definition}
Seja $F$ um corpo estendendo $E$ e $a \in F$. Dizemos que $a \in F$ é um elemento algébrico sobre $E$ se existe um polinômio não nulo $p \in E[x]$ tal que $p(a)=0$. Caso contrário, $a$ é dito transcendente sobre $E$.
\end{definition}

\begin{exemplo}
    Todo elemento de $a \in E$ é algébrico sobre $E$, pois é raiz de $x-a$.
\end{exemplo}

\begin{exemplo}
    Considere $\mathbb Q\subseteq \mathbb C$. Temos que $i$ é algébrico sobre $\mathbb Q$, pois é raiz de $x^2+1$.
\end{exemplo}

\begin{exemplo}
    Se $K$ é um corpo, seja $K(x)$ o corpo de frações do domínio $K[x]$. Identificando $K\subseteq K[x]\subseteq K(x)$, temos que $x \in K(x)$ é transcendente sobre $K$.
\end{exemplo}

\begin{definition}
    Seja $F$ uma extensão de um corpo $E$ e $A \subseteq F$.

    Define-se $E[A]$ como sendo o menor subanel de $F$ contendo $E$ e $A$.

    Define-se $E(A)$ como sendo o menor subcorpo de $F$ contendo $E$ e $A$.

    Se $A=\{a_1, \dots, a_n\}$, escrevemos $E[a_1, \dots, a_n]$ e $E(a_1, \dots, a_n)$, respectivamente.
\end{definition}

É claro que precisamos ver que $E[A]$ e $E(A)$ estão bem definidos.

\begin{lemma}
    Seja $F$ uma extensão de um corpo $E$ e $a \in F$. Então $E[A]$ e $E(A)$ estão bem definidos.
\end{lemma}

\begin{proof}
    Lembremos que a interseção de subanéis é um subanel, e que a interseção de subcorpos é um subcorpo. Assim, sejam:

    \[E[A]=\bigcap\{K\subseteq F: K \text{ é subanel de } E \text{ e } A\cup E K\}\]

    \[E(A)=\bigcap\{K\subseteq F: K \text{ é subcorpo de } E \text{ e } A\cup E K\}\]

    Segue que $E[A]$ e $E(A)$ são, respectivamente, um subanel e um subcorpo de $F$ contendo $E$ e $a$.
    É claro que se $K$ é outro subanel (subcorpo) de $F$ contendo $E$ e $a$, então $K\subseteq E[A]$ (ou $K\subseteq E(A)$). 

    Se $L$ é outro subanel mínimo dentre os subanéis que contém $E$ e $A$, então $L\subseteq E[A]$ pela definição de $E[A]$, e $L\supseteq E[A]$ pela definição de $L$.

    Analogamente, se $L$ é outro subcorpo mínimo dentre os subcorpos que contém $E$ e $A$, então $L\subseteq E(A)$ pela definição de $E(A)$, e $L\supseteq E(A)$ pela definição de $L$.
\end{proof}
Observação: na definição acima, $E[A]$ e $E(A)$ dependem também de $F$, apesar de tal fato não estar explícito na notação.
\begin{lemma}
Seja $F/E$ uma extensão de corpos e $A, B \subseteq F$. Então $E[A\cup B]=E[A][B]$ e $E(A\cup B)=E(A)(B)$
\end{lemma}
\begin{proof}
    Temos que $E[A\cup B]$ é o menor subanel de $F$ que contém $E$ e $A\cup B$. Temos que $E[A][B]$ contém $E[A]$ e $B$, logo, contém $E$, $A$ e $B$, assim, $E[A\cup B]\subseteq E[A][B]$.

    Reciprocamente, $E[A][B]$ é o menor subanel de $F$ que contém $E[A]$ e $B$. Temos que $E[A]\subseteq E[A\cup B]$, pois $E[A\cup B]$ contém $E$ e $A$, e $E[A][B]$ contém $B$, logo, $E[A][B]\subseteq E[A\cup B]$.

    Analogamente, $E(A\cup B)=E(A)(B)$.
\end{proof}

\begin{lemma}
    Seja $F/E$ uma extensão de corpos e $C$ um gerador de $F/E$. Então $F=\bigcup\{E(A): A \in [C]^{<\infty}\}$. Analogamente, $E[C]=\bigcup\{E[A]: A \in [C]^{<\infty}\}$.
\end{lemma}

\begin{proof}
    Se $A, B\subseteq C$ são finitos, então $E(A)\cup (B)\subseteq E(A\cup B)$. Logo, $K=\{E(A): A \in [C]^{<\infty}\}$ é uma união dirigida de corpos, e, portanto, um corpo.

    Como cada $E(A)$ com $A\in [C]^{<\infty}$ é tal que $E(A)\subseteq E(C)$, então $K\subseteq F=E(C)$.
    Por outro lado, para todo $c \in C$, temos que $c \in E(c)\subseteq E(C)$.
    Assim, $E(C)\subseteq K$.

    A outra afirmação é similar.
\end{proof}

\begin{lemma}
    Seja $F/E$ uma extensão de corpos e $a \in F$. Então $E[a]=\{p(a): p \in E[x]\}$.
\end{lemma}

\begin{proof}
    Temos que $\{p(a): p \in E[x]\}$ é a imagem de $E[x]$ pela avaliação em $a$, e, portanto, é um subanel de $F$ que claramente contém $E$ e $A$. Pela definição de $p(a)$, é imediato que qualquer subanel de $F$ que contém $E$ e $a$ também contém $\{p(a): p \in E[x]\}$.
\end{proof}
\begin{definition}
Dizemos que uma extensão $F/E$ é finitamente gerada por $A\subseteq F$ se $F=E(A)$.

Dizemos que $F/E$ é finitamente gerada e existe $A\subseteq F$ finito tal que $F=E(A)$.

Finalmente, dizemos que $F/E$ é de grau finito se $[F:E]$ é finito.
\end{definition}
\section{Extensões algébricas}
Nesta seção, discutiremos extensões e elementos algébricos, mencionando elementos transcendentes conforme necessário.
\begin{definition}
Seja $E$ um corpo.
Uma extensão algébrica de $E$ é um corpo $F$ contendo $E$ tal que todo elemento de $F$ é algébrico sobre $E$.
\end{definition}

\begin{prop}
    Seja $F$ um corpo estendendo $E$ e $a\in F$ um elemento algébrico de $E$. Então:
    \begin{enumerate}
        \item $\{p \in E[x]: p(a)=0\}$ é um ideal maximal de $E[x]$.
        \item $\{p \in E[x]: p(a)=0\}$ possui um único gerador mônico $q$, e este é o único polinômio mônico irredutível que anula $a$.
        \item $E[a]=E(a)$.
        \item $E[a]\approx E[x]/(q)$.
    \end{enumerate}
\end{prop}
\begin{proof}
    Seja $I=\{p \in E[x]: p(a)=0\}$. $I$ é o núcleo da avaliação em $a$, e, portanto, é um ideal de $E[x]$.
    Pelo primeiro teorema do isomorfismo, $E[x]/I$ é isomorfo à $E[a]$, que é um subanel de um corpo, e, portanto, um domínio de integridade. Logo, $I$ é primo.
    Como $I$ é também não nulo, $I$ é maximal.
    Logo, $E[x]$ é um corpo, e este é o menor subanel de $F$ contendo $E$ e $a$, logo, é também o menor subcorpo.

    Seja $q$ um gerador de $I$, que podemos supor ser mônico, pois $I$ é um ideal.
    Ele é irredutível, pois $I=(q)$ é primo não nulo, e, portanto, $q$ é primo.
    Se $p$ é um polinômio irredutível que anula $a$, temos que $p=qx$ para algum $x \in E[x]$. Como $p$ é irredutível, $q \in E$ ou $x \in E$. Como $q \notin E$, temos que $x \in E$, e, assim, o coeficiente dominante de $p$ é o de $qx$, que é $x$, logo, $x=1$.
\end{proof}
\begin{corol}
    Sejam $F/E$ e $F'/E$ extensões de corpos. Seja $p\in E[x]$ um polinômio irredutível para o qual $p(a)=0$ e $p(b)=0$, com $a \in F$ e $b \in F'$.
    Então $E[a]$ e $E[b]$ são isomorfos.
\end{corol}
\begin{proof}
    Seja $q$ um polinômio mônico associado à $p$. Temos que $(p)=(q)$. Logo, $E[a]\approx E[x]/(p)\approx E[b]$.
\end{proof}
\begin{prop}
    Seja $F$ uma extensão de um corpo $E$ e $a \in F$ um elemento algébrico sobre $E$. Então $[E[a]:E]$ é o grau do único polinômio mônico irredutível que anula $a$, e uma base é $(1, a, \dots, a^{n-1})$, onde $n=[E[a]:E]$.
\end{prop}
\begin{proof}
    Temos que $(1, a, \dots, a^{n-1})$ é L.I., pois se $t_0, \dots, t_{n-1} \in E$ tais que $t_0+t_1a+\dots+t_{n-1}a^{n-1}=0$, sendo $p(x)=t_0+t_1x+\dots+t_{n-1}x^{n-1}$, temos que $p(a)=0$. Como $\gr p<n$, o único modo de termos $p(a)=0$ é com $p=0$. Logo, $t_0=t_1=\dots=t_{n-1}=0$.

    Agora seja $z \in E[a]$ um elemento arbitrário. Ele é da forma $z=p(a)$, com $p \in E[x]$. Seja $q$ o único polinômio mônico irredutível que anula $a$. Então $p=qa+r$ para alguns $r, a \in E[x]$ com $\gr r<\gr q$
    
    Logo, $z=p(a)=q(a)r(a)+r(a)=r(a)$. Sendo $r(x)=t_0+t_1x+\dots+t_{n-1}x^{n-1}$, temos que $z=r(a)=t_0+t_1a+\dots+t_{n-1}a^{n-1} \in \text{span}(1, a, \dots, a^{n-1})$.

\end{proof}

\begin{prop}
    Seja $F/E$ uma extensão de corpos de grau finito. Então $F/E$ é algébrica e finitamente gerada.
\end{prop}
\begin{proof}
    Seja $n=[F:E]$ e $a \in F/E$. Então $(1, a, a^2, \dots, a_{n})$ é LD, logo existem $a_0, \dots, a_n \in E$, não todos nulos, tais que $a_0+a_1a+\dots+a_na^n=0$, sendo $p(x)=a_0+a_1x+\dots+a_nx^n$, temos que $p(a)=0$.
    Assim, $a$ é algébrico sobre $E$.

    Seja $(b_1, \dots, b_n)$ uma base de $F$ como $E$-espaço vetorial. Então $F=E[b_1, \dots, b_n]$, logo, $F/E$ é finitamente gerada.
\end{proof}


\begin{corol}
    Seja $F/E$ uma extensão e $a \in F$ um elemento algébrico sobre $E$. Então $E(a)/E$ é uma extensão algébrica.
\end{corol}
\begin{proof}
    Segue da proposição anterior, pois $E(a)/E$ tem grau finito.
\end{proof}

Uma recíproca parcial vale:

\begin{prop}
Seja $F/E$ uma extensão algébrica. São equivalentes:

\begin{enumerate}[label=(\roman*)]
    \item $F/E$ é finitamente gerada.
    \item $F/E$ tem grau finito.
\end{enumerate}
\end{prop}
\begin{proof}
    Já vimos que (ii) implica (i). Basta ver que (i) implica (ii).

    Provaremos a proposição por indução no número de geradores de $F=E(a_1, \dots, a_n)$.

    Se $F=E(\emptyset)$, temos que $F=E$.
    Se $F=E(a)$, vimos que $[E(a):E]$ é o grau do polinômio mônico irredutível que anula $a$. Logo, $F/E$ é de grau finito.

    Suponha que a tese vale para $n$ geradores. Então vale para $n+1$:

    Temos que $F=E(a_1, \dots, a_n, a_{n+1})=E(a_1, \dots, a_n)(a_{n+1})$.
    Por hipótese de indução, temos que $[E(a_1, \dots, a_n):E]$ é finito, e, como no caso $n=1$, $[E(a_1, \dots, a_n)(a_{n+1}):E(a_1, \dots, a_n)]$ é finito.
    Logo, $[F:E]=[E(a_1, \dots, a_n):E][E(a_1, \dots, a_n)(a_{n+1}):E(a_1, \dots, a_n)]$ é finito.
\end{proof}



\begin{prop}
    Seja $F/E$ uma extensão de corpos e $A\subseteq F$ uma coleção de elementos algébricos sobre $E$.
    Então $E[A]=E(A)$ e $E(A)/E$ é uma extensão algébrica.
\end{prop}

\begin{proof}
    Primeiro, provaremos a proposição para $A$ finito por indução no número de elementos de $A$.

    Temos que $E[\emptyset]=E(\emptyset)=E$ e $E/E$ é uma extensão algébrica.

    Para o passo indutivo, suponha que a tese vale para conjuntos $A$ de $n$ elementos. Mostraremos que valem para $A$ de $n+1$ elementos.

    Seja $A\subseteq F$ com $n+1$ elementos e seja $B=A\setminus \{a\}$, onde $a \in A$ é arbitrário.
    Por hipótese de indução, $E[B]=E(B)$.
    Seja $L=E(B)$. Como $a$ é algébrico sobre $E$, então $a$ é algébrico sobre $L$.
    Assim, $E[A]=E[B][a]=L[a]=L(a)=E(B)(a)=E(A)$.
    
    Além disso, $[E(B):E]$ é finito por hipótese de indução, e $[E(A):E(B)]$ é finito, logo $[E(A):E]$ é finito.


    Resta provar o teorema caso $A$ seja infinito.
    Temos que $E(A)=\bigcup\{E(B): B \in [A]^{<\infty}\}=\bigcup\{E[B]: B \in [A]^{<\infty}\}=E[A]$.

    $E(A)$ é uma extensão algébrica, pois dado $b \in E(A)$, existe $B\in [A]^{<\infty}$ tal que $b \in E(B)$, e $E(b)$ é uma extensão algébrica.
\end{proof}

\begin{prop}
    Seja $F/K$ e $K/E$ extensões de corpos. Então $F/E$ é uma extensão algébrica se, e somente se $F/K$ e $K/E$ são extensões algébricas.
\end{prop}
\begin{proof}
    Suponha que $F/E$ é uma extensão algébrica. Então $F/K$ é uma extensão algébrica, pois $E[x]\subseteq K[x]$, e $K/F$ é uma extensão algébrica, pois $K\subseteq F$.

    Reciprocamente, suponha que $F/K$ e $K/E$ são extensões algébricas. Fixe $a \in F$.
    Temos que existem $k_0, \dots, k_n \in K$ tais que $k_0+k_1a+\dots+k_na^n=0$. Como $K/E$ é algébrica, sendo $A=\{k_0, \dots, k_n\}$, temos que $E(A)$ é uma extensão algébrica e finitamente gerada de $E$, e, portanto, tem dimensão finita $[E(A):E]$. Como $k_0, \dots, k_n \in E(A)$, temos que $a$ é algébrico sobre $E(A)$, e, portanto, $[E(A)(a):E(A)]$ é finito. Logo, $[E(A)(a):E]$ é finito, o que implica que $a$ é algébrico sobre $E$.
\end{proof}

\begin{prop}
    Seja $F/E$ uma extensão de corpos e $K$ o conjunto dos elementos algébricos de $F$ sobre $E$. Então $K$ é um corpo e todo elemento de $F\setminus K$ é transcendente sobre $K$ (em outras palavras, todo elemento algébrico sobre $K$ está em $K$).
\end{prop}

\begin{proof}
    Temos que $1 \in K$.
    Se $a, b \in K$ e $b \neq 0$, então $E(a, b)$ é uma extensão algébrica de $E$, e $a-b, ab, b^{-1} \in E(a, b)$, logo, $a-b, ab, b^{-1}\in K$.

    Assim, $K$ é um subcorpo de $F$.

    Se $a \in F$ é algébrico sobre $K$, então $a \in K$: temos que $K(a)/K$ é uma extensão algébrica, e $K/F$ é uma extensão algébrica, logo, $K(a)/F$ é uma extensão algébrica. Como $a \in K(a)$, temos que $a$ é algébrico sobre $F$, e, portanto, $a \in K$.
\end{proof}

\section{Elementos transcendentes}
Como uma interseção arbitrária de subcorpos é um subcorpo, $E(a)$ está bem definido e vale que $E[a]\subseteq E(a)$. Vimos que se $a$ é algébrico sobre $E$, então $E(a)=E[a]$. Abaixo, caracterizaremos $E(a)$ quando $a$ é transcendente sobre $E$.
\begin{prop}
        Seja $F$ uma extensão de um corpo $E$ e $a \in F$ um elemento transcendente sobre $E$. Então $E(a)=\{p(a)q(a)^{-1}: p, q \in E[x], q\neq 0\}$ é isomorfo à $\Frac(E[x])$.
\end{prop}
\begin{proof}
    Seja $h:E[x] \rightarrow E(a)$ a avaliação em $a$. Temos que $h$ é um homomorfismo injetor, logo, existe um único homomorfismo injetor $g:\Frac(E(x))\rightarrow E(a)$ tal que $g(p/1)=h(p)=p(a)$ para todo $p \in E[x]$.

    Conforme estudado, $g(p/q)=h(p)h(q)^{-1}$ para todo $p, q \in E[x]$ com $q\neq 0$.
    Logo, a imagem de $g$ é $\{p(a)q(a)^{-1}: p, q \in E[x], q\neq 0\}$, que é um subanel de $E(a)$ isomorfo a $\Frac(E[x])$, e, portanto, um subcorpo de $E(a)\subseteq F$. Como $E(a)$ é o menor subcorpo de $F$ contendo $E$ e $a$, temos que $E(a)=\{p(a)q(a)^{-1}: p, q \in E[x], q\neq 0\}$.
\end{proof}
 
\section{Construtibilidade com Régua e Compasso}


\begin{definition}
    Se $p \in \mathbb R^2$ e $s>0$, denotamos por $\cir(p, s)=\{a \in \mathbb R^2: \|p-a\|=s\}$.

    Se $p, q \in \mathbb R^2$ com $p\neq q$, denotamos por $\reta(p, q)=\{p+t(q-p): t \in \mathbb R\}$ a única reta que passa por $p$ e $q$.

    Finalmente, se $C\subseteq \mathbb R^2$, define-se $\dist(C)=\{\|a-b\|: a, b \in C, a\neq b\}$.
\end{definition}

A definição a seguir será utilizada apenas nesta seção.
\begin{definition}
    Seja $\mathbb{P}_0=\{(0, 0), (1, 0)\}$.
    Definido $\mathbb P_n$, definimos $\mathbb{P}_{n+1}$ como sendo o conjunto dos pontos $p \in \mathbb R^2$ que satisfazem uma das seguintes condições:

    \begin{itemize}
        \item $p \in \mathbb{P}_n$.
        \item $p\in \reta(a, b)\cap \reta(c, d)$, onde $\reta(a, b)\neq \reta(c, d)$ e $a, b, c, d \in \mathbb{P}_n$ com $a\neq b$ e $c\neq d$.
        \item $p\in \cir(a, s)\cap \cir(a', s')$, onde $a, a'\in \mathbb{P}_n$ com $a\neq a'$ e $s, s' \in \dist \mathbb P_n$
        \item $p\in \cir(a, s)\cap \reta(b, c)$, onde $a, b, c \in \mathbb{P}_n$ com $b\neq c$ e $s \in \dist \mathbb P_n$.
    \end{itemize}

    Seja $\mathbb P=\bigcup_{n\in \mathbb N} \mathbb P_n$.
    Seja $\mathbb K=\{x \in \mathbb R: (x, 0) \in \mathbb P\}$.
\end{definition}

\begin{lemma}
   $(0, 1), (1, 1) \in \mathbb P$.
\end{lemma}
\begin{proof}
    Temos que $1 \in \dist(\mathbb P_0)$, $\reta((0, 0), (1, 0))\cap \cir((0, 0), 1)=\{(-1, 0), (1, 0)\}$ e $\reta((0, 1), (1, 0))\cap \cir((1, 0), 1)=\{(0, 0), (2, 0)\}$. Logo, $(2, 0), (0, 1)\in \mathbb P_1$ e $2 \in \dist(\mathbb P_1)$.

    Temos que $(0, \sqrt 3) \in \cir((-1, 0), 2)\cap \cir((1, 0), 2)$, logo, $(0, \sqrt 3) \in  \mathbb P_2$.

    Agora note que $(0, 1) \in \cir((0, 0), 1)\cap \reta((0, 0), (0, sqrt 3))$ e que $(1, 1)\in \cir((1, 0), 1)\cap \cir((0, 1), 1)$.
\end{proof}

\begin{lemma}
   Se $(x, y) \in \mathbb P$, então $(y, x) \in \mathbb P$.
\end{lemma}
\begin{proof}
    Suponha que $(x, y) \in \mathbb P$.
    
    Tome $n$ tal que $(x, y), (1, 1) \in \mathbb P_n$.

    Primeiramente, suponha que $x+y\neq 0$. Seja $s=\|(x, y)\|\neq 0$. Note que $(x+y, x+y)\in \reta((0, 0), (1, 1))\cap \cir((x, y), s)$ e que $(y, x) \in \cir((x+y, x+y), s)\cap \cir((0, 0), s)$.

    Caso $x=-y\neq 0$, seja $s=\|(x, y)\|\neq 0$. Então $(y, x)\in \reta((0, 0), (x, y))\cap \reta(0, 0, (x, y))$.

    Caso $x=-y=0$, é trivial.
\end{proof}

\begin{lemma}
Se $(x, y) \in \mathbb P$, então $(x, 0), (0, y) \in \mathbb P$.
\end{lemma}

\begin{proof}
Podemos supor que $x, y\neq 0$. Seja $s=\|(x, y)\|$. Então $(2x, 0) \in \cir((x, y), s)\cap \reta((0, 0), (1, 0))$ e $(0, 2y) \in \cir((x, y), s)\cap \reta((0, 0), (0, 1))$. Logo, $(x, 0), (0, y)\in \mathbb P$. Finalmente, $(x, 0)\in \cir((0, 0), s)\cap \cir((2x, 0), s)$ e $(0, y)\in \cir((0, 0), s)\cap \cir((0, 2y), s)$.
\end{proof}
\begin{prop}
    Sejam $x, y \in \mathbb R$. São equivalentes:
    \begin{enumerate}[label=(\roman*)]
        \item $(x, y)\in \mathbb P$.
        \item $x, y \in \mathbb K$.
        \item $|x|, |y| \in \dist \mathbb P\cup\{0\}$.
    \end{enumerate}
    
\end{prop}
\begin{proof}
    (i) $\Rightarrow$ (ii): seja $p=(x, y) \in \mathbb P$. Fixe $n$ tal que $(x, y)\in \mathbb P_n$. É claro que $x\in \mathbb K$. Como $(y, x)\in \mathbb P$, segue que $y \in \mathbb K$.

    Caso 1: $x=0$. Nesse caso, $y=\|(0, 0)-(0, y)\|\in \dist() \cir((0, 0), (0, y))$

    (ii) $\Rightarrow$ (i): existem $u, v$ com $(x, v), (u, y) \in \mathbb P$. Logo, $(x, 0)$ e $(0, y)$ estão em $\mathbb P$. Se $x=0$ ou $y=0$, segue a tese. Caso contrário, $|x|, |y|\in \dist(\mathbb P)$. Finalmente, note que $(x, y)\in \cir((x, 0), |y|)\cap \cir((0, y), |x|)$.

    (iii) $\Rightarrow$ (ii): Se $x, y\neq 0$, temos que $(x, 0) \in \circ((0, 0), |x|)\cap \reta((0, 0), (1, 0))$ e $(0, y) \in \cir((0, 0), |y|)\cap \reta((0, 0), (0, 1))$.

    (ii) $\Rightarrow$ (iii): Se $x=0$, então $x \in \mathbb K$. Se $|x| \in \dist \mathbb P$, então $(x, 0)\in \cir((0, 0), |x|)\cap \reta((0, 0), (1, 0))$, logo $x \in \mathbb K$.
\end{proof}

\begin{prop}
    $\mathbb K$ é um subcorpo de $\mathbb R$.
\end{prop}

\begin{proof}
    Temos que $1, 0 \in \mathbb K$.

    Se $a, b \in \mathbb K$ e $a>b$, então $a-b, b-a \in \mathbb K$: temos que $(a, 0), (b, 0) \in \mathbb P$, então $b-a \in \dist(\mathbb P)\subseteq \mathbb K$, e $(a-b, 0) \in \cir((0, 0), b-a)\cap \reta((0, 0), (1, 0))$.

    Note que isso implica que se $a, b \in \mathbb K$, então $b-a\in \mathbb K$. Em particular, $-a \in \mathbb K$.

    Se $a, b \in \mathbb K$ e $b\neq 0$, então $-b \in \mathbb K$, e, portanto, $(a, -b)\in \mathbb P$. Note que $\reta((0, 0), (1, 0))\cap \reta((0, 1), (a, -b))=\{(\frac ab, 0)\}$, logo, $\frac ab \in \mathbb K$.

    Se $a, b \in \mathbb K$, então $ab \in \mathbb K$: se $b=0$, $ab=0\in \mathbb K$. Se $b\neq 0$, temos que $\frac 1b\in \mathbb K$, logo, $\frac{a}{\frac1b}=ab\in \mathbb K$.
\end{proof}


\begin{prop}
    $\mathbb K$ é o menor subcorpo de $\mathbb R$ tal que para todo $c \in \mathbb K$, se $c\geq 0$ então $\sqrt c \in \mathbb K$.
\end{prop}
\begin{prop}
    $\mathbb K$ é um subcorpo de $\mathbb R$ fechado por raízes quadradas de números não negativos.
\end{prop}

\begin{proof}
    Temos que $1, 0 \in \mathbb K$.

    Se $a, b \in \mathbb K$ e $a>b$, então $a-b, b-a \in \mathbb K$: temos que $(a, 0), (b, 0) \in \mathbb P$, então $b-a \in \dist(\mathbb P)\subseteq \mathbb K$, e $(a-b, 0) \in \cir((0, 0), b-a)\cap \reta((0, 0), (1, 0))$.

    Note que isso implica que se $a, b \in \mathbb K$, então $b-a\in \mathbb K$. Em particular, $-a \in \mathbb K$.

    Se $a, b \in \mathbb K$ e $b\neq 0$, então $-b \in \mathbb K$, e, portanto, $(a, -b)\in \mathbb P$. Note que $\reta((0, 0), (1, 0))\cap \reta((0, 1), (a, -b))=\{(\frac ab, 0)\}$, logo, $\frac ab \in \mathbb K$.

    Se $a, b \in \mathbb K$, então $ab \in \mathbb K$: se $b=0$, $ab=0\in \mathbb K$. Se $b\neq 0$, temos que $\frac 1b\in \mathbb K$, logo, $\frac{a}{\frac1b}=ab\in \mathbb K$.

    Finalmente, se $a\geq 0  \in \mathbb K$, então sendo $l=\frac{c+1}{2}$, temos que $(1, \sqrt c)\in \cir((l, 0), l)\cap \reta((1, 0), (1, 1))$, pois $(1, 0)+\sqrt c((1, 1)-(1, 0))=(1, \sqrt c)$, e $(l-1)^2+(\sqrt c)^2=l^2-2l+1+c=l^2-c-1+1+c=l^2$. Assim, $\sqrt c \in \mathbb K$.
\end{proof}
\begin{prop}
    $\mathbb K$ é o menor subcorpo de $\mathbb R$ fechado por raízes quadradas de números não negativos.
\end{prop}
\begin{proof}
    Devemos ver que se $L$ é um subcorpo de $\mathbb R$ tal que para todo $c \in \mathbb K$, se $c\geq 0$ então $\sqrt c \in \mathbb K$, então $\mathbb K\subseteq \mathbb L$.

    Devemos ver que para todo $(x, y)\in \mathbb P$, $x \in L$. Para tanto, mostraremos que para todo $n$ e para todo $(x, y)\in \mathbb P_n$, $x \in L$.

    Temos que $\mathbb P_0=\{(0, 0), (1, 0)\}$. Como $0, 1 \in L$, a tese vale para a base $n=0$.

    Suponha que a tese vale para $n$. Veremos que vale para $n+1$.

    Seja $(x, y) \in \mathbb P_{n+1}$.

    \textbf{Caso 1:} $(x, y)\in \mathbb P_n$. Nesse caso, $x \in L$ por hipótese de indução.

    \textbf{Caso 2:} $(x, y)\in \reta(a, b)\cap \reta(c, d)$, onde $\reta(a, b)\neq \reta(c, d)$ e $a, b, c, d \in \mathbb P_n$ com $a\neq b$ e $c\neq d$. Por hipótese, as coordenadas de $a, b, c, d$ estão em $L$.

    Nesse caso, existe $t \in \mathbb R$ tal que $(x, y)=a+tb=c+td$, e, portanto, $a-c=t(d-b)$. Se $d=b$, segue que $a=c$, logo, $\reta(a, b)= \reta(c, d)$, o que é absurdo. Logo, $d\neq b$.

    Escreva $a=(a_1, a_2)$, $b=(b_1, b_2)$, $c=(c_1, c_2)$ e $d=(d_1, d_2)$. Como $b\neq d$, temos $b_1\neq d_1$ ou $b_2\neq d_2$. Se $b_1\neq d_1$, segue que $\frac{a_1-c_1}{b_1-d_1}=t\in L$, e, portanto, $x=a_1+tb_1\in L$. Analogamente, se $b_2\neq d_2$, $\frac{a_2-c_2}{b_2-d_2}$, e $x=a_1+tb_1\in L$.

    \textbf{Caso 3:} $(x, y)\in \reta(a, b)\cap \cir(c, s)$, onde $a, b \in \mathbb P_n$ com $a\neq b$ e $c \in \mathbb P_n$ com $s \in \dist(\mathbb P_n)$. Sejam $a=(a_1, a_2)$, $b=(b_1, b_2)$ e $c=(c_1, c_2)$.

    Nesse caso, existe $t \in \mathbb R$ tal que $(x, y)=a+tb$ e $\|a+tb-c\|=s$. Isso ocorre se, e somente se $(a_1+tb_1-c_1)^2+(a_2+tb_2-c_2)^2=s^2$.
    Da fórmula quadrática, $t$, que existe, é uma combinação de somas, diferenças, produtos e raízes quadradas de $a_1, b_1, c_1, a_2, b_2, c_2, s^2$, que são todos, por hipótese de indução, elementos de $L$. Logo, $t \in L$. Assim, $x=a_1+tb_1\in L$.

    \textbf{Caso 4:} $(x, y)\in \cir(a, s)\cap \cir(b, s')$, onde $a, b \in \mathbb P_n$ com $a\neq b$ e $s, s' \in \dist(\mathbb P_n)$. Sejam $a=(a_1, a_2)$ e $b=(b_1, b_2)$.
    Nesse caso, $(x-a_1)^2+(y-a_2)^2=s^2$ e $(x-b_1)^2+(y-b_2)^2=s'^2$. Da primeira equação, $x-a_1=\sqrt(s^2-(y-a_2)^2)$ ou $x-a_1=-\sqrt(s^2-(y-a_2)^2)$. Em qualquer caso, é claro que $x \in L$.
\end{proof}


\begin{lemma}
    $\mathbb K=\bigcup_{n \in \mathbb N} K_n$, onde $(K_n: n \in \mathbb N)$ é a sequência de conjuntos definida recursivamente por $K_0=\{1\}$ e $K_{n+1}=K_n\cup\{a-b, ab, a^{-1}: a, b \in K_n\}\cup\{b^{-1}: b\in K_n\setminus\{0\}\}\cup \{\sqrt b: b\geq 0, b \in \mathbb K_n\}$.
\end{lemma}

\begin{proof}
    Por indução, em $m$, vemos que para todo $m, n$ com $m\leq n$, $K_m\subseteq K_n$.

    Também por indução, vemos que se $L$ é um subcorpo de $\mathbb R$ fechado por raízes quadradas de números não negativos, então para todo $n\in \mathbb N$, $K_n\subseteq L$.

    Resta ver que $K=\bigcup_{n \in \mathbb N}K_n$ é um subcorpo de $\mathbb R$. Para tanto, note que $1 \in K$. Se $b\in K\setminus \{0\}$, então existe $n \in K$ tal que $b\in K_n$. Logo, $b^{-1} \in K_{n+1}$. Similarmente, se- $b\geq 0$ e $b \in K$, então, $\sqrt b \in K$. Finalmente, se $a, b \in K$, existem $n, m$ com $a\in K_n$ e $b\in K_m$. Sendo $k=\max(n, m)$, temos que $a-b, ab \in K_{k+1}$.     
\end{proof}

\begin{prop}
    Todo elemento de $\mathbb K$ tem grau sobre $\mathbb Q$ igual a $2^i$ para algum $i\in \mathbb N$. Mais especificamente, se $a \in \mathbb K$, então existem $(a_0, \dots, a_n)\in \mathbb K$ com $a=a_n$, com $[\mathbb Q(a_0): \mathbb Q]\in \{1, 2\}$ e $[\mathbb Q(a_0, \dots, a_i)(a_{i+1}):\mathbb Q(a_0, \dots, a_i)]\in \{0, 1\}$ para todo $i<n$
\end{prop}
\begin{proof}
    Primeiro, note que a primeira parte da proposição é implicada pela segunda, já que dado $\mathbb a \in \mathbb K$, se existem $a_0, \dots, a_n\in \mathbb K$ com $a=a_n$, com $[\mathbb Q(a_0): \mathbb Q]\in \{1, 2\}$ e $[\mathbb Q(a_0, \dots, a_i)(a_{i+1}):\mathbb Q(a_0, \dots, a_i)]\in \{0, 1\}$ para todo $i<n$, então $[\mathbb Q(a_0, \dots, a_n):\mathbb Q]$ é uma potência de $2$, e $[\mathbb Q(a):\mathbb Q]$ divide esta potência.

    Para provar a segunda afirmação, procederemos por indução em $n$ para mostrar que para todo $a \in K_n$, existem $a_0, \dots, a_n\in K_n$ com $a=a_n$, com $[\mathbb Q(a_0): \mathbb Q]\in \{1, 2\}$ e $[\mathbb Q(a_0, \dots, a_i)(a_{i+1}):\mathbb Q(a_0, \dots, a_i)]\in \{0, 1\}$ para todo $i<n$.

    Base: que $K_0=\{1\}$, e $[\mathbb Q(1):\mathbb Q]=1$. Considere a sequência unitária $(1)$.

    Passo indutivo: suponha que a hipótese é válida para $n$. Mostraremos que ela é válida para $n+1$.

    Seja $c \in K_{n+1}$. Se $a \in K_n$, segue da hipótese indutiva.

    Caso $c=a-b$ ou $c=ab$, com $a, b \in K_n$, por hipótese indutiva, existem $a_0, \dots, a_n\in K_n$ com $a=a_n$, com $[\mathbb Q(a_0): \mathbb Q]\in \{1, 2\}$ e $[\mathbb Q(a_0, \dots, a_i)(a_{i+1}):\mathbb Q(a_0, \dots, a_i)]\in \{0, 1\}$ para todo $i<n$. Similarmente para $b$, existem $b_0, \dots, b_m\in K_n$ com $b=b_m$, com $[\mathbb Q(b_0): \mathbb Q]\in \{1, 2\}$ e $[\mathbb Q(b_0, \dots, b_i)(b_{i+1}):\mathbb Q(b_0, \dots, b_i)]\in \{0, 1\}$ para todo $i<m$.

    Considere $(a_0, \dots, a_n, b_0, \dots, b_m, c)$. Temos que $[\mathbb Q(a_0, \dots, a_n)(b_0):\mathbb Q(a_0, \dots, a_n)]$ é o menor grau de um polinômio não nulo com coeficientes em $\mathbb Q(a_0, \dots, a_n)$ que anula $b_0$. Ora, como $[\mathbb Q(b_0):\mathbb Q]$ é $1$ ou $2$, segue que $[\mathbb Q(a_0, \dots, a_n)(b_0):\mathbb Q(a_0, \dots, a_n)]$ também é $1$ ou $2$.
    Similarmente, $[\mathbb Q(a_0, \dots, a_n, b_0, \dots, b_{j+1}):\mathbb Q(a_0, \dots, a_n, b_0, \dots, b_{j})]$ é $1$ ou $2$.
    Finalmente, note que $[\mathbb Q(a_0, \dots, a_n, b_0, \dots, b_m)(c):\mathbb Q(a_0, \dots, a_n, b_0, \dots, b_m)]=1$,  pois $c \in \mathbb Q(a_0, \dots, a_n, b_0, \dots, b_m)$.

    Se $c=a^{-1}$ (ou $c=\sqrt a$), com $a\in K_n\setminus\{0\}$ (respectivamente, $\sqrt a=c$). Por hipótese indutiva, existem $a_0, \dots, a_n\in K_n$ com $a=a_n$, com $[\mathbb Q(a_0): \mathbb Q]\in \{1, 2\}$ e $[\mathbb Q(a_0, \dots, a_i)(a_{i+1}):\mathbb Q(a_0, \dots, a_i)]\in \{0, 1\}$ para todo $i<n$. Considere $a_{n+1}=a^{-1}$. Então, caso $c=a^{-1}$ $[\mathbb Q(a_0, \dots, a_n)(a_{n+1}):\mathbb Q(a_0, \dots, a_n)]=1$, pois $a_{n+1}\in \mathbb Q(a_0, \dots, a_n)$, e, caso $c=\sqrt a$, temos que $x^2-a$ anula $c$, logo $[\mathbb Q(a_0, \dots, a_n)(a_{n+1}):\mathbb Q(a_0, \dots, a_n)]$ é $1$ ou $2$.
\end{proof}

\begin{definition}
Dizemos que um ângulo $\alpha$ é construtível com régua e compasso se existem $a, b, c \in \mathbb P$ tais que a medida do ângulo $\angle{abp}$ é $\alpha$.
\end{definition}

Da geometria analítica, temos que:
\begin{prop}
    Um ângulo $\alpha$ é construtível com régua e compasso se, e somente se $(\cos \alpha, \sin \alpha) \in \mathbb P$.
\end{prop}
Disso, segue que:
\begin{prop}
    Um ângulo $\alpha$ é construtível com régua e compasso se, e somente se $\cos \alpha \in \mathbb K$.
\end{prop}
\begin{proof}
Se $\cos \alpha \in \mathbb K$, temos que $\sin \alpha=\sqrt(1+\cos \alpha^2)$ ou $\sin \alpha=-\sqrt(1+\cos \alpha^2)$. Ambos esses números estão em $\mathbb K$.
\end{proof}

\begin{prop}
    É impossível trissectar, com régua e compasso, um ângulo de $60$ graus ($\pi/3$).
\end{prop}
\begin{proof}
    Da trigonometria, sabemos que para todo $\theta, \theta'$, temos que $\cos(\theta+\theta')=\cos(\theta)\cos(\theta')-\sin(\theta)\sin(\theta')$ e $\sin(\theta+\theta')=\sin(\theta)\cos(\theta')+\cos(\theta)\sin(\theta')$.

    Pondo $\theta'=2\theta$ na primeira equação, segue que:
    \begin{align*}
        \cos(3\theta)=&\cos(\theta)\cos(2\theta)-\sin(\theta)\sin(2\theta) \\
        =&\cos(\theta)(\cos^2(\theta)-\sin^2(\theta))-\sin(\theta)(2\sin(\theta)\cos(\theta)) \\
        =&\cos(\theta)(\cos^2(\theta)-\sin^2(\theta)-2\sin^2(\theta)) \\
        =&\cos(\theta)(\cos^2(\theta)-3\sin^2(\theta)) \\
        =&\cos(\theta)(\cos^2(\theta)-3(1-\cos^2(\theta))) \\
        =&\cos(\theta)(4\cos^2(\theta)-3) \\
        =&4\cos^3(\theta)-3\cos(\theta).
    \end{align*}    

    Pondo $\theta=\pi/9$, segue que $\cos(\pi/3)=\frac{1}{2}=4\cos^3(\pi/9)-3\cos(\pi/9)$. Assim, $p(x)=8x^3-6x-1$ anula $\cos(\pi/6)$, e, portanto, $\cos(\pi/9)$ é algébrico sobre $\mathbb Q$.
    $p(x)$ é irredutível sobre $\mathbb Q$: se não for, ele possui um divisor de grau $1$ (pois seu grau é $3$). Porém, se $a, b \in \mathbb Z$, $b\neq 0$ e $1\in \MDC(a, b)$, então $a|1$ e $b|8$. Note que nenhum elemento de $\{1, -1, \frac{1}{2}, -\frac{1}{2}, \frac{1}{4}, -\frac{1}{4}, \frac{1}{8}, -\frac{1}{8}\}$ é raiz de $p(x)$, logo, $p(x)$ não possui raízes racionais.

    Como $\cos(\pi/9)$ é algébrico sobre $\mathbb Q$ de grau $3$ e $3$ não é uma potência de $2$, segue que $\cos(\pi/9)$ não está em $\mathbb K$. Logo, $\pi/9$ não é construtível com régua e compasso.
\end{proof}
\begin{prop}
    É impossível construir, com régua e compasso, um polígono regular de $18$ lados.
\end{prop}
\begin{proof}
    Os ângulos internos de tal polígono tem $160$ graus. Logo, o externo tem $20$ graus. Como vimos, ângulos de $20$ graus não são construtíveis com régua e compasso.
\end{proof}

\begin{prop}
    É impossível construir, com régua e compasso, um quadrado com a área de um círculo de raio $1$.
\end{prop}
\begin{proof}
Tal cículo tem área $\pi$. Um quadrado com área $\pi$ tem lado $\sqrt \pi$. Porém, $\sqrt \pi$ não está em $\mathbb K$, pois é transcendente (caso contrário $\pi$ seria algébrico).
\end{proof}

\begin{prop}
É impossível construir, com régua e compasso, a base de um cubo de volume $2$.
\end{prop}
\begin{proof}
    Para tanto, teríamos uma reta de comprimento $\sqrt[3]{2}$, número anulado por $x^3-2$, que é irredutível sobre $\mathbb Q$ (pois não possui raízes racionais). Logo, $\sqrt[3]{2}$ não está em $\mathbb K$, pois $[\mathbb Q(\sqrt[3]{2}):\mathbb Q]=3$ e $3$ não é uma potência de $2$.
\end{proof}
\section{Exercícios}
\begin{exer}
    Seja $F$ um corpo estendendo $E$ e $a \in F$ um elemento transcendente sobre $E$. Mostre que $E[a]$ é isomorfo ao anel de polinômios $E[x]$.
\end{exer}

\begin{exer}
    Prove ou dê um contra-exemplo: se $F/E$ é uma extensão algébrica e $K$ é um corpo intermediário entre $E$ e $F$, então $F/K$ e $K/E$ são extensões algébricas.
\end{exer}
