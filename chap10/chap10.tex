\chapter{Extensão de corpos}

\section{Elementos algébricos e transcendentes}
\begin{definition}
Dizemos que um corpo $K$ estende um corpo $L$ se $L\subseteq K$ e $L$ é um subanel (subcorpo) de $K$. Escrevemos que $K/L$ é uma extensão de corpos.
\end{definition}

Lembremos que, da álgebra linear, se $K, L$ são corpos e $K$ estende $L$ então $K$ é um $L$-espaço vetorial:

\begin{prop}
    Sejam $K$ um corpo estendendo $L$. Então $K$ é, com a multiplicação usual, um $L$-espaço vetorial.
\end{prop}

\begin{proof}
    $(K, +)$ é claramente um grupo abeliano. Além disso, se $u, v \in L$ e $a, b \in K$, temos que $u(a+b)=ua+ub$, que $(u+v)a=ua+uv$, que $1a=a$ e que $(uv)a=u(va)$.
\end{proof}

\begin{definition}
Na notação anterior, a dimensão de $K$ como $L$-espaço vetorial é denotada por $[K:L]$.
\end{definition}

\begin{lemma}
Seja $E, F, K$ corpos com $K$ estendendo $F$ e $F$ estendendo $K$. Então $[K:E]$ é finito se, e somente se $[K:F]$ e $[F:E]$ são finitos, e, nesse caso, $[K:E]=[K:F][F:E]$.
\end{lemma}

\begin{proof}
    Primeiro, suponha que $[K:F]$ e $[F:E]$ são finitos. Seja $\mathcal B=(b_i: 1\leq i\leq n)$ uma base de $K$ como $F$-espaço vetorial e $\mathcal C=(c_j: 1\leq j\leq m)$ uma base de $F$ como $E$-espaço vetorial.

    Seja $\mathcal D=(b_ic_j: 1\leq i\leq n, 1\leq j\leq m)$. Afirmo que $\mathcal D$ é $E$-base de $K$.

    $\mathcal D$ é L.I.: com efeito, sejam $x_{ij}\in E$ tais que $\sum\left(x_{ij}b_ic_j: 1\leq i\leq n, 1\leq j\leq m\right)=0$. Então $\sum_{i=1}^n\left(\sum_{j=1}^mx_{ij}c_j\right)b_i=0$. Como $\mathcal B$ é L.I., cada $\sum_{j=1}^m x_{ij}c_j=0$. Como $\mathcal C$ é L.I., cada $x_{ij}=0$. Portanto, $\mathcal D$ é L.I.

    $\mathcal D$ é gerador: dado $k \in K$, existem $y_1, \dots, y_n$ tais que $k=\sum_{i=1}^n y_i b_i$. Como $\mathcal C$ é gerador, existem $x_{ij}\in E$ tais que $y_i=\sum_{j=1}^m x_{ij}c_j$ para cada $1\leq i\leq k$. Portanto, $k=\sum_{i=1}^n\left(\sum_{j=1}^mx_{ij}c_j\right)b_i=\sum_{i=1}^n\sum_{j=1}^mx_{ij}c_jb_i$, o que mostra que $\mathcal D$ é gerador.

    Note que $\mathcal D$ é uma família de tamanho $mn=[K:F][F:E]$.

    Reciprocamente, suponha que $[K:E]$ é finito. Como $F$ é um subespaço de $K$ como $E$-espaço vetorial, temos que $[F:E]\leq [K:E]$. Ademais, uma base de $K$ como $E$-espaço vetorial é um gerador de $K$ como $F$-espaço vetorial, e, assim, esta contém uma base (finita). Logo, $[K:F]\leq [K:E]$.
\end{proof}

\begin{definition}
Seja $F$ um corpo estendendo $E$ e $a \in F$. Dizemos que $a \in F$ é um elemento algébrico sobre $E$ se existe um polinômio não nulo $p \in E[x]$ tal que $p(a)=0$. Caso contrário, $a$ é dito transcendente sobre $E$.
\end{definition}

\begin{exemplo}
    Todo elemento de $a \in E$ é algébrico sobre $E$, pois é raiz de $x-a$.
\end{exemplo}

\begin{exemplo}
    Considere $\mathbb Q\subseteq \mathbb C$. Temos que $i$ é algébrico sobre $\mathbb Q$, pois é raiz de $x^2+1$.
\end{exemplo}

\begin{exemplo}
    Se $K$ é um corpo, seja $K(x)$ o corpo de frações do domínio $K[x]$. Identificando $K\subseteq K[x]\subseteq K(x)$, temos que $x \in K(x)$ é transcendente sobre $K$.
\end{exemplo}

\begin{definition}
    Seja $F$ uma extensão de um corpo $E$ e $a \in F$.

    Define-se $E[a]=\{p(a): p \in E[x]\}$.
\end{definition}

Observação: na definição acima, $E[a]$ e $E(a)$ dependem também de $F$, apesar de tal fato não estar explícito na notação.
\begin{lemma}
    Na notação acima, $E[a]$ é o menor subanel de $F$ contendo $E$ e $a$.
\end{lemma}
\begin{proof}
    Temos que $E[a]$ é a imagem de $E[x]$ pela avaliação em $a$, e, portanto, é um subanel de $F$.
    Qualquer subanel de $F$ contendo $E, a$ contém $E[a]$, pois se $S$ é um subanel de $F$ contendo $E, a$, então dado $p\in E[x]$, sendo $p(x)=\sum_{i=0}^n a_ix^i$, temos que $p(a)=\sum_{i=0}^n a_i a^i \in S$.
\end{proof}
\begin{prop}
    Seja $F$ um corpo estendendo $E$ e $a\in F$ um elemento algébrico de $E$. Então:

    \begin{enumerate}
        \item $\{p \in E[x]: p(a)=0\}$ é um ideal maximal de $E[x]$.
        \item $\{p \in E[x]: p(a)=0\}$ possui um único gerador mônico $q$, e este é o único polinômio mônico irredutível que anula $a$.
        \item $E[a]$ é o menor subcorpo de $F$ contendo $E$ e $a$.
        \item $E[a]\approx E[x]/(q)$.
    \end{enumerate}
\end{prop}
\begin{proof}
    Seja $I=\{p \in E[x]: p(a)=0\}$. $I$ é o núcleo da avaliação em $a$, e, portanto, é um ideal de $E[x]$.
    Pelo primeiro teorema do isomorfismo, $E[x]/I$ é isomorfo à $E[x]$, que é um subanel de um corpo, e, portanto, um domínio de integridade. Logo, $I$ é primo.
    Como $I$ é também não nulo, $I$ é maximal.
    Logo, $E[x]$ é um corpo, e este é o menor subanel de $F$ contendo $E$ e $a$, logo, é também o menor subcorpo.

    Seja $q$ um gerador de $I$, que podemos supor ser mônico pois $I$ é um ideal.
    Ele é irredutivel, pois $I=(q)$ é primo não nulo, e, portanto, $q$ é primo.
    Se $p$ é um polinômio irredutível que anula $a$, temos que $p=qx$ para algum $x \in E[x]$. Como $p$ é irredutível, $q \in E$ ou $x \in E$. Como $q \notin E$, temos que $x \in E$, e, assim, o coeficiente dominante de $p$ é o de $qx$, que é $x$, logo, $x=1$.
\end{proof}
\begin{corol}
    Sejam $F/E$ e $F'/E$ extensões de corpos. Seja $p\in E[x]$ um polinômio irredutível para o qual $p(a)=0$ e $p(b)=0$, com $a \in F$ e $b \in F'$.
    Então $E[a]$ e $E[b]$ são isomorfos.
\end{corol}
\begin{proof}
    Seja $q$ um polinômio mônico associado à $p$. Temos que $(p)=(q)$. Logo, $E[a]\approx E[x]/(p)\approx E[b]$.
\end{proof}

\begin{definition}
    Seja $F$ uma extensão de um corpo $E$ e $a \in F$ um elemento algébrico sobre $E$. Então $[E[a]:E]$ é o grau do único polinômio mônico irredutível que anula $a$, e uma base é $(1, a, \dots, a^{n-1})$, onde $n=[E[a]:E]$.
\end{definition}
\begin{proof}
    Temos que $(1, a, \dots, a^{n-1})$ é L.I., pois se $t_0, \dots, t_{n-1} \in E$ tais que $t_0+t_1a+\dots+t_{n-1}a^{n-1}=0$, sendo $p(x)=t_0+t_1x+\dots+t_{n-1}x^{n-1}$, temos que $p(a)=0$. Como $\gr p<n$, o único modo de termos $p(a)=0$ é com $p=0$. Logo, $t_0=t_1=\dots=t_{n-1}=0$.

    Agora seja $z \in E[a]$ um elemento arbitrário. Ele é da forma $z=p(a)$, com $p \in E[x]$. Seja $q$ o único polinômio mônico irredutível que anula $a$. Então $p=qa+r$ para alguns $r, a \in E[x]$ com $\gr r<\gr q$
    
    Logo, $z=p(a)=q(a)r(a)+r(a)=r(a)$. Sendo $r(x)=t_0+t_1x+\dots+t_{n-1}x^{n-1}$, temos que $z=r(a)=t_0+t_1a+\dots+t_{n-1}a^{n-1} \in \text{span}(1, a, \dots, a^{n-1})$.

\end{proof}
Como uma interseção arbitrária de subcorpos é um subcorpo, $E(a)$ está bem definido e vale que $E[a]\subseteq E(a)$. Vimos que se $a$ é algébrico sobre $E$, então $E(a)=E[a]$. Abaixo, caracterizaremos $E(a)$ quando $a$ é transcendente sobre $E$.
\begin{prop}
        Seja $F$ uma extensão de um corpo $E$ e $a \in F$ um elemento transcendente sobre $E$. Então $E(a)=\{p(a)q(a)^{-1}: p, q \in E[x], q\neq 0\}$ é isomorfo à $\Frac(E[x])$.
\end{prop}
\begin{proof}
    Seja $h:E[x] \rightarrow E(a)$ a avaliação em $a$. Temos que $h$ é um homomorfismo injetor, logo, existe um único homomorfismo injetor $g:\Frac(E(x))\rightarrow E(a)$ tal que $g(p/1)=h(p)=p(a)$ para todo $p \in E[x]$.

    Conforme estudado, $g(p/q)=h(p)h(q)^{-1}$ para todo $p, q \in E[x]$ com $q\neq 0$.
    Logo, a imagem de $g$ é $\{p(a)q(a)^{-1}: p, q \in E[x], q\neq 0\}$, que é um subanel de $E(a)$ isomorfo a $\Frac(E[x])$, e, portanto, um subcorpo de $E(a)\subseteq F$. Como $E(a)$ é o menor subcorpo de $F$ contendo $E$ e $a$, temos que $E(a)=\{p(a)q(a)^{-1}: p, q \in E[x], q\neq 0\}$.
\end{proof}
 
\begin{definition}
    Seja $E$ um corpo e $E[x]$ seu anel de polinômios. Define-se $E(x)=\Frac(E[x])$, o corpo de frações de $E[x]$.
\end{definition}

\begin{prop}
    Seja $F$ uma extensão de um corpo $E$ e $a \in F$ um elemento transcendente sobre $E$. Então $[E(a):E]$ é infinito.
\end{prop}
\begin{proof}
    Basta ver que $(1, a, a^2, \dots)$ é L.I. em $E(a)$, uma vez que para todos $s_0, \dots, s_n \in E$, se $\sum_{i=0}^n s_ia^i=0$, temos que $p(x)=s_0+s_1x+\dots+s_nx^n$ é um polinômio com $p(a)=0$, o que nos dá $p=0$ e, portanto, $s_0=s_1=\dots=s_n=0$.
\end{proof}
\section{Extensões algébricas}
\begin{definition}
Seja $E$ um corpo. Uma extensão algébrica de $E$ é um corpo $F$ contendo $E$ tal que todo elemento de $F$ é algébrico sobre $E$.
\end{definition}
\section{Exercícios}
\begin{exer}
    Seja $F$ um corpo estedendo $E$ e $a \in F$ um elemento transcendente sobre $E$. Mostre que $E[a]$ é isomorfo ao anel de polinômios $E[x]$.
\end{exer}
