
\chapter{Anéis e subanéis}
Nesta seção, começaremos a discutir a noção matemática de anel, uma das principais estruturas que serão estudadas.
\begin{definition}[Anel]
    Um anel é uma $4$-upla $(A, +, \cdot, 0, 1)$ conjunto $A$ com duas operações binárias, adição e multiplicação, denotadas por $+$ e $\cdot$, tais que:
    \begin{itemize}
        \item $(A, +, 0)$ é um grupo abeliano.
        \item (\textbf{Associatividade}) Para todo $a, b \in A$, temos $(a \cdot b)\cdot c = a\cdot(b\cdot c)$.
        \item (\textbf{Elemento identidade}) $\forall a \in A$ $1 \cdot a = a \cdot 1 = a$.
        \item (\textbf{Propriedades distributivas}) Para todos $a, b, c, \in A$, temos:
        \begin{align*}
            a \cdot (b + c) &= a \cdot b + a \cdot c, \text{ e}\\
            (a + b) \cdot c &= a \cdot c + b \cdot c
        \end{align*}
    \end{itemize}
    Se, adicionalmente, a seguinte propriedade é satisfeita, o anel é chamado de \emph{comutativo}.
    \begin{itemize}
        \item (\textbf{Comutatividade}) $\forall a, b \in A$ $a \cdot b = b \cdot a$.
    \end{itemize}
\end{definition}

Algumas observações:
\begin{itemize}
    \item Como em grupos, ao discursar sobre anéis é comum omitir as operações, referindo-se apenas ao conjunto $A$.
    \item Ao discursar sobre anéis, e a exemplo do que foi feito ao enunciar as propriedades distributivas, são utilizadas as convenções usuais sobre precedência de operações envolvidas por parênteses.
    Assim, $a + b \cdot c$ é interpretado como $a + (b \cdot c)$.
    \item Há textos que definem anéis sem incluir o elemento identidade $1$.
    Nestes textos, a definição acima dá nome ao que chamam de \emph{anéis com identidade}, ou \emph{anéis com 1}.
    Nesse curso, não usaremos essa convenção, de modo que \textbf{todos nossos anéis possuem identidade}.
    De modo similar, alguns textos definem anéis como sendo comutativos. Também não adotaremos essa convenção.
    \textbf{Os nossos anéis podem ser não comutativos}.
    \item A definição de anel não exige que $0=1$.
    \item $0$ é chamado de elemento nulo, e $1$ de elemento identidade.
\end{itemize}

\begin{prop}[Propriedade multiplicativa do $0$]
    Seja $A$ um anel.
    Então $\forall a \in A$ $0 \cdot a = a \cdot 0 = 0$.
\end{prop}
\begin{proof}
Provaremos a primeira afirmação.
A segunda é análoga e fica como exercício.

Temos que $0\cdot a=(0+0)\cdot a=0\cdot a +0\cdot a$.
Cancelando, segue que $0=0\cdot a$.
\end{proof}

\begin{prop}[Anel trivial]
    Seja $A={x}$ um conjunto qualquer.
    Defina $x\cdot x=x=x+x=0=1$.
    Então $(A, +, \cdot, 0, 1)$ é um anel.
    Um anel dessa forma é chamado de \emph{anel trivial}.
    
    Além disso, se $A$ é um anel tal que $0=1$, então $A$ é um anel trivial.
\end{prop}
\begin{proof}
    A primeira afirmação (de que $A$ como acima é um anel) fica como exercício.

    Para a segunda afirmação, assuma que $A$ é um anel tal que $0=1$.
    Fixe $a \in A$ qualquer.
    Então $a=a\cdot1=a\cdot0=0$, ou seja, $a=0$.
    Assim, $A$ é o conjunto unitário $\{0\}$, que é um anel trivial.
\end{proof}

\begin{prop}[Regras de sinal II]\label{prop:regraSinal2}
    Seja $A$ um anel e $a, b \in A$. Então:
    \begin{enumerate}[label=\alph*)]
        \item $(-a)b=a(-b)=-(ab)$\label{prop:regraSinal2_A}
        \item $(-a)(-b)=ab$.\label{prop:regraSinal2_B}
        \item $(-1)a=-a$.\label{prop:regraSinal2_C}
    \end{enumerate}
\end{prop}
\begin{proof}
    \ref{prop:regraSinal2_A}: Temos que $ab+(-a)b=(-a)b+ab=[-a+a]b=0b=0$. Assim, $(-a)b=-(ab)$. Analogamente, $a(-b)=-(ab)$.

    \ref{prop:regraSinal2_B}: Temos que $(-a)(-b)=-[a(-b)]=-[-(ab)]=ab$ pela regra anterior.

    \ref{prop:regraSinal2_C}: Temos que $(-1)a=-(1a)=-a$.
\end{proof}

\section{Elementos invertíveis}
\begin{definition}[Elemento invertível]
    Seja $A$ um anel.
    Um elemento $a \in A$ é dito \emph{invertível}, ou uma \emph{unidade} se $\exists b \in A$ tal que $a \cdot b = b \cdot a = 1$.
    
    O conjunto de todas das unidades de $A$ é denotado por $A^*$.
\end{definition}

\begin{definition}
    Seja $A$ um anel.
    Então, se $a \in A^*$, existe um \textbf{único} $b \in A$ tal que $a \cdot b = b \cdot a = 1$. Este elemento é denotado por $a^{-1}$, e é chamado de \emph{inverso} de $a$.
\end{definition}

Observação: para que a definição acima faça sentido, é necessário mostrar que se $a$ é unidade, existe um \textbf{único} $b \in A$ tal que $a \cdot b = b \cdot a = 1$.
A existência é garantida pela definição de unidade, e a demonstração da unicidade é análoga à da unicidade do inverso em grupos (Proposição \ref{prop:inverso_unico_grupo}), ficando como exercício.

\begin{prop}
Seja $A$ um anel. Para todos $a, b \in A^*$, temos:
\begin{enumerate}[label=\alph*)]
    \item $ab\in A^U$ e $(ab)^{-1}=b^{-1}a^{-1}$.\label{prop:unidadeProduto_a}
    \item $a^{-1}\in A^U$ e $(a^{-1})^{-1}=a$.\label{prop:unidadeProduto_b}
    \item $1^{-1}=1$.\label{prop:unidadeProduto_c}
\end{enumerate}
Além disso, $A^*$ é, com a restrição da operação de multiplicação do anel, um grupo com identidade $1$. Caso $A$ seja abeliano, $A^*$ é um grupo abeliano.
\end{prop}
\begin{proof}
    \ref{prop:unidadeProduto_a}: Sejam $a, b \in A^*$. Pela associatividade, $(ab)(b^{-1}a^{-1})=1=(b^{-1}a^{-1})(ab)$, logo, pela unicidade do inverso, $(ab)^{-1}=b^{-1}a^{-1}$.

    \ref{prop:unidadeProduto_b}: Seja $a \in A^*$. Temos que $a^{-1}a=1=a(a^{-1})$, logo, pela unicidade do inverso, $(a^{-1})^{-1}=a$.

    \ref{prop:unidadeProduto_c}: Note que $1\cdot 1=1=1\cdot 1$, logo, pela unicidade do inverso, $1^{-1}=1$.

    A última afirmação é imediata e fica como exercício.
\end{proof}

Abaixo, segue a definição de anel de divisão e corpo.
A noção de corpo será uma das noções mais importantes deste texto.
\begin{definition}[Anel de divisão]
Um \emph{anel de divisão} é um anel não trivial para o qual todo elemento não nulo é invertível.
Um \emph{corpo} é um anel de divisão comutativo.
\end{definition}

\begin{exer}
    Mostre que um anel $A$ é um anel de divisão se, e somente se $A^*=A\setminus\{0\}$.
\end{exer}
\begin{definition}
    Um domínio de integridade é um anel comutativo não trivial $A$ tal que $\forall a, b \in A$, se $ab=0$, então $a=0$ ou $b=0$.
\end{definition}

\begin{prop}
    Seja $K$ um corpo.
    Então $K$ é um domínio de integridade.
\end{prop}
\begin{proof}
Sabemos que $K$ é um anel comutativo não trivial.
Sejam $a, b \in K$ tais que $ab=0$.
Se $a=0$, então segue a tese.
Caso contrário, como $K$ é um corpo, $a^{-1}$ existe.
Assim, temos que $b=(a^{-1}a)b=a^{-1}(ab)=0$, logo, $b=0$.
\end{proof}


\section{Subanéis}
Em Matemática, é comum que as estruturas estudadas possuam uma noção de subestrutura.
Em geral, uma subestrutura de uma estrutura data é um subconjunto desta que seja, de forma natural, uma estrutura da mesma natureza daquela.

Veremos que, quando tratamos de anéis, nem todo subconjunto pode ser visto como uma subestrutura.

\begin{definition}[Subanel]
    Seja $A$ um anel e $B \subseteq A$. Dizemos que $B$ é subanel de $A$ se, e somente se $(B, +|_{B^2}, \cdot|_{B^2}, 0_A, 1_A)$ é um anel, onde $+|_{B^2}:B^2\rightarrow B$ e $\cdot|_{B^2}:B^2\rightarrow B$ são as restrições das operações de $A$ à $B^2$.
\end{definition}

Na definição acima, estamos pedindo que $B$ seja um subconjunto de $A$ que possua as mesmas operações que $A$, e que essas operações sejam restritas a $B$ e satisfaçam todas as cláusulas da definição de anel. Aparentemente, na prática, provar que um dado subconjunto de $A$ é um subanel pode parecer uma tarefa longa. Porém, a seguinte proposição encurta esta tarefa significativamente: 

\begin{definition}[Subanel]
    Seja $A$ um anel e $B\subseteq A$. Então $B$ é um subanel de $A$ se, e somente se:
    \begin{itemize}
        \item $1_A \in B$
        \item Para todos $a, b \in B$, $a-b \in B$.
        \item Para todos $a, b \in B$, $ab\in B$
    \end{itemize}

    Além disso, caso $B$ seja um subanel de $A$, os opostos aditivos de $B$ são os mesmos que os de $A$, ou seja, que $-b \in B$ para todo $B \in B$.
\end{definition}

\begin{proof}
    Primeiro, notemos suponhamos que $B$ seja um subanel de $A$. Então $B$ é fechado por $+, \cdot$ e $1_A\in B$. Resta apenas ver que para todos $a, b \in B$, $a-b \in B$.
    Como $B$ é fechado por soma, basta provar a última afirmação: que para todo $b \in B$, $-b \in B$.
    Fixe $b \in B$. Como $(B, +|_B^2, 0_A)$ é um grupo abeliano, existe $x \in B$ tal que $b+x=0_B$. Então, em $a$, segue que $b+x=x+b=0_A$. Pela unicidade dos opostos em $A$, segue que $-b=x\in B$.

    Reciprocamente, provaremos que se $B$ possui $1_B$ como elemento e é fechado por diferença e por produto, então $B$ é um subanel de $A$. Iniciaremos verificando que $B$ é fechado por soma, por opostos e que tem $0_A$ como elemento.

    Como $1_A$ é elemento de $B$, temos que $0_A=1_A-1_A\in B$. Assim, $B$ possui $0_A$ como elemento. Agora, dado $b \in B$, $0_A-b=-b \in B$, o que mostra que $B$ é fechado por opostos. Finalmente, dados $a, b \in B$, $a-(-b)=a+b\in B$, o que mostra que $B$ é fechado para soma.

    As propriedades associativas, comutativas, distributivas e de identidade valem em $B$, pois valem em $A$ e as operações de $B$ são as mesmas de $A$, restritas. Para finalizar, basta observar que dado $a \in B$, $(-a)\in B$, como já mostrado, e que $a+(-a)=(-a)+a=0_A$, o que mostra que $B$ possui opostos aditivos.
\end{proof}

\begin{exemplo}
$\mathbb N$ não é um subanel de $\mathbb Z$, pois $-1 \notin \mathbb Z$.
Porém, note que $\mathbb N$ tem $1$ e é fechado por soma e produto, o que mostra que na proposição anterior, a expressão $a-b$ não pode ser substituída por $a+b$.
\end{exemplo}

\begin{exemplo}[Subanel trivial]
    Para todo $A$, temos que $A$ é subanel de si mesmo.
\end{exemplo}

\begin{exemplo}
O único subanel de $\mathbb Z$ é $\mathbb Z$: se $B$ é um subanel de $\mathbb Z$, então $0, 1 \in B$.
Por indução, para todo $n\geq 1$ temos que $n \in \mathbb B$: com efeito, $1\in B$, e, se $n \in B$, $n+1\in B$, logo vale o passo indutivo. Finalmente, $-n\in B$ para todo $n\geq 1$. Como $\mathbb Z=\{0\}\cup\{n \in \mathbb Z: n\geq 1\}\cup \{-n \in \mathbb Z: n\geq 1\}$, temos que $B=\mathbb Z$.
\end{exemplo}

Como as operações de um subanel são as mesmas de um anel, um subanel de um anel comutativo é comutativo.

\begin{prop}
    Subanéis de aneis comutativos são comutativos.
\end{prop}
\begin{proof}
Seja $A$ um anel comutativo e $B$ um subanel de $A$. Para todos $a, b \in B$, temos que o produto $a\cdot b$ em $B$ é dado pelo produto (comutativo) $a\cdot b$ em $A$, logo $a\cdot b=b\cdot a$.
\end{proof}
