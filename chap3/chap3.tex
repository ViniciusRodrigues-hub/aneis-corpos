\chapter{Anéis e subanéis}
Nesta seção, iniciaremos o estudo dos anéis e de estruturas relacionadas. Apresentaremos as definições dessas estruturas e suas propriedades mais elementares.

\section{A definição de anel}
No Capítulo 2, conhecemos, por alto, a definição de grupo.
Um grupo é um conjunto munido de uma operação binária que satisfaz algumas propriedades.
Grupos pode ser Abelianos ou não Abelianos, e, quando é Abeliano, lembra-nos da adição de inteiros.

Porém, estuturas como as dos números inteiros, racionais e reais parecem não ter sua estrutura algébrica completamente capturada pela noção de grupo Abeliano, pois possuem também outra operação binária -- a multiplicação.
Esta operação se relaciona com a soma através das propriedades distributivas. A noção de anel surge para capturar parte destas ideias, generalizando o estudo das estruturas mencionadas.
\begin{definition}[Anel]
    Um anel é uma $5$-upla $(A, +, \cdot, 0, 1)$ conjunto $A$ com duas operações binárias, adição e multiplicação, denotadas por $+$ e $\cdot$, tais que:
    \begin{itemize}
        \item $(A, +, 0)$ é um grupo abeliano.
        \item (\textbf{Associatividade}) Para todo $a, b \in A$, temos $(a \cdot b)\cdot c = a\cdot(b\cdot c)$.
        \item (\textbf{Elemento identidade}) $\forall a \in A$ $1 \cdot a = a \cdot 1 = a$.
        \item (\textbf{Propriedades distributivas}) Para todos $a, b, c, \in A$, temos:
        \begin{align*}
            a \cdot (b + c) &= a \cdot b + a \cdot c, \text{ e}\\
            (a + b) \cdot c &= a \cdot c + b \cdot c
        \end{align*}
    \end{itemize}
    Se, adicionalmente, a seguinte propriedade é satisfeita, o anel é chamado de \emph{comutativo}.
    \begin{itemize}
        \item (\textbf{Comutatividade}) $\forall a, b \in A$ $a \cdot b = b \cdot a$.
    \end{itemize}
\end{definition}

Algumas observações:
\begin{itemize}
    \item Como em grupos, ao discursar sobre anéis é comum omitir as operações, referindo-se apenas ao conjunto $A$.
    \item Ao discursar sobre anéis, e a exemplo do que foi feito ao enunciar as propriedades distributivas, são utilizadas as convenções usuais sobre precedência de operações envolvidas por parênteses.
    Assim, $a + b \cdot c$ é interpretado como $a + (b \cdot c)$.
    \item Há textos que definem anéis sem incluir o elemento identidade $1$.
    Nestes textos, a definição acima dá nome ao que chamam de \emph{anéis com identidade}, ou \emph{anéis com 1}.
    Nesse curso, não usaremos essa convenção, de modo que \textbf{todos nossos anéis possuem identidade}.
    De modo similar, alguns textos definem anéis como sendo comutativos. Também não adotaremos essa convenção.
    \textbf{Os nossos anéis podem ser não comutativos}.
    \item A definição de anel não exige que $0=1$.
    \item $0$ é chamado de elemento nulo, e $1$ de elemento identidade.
\end{itemize}

\begin{prop}[Propriedade multiplicativa do $0$]
    Seja $A$ um anel.
    Então $\forall a \in A$ $0 \cdot a = a \cdot 0 = 0$.
\end{prop}
\begin{proof}
Provaremos a primeira afirmação.
A segunda é análoga e fica como exercício.

Temos que $0\cdot a=(0+0)\cdot a=0\cdot a +0\cdot a$.
Cancelando, segue que $0=0\cdot a$.
\end{proof}

\begin{prop}[Anel trivial]
    Seja $A={x}$ um conjunto qualquer.
    Defina $x\cdot x=x=x+x=0=1$.
    Então $(A, +, \cdot, 0, 1)$ é um anel.
    Um anel dessa forma é chamado de \emph{anel trivial}.
    
    Além disso, se $A$ é um anel tal que $0=1$, então $A$ é um anel trivial.
\end{prop}
\begin{proof}
    A primeira afirmação (de que $A$ como acima é um anel) fica como exercício.

    Para a segunda afirmação, assuma que $A$ é um anel tal que $0=1$.
    Fixe $a \in A$ qualquer.
    Então $a=a\cdot1=a\cdot0=0$, ou seja, $a=0$.
    Assim, $A$ é o conjunto unitário $\{0\}$, que é um anel trivial.
\end{proof}

Todo anel satisfaz as conhecidas regras de sinais referentes à multiplicação e adição, como:
\begin{prop}[Regras de sinal II]\label{prop:regraSinal2}
    Seja $A$ um anel e $a, b \in A$. Então:
    \begin{enumerate}[label=\alph*)]
        \item $(-a)b=a(-b)=-(ab)$\label{prop:regraSinal2_A}
        \item $(-a)(-b)=ab$.\label{prop:regraSinal2_B}
        \item $(-1)a=-a$.\label{prop:regraSinal2_C}
    \end{enumerate}
\end{prop}
\begin{proof}
    \ref{prop:regraSinal2_A}: Temos que $ab+(-a)b=(-a)b+ab=[-a+a]b=0b=0$. Assim, $(-a)b=-(ab)$. Analogamente, $a(-b)=-(ab)$.

    \ref{prop:regraSinal2_B}: Temos que $(-a)(-b)=-[a(-b)]=-[-(ab)]=ab$ pela regra anterior.

    \ref{prop:regraSinal2_C}: Temos que $(-1)a=-(1a)=-a$.
\end{proof}
\section{Anéis de Matrizes}
Matrizes são objetos muito importantes na matemática, sendo amplamente utilizadas na Álgebra Linear.

Nesta seção, construiremos os anéis de matrizes com coeficientes em um anel arbitrário.

\begin{definition}
    Seja $A$ um anel e $n, m$ inteiros positivos. O conjunto $M_{n\times m}(A)$ é o conjunto de matrizes $n\times m$ cujos coeficientes estão em $A$. Formalmente, $M_{n\times m}$ é o conjunto de todas as famílias $(a_{ij})_{i,j}=(a_{ij}: (i, j) \in \{1, \dots, n\}\times \{1, \dots, m\})$. Quando conveniente, representamos a tal matriz de qualquer uma das duas formas a seguir:
\begin{multicols}{2}
    \centering
    \begin{equation*}
        \begin{pmatrix}
            a_{11} & \cdots & a_{1m}\\
            \vdots & \ddots & \vdots\\
            a_{n1} & \cdots & a_{nm}
        \end{pmatrix}
    \end{equation*}

    \begin{equation*}
        \begin{bmatrix}
            a_{11} & \cdots & a_{1m}\\
            \vdots & \ddots & \vdots\\
            a_{n1} & \cdots & a_{nm}
        \end{bmatrix}
    \end{equation*}
\end{multicols}

Se $(a_{i, j})_{i,j}$ e $(b_{i, j})_{i,j}$ são matrizes $n\times m$ em $M_{n\times m}(A)$, definimos sua \emph{soma} como $(a_{i, j})+(b_{i, j})_{i,j}=(a_{i, j}+b_{i, j})_{i,j}$.
Em outra notação:

\begin{equation*}
    \begin{pmatrix}
        a_{11} & \cdots & a_{1m}\\
        \vdots & \ddots & \vdots\\
        a_{n1} & \cdots & a_{nm}
    \end{pmatrix}
    +
    \begin{pmatrix}
        b_{11} & \cdots & b_{1m}\\
        \vdots & \ddots & \vdots\\
        b_{n1} & \cdots & b_{nm}
    \end{pmatrix}
    =
    \begin{pmatrix}
        a_{11}+b_{11} & \cdots & a_{1m}+b_{1m}\\
        \vdots & \ddots & \vdots\\
        a_{n1}+b_{n1} & \cdots & a_{nm}+b_{nm}
    \end{pmatrix}
\end{equation*}

Se $(a_{ij})_{i, j}\in M_{n\times m}(A)$ e $(b_{ij})_{i, j}\in M_{m\times p}(A)$, definimos o produto de matrizes como $(a_{ij})_{i,j}\cdot(b_{ij})_{i,j}=(c_{ij})_{i,j}$, onde $c_{ij}=\sum_{k=1}^m a_{ik}b_{kj}$.
Em outra notação:
\begin{equation*}
    \begin{pmatrix}
        a_{11} & \cdots & a_{1m}\\
        \vdots & \ddots & \vdots\\
        a_{n1} & \cdots & a_{nm}
    \end{pmatrix}
    \cdot
    \begin{pmatrix}
        b_{11} & \cdots & b_{1p}\\
        \vdots & \ddots & \vdots\\
        b_{m1} & \cdots & b_{mp}
    \end{pmatrix}
    =
    \begin{pmatrix}
        c_{11} & \cdots & c_{1p}\\
        \vdots & \ddots & \vdots\\
        c_{n1} & \cdots & c_{np}
    \end{pmatrix}
\end{equation*}

A matriz nula de $M_{n\times m}(A)$ é a matriz cuja todas as entradas são $0\in A$, e é denotada por $0_{n\times m}$, ou, simplesmente, $0$.

Caso $n=m$, abreviamos $M_{n\times n}(A)$ como $M_n(A)$.
\end{definition}

Sobre a aditividade, independente de $m, n$, sempre temos um grupo Abeliano:

\begin{prop}
Seja $A$ um anel e $n, m \in \mathbb{N}$. Então, o conjunto $M_{n \times m}(A)$, munido da operação de soma de matrizes, é um grupo abeliano.
\end{prop}

\begin{proof}
    Sejam $(a_{ij})_{i,j}, (b_{ij})_{i,j}, (c_{ij})_{i,j} \in M_{n \times m}(A)$. Mostraremos que $(M_{n \times m}(A), +)$ satisfaz as propriedades de um grupo abeliano:
    
    \begin{enumerate}
        \item \textbf{Fechamento:} Para todos $(a_{ij})_{i,j}, (b_{ij})_{i,j} \in M_{n \times m}(A)$, temos que $(a_{ij} + b_{ij})_{i,j} \in M_{n \times m}(A)$, pois $A$ é fechado sob adição.
    
        \item \textbf{Associatividade:} para todos $(a_{ij})_{i,j}, (b_{ij})_{i,j}, (c_{ij})_{i,j} \in M_{n \times m}(A)$, temos:
        \[
        \big((a_{ij}) + (b_{ij})\big) + (c_{ij}) = (a_{ij} + b_{ij}) + c_{ij} = a_{ij} + (b_{ij} + c_{ij}) = (a_{ij}) + \big((b_{ij}) + (c_{ij})\big).
        \]
    
        \item \textbf{Elemento neutro:} A matriz nula é o elemento neutro.
        Com efeito, dado $(a_{ij})_{i,j} \in M_{n \times m}(A)$, temos:
        \[
        (a_{ij}) + 0_{m\times n} = (a_{ij} + 0) = (a_{ij}).
        \]
    
        \item \textbf{Elemento inverso:} Para cada $(a_{ij})_{i,j} \in M_{n \times m}(A)$, a matriz $(-a_{ij})_{i,j}$, é oposto aditivo, pois:
        \[
        (a_{ij}) + (-a_{ij}) = (a_{ij} +(-a_{ij})) = 0
        \]
    
        \item \textbf{Comutatividade:} A soma de matrizes é comutativa, pois, para todos $(a_{ij})_{i,j}, (b_{ij})_{i,j} \in M_{n \times m}(A)$, temos:
        \[
        (a_{ij}) + (b_{ij}) = (a_{ij} + b_{ij}) = (b_{ij} + a_{ij}) = (b_{ij}) + (a_{ij}).
        \]
    \end{enumerate}
    
    Portanto, $(M_{n \times m}(A), +)$ é um grupo abeliano.
\end{proof}

A multiplicação de matrizes é associativa e distributiva sobre a soma. Formalmente:

\begin{prop}
Seja $A$ um anel e $n, m, p, q \geq 1$. Então:
\begin{enumerate}[label=\alph*)]
\item (\textbf{Associatividade}) Para todos $(a_{ij})_{i,j} \in M_{n \times m}(A)$, $(b_{jk})_{j,k} \in M_{m \times p}(A)$ e $(c_{kl})_{k,l} \in M_{p \times q}(A)$, temos:
\[
\big((a_{ij}) \cdot (b_{jk})\big) \cdot (c_{kl}) = (a_{ij}) \cdot \big((b_{jk}) \cdot (c_{kl})\big).
\]

\item (\textbf{Distributividade}) Para todos $(a_{ij})_{i,j} \in M_{n \times m}(A)$, $(b_{jk})_{j,k}, (c_{jk})_{j,k} \in M_{m \times p}(A)$, temos:
\[
(a_{ij}) \cdot\left((b_{jk}) + (c_{jk})\right)
= (a_{ij}) \cdot (b_{jk}) + (a_{ij}) \cdot (c_{jk}).
\]
E, para todos $(a_{ij})_{i,j}, (b_{ij})_{i,j} \in M_{n \times m}(A)$ e $(c_{jk})_{j,k} \in M_{m \times p}(A)$, temos:
\[
\left((a_{ij}) + (b_{ij})\right) \cdot (c_{jk})
= (a_{ij}) \cdot (c_{jk}) + (b_{ij}) \cdot (c_{jk}).
\]
\end{enumerate}
\end{prop}
\begin{proof}
    \textbf{a)} Sejam $(a_{ij})_{i,j} \in M_{n \times m}(A)$, $(b_{jk})_{j,k} \in M_{m \times p}(A)$ e $(c_{kl})_{k,l} \in M_{p \times q}(A)$. Considere o elemento $(i,l)$ da matriz resultante de $\big((a_{ij}) \cdot (b_{jk})\big) \cdot (c_{kl})$.
    Pela propriedade distributiva, temos:
    \[
    \sum_{k=1}^p \left( \sum_{j=1}^m a_{ij} b_{jk} \right) c_{kl}
    =\sum_{k=1}^p \left( \sum_{j=1}^m a_{ij} b_{jk}c_{kl} \right) .
    \]
    Comutando os somatórios e novamente pela propriedade distributiva, isso é:
    \[
    \sum_{j=1}^m\left( \sum_{k=1}^p  a_{ij} b_{jk} c_{kl} \right),
    =\sum_{j=1}^m a_{ij} \left( \sum_{k=1}^p b_{jk} c_{kl} \right),
    \]
    que é exatamente o elemento $(i,l)$ da matriz $(a_{ij}) \cdot \big((b_{jk}) \cdot (c_{kl})\big)$. Assim, a associatividade é satisfeita.

    \textbf{b)} Para a distributividade, considere $(a_{ij})_{i,j} \in M_{n \times m}(A)$, $(b_{jk})_{j,k}, (c_{jk})_{j,k} \in M_{m \times p}(A)$. O elemento $(i,k)$ da matriz resultante de $(a_{ij}) \cdot \big((b_{jk}) + (c_{jk})\big)$ é dado por:
    \[
    \sum_{j=1}^m a_{ij} (b_{jk} + c_{jk})
    =\sum_{j=1}^m (a_{ij}b_{jk} + a_{ij}c_{jk})
    =\sum_{j=1}^m a_{ij}b_{jk} + \sum_{j=1}^ma_{ij}c_{jk}
    \]

    Isso corresponde ao elemento $(i,k)$ da matriz $(a_{ij}) \cdot (b_{jk}) + (a_{ij}) \cdot (c_{jk})$. A outra distributividade é provada de forma análoga.
\end{proof}

Como o produto de uma matriz de $M_{n\times m}(A)$ com uma matriz de $M_{m\times p}(A)$ é uma matriz de $M_{n\times p}(A)$, em geral, não há uma propriedade de fechamento para o produto de matrizes.

Lembremos que a matriz identidade de $M_{n\times n}(A)$ é a matriz cujos elementos da diagonal principal são $1$ e os demais são $0$.
Utilizando a notação do delta de Kronecker, em que $\delta_{ij}$ é $1$ caso $i=j$ e $0$ caso contrário, a matriz identidade é a matriz $I_n=(\delta_{ij})_{i,j}\in M_{n}(A)$.

Porém, tal fato acontece para matrizes quadradas. De fato, temos:

\begin{prop}[Anéis de matrizes]
    Seja $A$ um anel e $n\geq 1$. Com as operações de soma e multiplicação definidas acima, e com a identidade $I_n$ como a matriz identidade de $M_{n}(A)$, o conjunto $M_{n}(A)$ é um anel, denominado \emph{anel das matrizes $n\times n$ de $A$}.

    Se $n\geq 2$ e $A$ é um anel não trivial, $M_n(A)$ não é comutativo.
\end{prop}

\begin{proof}
Para a verificação das propriedades de anel, resta apenas ver que a matriz identidade $I_n$ é uma identidade multiplicativa.
Com efeito, dado $(a_{ij})_{i,j} \in M_n(A)$, temos:
\begin{align*}
    (a_{ij}) \cdot I_n &= \left( \sum_{k=1}^n a_{ik} \delta_{kj} \right)_{i,j} \\
    &= \left( a_{ij} \right)_{i,j},
\end{align*}

e:

\begin{align*}
    I_n \cdot (a_{ij}) &= \left( \sum_{k=1}^n \delta_{ik} a_{kj} \right)_{i,j} \\
    &= \left( a_{ij} \right)_{i,j}.
\end{align*}

Para a última afirmação, considere $(a_{ij})_{i,j}, (b_{ij})_{i,j} \in M_n(A)$ definidos por:
\begin{multicols}{2}
\centering
\[a_{ij}=\begin{cases}
    1 & \text{se } i=j=1\\
    0 & \text{caso contrário}
\end{cases}\]

\[b_{ij}=\begin{cases}
    1 & \text{se } i=1, j=n\\
    0 & \text{caso contrário}
\end{cases}\]
\end{multicols}

Temos que o elemento $(1,n)$ da matriz $(a_{ij})(b_{ij})$ é dado por $\sum_{k=1}^n a_{1k}b_{kn}=1$, enquanto o elemento $(1,n)$ da matriz $(b_{ij})(a_{ij})$ é dado por $\sum_{k=1}^n b_{1k}a_{kn}=1$.
\end{proof}

Assim, os anéis de matrizes nos dão uma ampla gama de anéis não comutativos.
\section{Domínios de integridade e divisores de zero}
O anel dos números inteiros, bem como o anel dos racionais reais, possuem a seguinte importante propriedade:
\begin{definition}
Seja $A$ um anel comutativo. Dizemos que $A$ é um \emph{domínio de integridade} se, e somente se, $\forall a, b \in A$, se $ab=0$, então $a=0$ ou $b=0$.
\end{definition}

Nem todos os anéis comutativos são domínios de integridade. Por exemplo, no anel dos inteiros módulo $4$, $\mathbb Z_4$, temos que $2\cdot 2=4=0$, e $2\neq 0$.

Divisores de zero são elementos não nulos que, multiplicados entre si, resultam em zero.

\begin{definition}
Sejam $A$ um anel. Um divisor de zero de $A$ é um elemento $a \in A$ não nulo para o qual exista $b \in A$ não nulo tal que $ab=0$ ou $ba=0$.
\end{definition}

Note que um domínio de integridade é um anel comutativo sem divisores de zero.

Divisores de zero são patológicos ao estudar a teoria de divisibilidade em anéis, assim, muitas vezes, eles são excluídos de tal estudo.

\section{Elementos invertíveis}
Um anel, com sua soma, é um grupo Abeliano, e, portanto, possui opostos aditivos. Porém, não necessita possuir opostos multiplicativos. Os elementos de um anel que possuem inversos no anel são os chamados \emph{elementos invertíveis} ou \emph{unidades}.
\begin{definition}[Elemento invertível]
    Seja $A$ um anel.
    Um elemento $a \in A$ é dito \emph{invertível}, ou uma \emph{unidade} se $\exists b \in A$ tal que $a \cdot b = b \cdot a = 1$.
    
    O conjunto de todas das unidades de $A$ é denotado por $A^*$.
\end{definition}

\begin{definition}
    Seja $A$ um anel.
    Então, se $a \in A^*$, existe um \textbf{único} $b \in A$ tal que $a \cdot b = b \cdot a = 1$. Este elemento é denotado por $a^{-1}$, e é chamado de \emph{inverso} de $a$.
\end{definition}

Observação: para que a definição acima faça sentido, é necessário mostrar que se $a$ é unidade, existe um \textbf{único} $b \in A$ tal que $a \cdot b = b \cdot a = 1$.
A existência é garantida pela definição de unidade, e a demonstração da unicidade é análoga à da unicidade do inverso em grupos (Proposição \ref{prop:group_uniqueInverse}, ficando como exercício.

\begin{prop}
Seja $A$ um anel. Para todos $a, b \in A^*$, temos:
\begin{enumerate}[label=\alph*)]
    \item $ab\in A^U$ e $(ab)^{-1}=b^{-1}a^{-1}$.\label{prop:unidadeProduto_a}
    \item $a^{-1}\in A^U$ e $(a^{-1})^{-1}=a$.\label{prop:unidadeProduto_b}
    \item $1^{-1}=1$.\label{prop:unidadeProduto_c}
\end{enumerate}
Além disso, $A^*$ é, com a restrição da operação de multiplicação do anel, um grupo com identidade $1$. Caso $A$ é um anel comutativo, $A^*$ é um grupo abeliano.
\end{prop}
\begin{proof}
    \ref{prop:unidadeProduto_a}: Sejam $a, b \in A^*$. Pela associatividade, $(ab)(b^{-1}a^{-1})=1=(b^{-1}a^{-1})(ab)$, logo, pela unicidade do inverso, $(ab)^{-1}=b^{-1}a^{-1}$.

    \ref{prop:unidadeProduto_b}: Seja $a \in A^*$. Temos que $a^{-1}a=1=a(a^{-1})$, logo, pela unicidade do inverso, $(a^{-1})^{-1}=a$.

    \ref{prop:unidadeProduto_c}: Note que $1\cdot 1=1=1\cdot 1$, logo, pela unicidade do inverso, $1^{-1}=1$.

    Se $A$ é um anel comutativo, então $A^*$ é um grupo abeliano, pois para todo $a, b \in A^*$, temos que $ab=ba$, logo $(ab)^{-1}=b^{-1}a^{-1}=a^{-1}b^{-1}$.
\end{proof}

\section{O anel dos números inteiros}
Espera-se que o estudante já possua traquejo com o anel dos números inteiros, incluindo contato com a noção formal de divisibilidade, o teorema fundamental da aritmética e a noção de congruência módulo $n$.

Primeiramente, reconheçamos que $\mathbb Z$ possui, além da estrutura de domínio de integridade, uma estrutura de ordem.

\begin{definition}
    Um anel ordenado é uma tupla $(A, +, \cdot, 0, 1, \leq)$ tal que $(A, +, \cdot, 0, 1)$ é um anel comutativo tal que $\leq$ é uma relação de ordem total (também chamada de ordem linear) em $A$, ou seja, que satisfaça:

    \begin{itemize}
        \item (Propriedade reflexiva) $\forall a \in A$, $a \leq a$.
        \item (Propriedade antissimétrica) $\forall a, b \in A$, se $a \leq b$ e $b \leq a$, então $a=b$.
        \item (Propriedade transitiva) $\forall a, b, c \in A$, se $a \leq b$ e $b \leq c$, então $a \leq c$.
        \item (Linearidade) $\forall a, b \in A$, $a \leq b$ ou $b \leq a$.
    \end{itemize}

    e tal que:

    \begin{itemize}
        \item (Compatibilidade da soma) $\forall a, b, c \in A$, se $a \leq b$, então $a+c \leq b+c$ e $ac \leq bc$.
        \item (Compatibilidade da multiplicação) $\forall a, b, c \in A$, se $a \leq b$ e $0 \leq c$, então $ac \leq bc$.
    \end{itemize}

    Nesse caso, dizemos que $a<b$ se $a\leq b$ e $a\neq b$.
    
    Os elementos positivos de $A$ são os elementos maiores do que $0$.
    
    Os negativos são os menores do que $0$.
\end{definition}

Assumiremos, sem demonstração (por fugir do escopo do texto), que existe uma estrutura $\mathbb Z=(\mathbb Z,+,\cdot,0,1, \leq)$ como abaixo:

\begin{definition}[Inteiros, anel ordenado]
$\mathbb Z=(\mathbb Z,+,\cdot,0,1, \leq)$ é um domínio de integridade ordenado cujos elementos positivos possuem a propriedade da boa ordenação:

Qualquer subconjunto não vazio de inteiros positivos possui elemento mínimo.
\end{definition}

Assumiremos todos os fatos elementares sobre $\mathbb Z$ que não foram provados, inclusive o fato de que $\mathbb Z=\{\dots, -2, -1, 0, 1, 2, \dots, \}$.
\section{Corpos e anéis de divisão}

Abaixo, segue a definição de anel de divisão e corpo.
A noção de corpo será uma das noções mais importantes deste texto.
\begin{definition}[Corpo e Anel de Divisão]
Um \emph{anel de divisão} é um anel não trivial para o qual todo elemento não nulo é invertível.
Um \emph{corpo} é um anel de divisão comutativo.
\end{definition}

Todo corpo é um domínio de integridade.
De fato:
\begin{prop}
    Seja $K$ um corpo.
    Então $K$ é um domínio de integridade.
\end{prop}
\begin{proof}
Sabemos que $K$ é um anel comutativo não trivial.
Sejam $a, b \in K$ tais que $ab=0$.
Se $a=0$, então segue a tese.
Caso contrário, como $K$ é um corpo, $a^{-1}$ existe.
Assim, temos que $b=(a^{-1}a)b=a^{-1}(ab)=0$, logo, $b=0$.
\end{proof}

Porém, nem todo domínio de integridade é um corpo: por exemplo, $\mathbb Z$ é um domínio de integridade que não é um corpo, pois $2$ não possui inverso multiplicativo em $\mathbb Z$.

\section{O corpo dos números reais}
Assim como fizemos com $\mathbb Z$, assumiremos a existência do corpo dos números reais, que é um corpo ordenado que satisfaz a propriedade de ser Dedekind-completo. Formalmente:

\begin{prop}
    O corpo dos números reais $\mathbb R$ é um corpo ordenado, e satisfaz a propriedade de ser Dedekind-completo.
    Ou seja, tal que para todo $A\subseteq \mathbb R$ não vazio, se $A$ é limitado superiormente (ou seja, se existe $a \in \mathbb R$ tal que $\forall x \in A, x\leq a$), então $A$ admite um supremo (um menor limitante superior, ou seja, existe $b \in \mathbb R$ tal que $\forall x \in A, x\leq b$ e $\forall c \in \mathbb R$, se $x\leq c$ para todo $x\in A$, então $b\leq c$).
\end{prop}

O estudo das propriedades dos números reais é um assunto central de um curso básico de Análise Real.

Nesse texto, detalharemos tais propriedades somente de acordo com nossa necessidade.

\section{O corpo dos números complexos}
A história dos números complexos remete à representar uma solução para a equação $x^2+1=0$, que não possui solução real.

A ideia é que adiciona-se em $\mathbb R$ um novo elemento, $i$, para o qual vale $i^2=-1$ e para o qual as demais propriedades operacionais de números reais são preservadas. Nesse anel, todo elemento se escreverá de forma única como $a+bi$, onde $a,bin \mathbb R$.

Apresentaremos uma construção a seguir.

\begin{definition}[Quaternions]
    Definimos $\mathbb C=\mathbb R^2$.
    
    Se $a \in \mathbb R$, identifique $a=(a, 0)$ e $i=(0, 1)$.

    Segue que, utilizando a linguagem de produto por escalar oriunda da álgebra linear, que para todo $x \in \mathbb H$, exisem únicos $a, b\in \mathbb R$ tais que $x=a+bi$.

    Em $\mathbb C$, definimos a soma coordenada-a-coordenada. Da Álgebra Linear, sabemos que isso nos dá um grupo Abeliano.

    Define-se também a multiplicação, inspirada pela discussão acima, como se segue: para $a, b, c, d \in \mathbb R$:

    \begin{equation*}
        (a, b)(u, v)=(au-bv, bu+av).
    \end{equation*}

    Ou, em outra notação:

    \begin{align*}
    (a+bi)(c+di)& \\
    &=(ac-bd)+(ad+bc)i
    \end{align*}
\end{definition}

\begin{prop}
    $\mathbb C$ é um corpo.
\end{prop}
\begin{proof}
    \begin{prop}
        $\mathbb H$ é um domínio de integridade.
    \end{prop}
    \begin{proof}
        $1$ é neutro multiplicativo: dado $a+bi=(a, b)\in \mathbb C$, pela definição, temos que $(1, 0)(a, b)=(a, b)$, pois as demais parcelas zeram. Analogamente, $(a, b)(1, 0)=(a, b)$.
        
        A multiplicação é associativa:
        Para $x, y, z \in \mathbb H$, tome $a, b, u, v, p, q \in \mathbb R$ com e $x=(a, b)$, $y=(u, v)$ e $z=(p, q)$.
        Temos que:
        \begin{align*}
            (xy)z &=(au-bv, bu+av)(p, q)\\
            &=((au-bv)p-(bu+av)q, (bu+av)p+(au-bv)q)\\
            &=(aup-bvp-buq+avq, bup+avp+auq-bvq)
        \end{align*}
        e $x(yz)$ é dado por:
        \begin{align*}
        x(yz) &=(a, b)(up-pv,uq+vp)\\
        &=(a(up-bv)-b(uq+vp), b(up-bv)+a(uq+vp))\\
        &=(aup-bvq-bup+avq, bup+avp+auq-bvq)
        \end{align*}
        Comparando, segue.

        A multiplicação é comutativa: Para $x, y \in \mathbb H$, temos que $x=(a, b)$ e $y=(u, v)$. Temos que:
        \begin{align*}
            xy &=(a, b)(u, v)
            =(au-bv, bu+av)\\
            &= (ua-vb, va+ub)\\
            &= (u, v)(a, b)\\
            &=yx.
        \end{align*}    
        A propriedade distributiva também é válida:

        Para $x, y, z \in \mathbb H$, temos que $x=(a, b)$, $y=(u, v)$ e $z=(p, q)$. Temos que:
        \begin{align*}
            x(y+z) &=(a, b)((u, v)+(p, q))\\
            &=(a, b)(u+p, v+q)\\
            &=(a(u+p)-b(v+q), b(u+p)+a(v+q))\\
            &=(au-bv, bu+av)+(ap-bq, bp+aq)\\
            &=xy+xz
        \end{align*}

        Finalmente, todo elemento distindo de $(0, 0)$ é invertível: seja $x=(a, b)\in \mathbb C$ tal que $x\neq 0$.
        Então, $a^2+b^2\neq 0$.
        Considere $y=(\frac{a}{a^2+b^2}, \frac{-b}{a^2+b^2})$.
        Calculemos $xy$:
        \begin{align*}
            xy &=(a, b)(\frac{a}{a^2+b^2}, \frac{-b}{a^2+b^2})\\
            &=(\frac{a^2}{a^2+b^2}+\frac{b^2}{a^2+b^2}, \frac{-ab}{a^2+b^2}+\frac{ab}{a^2+b^2})\\
            &=\left(\frac{a^2+b^2}{a^2+b^2}, 0\right)\\
            &=1.
        \end{align*}
    \end{proof}
\end{proof}
\section{O Anel dos Quaternions}
Discutimos as noções de corpo e de anel de divisão.
Por definição, todo corpo é um anel de divisão.
Um dos primeiros exemplos de um anel de divisão que não é um corpo é o anel dos quaternions $\mathbb H$, que descreveremos abaixo.

A ideia é que adiciona-se em $\mathbb R$ três elementos distintos: $i, j, k$, para os quais valem as propriedades de que $i^2=j^2=k^2=-1$, e $ij=k$, $jk=i$ e $ki=j$, e para o qual as demais propriedades operacionais de números reais são preservadas. Nesse anel, todo elemento se escreverá de forma única como $a+bi+cj+dk$, onde $a,b,c,d\in \mathbb R$.

Apresentaremos uma construção a seguir. Antes disso, note que, como $k=ij$, multiplicando ambos os lados por $i$ à esquerda, supondo que a propriedade associativa ainda valha, temos que $ik=-j$.

Multiplicando por $j$ à direita, temos que $kj=-1$.

Além disso, multiplicando por $i=jk$ à esquerda por $j$, temos que $ji=-1$. Assim, temos que $ij=k$, $jk=i$, $ki=j$, $ji=-k$, $kj=-i$ e $ik=-j$.

Assumindo que $-i\neq i$, $-j\neq j$ e $-k\neq k$, temos que $i,j,k$ vêmos que a nossa estrutura deverá ser não comutativa.

\begin{definition}[Quaternions]
    Definimos $\mathbb H=\mathbb R^4$.
    
    Se $a \in \mathbb R$, seja $a=(a, 0, 0, 0)$, $i=(0, 1, 0, 0)$, $j=(0, 0, 1, 0)$ e $k=(0, 0, 0, 1)$.

    Segue que, utilizando a linguagem de produto por escalar oriunda da álgebra linear, que para todo $x \in \mathbb H$, exisem únicos $a, b, c, d \in \mathbb R$ tais que $x=a+bi+cj+dk$.

    Em $\mathbb H$, definimos a soma coordenada-a-coordenada. Da Álgebra Linear, sabemos que isso nos dá um grupo Abeliano.

    Define-se também a multiplicação, inspirada pela discussão acima, como se segue: para $a, b, c, d, u, v, z, w \in \mathbb R$:

    \begin{equation*}
        (a, b, c, d)(u, v, z, w)=(au-bv-cz-dw, av+bu+cw-dz, az+bw-cu+dv, aw+bz+cv-du).
    \end{equation*}

    Ou, em outra notação:

    \begin{align*}
    (a+bi+cj+dk)(u+vi+zj+kw)& \\
    &=(au-bv-cz-dw)+(av+bu+cw-dz)i\\
    &+(az+bw-cu+dv)j+(aw+bz+cv-du)k.
    \end{align*}
\end{definition}

Note que, com isso, temos $i^2=j^2=k^2=-1$, $ij=k$, $jk=i$ e $ki=j$, além de $i\neq -i$, $j\neq -j$ e $k\neq -k$.

Porém, $\mathbb H$ é um anel de divisão. Primeiro, provaremos que:

\begin{prop}
    $\mathbb H$ é um domínio de integridade.
\end{prop}
\begin{proof}
    $1$ é neutro multiplicativo: dado $a+bi+cj+dk=(a, b, c, d)\in \mathbb H$, pela definição, temos que $(1, 0, 0, 0)(a, b, c, d)=(a, b, c, d)$, pois as demais parcelas zeram. Analogamente, $(a, b, c, d)(1, 0, 0, 0)=(a, b, c, d)$.
    
    A multiplicação é associativa:
    Para $x, y, z \in \mathbb H$, temos que $x=(a, b, c, d)$, $y=(u, v, z, w)$ e $z=(p, q, r, s)$. Temos que:
    \begin{align*}
        (xy)z &=(au-bv-cz-dw, av+bu+cw-dz, az+bw-cu+dv, aw+bz+cv-du)(p,q,r,s)
    \end{align*}
    e $x(yz)$ é dado por:
    \begin{align*}
    x(yz) &=(a, b, c, d)(up-vq-zr-sw, uq+vp+zs-tw, ur+vq-pw+zt, us+vq+pw-qt)\\
    \end{align*}
    Expandindo os últimos produtos e comparando-os, vê-se que são iguais. Os detalhes ficam a cargo do leitor.

    De maneira igualmente trabalhosa, porém mecânica, verifica-se às duas propriedades distributivas.
\end{proof}

Mais interessante é demonstrar que $\mathbb H$ é um anel de divisão. Para isso, precisamos mostrar que todo elemento não nulo de $\mathbb H$ é invertível.

\begin{prop}
    $\mathbb H$ é um anel de divisão.
\end{prop}
\begin{proof}
    Fica a cargo do leitor. Para um guia, ver o Exercício~\ref{exer:quaternion}
\end{proof}
\section{Subanéis}
Em Matemática, é comum que as estruturas estudadas possuam uma noção de subestrutura.
Em geral, uma subestrutura de uma estrutura data é um subconjunto desta que seja, de forma natural, uma estrutura da mesma natureza daquela.

Veremos que, quando tratamos de anéis, nem todo subconjunto pode ser visto como uma subestrutura.

\begin{definition}[Subanel]
    Seja $A$ um anel e $B \subseteq A$. Dizemos que $B$ é subanel de $A$ se, e somente se $(B, +|_{B^2}, \cdot|_{B^2}, 0_A, 1_A)$ é um anel, onde $+|_{B^2}:B^2\rightarrow B$ e $\cdot|_{B^2}:B^2\rightarrow B$ são as restrições das operações de $A$ à $B^2$.
\end{definition}

Na definição acima, estamos pedindo que $B$ seja um subconjunto de $A$ que possua as mesmas operações que $A$, e que essas operações sejam restritas a $B$ e satisfaçam todas as cláusulas da definição de anel. Aparentemente, na prática, provar que um dado subconjunto de $A$ é um subanel pode parecer uma tarefa longa. Porém, a seguinte proposição encurta esta tarefa significativamente: 

\begin{prop}[Subanel]
    Seja $A$ um anel e $B\subseteq A$. Então $B$ é um subanel de $A$ se, e somente se:
    \begin{itemize}
        \item $1_A \in B$
        \item Para todos $a, b \in B$, $a-b \in B$.
        \item Para todos $a, b \in B$, $ab\in B$
    \end{itemize}

    Além disso, caso $B$ seja um subanel de $A$, os opostos aditivos de $B$ são os mesmos que os de $A$, ou seja, que $-b \in B$ para todo $B \in B$.
\end{prop}

\begin{proof}
    Primeiro, notemos suponhamos que $B$ seja um subanel de $A$. Então $B$ é fechado por $+, \cdot$ e $1_A\in B$. Resta apenas ver que para todos $a, b \in B$, $a-b \in B$.
    Como $B$ é fechado por soma, basta provar a última afirmação: que para todo $b \in B$, $-b \in B$.
    Fixe $b \in B$. Como $(B, +|_B^2, 0_A)$ é um grupo abeliano, existe $x \in B$ tal que $b+x=0_B$. Então, em $a$, segue que $b+x=x+b=0_A$. Pela unicidade dos opostos em $A$, segue que $-b=x\in B$.

    Reciprocamente, provaremos que se $B$ possui $1_B$ como elemento e é fechado por diferença e por produto, então $B$ é um subanel de $A$. Iniciaremos verificando que $B$ é fechado por soma, por opostos e que tem $0_A$ como elemento.

    Como $1_A$ é elemento de $B$, temos que $0_A=1_A-1_A\in B$. Assim, $B$ possui $0_A$ como elemento. Agora, dado $b \in B$, $0_A-b=-b \in B$, o que mostra que $B$ é fechado por opostos. Finalmente, dados $a, b \in B$, $a-(-b)=a+b\in B$, o que mostra que $B$ é fechado para soma.

    As propriedades associativas, comutativas, distributivas e de identidade valem em $B$, pois valem em $A$ e as operações de $B$ são as mesmas de $A$, restritas. Para finalizar, basta observar que dado $a \in B$, $(-a)\in B$, como já mostrado, e que $a+(-a)=(-a)+a=0_A$, o que mostra que $B$ possui opostos aditivos.
\end{proof}

\begin{exemplo}
$\mathbb N$ não é um subanel de $\mathbb Z$, pois $-1 \notin \mathbb Z$.
Porém, note que $\mathbb N$ tem $1$ e é fechado por soma e produto, o que mostra que na proposição anterior, a expressão $a-b$ não pode ser substituída por $a+b$.
\end{exemplo}

\begin{exemplo}[Subanel trivial]
    Para todo $A$, temos que $A$ é subanel de si mesmo.
\end{exemplo}

\begin{exemplo}
O único subanel de $\mathbb Z$ é $\mathbb Z$: se $B$ é um subanel de $\mathbb Z$, então $0, 1 \in B$.
Por indução, para todo $n\geq 1$ temos que $n \in \mathbb B$: com efeito, $1\in B$, e, se $n \in B$, $n+1\in B$, logo vale o passo indutivo. Finalmente, $-n\in B$ para todo $n\geq 1$. Como $\mathbb Z=\{0\}\cup\{n \in \mathbb Z: n\geq 1\}\cup \{-n \in \mathbb Z: n\geq 1\}$, temos que $B=\mathbb Z$.
\end{exemplo}

Como as operações de um subanel são as mesmas de um anel, um subanel de um anel comutativo é comutativo.

\begin{prop}
    Subanéis de aneis comutativos são comutativos.
\end{prop}
\begin{proof}
Seja $A$ um anel comutativo e $B$ um subanel de $A$. Para todos $a, b \in B$, temos que o produto $a\cdot b$ em $B$ é dado pelo produto (comutativo) $a\cdot b$ em $A$, logo $a\cdot b=b\cdot a$.
\end{proof}

\section{O centro de um anel}
Apesar de nem todo anel ser comutativo, todos os anéis possuem elementos que comutam com qualquer outro elemento -- ao menos o elemento $1$.

O centro do anel é o conjunto de tais elementos.

\begin{definition}[Centro de um anel]
    Seja $A$ um anel.

    O \emph{centro} de $A$, denotado por $Z(A)$, é o conjunto dos elementos de $A$ que comutam com todos os outros elementos de $A$.

    Formalmente, $Z(A)=\{a \in A: \forall b \in A, ab=ba\}$.
\end{definition}

O centro de um anel sempre é um subanel.

\begin{prop}
    Para todo anel $A$, o conjunto $Z(A)$ é um subanel de $A$.
\end{prop}

\begin{proof}
    Temos que $1 \in Z(A)$ pois para todo $b \in A$, $1a=a1=a$.

    Se $a, a' \in A$, temos que $aa' \in Z(A)$ pois para todo $b \in A$, $(aa')b=a(a'b)=a(ba')=(ab)a'=(ba)a'=b(a'a)$.

    Finalmente, se $a, a' \in A$, temos que $a-a' \in Z(A)$, pois para todo $b \in A$, $(a-a')b=ab-a'b=ba-ba'=b(a-a')$.
\end{proof}



\section{Exercícios}

\begin{exer} Seja $R$ um anel com identidade e seja $S$ um subanel de $R$ que contém a identidade de $R$.
Prove que se $u$ é uma unidade em $S$, então $u$ é uma unidade em $R$.
Apresente um exemplo que demonstre que a recíproca é falsa.

\end{exer} 
\begin{exer}
    Seja $A$ um anel. Mostre que um anel $A$ é um anel de divisão se, e somente se $A^*=A\setminus\{0\}$.
\end{exer}

\begin{exer}
    No anel dos quaternions $\mathbb H$, identifique $x \in \mathbb R$ com $(x, 0, 0, 0)=x+0i+0j+0k$.

    Mostre que $\mathbb R=Z(\mathbb H)$.

    (Dica: após mostrar que $\mathbb R\subseteq Z(\mathbb H)$, tome um elemento arbitrário de $Z(\mathbb H)$ e estude sua multiplicação por $i$, $j$ e $k$.)
\end{exer}
\begin{exer}\label{exer:quaternion}
    No anel dos quaternions, dado $q \in \mathbb H$, seu conjugado é definido como $\bar q = a + bi + cj + dk$.

    \begin{enumerate}[label=\alph*)]
        \item Calcule  $q\bar q$ e $q\bar q$.
        \item Prove que, se $q\neq 0$,  $\bar q(q\bar q)^{-1}$ é inverso multiplicativo de $q$.
        Conclua que $\mathbb H$ é anel de divisão.
    \end{enumerate}
\end{exer}
\begin{exer}
    Seja $A$ um anel. Prove que se $q \in Z(A)$ e $q$ é uma unidade, então $q^{-1} \in Z(A)$.
    Utilize esse fato para provar que o centro de qualquer anel de divisão é um corpo.
\end{exer}

\begin{exer}
    Seja $\mathbb Z[i] = \{m+in: m, n \in \mathbb Z\}\subseteq \mathbb C$ (o conjunto dos inteiros de Gauss).
    Mostre que $\mathbb Z[i]$ é um subanel de $\mathbb C$, e que é um domínio de integridade.
\end{exer}
