
\section{Anéis}
Nesta seção, começaremos a discutir a noção matemática de anel, uma das principais estruturas que serão estudadas.
\begin{definition}[Anel]
    Um anel é uma $4$-upla $(A, +, \cdot, 0, 1)$ conjunto $A$ com duas operações binárias, adição e multiplicação, denotadas por $+$ e $\cdot$, tais que:
    \begin{itemize}
        \item $(A, +, 0)$ é um grupo abeliano.
        \item (\textbf{Associatividade}) Para todo $a, b \in A$, temos $(a \cdot b)\cdot c = a\cdot(b\cdot c)$.
        \item (\textbf{Elemento identidade}) $\forall a \in A$ $1 \cdot a = a \cdot 1 = a$.
        \item (\textbf{Propriedades distributivas}) Para todos $a, b, c, \in A$, temos:
        \begin{align*}
            a \cdot (b + c) &= a \cdot b + a \cdot c, \text{ e}\\
            (a + b) \cdot c &= a \cdot c + b \cdot c
        \end{align*}
    \end{itemize}
    Se, adicionalmente, a seguinte propriedade é satisfeita, o anel é chamado de \emph{comutativo}.
    \begin{itemize}
        \item (\textbf{Comutatividade}) $\forall a, b \in A$ $a \cdot b = b \cdot a$.
    \end{itemize}
\end{definition}

Algumas observações:
\begin{itemize}
    \item Como em grupos, ao discursar sobre anéis é comum omitir as operações, referindo-se apenas ao conjunto $A$.
    \item Ao discursar sobre anéis, e a exemplo do que foi feito ao enunciar as propriedades dsitributivas, são utilizadas as convenções usuais sobre precedência de operações envolvidas por parênteses. Assim, $a + b \cdot c$ é interpretado como $a + (b \cdot c)$.
    \item Há textos que definem anéis sem incluir o elemento identidade $1$. Nestes textos, a definição acima dá nome ao que chamam de \emph{anéis com identidade}, ou \emph{anéis com 1}. Nesse curso, não usaremos essa convenção, de modo que \textbf{todos nossos anéis possuem identidade}. De modo similar, alguns textos definem anéis como sendo comutativos. Também não adotaremos essa convenção. \textbf{Os nossos anéis podem ser não comutativos}.
    \item A definição de anel não exige que $0=1$.
    \item $0$ é chamado de elemento nulo, e $1$ de elemento identidade.
\end{itemize}

\begin{prop}[Propriedade multiplicativa do $0$]
    Seja $A$ um anel. Então $\forall a \in A$ $0 \cdot a = a \cdot 0 = 0$.
\end{prop}
\begin{proof}
Provaremos a primeira afirmação. A segunda é análoga e fica como exercício.

Temos que $0\cdot a=(0+0)\cdot a=0\cdot a +0\cdot a$. Cancelando, segue que $0=0\cdot a$.
\end{proof}

\begin{prop}[Anel trivial]
    Seja $A={x}$ um conjunto qualquer. Defina $x\dot x=x=x+x=0=1$. Então $(A, +, \cdot, 0, 1)$ é um anel. Um anel dessa forma é chamado de \emph{anel trivial}.
    
    Além disso, se $A$ é um anel tal que $0=1$, então $A$ é um anel trivial.
\end{prop}
\begin{proof}
    A primeira afirmação (de que $A$ como acima é um anel) fica como exercício.

    Para a segunda afirmação, assuma que $A$ é um anel tal que $0=1$. Fixe $a \in A$ qualquer. Então $a=a.1=a.0=0$, ou seja, $a=0$. Assim, $A$ é o conjunto unitário $\{0\}$, que é um anel trivial.
\end{proof}

\begin{prop}[Regras de sinal II]\label{prop:regraSinal2}
    Seja $A$ um anel e $a, b \in A$. Então:
    \begin{enumerate}[label=\alph*)]
        \item $(-a)b=a(-b)=-(ab)$\label{prop:regraSinal2_A}
        \item $(-a)(-b)=ab$.\label{prop:regraSinal2_B}
        \item $(-1)a=-a$.\label{prop:regraSinal2_C}
    \end{enumerate}
\end{prop}
\begin{proof}
    \ref{prop:regraSinal2_A}: Temos que $ab+(-a)b=(-a)b+ab=[-a+a]b=0b=0$. Assim, $(-a)b=-(ab)$. Analogamente, $a(-b)=-(ab)$.
    \ref{prop:regraSinal2_B}: Temos que $(-a)(-b)=-[a(-b)]=-[-(ab)]=ab$ pela regra anterior.
    \ref{prop:regraSinal2_C}: Temos que $(-1)a=-(1a)=-a$.

\end{proof}

\section{Elementos invertíveis}
\begin{definition}[Elemento invertível]
    Seja $A$ um anel. Um elemento $a \in A$ é dito \emph{invertível}, ou uma \emph{unidade} se $\exists b \in A$ tal que $a \cdot b = b \cdot a = 1$.
    
    O conjunto de todas das unidades de $A$ é denotado por $A^*$.
\end{definition}

\begin{definition}
    Seja $A$ um anel. Então, se $a \in A^*$, existe um \textbf{único} $b \in A$ tal que $a \cdot b = b \cdot a = 1$. Este elemento é denotado por $a^{-1}$, e é chamado de \emph{inverso} de $a$.
\end{definition}

Observação: para que a definição acima faça sentido, é necessário mostrar que se $a$ é unidade, existe um \textbf{único} $b \in A$ tal que $a \cdot b = b \cdot a = 1$. A existência é garantida pela definição de unidade, e a demonstração da unicidade é análoga à da unicidade do inverso em grupos (Proposição \ref{prop:inverso_unico_grupo}), ficando como exercício.

\begin{prop}
Seja $A$ um anel. Para todos $a, b \in A^*$, temos:
\begin{enumerate}[label=\alph*)]
    \item $ab\in A^U$ e $(ab)^{-1}=b^{-1}a^{-1}$.\label{prop:unidadeProduto_a}
    \item $a^{-1}\in A^U$ e $(a^{-1})^{-1}=a$.\label{prop:unidadeProduto_b}
    \item $1^{-1}=1$.\label{prop:unidadeProduto_c}
\end{enumerate}
Além disso, $A^*$ é, com a restrição da operação de multiplicação do anel, um grupo com identidade $1$. Caso $A$ seja abeliano, $A^*$ é um grupo abeliano.
\end{prop}
\begin{proof}
    \ref{prop:unidadeProduto_a}: Sejam $a, b \in A^*$. Pela associatividade, $(ab)(b^{-1}a^{-1})=1=(b^{-1}a^{-1})(ab)$, logo, pela unicidade do inverso, $(ab)^{-1}=b^{-1}a^{-1}$.

    \ref{prop:unidadeProduto_b}: Seja $a \in A^*$. Temos que $a^{-1}a=1=a(a^{-1})$, logo, pela unicidade do inverso, $(a^{-1})^{-1}=a$.

    \ref{prop:unidadeProduto_c}: Note que $1\cdot 1=1=1\cdot 1$, logo, pela unicidade do inverso, $1^{-1}=1$.

    A última afirmação é imediata e fica como exercício.
\end{proof}

\begin{definition}
Um anel de divisão é um anel não trivial para o qual todo elemento não nulo é invertível. Um corpo é um anel de divisão comutativo.
\end{definition}

\begin{exer}
    Mostre que um anel $A$ é um anel de divisão se, e somente se $A^*=A\setminus\{0\}$.
\end{exer}
\begin{definition}
    Um domínio de integridade é um anel comutativo não trivial $A$ tal que $\forall a, b \in A$, se $ab=0$, então $a=0$ ou $b=0$.
\end{definition}

\begin{prop}
    Seja $K$ um corpo. Então $K$ é um domínio de integridade.
\end{prop}
\begin{proof}
Sabemos que $K$ é um anel comutativo não trivial. Sejam $a, b \in K$ tais que $ab=0$. Se $a=0$, então segue a tese. Caso contrário, como $K$ é um corpo, $a^{-1}$ existe. Assim, temos que $b=(a^{-1}a)b=a^{-1}(ab)=0$, logo, $b=0$.
\end{proof}
\section{Ideais}
\begin{definition}[Ideal à esquerda]
    Seja $A$ um anel. Um subconjunto $I \subseteq A$ é dito \emph{ideal à esquerda} se:
    \begin{itemize}
        \item $0 \in I$.
        \item Para todos $a, b \in I$, temos $a+b\in I$.
        \item $\forall a \in A$ e $\forall b \in I$, temos $ab \in I$.
    \end{itemize}
\end{definition}

\begin{definition}[Ideal à direita]
    Seja $A$ um anel. Um subconjunto $I \subseteq A$ é dito \emph{ideal à direita} se:
    \begin{itemize}
        \item $0 \in I$.
        \item Para todos $a, b \in I$, temos $a+b\in I$.
        \item $\forall a \in I$ e $\forall b \in A$, temos $ab \in I$.
    \end{itemize}
\end{definition}

\begin{definition}[Ideal]
    Seja $A$ um anel. Um subconjunto $I \subseteq A$ é dito \emph{ideal} se for um ideal à esquerda e um ideal à direita. Ou seja, $I$ é um ideal se:
    \begin{itemize}
        \item $0 \in I$.
        \item Para todos $a, b \in I$, temos $a+b\in I$.
        \item $\forall a \in A$ e $\forall b \in I$, temos $ab \in I$.
        \item $\forall a \in I$ e $\forall b \in A$, temos $ab \in I$.

    \end{itemize}
\end{definition}

\begin{prop}[Ideal trivial]Seja $A$ um anel. Então $\{0\}$ e $A$ são ideais de $A$. Estes ideais são chamados de \emph{ideais principais}
\end{prop}
\begin{proof}
    Exercício.
\end{proof}

Note que se $A$ é um anel comutativo, então $I$ é um ideal à esquerda se, e somente se, $I$ é um ideal à direita. Assim, em anéis comutativos, a noção de ideal é equivalente à de ideal à esquerda ou à de ideal à direita.

\begin{prop}[Interseção de ideais]
    Seja $A$ um anel e $\mathcal F$ uma coleção não vazia de ideais à esquerda de $A$. Então $\bigcap_{I \in \mathcal F}I=\bigcap \mathcal F$ é um ideal de $A$. O mesmo vale para ideais à direita e ideais.
\end{prop}
\begin{proof}
    Provaremos para ideais à esquerda. A prova para ideais à direita é análoga e fica como exercício.

    Seja $I=\bigcap \mathcal F$. Então $0 \in I$, pois $0 \in I$ para todo $I \in \mathcal F$.

    Sejam $a, b \in I$. Então, para todo $I \in \mathcal F$, temos que $a, b \in I$, logo, $a+b\in I$. Assim, $a+b\in \bigcap \mathcal F$.

    Finalmente, seja $a \in A$ e $b \in I$. Então, para todo $I \in \mathcal F$, temos que $b \in I$, logo, $ab\in I$. Assim, $ab\in \bigcap \mathcal F$.
\end{proof}

\begin{prop}[Ideal gerado]
    Seja $A$ um anel comutativo e $B\subseteq A$ um conjunto não vazio. Então, o conjunto $I=\{a_1b_1+\cdots+a_nb_n: n\geq 1, a_i \in A, b_i \in B\}$ é o menor ideal à esquerda $A$ que contém $B$ (ou seja, além de ser um ideal contendo $B$, se $J$ é qualquer ideal contendo $B$, então $I\subseteq J$). O ideal $I$ é chamado de \emph{ideal gerado por $B$}, e denotado por $\langle B \rangle$.
    
    Se $B=\{x_0, \dots, x_n\}$, então abreviamos $\langle B \rangle$ como $\langle x_0, \dots, x_n \rangle$.
\end{prop}
\begin{proof}
    Primeiro, verificaremos que $I$ é um ideal.

    $0 \in I$, pois $0=0b$ para todo $b \in B$.

    Sejam $x, y \in I$. Então existem $n, m\geq 1$, $a_1, \dots, a_n \in A$, $b_1, \dots, b_n \in B$, $c_1, \dots, c_m \in A$ e $d_1, \dots, d_m \in B$ tais que $x=a_1b_1+\cdots+a_nb_n$ e $y=c_1d_1+\cdots+c_md_m$. Assim, $x+y=(a_1b_1+\cdots+a_nb_n)+(c_1d_1+\cdots+c_md_m)=(a_1b_1+\cdots+a_nb_n)+(c_1d_1+\cdots+c_md_m) \in I$.

    Finalmente, seja $a \in A$ e $b \in I$. Então existem $n\geq 1$, $a_1, \dots, a_n \in A$ e $b_1, \dots, b_n \in B$ tais que $b=a_1b_1+\cdots+a_nb_n$. Assim, $ab=(a_1b_1+\cdots+a_nb_n)a=a_1(b_1a)+\cdots+a_n(b_na) \in I$.

    Agora, seja $J$ um ideal de $A$ que contém $B$. Então, como $J$ é um ideal de $A$, temos que $\forall a_i\in A$, $\forall b_i\in B$, temos que $(a_i b_i)\in J$. Logo, $I\subseteq J$. Portanto, $I$ é o menor ideal de $A$ que contém $B$.
\end{proof}

Observação: note que o menor ideal contendo $B=\emptyset$ é o ideal nulo, $\{0\}$.

\begin{prop}[Ideal principal]
    Seja $A$ um anel. Para todo $x \in A$, o conjunto $xA=\{xa:a \in A\}$ é um ideal à direita de $A$. O ideal $xA$ é chamado de \emph{ideal principal à direita gerado por $x$}.
    Analogamente, o conjunto $Ax=\{ax:a \in A\}$ é um ideal à esquerda de $A$, e é chamado de \emph{ideal principal à esquerda gerado por $x$}.
    Se $A$ é comutativo, o ideal $xA=Ax$ é chamado de \emph{ideal principal gerado por $x$}.
\end{prop}
\begin{proof}
Mostraremos que $xA$ é um ideal à direita. As demais afirmações ficam como exercício.

Note que $0 \in xA$, pois $x0=0$.

Sejam $a, b \in xA$. Então, existem $a_1, a_2 \in A$ tais que $a=xa_1$ e $b=xa_2$. Assim, $a+b=xa_1+xa_2=x(a_1+a_2) \in xA$.

Finalmente, seja $a \in A$ e $b \in xA$. Então, existe $b_1 \in A$ tal que $b=xb_1$. Assim, $ab=(xa)b_1=x(ab_1) \in xA$.
\end{proof}
\begin{definition}[Ideal principal]
    Seja $A$ um anel. Para todo $x \in A$, o conjunto $xA=\{xa:a \in A\}$ é um ideal à esquerda de $A$. O ideal $xA$ é chamado de \emph{ideal principal à esquerda gerado por $x$}.
    Analogamente, o conjunto $Ax=\{ax:a \in A\}$ é um ideal à direita de $A$, e é chamado de \emph{ideal principal à direita gerado por $x$}.
    Se $A$ é comutativo, o ideal $xA=Ax$ é chamado de \emph{ideal principal gerado por $x$}.
\end{definition}

Observação: note que, comparando as definições, se $A$ é um anel comutativo com unidade, $xA=\langle x\rangle$.

Notemos que ideais triviais são principais à esquerda e à direita, pois $0A=\{0\}=A0$ e $A1=A=1A$.

\begin{definition}[Domínio de ideais principais]
    Um domínio de ideais principais (DIP), ou anel principal, é um domínio de integridade $A$ tal que todo ideal de $A$ é principal.
\end{definition}

\begin{prop}[Ideais de um corpo são triviais]
    Todo ideal de um corpo é trivial. Em particular, todo corpo é um DIP. Reciprocamente, se $A$ é um anel comutativo não trivial cujo todo ideal é trivial, então $A$ é um corpo.
\end{prop}
\begin{proof}
Seja $K$ um corpo e $I$ um ideal de $K$. Se $I=\{0\}$, então $I$ é trivial. Se $I\neq \{0\}$, então existe $a \in I$ tal que $a \neq 0$. Daí $1=a^{-1}a=\in I$. Logo, para todo $k \in K$, $k=1k\in I$.

Para a recíproca, seja $A$ um anel comutativo não trivial tal que todo ideal de $A$ é trivial, e fixe $x \in A\setminus \{0\}$. Como $Ax$ é um ideal trivial e $0\neq x \in Ax$, temos que $Ax=A$. Logo, existe $a \in A$ tal que $ax=1$. Assim, $x$ é invertível. Portanto, $A$ é um corpo.
\end{proof}

\begin{prop}[$\mathbb Z$ é um DIP que não é um corpo]
    Seja $I$ um ideal de $\mathbb Z$. Veremos que $I$ é um ideal principal. Se $I=\{0\}$, então $I$ é principal. Caso contrário, $I$ contém ao menos um elemento positivo, já que, sendo $x\in I\setminus\{0\}$, temos que $-x \in I$ e um dos $x, -x$ é positivo.


    Seja $n$ o menor inteiro positivo de $I$. Afirmamos que $I=n\mathbb Z$. De fato, se $x \in I$, então escreva $x=qn+r$, onde $q,r \in \mathbb Z$ e $0\leq r<n$. Como $x \in I$, temos que $r=x-qn \in I$. Assim, $r=0$, ou violaríamos a minimalidade de $n$. Logo, $x=qn\in n\mathbb Z$. Portanto, $I\subseteq n\mathbb Z$. Como $n\mathbb Z=\langle n\rangle$ e $n \in I$, temos que $n\mathbb Z\subseteq I$, o que completa a prova.
\end{prop}
\section{Subanéis}
\begin{definition}[Subanel]
    Seja $A$ um anel. Um subanel de $A$ é um conjunto $B\subseteq A$, com as operações de $A$ restritas à $B$ e com mesmo $0$ e $1$ de $A$, é um anel. Note que, para isso, é necessário e suficiente que:
    
    \begin{itemize}
        \item $1 \in B$
        \item $a-b \in B$ para todo $a, b \in B$.
        \item $ab \in B$ para todo $a, b \in B$.
    \end{itemize}

\end{definition}