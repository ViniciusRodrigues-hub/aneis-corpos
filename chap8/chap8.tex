\chapter{Divisibilidade em anéis}
Neste capítulo, estudaremos a noção de divisibilidade em anéis.
Tal noção é uma generalização da noção de divisibilidade em $\mathbb Z$.

Trataremos de divisibilidade apenas em anéis comutativos, dando particular atenção aos domínios de integridade.

Neste capítulo, estudaremos diversos tipos de domínios de integridade que capturam boa parte das propriedades de divisibilidade dos números inteiros.

Uma esquematização encontra-se no diagrama abaixo:

\begin{center}
    \begin{tikzcd}
        \text{Domínios de integridade} \\ \text{Domínios de MDC} \arrow[u] \\ \text{Domínios Fatoriais} \arrow[u] \\ \text{Domínios de ideais principais} \arrow[u] \\ \text{Domínios Euclideanos} \arrow[u] \\ \text{Corpos} \arrow[u]\\
    \end{tikzcd}
\end{center}

\section{Definição de divisibilidade}

\begin{definition}
Seja $R$ um anel comutativo. Definimos a relação de divisibilidade, $\mid $, em $R$, como se segue:

Para $a, b \in R$, dizemos que $a\mid b$ ($a$ divide $b$) se existe $c \in R$ tal que $b=ac$.
\end{definition}
Algumas propriedades básicas:
\begin{prop}
    Seja $R$ um anel comutativo.
    Então a relação de divisibilidade $\mid $ em $R$ é uma pré-ordem, ou seja, é reflexiva e transitiva. Além disso, $1$ é elemento mínimo, e $0$, elemento máximo
\end{prop}

\begin{proof}Sejam $a, b, c \in R$.
    Temos que $a\mid a$, pois $a=1\cdot a$.

    Se $a\mid b$ e $b\mid c$, existem $e, f \in R$ tais que $b=ae$ e $c=bf$.
    Logo, $c=bf=aef=a(ef)$, o que implica em que $a\mid c$.

    Temos que $0$ é elemento máximo, já que para todo $a \in R$, $0=0a$.
    Por outro lado, $1$ é elemento mínimo, já que para todo $a \in R$, $a=1a$.
\end{proof}

Note que $a\mid 0$ não implica, pelas nossas definições, que $a$ é divisor de zero, uma vez que necessitaríamos da existência de $b \in R$ não nulo tal que $ab=0$.
Divisores de zero geram diversas patologias na teoria da divisibilidade, e estas não serão objeto primário de nosso estudo.

Assim, nos restringiremos aos anéis comutativos que não possuem divisores de zero, ou seja, aos domínios de integridade.
Note que, nesses domínios, se $b\neq 0$ e $b\mid a$, então existe um \emph{único} $c\in R$ tal que $a=bc$.
Este elemento $c$ é, muitas vezes, chamado de quociente de $a$ por $b$.

Outra vantagem é a caracterização de $a\mid b$ e $b\mid a$ a seguir:
\begin{prop}
    Seja $R$ um anel domínio de integridade. Se $a, b\in R$, são equivalentes:

    \begin{enumerate}
        \item $a\mid b$ e $b\mid a$.
        \item Existe $u \in R$ invertível tal que $a=ub$.
    \end{enumerate}

    \begin{proof}
        Primeiro, suponha que $a\mid b$ e $b\mid a$.
        Temos que existem $c, d$ com $a=cb$ e $b=da$.
        Substituindo, temos que $b=dcb$. Cancelando, $1=dc$. Assim, $c$ é invertível.

        Reciprocamente, como $u$ é invertível, $a=ub$ e $u^{-1}a=b$, logo, $a\mid b$ e $b\mid a$.
    \end{proof}
\end{prop}

Com isso, definimos:

\begin{definition}
Seja $R$ um anel comutativo.
Dizemos que elementos $a, b \in R$ são associados se existe $u \in R$ invertível tal que $a=ub$.
\end{definition}

A relação de ser associado é uma relação de equivalência:

\begin{lemma}
    Seja $R$ um anel comutativo. A relação de ser associado é uma relação de equivalência em $R$.
\end{lemma}
\begin{proof}
    Seja $a, b, c \in R$.
    \begin{itemize}
        \item Reflexividade: $a$ é associado a si, pois $a=1\cdot a$ e $1 \in R^*$.
        \item Simetria: Se $a$ é associado a $b$, então existe $u$ invertível tal que $a=ub$. Logo, $b=u^{-1}a$, e, portanto, $b$ é associado a $a$.
        \item Transitividade: Se $a$ é associado a $b$ e $b$ é associado a $c$, então existem $u, v$ invertíveis tais que $a=ub$ e $b=vc$.
        Logo, temos que $a=uvc$, e, portanto, $a$ é associado a $c$, já que $uv\in R^*$.
    \end{itemize}
\end{proof}

A relação de divisibilidade tem ligações com propriedades de ideais.
Enunciaremos a primeira delas a seguir:

\begin{prop}
Seja $R$ um anel comutativo e $a, b \in R$.
São equivalentes:
\begin{enumerate}[label=\alph*)]
    \item $a\mid b$.
    \item $b \in \langle a \rangle$.
    \item $\langle b\rangle \subseteq \langle a \rangle$.
\end{enumerate}
\end{prop}
\begin{proof}
    Primeiro, suponha que $a\mid b$.
    Então existe $c \in R$ tal que $b=ac$.
    Como $\langle a=\{ac: c \in R\}$, segue b).

    Agora suponha que $b \in \langle a\rangle$.
    Como $\langle b\rangle$ é o menor ideal que contém $b$ e $\langle a\rangle$ é um ideal que contém $b$, segue que $\langle b\rangle \subseteq \langle a\rangle$.

    Agora suponha que $\langle b\rangle \subseteq \langle a\rangle$.
    Em particular, $b \in \langle a\rangle$.
    Pela definição de $\langle a \rangle$, existe $c \in R$ tal que $b=ac$, e, assim, $a\mid b$.
\end{proof}

\section{Mínimo múltiplo comum e Máximo divisor comum}

Nesta seção, definimos a noção de mínimo múltiplo comum e máximo divisor comum de dois elementos em um domínio de integridade.
\begin{definition}
Seja $R$ um anel comutativo e $a_0, \dots, a_n \in R$ não nulos.

Um mínimo múltiplo comum de $a_0, \dots, a_n$ é, se existe, um elemento $m\in R$ tal que:
\begin{itemize}
    \item $a_0\mid m$, $\dots$, $a_n\mid m$
    \item Se $c\in R$ é tal que $a_0\mid c$, $\dots$, $a_n\mid c$, então $m\mid c$.
\end{itemize}

Um máximo divisor comum de $a_0, \dots, a_n$ é, se existe, um elemento $d\in R$ tal que:
\begin{itemize}
    \item $d\mid a_0$, $\dots$, $d\mid a_n$.
    \item Se $c\in R$ é tal que $c\mid a_0$, $\dots$,  $c\mid a_n$, então $c\mid d$.
\end{itemize}

O conjunto de todos os MMC's de $a, b$ é $\MMC(a_0, \dots, a_n)$.
O conjunto de todos os MDC's de $a, b$ é $\MDC(a_0, \dots, a_n)$.
\end{definition}

Note que, pela simetria da definição, $\MDC(a, b)=\MDC(b, a)$ e $\MMC(a, b)=\MMC(b, a)$.
É importante ressaltar que, no geral, $\MMC(a_0, \dots, a_n)$ e $\MDC(a_0, \dots, a_n)$ podem ser vazios, e, geralmente, quando não vazios, não são unitários.
De fato:
\begin{lemma}
    Seja $R$ um domínio de integridade e $a \in R$ não nulo.

    Sejam $a_0, \dots, a_n \in R$.
    Então todos os elementos de $\MDC(a_0, \dots, a_n)$ são associados entre si.
    Analogamente, todos os elementos de $\MMC(a_0, \dots, a_n)$ são associados entre si.

    Reciprocamente, todo elemento de $R$ associado a algum elemento de $\MDC(a_0, \dots, a_n)$ é um MDC de $a_0, \dots, a_n$, e todo elemento de $R$ associado a algum elemento de $\MMC(a_0, \dots, a_n)$ é um MMC de $a_0, \dots, a_n$.
\end{lemma}
\begin{proof}
Sejam $d, d'$ máximos divisores comuns de $a_0, \dots, a_n$.
Então, $d\mid a_0$, $\dots$, $d\mid a_n$, $d'\mid a_0$, $\dots$, $d'\mid a_n$.
Logo, $d\mid d'$ e $d'\mid d$.
Portanto, $d$ e $d'$ são associados entre si.

Similarmente, sejam $m, m'$ mínimos múltiplos comuns de $a_0, \dots, a_n$.
Então, $m\mid a_0$, $\dots$, $m\mid a_n$, $m'\mid a_0$, $\dots$, $m'\mid a_n$.
Logo, $m\mid m'$ e $m'\mid m$.
Portanto, $m$ e $m'$ são associados entre si.

Para a recíproca, se $d$ é MDC de $a_0, \dots, a_n$ e $d'$ é associado à $d$, então $d'\mid a_0$, $\dots$, $d'\mid a_n$, pois $d\mid a_0$, $\dots$, $d\mid a_n$ e $d'\mid d$.
Se $x\in R$ é tal que $x\mid a_0$, $\dots$, $x\mid a_n$, então $d'\mid d$ e $d\mid x$, logo $d'\mid x$.

Similarmente, se $m$ é MMC de $a, b$ e $m'$ é associado à $m$, então $m'$ é MMC de $a, b$.
\end{proof}
No geral, em domínios de integridade, podem existir pares de elementos sem MMC ou MDC.
Porém:

\begin{lemma}
Sejam $a, b \in R$ tais que $a\mid b$.
Então $a \in \MDC(a, b)$ e $b \in \MDC(a, b)$.

Em particular, $0 \in \MDC(a, 0)$ e $a \in \MMC(a, 0)$.
Além disso, se $a, b \neq 0$, temos $0 \notin \MMC(a, b)$. e $0 \notin \MDC(a, b)$.
\end{lemma}

\begin{proof}
    Suponha que $a\mid b$.
    Temos que $a\mid b$ e $a\mid a$, logo, $a$ é divisor comum de $a, b$.
    Se $r$ é divisor comum de $a, b$, temos que $r\mid a$, logo, $a \in \MDC(a, b)$.

    Temos que $a\mid b$ e $b\mid b$, logo, $b$ é múltiplo comum de $a, b$.
    Se $r$ é múltiplo comum de $a, b$, temos que $b\mid r$, logo, $b \in \MMC(a, b)$.

    Para a última afirmação, se $a, b \neq 0$, temos que $0 \notin \MDC(a, b)$, já que $0$ divide apenas $0$.
    Além disso, $0 \notin MMC(a, b)$, pois $ab\neq 0$ é múltiplo comum de $a, b$ e não é múltiplo de $0$.
\end{proof}


\section{Elementos primos e irredutíveis}
Os números inteiros possuem uma classe muito importante de números: a dos primos.
As definições abaixo generalizam a noção de primo.

\begin{prop}[Elementos primos]
    Seja $R$ um anel comutativo
    Dizemos que $p \in R$ é um elemento primo se $p\notin R^*$, $p\neq 0$, e, para todos $a, b \in R$, se $p\mid ab$, então $p\mid a$ ou $p\mid b$.
\end{prop}

\begin{prop}[Elementos irredutíveis]
    Seja $R$ um anel comutativo.
    Dizemos que $p \in R$ é um elemento irredutível se $p\notin R^*$, $p\neq 0$, e, para todos $a, b \in R$, se $p=ab$, então $a\in R^*$ ou $b\in R^*$.
\end{prop}

Elementos primos se relacionam com ideais primos, como vemos a seguir:
\begin{prop}
    Seja $R$ um anel comutativo e $p\neq 0$.
    Então, $p$ é primo se, e somente se $\langle p\rangle$ é um ideal primo.
\end{prop}
\begin{proof}
    Primeiro, suponha que $p$ é primo.
    Segue que $\langle p\rangle=\{ap: a \in R\}$ não é $0$, pois $p \in R$, e não é $R$, pois $p\notin R^*$.
    Agora, seja $a, b \in R$ tais que $ab \in \langle p\rangle$.
    Então $p\mid ab$.
    Logo, $p\mid a$ ou $p\mid b$, o que implica que $a\in \langle p\rangle$ ou $b \in \langle p\rangle$.

    Reciprocamente, suponha que $\langle p\rangle$ é um ideal primo.
    Veremos que $p$ é primo.
    Temos que $p\neq 0$ e $p$ não é invertível (pois $\langle p\rangle\neq R$).
    Agora, seja $a, b \in R$ tais que $p\mid ab$.
    Logo, $ab \in \langle p\rangle$.
    Como $\langle p\rangle$ é primo, temos que $a \in \langle p\rangle$ ou $b \in \langle p\rangle$.
    Logo, $p\mid a$ ou $p\mid b$.
\end{proof}

A seguinte proposição relaciona primos e irredutíveis em domínios de integridade.

\begin{prop}
    Seja $R$ um domínio de integridade
    Então, se $p \in R$ é primo, então $p$ é irredutível.
\end{prop}
\begin{proof}
    Seja $p \in R$ primo.
    Para ver que $p$ é irredutível, fixe $a, b \in R$ e suponha que $p=ab$.
    
    Como $ab=p$, temos que $p\mid ab$.
    Logo, $p\mid a$ ou $p\mid b$.
    Supondo $p\mid a$, temos que $a=pc$ para algum $c \in R$.
    Logo, $p=pcb$, e, portanto, $1=cb$, o que mostra que $b \in R^*$.

    O caso em que $p\mid b$ é análogo.
\end{proof}
A recíproca vale em domínios de ideais principais.

\begin{prop}
    Seja $R$ um domínio de ideais principais.
    Então, se $p \in R$ é irredutível, $p$ é primo.
\end{prop}
\begin{proof}
    Seja $p \in R$ irredutível.
    Veremos que $\langle p\rangle$ é primo.
    Para tanto, basta ver que $\langle p\rangle$ é maximal.
    Sabemos que esse ideal não é $R$.
    Suponha que existe $a \in R$ é tal que $\langle p\rangle \subseteq \langle a\rangle$.

    Então $p=ab$ para algum $b \in R$.
    Como $p$ é irredutível, temos que $a \in R^*$ ou $b \in R^*$.
    Se $a \in R^*$, então $\langle a\rangle=R$.
    Se $b \in R^*$, então $a=p\cdot b^{-1}\in \langle p\rangle$, e, portanto, $\langle p\rangle=\langle a\rangle$.
\end{proof}

\begin{lemma}
Seja $R$ um domínio de integridade e $p$ um elemento irredutível.

Se $q \in R$ é associado à $p$, então $q$ é irredutível.
\end{lemma}
\begin{proof}
    Seja $u$ invertível tal que $q=pu$.

    Primeiro, note que $q$ não é invertível, do contrário, teríamos que $p=qu^{-1} \in R^*$.
    Além disso, $q\neq 0$ já que $u, p\neq 0$.

    Se $a, b \in R$ tão tais que $q=ab$, então $pu=ab$.
    Então, $p=(u^{-1}a)b$.
    Assim, $u^{-1}a$ é invertível ou $b$ é invertível.
    Se $b$ é invertível, segue a tese.
    Se $u^{-1}a=v$ é invertível, então $a=uv$ também é, e segue a tese.
\end{proof}

\begin{lemma}
    Seja $R$ um domínio de integridade e $p$ um elemento primo.
    
    Se $q \in R$ é associado à $p$, então $q$ é primo.
\end{lemma}
    
\begin{proof}
    Como na proposição anterior, temos que $q\notin R^*$ e $q\neq 0$.

    Se $a, b \in R$ tão tais que $q\mid ab$, como $p\mid q$, segue que $p\mid ab$.
    Logo, $p\mid a$ ou $p\mid b$.
    Como $q\mid p$, segue que $q\mid a$ ou $a\mid b$.
\end{proof}



\section{Domínios de MDC}
Em $\mathbb Z$, sabemos que quaisquer dois elementos possuem um mdc (máximo divisor comum) e um mmc (mínimo múltiplo comum).

Nesta seção, estudaremos a classe de domínios de integridade que generaliza essa propriedade.

\begin{definition}
    Um domínio de integridade $R$ é um domínio de MDC se para todos $a, b \in R$ existe um mdc de $a$ e $b$.
\end{definition}

\begin{lemma}
    Seja $R$ um domínio de MDC. Então para todos $a_0, \dots, a_n \in R$, $\MDC(a_0, \dots, a_n)\neq \emptyset$.
\end{lemma}
\begin{proof}
Indução em $n$.

Para $n=0$, temos que $a_0\in \MDC(a_0)$.

Suponha que a tese vale para $n$.
Veremos que vale para $n+1$.
Com efeito, sejam dados $a_0, \dots, a_{n+1}\in R$. Fixe $m \in \MDC(a_0, \dots, a_n)$.
Tome $k \in \MDC(m, a_{n+1})$.

Temos que $k\mid m$, logo $k\mid a_0, \dots, k\mid a_n$, e $k\mid a_{n+1}$.

Agora, se $x \in R$ é tal que $x\mid a_0, \dots, x\mid a_{n+1}$, então $x\mid m$ e $x\mid a_{n+1}$, logo $x\mid k$.
\end{proof}

Como primeira propriedade, temos:

\begin{prop}
    Seja $R$ um domínio de MDC e $a_0, \dots, a_n\in R$.
    Então, para todo $x \in R$, temos $x\MDC(a_0, \dots, a_n)=\MDC(xa_0, \dots, xa_n)$.
\end{prop}
\begin{proof}
    Se $x=0$, a igualdade nos diz que $\{0\}=\{0\}$.
    Assim, suponha que $x\neq 0$.

    Seja $m\in \MDC(a_0, \dots, a_n)$ e $\bar m \in \MDC(xa_0,\dots xa_n)$.
    Veremos que $xm$ e $\bar m$ são associados.

    Como $m\in \MDC(a_0, \dots, a_n)$, temos que $m\mid a_0, \dots, m\mid a_n$.
    Logo, $xm\mid xa_0, \dots, xm\mid xa_n$.
    Assim, $xm\mid \bar m$.

    Para a recíproca, temos que $x\mid xa_0, \dots, x\mid xa_n$. Segue que $x\mid \bar m$.
    Escreva $\bar m=xc$ para algum $c \in R$.
    Temos que $xc\mid xa_0, \dots, xc\mid xa_n$.
    Cancelando, $c\mid a_0, \dots, c\mid a_n$.
    Logo, $c\mid m$. Assim, $\bar m=xc\mid xm$.
\end{proof}

Poderíamos definir também a noção de domínio de MMC, mas isso não é necessário:

\begin{prop}
    Seja $R$ um domínio de integridade.
    Então $R$ é um domínio de MDC se, e somente se para todos $a, b \in R$, $\MMC(a, b)\neq \emptyset$.

    Além disso, nesses casos, se $a, b \neq 0$ e $m\in \MMC(a, b)$, então o quociente de $ab$ por $m$ é um MDC de $a$ e $b$.

    Analogamente, se $a, b \neq 0$ e $d\in \MDC(a, b)$, então o quociente de $ab$ por $d$ é um MMC de $a$ e $b$.
\end{prop}

\begin{proof}
    Primeiro, suponha que $R$ é um domínio de MDC.
    Fixe $a, b \in R$,
    veremos que existe um MMC de $a, b$.
    Se $a=0$ ou $b=0$, então $0 \in MMC(a, b)$.
    Assim, suponha que $a\neq 0\neq b$.
    Segue que $0 \notin \MDC(a, b)$.

    Seja $d \in \MDC(a, b)$.
    Escreva $a=da'$ e $b=db'$.
    Seja $m=da'b'=ab'=ba'$.
    Está claro que $a\mid m$ e $b\mid m$, e $ab=md$.

    Suponha que $x\in R$ é tal que $a\mid x$ e $b\mid x$.
    Então $md=ab\mid xb$ e $md=ab\mid xa$.
    Como $xd\in \MDC(ax, bx)$, temos que $xd\mid md$.
    Como $d\neq 0$, temos que $x\mid m$.
    Isso conclui que $m \in \MMC(a, b)$.

    Reciprocamente, suponha que para todos $a, b \in R$, $\MMC(a, b)\neq \emptyset$. Veremos que para todos $a, b \in R$, $\MDC(a, b)\neq \emptyset$.
    Se $a=0$ ou $b=0$, então $0 \in \MDC(a, b)$.
    Assim, vamos supor $a, b \neq 0$.

    Fixe $m \in \MMC(a, b)$.
    Temos que $a\mid m$ e $b\mid m$, que $ab\mid am$ e $ab\mid bm$.
    Escreva $m=aa'=bb'$.
    Como $ab$ é múltiplo de $a$ e $b$, temos que $m\mid ab$.
    Escreva $ab=dm$.
    Afirmamos que $d \in \MDC(a, b)$.
    
    Com efeito, $ab=dm=daa'=dbb'$, logo, $b=da'$ e $b=db'$.
    Assim, $d\mid a$ e $d\mid b$.
    Agora seja $x\in R$ tal que $x\mid a$ e $x\mid b$.
    Temos que $x\mid ab$.
    Escreva $ab=xy$, $a=\bar ax$, $b=\bar by$.
    Então $ab=\bar axb=\bar bxa=xy$, assim, $\bar ab=\bar ba=y$.
    Logo, $b\mid y$ e $a\mid y$, e, assim, $m\mid y$.

    Escreva $y=my'$.
    Segue que $ab=xy=xmy'=dm$.
    Cancelando $m$, segue que $xy'=d$, ou seja, $x\mid d$.
\end{proof}

\begin{corol}
    Seja $R$ um domínio de MDC.
    Então para todos $a, b, x \in R$, temos que $x\MMC(a, b)=\MMC(xa, xb)$.
\end{corol}
\begin{proof}
Seja $x=0$ a igualdade é $\{0\}=\{0\}$.
Se $a=0$, temos que $xb$ está em ambos os conjuntos.
Se $b=0$, temos que $xa$ está em ambos os conjuntos.

Assim, suponha $a, b, x \neq 0$.
Seja $d\in \MDC(a, b)$.
Escreva $md=ab$.
Assim $(xm)(xd)=(xa)(xb)$ e $xd \in MDC(a, b)$.
Logo, $(xm)\in \MMC(xa, xb)$.
\end{proof}

\begin{lemma}
    Em um domínio de MDC, todo elemento irredutível é primo.
    \end{lemma}
    
\begin{proof}
    Seja $p \in R$ um elemento irredutível.
    Sejam $a, b \in R$ com $p\mid ab$.
    Veremos que $p\mid a$ ou $p\mid b$.

    Seja $d\in\MDC(p, a)$.
    Então $bd\in \MDC(pb, ab)$.
    Como $p\mid pb$ e $p\mid ab$, segue que $p\mid bd$.

    Como $d\mid p$, então $d$ é invertível ou é associado à $p$.

    Se $d$ é associado à $p$, temos que $p\mid d$, e, portanto, $p\mid a$.

    Se $d$ é invertível, temos, de $p\mid bd$, que $p\mid b$.
\end{proof}

Finalmente, em um domínio de MMCs de quantidades finitas arbitrárias de elementos existem.

\begin{prop}
    Seja $R$ um domínio de MDC e $a_0, \dots, a_n \in R$.
    Então $\MMC(a_0, \dots, a_n)\neq \emptyset$ não são vazios.
\end{prop}

\begin{proof}
Provaremos por indução em $n$.
Para $n=0$, note que $a \in \MMC(a)$.
Suponha que a proposição vale para $n$. Sejam $a_1, \dots, a_{n+1} \in R$.
Seja $m\in \MMC(a_1, \dots, a_n)$.

Seja $m' \in \MMC(a_{n+1}, m)$.
Afirmamos que $m'\in \MMC(a_1, \dots, a_{n+1})$.

$m'$ é múltiplo comum de $a_1, \dots, a_{n+1}$: temos que $a_{n+1}\mid m'$, $m\mid m'$, logo $a_1\mid m'$, $\dots$, $a_n\mid m'$.

Se $x$ é múltiplo comum de $a_1, \dots, a_{n+1}$, então $m\mid x$ e $a_{n+1}\mid x$, logo $m'\mid x$.
\end{proof}
\section{Domínios de Fatoração Única}
Domínios de Fatoração Única, também conhecidos como Domínios Fatoriais, ou Anéis Fatoriais, são domínios de integridade que capturam outra propriedade dos números inteiros:
a do Teorema Fundamental da Aritmética.

\begin{definition}
    Um Domínio de Fatoração Única (DFU) é um domínio de integridade $R$ que satisfaz as condições abaixo.
    
    \begin{enumerate}
        \item Para todo $a\in R\setminus \{0\}$ não invertível, existe um inteiro $n\geq 1$, irredutíveis $p_1, \dots, p_n$ tais que $a=p_1\cdots p_n$.
        \item Para todos $m, n\geq 1$ e irredutíveis $q_1, \dots, q_m, p_1, \dots, p_m$, se $q_1\dots q_m p_1\dots p_m$, então $n=m$ e existe uma permutação $\sigma:\{1\dots, n\}\rightarrow \{1, \dots, n\}$ tal que $p_i$ é associado a $q_{\sigma(i)}$ para todo $i=1, \dots, n$.
    \end{enumerate}
\end{definition}

\begin{exemplo}
    Conforme argumentaremos mais a frente, $\mathbb Z$ é um domínio de fatoração única. Note que $2, 3, -2, -3$ são irredutíveis em $\mathbb Z$, e que $6=2\cdot 3=(-2)\cdot(-3)$. Perceba que $2$ é associado à $-2$ e $3$ é associado à $-3$.
\end{exemplo}

\begin{lemma}
    Seja $R$ um DFU. Se $n, m\geq 1$ e $p_1, \dots, p_n, q_1, \dots, q_m$ são elementos irredutíveis e $(p_1\dots p_n)|(q_1\dots q_m)$, então existe $\sigma:\{1, \dots, n\}\rightarrow\{1, \dots, m\}$ injetora tal que $p_i$ é associado a $q_{\sigma(i)}$ para todo $i \in \{1, \dots, n\}$. 
\end{lemma}
\begin{proof}
    Seja $x$ tal que $q_1 \dots q_m=x p_1\dots, p_n$. Temos que $x\neq 0$, pois $q_1\dots q_n\neq 0$, uma vez que $R$ é um domínio.

    Se $x$ é invertível, $(xp_1)$ é associado à $p_1$, logo, $x p_1$ é irredutível. Como $R$ é um DFU, $n=m$ e existe $\sigma$ permutação de $\{1, \dots, n\}$ tal que $xp_1$ é associado à $q_{\sigma(1)}$ e, para $2\leq i\leq n$, $p_i$ é associado à $q_i$. Como $p_1$ é associado à $xp_1$, segue a tese.

    Se $x$ não é invertível escreva $x=p_{n+1}\dots p_{n+k}$, com $x_1, \dots, x_k$ irredutíveis. Temos que $q_1\dots q_m=p_1\dots p_{n+k}$. Como $R$ é um DFU, $k+n=m$ e existe $\sigma$ permutação em $n+k=m$ tal que $p_i$ é associado à $q_{\sigma(i)}$ para todo $i<k$. Restringindo-se $\sigma$ à $\{1, \dots, n\}$, segue a tese.
\end{proof}

\begin{corol}\label{corol:DFUprime}
Em um DFU, todo elemento irredutível é primo.
Assim, em um DFU, todo elemento não nulo e não invertível se escreve como um produto de primos.
\end{corol}

Para a próxima proposição, observe que se $R$ é um domínio, $a, b, c, d \in R\setminus \{0\}$ e se $a$ é associado à $b$ e $a\neq 0$, então $c\mid d$ se, e somente se $ac\mid ad$.
\begin{prop}
    Todo DFU é um domínio de MDC.
\end{prop}
\begin{proof}
Seja $R$ um DFU.
Se $a\mid b$ ou $b\mid a$, temos que o $\MDC(a, b)\neq \emptyset$.
Assim, vamos supor que $a\nmid b$ e $b\nmid a$. Em particular, $a, b \neq 0$ e $a, b \notin R^*$.

Escreva $a=p_1\dots p_n$ e $b=q_1\dots q_m$ com $p_1, \dots, p_n$ e $q_1, \dots, q_m$ irredutíveis.

Reordenando, podemos transformar $(p_1, \dots, p_n)$ nas sequências $(c_1, \dots, c_k)$ e $(c_{k+1}, \dots, c_{k+l})$, e $(q_1, \dots, q_m)$ nas sequências $(c_1', \dots, c_k')$ e $(c_{k+1}', \dots, c_{k+l'}')$ de modo que:

\[a=\prod_{i=1}^{k+l}c_i,\]
\[b=\prod_{i=1}^{k+l'}c_i'.\]

Onde cada $c_i$ é associado a $c_i'$ para $1\leq i\leq k$ e nenhum $c_i$ é associado a nenhum $c_j'$ para $k+1\leq i\leq k+l$ e $k+1\leq j\leq k+l'$.
Note que $a\nmid b$ e $b\nmid  a$, temos que as famílias $(c_{k+1}, \dots, c_{k+l})$ e $(c_{k+1}', \dots, c_{k+l'}')$ são não vazias.


Afirmamos que $c=\prod_{i \in I} c_i$ é um MDC de $a$ e $b$.

Está claro que $c\mid a$ e $c\mid b$.
Suponha que $d\mid a$ e $d\mid b$.
Devemos ver que $d\mid c$.
Temos que $d\neq 0$, e se $d$ é invertível, segue a tese.

Suponha por absurdo que $d\nmid  c$.

Se $d$ não é invertível, escreva $d=d_1\dots d_t$ com $d_i$ irredutíveis. Veremos, por indução, que para todo $s\in\{1, \dots, t\}$, que $d_1\dots d_s|c$, o que concluirá a prova.

Para $s=1$, temos que $d_1|c_1\dots c_{k+l}$ e $d_1|c_1'\dots c_{k+l}'$. Pelo lema anterior, existem $i, j$ tais que $d_1$ é associado à $c_i$ e $d_1$ é associado à $c_j'$. Se $i, j>k$ temos que $c_i$ é associado à $c_j'$, o que é um absurdo. Assim, $i\leq k$ ou $j\leq k$, e, em qualquer caso, temos $d_1|c$.

Suponha que a hipótese vale para $s<t$. Veremos que vale para $s+1$.

Temos que $d_1\dots d_s|c$, logo, pelo lema anterior, existe $\sigma:\{1, \dots, s\}\rightarrow\{1, \dots, k\}$ injetora tal que cada $d_i$ é associado à $c_{\sigma(i)}$, e, consequentemente, à $c_{\sigma(i)}'$. Como $d_1\dots d_sd_{s+1}|c_1\dots c_{k+l}$, cancelando os elementos associados por $\sigma$, temos que existe $\bar i$ fora da imagem de $\sigma$ tal que $d_{s+1}$ é associado à $c_{\bar i}$. Analogamente, existe $\bar j$ fora da imagem de $\sigma$ tal que $d_{s+1}$ é associado à $c_{\bar j}'$. Se $\bar i, \bar j>k$, temos um absurdo. Logo, sem perda de generalidade, $\bar i\leq k$ e $d_1\dots d_{s+1}|c_{\sigma(1)}\dots c_{\sigma(s)}c_{\bar j}$ e $c_{\sigma(1)}\dots c_{\sigma(s)}c_{\bar j}|c$.
\end{proof}


A recíproca do Corolário~\ref{corol:DFUprime} é verdadeira. Primeiro, precisamos:

\begin{lemma}
Seja $R$ um domínio de integridade, $p\in R$ um elemento primo e $q_1, \dots, q_n\in R$ irredutíveis.
Se $p\mid q_1\cdots q_n$, então existe $i$ tal que $p$ é associado à $q_i$.
\end{lemma}
\begin{proof}
Seguimos por indução em $n$.
Para $n=1$, se $p\mid q_1$, temos que $q_1=pu$ para algum $u\in R$.
Assim, $p\in R^*$ ou $u\in R^*$.
Como $p \notin R^*$, segue que $u \in R^*$, logo, $p$ é associado à $q_1$.

Para o passo indutivo, suponha que a tese é verdadeira para $n$ e que $p\mid (q_1\dots q_{n+1})$.
Temos que $p\mid (q_1\dots q_{n})$ ou $p\mid q_{n+1}$.
As hipóteses indutivas concluem a prova.
\end{proof}

\begin{prop}
Suponha que $R$ é um domínio em que todo elemento não nulo, não invertível se escreve como um produto de primos.
Então $R$ é um DFU.
\end{prop}
\begin{proof}
    Primeiro, verificaremos que todo irredutível é primo.
    Suponha que $q$ é irredutível.
    Escreva $q=p_1\cdots p_n$, com $p_1, \dots, p_n$ primos.
    Temos que cada $p_i$ divide $q$.
    Assim, pelo lema, existe $i$ tal que $p_i$ é associado a $q$.
    Em particular, $q$ é primo.


    Para provar a proposição, mostraremos, por indução em $k\geq 1$, que para todos $n, m$ com $1\leq n, m\leq k$ e $p_1, \dots, p_n, q_1, \dots, q_m$  irredutíveis, se $p_1\dots p_n$ é associado à $q_1\dots q_m$, então $n=m$ e existe uma permutação $\sigma:\{1, \dots, n\}\rightarrow \{1, \dots, n\}$ tal que $p_i$ é associado a $q_{\sigma(i)}$ para todo $i=1, \dots, n$.

    Para $k=1$, temos que $n=m=1$ e $p_1$ é associado à $q_1$.

    Para o passo indutivo, suponha que a tese é verdadeira para $k$. Sejam $p_1, \dots, p_{n}, q_1, \dots, q_{m}$ irredutíveis (primos) com $1\leq n, m\leq k+1$ tais que $p_1\dots p_{n}=q_1\dots q_{m}$.
    Se $m, n\leq k$, a hipótese indutiva conclui a prova.
    Sem perda de generalidade, vamos supor $n=k+1$.
    Assim, $(p_1\cdots p_{k})p_{k+1}=q_1\cdots q_{m}$.

    Como $p_{k+1}\mid q_1\cdots q_m$, segue do lema anterior que existe $i$ tal que $p_{k+1}$ é associado a $q_i$.

    Notemos que $m\geq 2$, caso contrário teremos que $p_1\dots p_k$ é invertível, o que implica que cada $p_i$ é invertível.

    Tome $\theta: \{1, \dots, m\}\rightarrow \{1, \dots, m\}$ de modo que  $\theta(m)=i$. Escreva $p_{k+1}=w q_i$ com $w \in R^*$.
    Temos que $p_1\cdots p_{k}w p_i=q_1\cdots q_{m}=q_{\theta(1)}\cdots q_{\theta(m)}$.
    Cancelando, segue que $wp_1\cdots p_{k}=q_{\theta(1)}\cdots q_{\theta(m-1)}$.
    Por hipótese indutiva, temos que $m-1=k$ e existe $\sigma: \{1, \dots, k\}\rightarrow \{1, \dots, k\}$ tal que $wp_1$ é associado à $q_{\theta(\sigma(1))}$, e, para $2\leq j\leq k-1$, $p_j$ é associado a $q_{\theta(\sigma(j))}$ para todo $j=1, \dots, k$.

    Seja $\sigma'=\theta\circ\sigma \cup \{(k+1, i)\}$, o que conclui a prova.
\end{proof}

\section{Domínios de Ideais Principais}
Nessa seção, veremos que todo DIP é um DFU, bem como o conhecido Lema de Bézout.
\begin{lemma}
Seja $R$ um domínio de ideais principais. Então todo elemento não nulo, não invertível, possui um divisor irredutível.
\end{lemma}
\begin{proof}
    Fique $a \in R$ como no enunciado e suponha por absurdo que $a$ não possui divisor irredutível. Recursivamente, construiremos uma sequência $(p_n: n \in \mathbb N)$ de elementos de $\mathbb R$ satisfazendo:
    \begin{enumerate}[label=\alph*)]
        \item $p_0=a$.
        \item $p_n\mid p$ para todo $n \in \mathbb N$
        \item $\langle p_n\rangle\neq R$.
        \item $\langle p_n\rangle\subsetneq \langle p_{n+1}\rangle$.
    \end{enumerate}
    O que nos dará um absurdo, por d).

    Para ver que tal sequência existe, seja $p_0=a$. Como $a$ não é invertível, $\langle p_n\rangle \neq R$, e $p_0\mid a$.

    Para o passo sucessor da recursão, construído $p_n$, como $\langle p_n\rangle\neq R$, sabemos que $p_n$ não é invertível.
    Além disso, $0\neq p_n$, pois $p_n\mid a$.
    Pelo mesmo motivo, $p_n$ não é irredutível.
    Logo, existem $b, c \in R$ tais que $p_n=bc$ e nem $b$, nem $c$ são invertíveis, nem associados à $p_n$.
    Seja $p_{n+1}=b$.

    Como $p_{n+1}\mid p_n$ e $p_n\mid p$, segue que $p_{n+1}\mid p$.
    Uma vez que $b$ não é invertível, $\langle p_{n+1}\rangle\neq R$.

    Como $p_{n+1}\mid p_n$, temos que $\langle p_n\rangle\subseteq \langle p_{n+1}\rangle$.
    Porém, a inclusão é própria, caso contrário, teríamos $p_{n+1}$ e $p_n$ são associados.
\end{proof}

\begin{lemma}
    Seja $R$ um domínio de ideais principais e $a\in R\setminus \{0\}$ não invertível.
    Então existem $p_1, \dots, p_n$ irredutíveis tais que $a=p_1\cdots p_n$.
\end{lemma}

\begin{proof}
    Suponha que não.
    Em particular, $a$ não é irredutível.
    Recursivamente, tomamos $(p_n: n \in \mathbb N)$ e $(a_n: n \in \mathbb N)$, com $a_n\in R$ satisfazendo:
    \begin{enumerate}[label=\alph*)]
        \item $a=(p_0\cdots p_n) a_n$.
        \item $a_{n}=a_{n+1}p_{n+1}$.
        \item Cada $p_n$ é irredutível.
    \end{enumerate}
    Para ver que isso é possível, como $a$ é não nulo, não invertível e não irredutível, existe um irredutível $p_0$ tal que $a=p_0a_0$.
    
    Para o passo indutivo, como $a\neq 0$ não é um produto de irredutíveis, $a_n$ não é invertível (pois $p_na_n$ seria também irredutível) e não nulo.
    Logo, existem $p_{n+1}$ e $a_{n+1}$ tais que $a_n=p_{n+1}a_{n+1}$ com $p_{n+1}$ irredutível.

    Isso completa a construção.
    Assim, $a_{n+1}\mid a_{n}$, mas $a_{n}\not \neq a_{n+1}$, ou teríamos que existe $b$ tal que $a_{n+1}=ba_n$, e, substituindo, seguiria que $a_n=a_nbp_{n+1}$, o que nos dá $p_{n+1}$ é invertível, um absurdo.

    Portanto, segue que para todo $n$, $\langle a_n\rangle\subsetneq \langle a_{n+1}\rangle$, uma contradição.
\end{proof}

\begin{corol}
    Todo domínio de ideais principais é um domínio de fatoração única.
\end{corol}

\begin{prop}[Lema de Bézout]
    Seja $R$ um domínio de ideais principais, $a_0, \dots, a_n \in R$ e $d\in \MDC(a_0, \dots, a_n)$
    Então existem $x_0, \dots, x_n \in R$ tais que $d=x_0a_0+\dots+x_na_n$ e $d$ gera o ideal $\langle a_0, \dots, a_n\rangle$.
\end{prop}
\begin{proof}
    Seja $d'$ um gerador do ideal $\langle a_0, \dots, a_n\rangle$.

    Para cada $i\leq n$, $d|a_i$, logo, $a_i \in \langle d\rangle$, assim, segue que $\langle a_0, \dots, a_n\rangle \subseteq \langle d\rangle$.

        Seja $d'$ um gerador do ideal $\langle a_0, \dots, a_n\rangle$. Temos que, para cada $i\leq n$, $d'|a_i$, logo, $d'|d$. Assim, $\langle d\rangle\subseteq \langle d'\rangle\subseteq = a_0, \dots, a_n\rangle$.

        Conclui-se que $\langle d\rangle = \langle a_0, \dots, a_n\rangle$. Em particular, $d \in \langle a_0, \dots, a_n\rangle$, o que nos dá a existência de $x_0, \dots, x_n \in R$ tais que $d=x_0a_0+\dots+x_na_n$.
\end{proof}
\section{Domínios Euclideanos}
O anel dos números inteiros possui uma propriedade muito importante:
dado $n\in \mathbb Z$ e $d>0$, existem únicos $n, r$ tais que $0\leq r < d$ e $a=nd+r$.

A noção de domínio Euclideano generaliza os anéis que possuem essa propriedade.

\begin{definition}
    Um domínio de integridade $R$ é um domínio Euclideano
    se existe uma função $\nu:R\setminus \{0\} \to \mathbb N=\{0, 1, \dots\}$ satisfazendo:
    \begin{itemize}
        \item para todos $a, b \in R$ com $b\neq 0$, existem $q, r \in R$ com $a=bq+r$ e ($r=0$ ou $\nu(r)<\nu(b)$), e,
        \item para todos $a, b \in R\setminus \{0\}$, $\nu(ab)\geq \nu(a)$.
    \end{itemize}

    Tal função $\nu$ é chamada de valoração, ou grau.
\end{definition}
Um primeiro resultado simples:
\begin{definition}
    Seja $R$ um domínio Euclideano e $\nu$ uma função de valoração.
    Então se $a, b \in R$ são associados, temos $\nu(a)=\nu(b)$.
\end{definition}
\begin{proof}
    Se $a$ e $b$ são associados, existe $u \in R^*$ tal que $a=ub$.
    Logo, $\nu(a)=\nu(ub)\geq \nu(b)$ e $\nu(b)=\nu(u^{-1}a)\geq \nu(a)$.
    Assim, $\nu(a)=\nu(b)$.
\end{proof}

\begin{exemplo}
    O anel dos inteiros $\mathbb Z$ é um domínio Euclideano, com $\nu(n)=|n|$ para $n\neq 0$.
    Primeiro, é claro que se $a, b\in \mathbb Z$ são não nulos, então $|ab|=|a||b|\geq |a|1=|a|$.

    Agora, sejam $a, b\in \mathbb Z$ com $b\neq 0$.
    Sabemos que existem $q, r \in \mathbb Z$ tais que $a=q|b| +r$ e $0\leq r < | b| $.
    Se $b>0$, isso conclui a prova.
    Se $b<0$, então $a=(-q)b+r$, e isso conclui a prova.
\end{exemplo}

\begin{exemplo}
    No geral, não podemos exigir a unicidade de $q, r$.
    De fato, em $\mathbb Z$, considere $a=3$, $b=2$.
    Então $3=1\cdot 2+1$ com $|1| <|2| $, mas também $3=2\cdot 2+(-1)$ com $|-1|<|2|$.
\end{exemplo}

Porém, temos o resultado a seguir.
Mais adiante, veremos que esse será o caso para anéis de polinômios sobre corpos munidos da função grau.


\begin{prop}
    Seja $R$ um domínio Euclideano e $\nu$ uma função de valoração tal que
    para todos $a, b\in R$ com $a, b, a+b\neq 0$, temos $\nu(a+b)\leq \max(\nu(a), \nu(b))$.
    
    Então para todos $a, b \in R$ com $b\neq 0$, existem únicos $q, r \in R$ tais que $a=bq+r$ e ($r=0$ ou $\nu(r)<\nu(b)$).
\end{prop}
\begin{proof}
    A existência de $q, r$ como acima vem da definição de domínios Euclideanos.
    Adicionalmente, sejam $q', r' \in R$ tais que $a=bq'+r'$ e $r'=0$ ou $\nu(r')<\nu(b)$.

    Temos que $q'b+r'=qb+r$.
    Se $r=r'$, segue que $q=q'$.
    Similarmente, se $q=q'$, segue que $r=r'$.
    Portanto, vamos supor por absurdo que $r\neq r'$ e $q\neq q'$.
    Assim, $r'-r=(q-q')b+(r-r')$.
    
    Se $r=0$, temos que $\nu(r')<\nu(b)\leq \nu((q-q')b)=\nu(r)$, o que é absurdo.

    Se $r'=0$, temos que $\nu(-r)=\nu(r)<\nu(b)\leq(\nu(q-q')b)=\nu(r)$, o que é absurdo.
    
    Finalmente, se $r, r'\neq 0$, temos que $\nu(r'-r)\leq \max(\nu(r), \nu(r'))<\nu(b)\leq(\nu(q-q')b)=\nu(r)$, o que é absurdo.
\end{proof}

\begin{prop}
    Todo corpo é um domínio Euclideano.
\end{prop}
\begin{proof}
Seja $K$ um corpo e considere $\nu:K\setminus \{0\}\to N$ dada por $\nu(x)=1$ para todo $x\in K\setminus \{0\}$.

    Então, para $a, b \in K\setminus \{0\}$, temos que $\nu(ab)=1=\nu(a)$,
    e, dados $a, b\in K$ com $b\neq 0$, temos que $a=(ab^{-1})b+0$.
\end{proof}

O resultado abaixo generaliza o que também já sabemos sobre $\mathbb Z$.
\begin{prop}
Todo domínio Euclideano é um domínio de ideais principais.
\end{prop}
\begin{proof}
    Seja $R$ um domínio Euclideano com valoração $\nu$ e $I$ um ideal em $R$.

    Se $I=\{0\}$, então $I$ é gerado por $0$.

    Se $I\neq \{0\}$, então existe $b\in I$ tal que $\nu(b)$ é mínimo.
    Afirmamos que $I=\langle b\rangle$.
    É claro que $\langle b\rangle\subseteq I$, restando verificar que $I\subseteq \langle b\rangle$.
    De fato, tome $a \in I$.
    Existem $q, r \in R$ tais que $a=bq+r$ e $r=0$ ou $\nu(r)<\nu(b)$.
    Se $r\neq 0$, temos que $\nu(r)<\nu(b)$, o que nos dá um absurdo, uma vez que $r=a-bq\in I$ e $b$ é o elemento de menor valoração em $I$.
    Assim, $r=0$, e, portanto, $a=bq\in \langle b\rangle$.
\end{proof}





\section{Exercícios}

\begin{exer}
    Prove, com suas próprias palavras e de modo que considere satisfatório, que a relação de ser associado, em um anel comutativo, é uma relação de equivalência.
\end{exer}

\begin{exer}
    Seja $D$ um domínio de integridade e $a, b \in R$ não nulos.
    Redija com suas palavras, de forma que considere satisfatória, uma demonstração para que quaisquer dois mínimos múltiplos comuns de $a$ e $b$ são associados entre si, e que quaisquer dois máximos divisores comuns de $a$ e $b$ são associados entre si, caso existam.
\end{exer}
\begin{exer}
    Seja $R$ um domínio Euclideano munido de função grau $\nu$ e $a \in R$ não nulo.
    Seja $m$ o menor valor assumido por $\nu$.
    Prove que $a$ é invertível se, e somente se $\nu(a)=m$.
\end{exer}

\begin{exer}
    Determine todas as unidades de $\mathbb Z[i]$.
\end{exer}

\begin{exer}
    Seja $R$ um domínio e $a, b, c, d \in R$, com $a, b$ não nulos e associados. Mostre que $c\mid d$ se, e somente se $ac\mid bd$.
\end{exer}
\begin{exer}
    Seja $R$ um domínio e $a, b, c, d \in R$, com $a, b$ não nulos e associados. Mostre que $ac$ é associado à $bd$ se, e somente se $c$ é associado à $d$.
\end{exer}

\begin{exer}
    Seja $R$ um domínio e $a, b \in R$ irredutíveis. Mostre que $a|b$ se, e somente se $a$ e $b$ são associados.
\end{exer}
\begin{exer}
    Seja $R$ um domínio e $a \in R$ irredutível. Mostre que para todo $x \in R$, $ax\notin R^*$.
\end{exer}

\begin{exer}
    Seja $R$ um domínio de fatoração única e $a_0, \dots, a_n \in R$ não todos nulos. Mostre que $1 \in \MDC(a_0, \dots, a_n)$ se, e somente se, para todo $p \in R$ irredutível existe $i \in \{0, \dots, n\}$ tal que $p\nmid a_i$.
\end{exer}
