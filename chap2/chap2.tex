\chapter{Noções de Grupos}


\section{Definição e Propriedades Básicas}

O principal objetivo deste texto é servir como texto para um estudo introdutório sobre anéis e corpos.
A noção de grupo é mais simples do que ambas essas estruturas, porém, necessita de ferramentas especiais para seu tratamento completo que fogem do escopo deste texto.
Assim, não é objetivo deste capítulo apresentar uma introdução ao estudo de grupos, mas sim apenas enunciar as principais definições e propriedades que utilizaremos ao longo do texto.

\begin{definition}
Um grupo é uma quadrupla $(G,\cdot,e)$, tal que $G$ é um conjunto, $\cdot$ é uma operação binária em $G$ e $0 \in G$, e satisfazem:

\begin{itemize}
    \item (\textbf{Propriedade associativa}) $\forall a, b, c \in G$ $(a \cdot b) \cdot c = a \cdot (b \cdot c)$.
    \item (\textbf{Elemento neutro}) $\forall a \in G$  $e \cdot a = a \cdot e = a$.
    \item (\textbf{Elemento inverso}) $\forall a \in G$ $\exists b \in G$ $a \cdot b = b \cdot a = e$.
\end{itemize}
Se, adicionalmente, a seguinte propriedade é satisfeita, o grupo é chamado de \emph{comutativo}, ou, mais comunmente, \emph{Abeliano}:
\begin{itemize}
    \item (\textbf{Comutatividade}) $\forall a, b \in G\, a \cdot b = b \cdot a$.
\end{itemize}
\end{definition}
Algumas observações importantes sobre a notação utilizada no estudo de grupos:
\begin{itemize}
\item Ao discursar sobre grupos, é comum omitir a operação e o elemento neutro, referindo-se apenas ao conjunto $G$.
\item Caso o grupo seja Abeliano, é comum que sua operação binária seja denotada por $+$ ou outro símbolo similar.
Nesse contexto, o elemento neutro é frequentemente denotado por $0$.
\item Caso o grupo não seja Abeliano, é comum que sua operação binária seja denotada por $\cdot$ ou outro símbolo similar.
Nesse contexto, o elemento neutro é frequentemente denotado por $e$, e a operação é frequentemente omitida, ou seja, $a \cdot b$ é frequentemente escrito como $ab$.
\end{itemize}

Alguns exemplos:

\begin{itemize}
    \item Com a soma usual, $\mathbb{Z, Q, R, C}$ são grupos Abelianos.
    \item Com a multiplicação usual, o círculo unitário complexo $\mathbb T=\{x \in \mathbb C: |x|=1\}$ é um grupo Abeliano com elemento neutro $1$.
    De fato, o produto de complexos é comutativo, associativo e tem $1$ como elemento neutro.
    Note que $1\in \mathbb T$ e $0\notin \mathbb T$.
Se $x \in \mathbb T$, o inverso multiplicativo de $x$ é dado por $\frac{\bar x}{|x|^2}$, onde $\bar x$ denota o conjugado de $x$.
    Como $|\bar x|=|x|=1$, segue que $\mathbb T$ tem todos os inversos de todos seus elementos.
    \item Os inteiros módulo $n$ ($n\geq 1$), dados por $\mathbb Z_n=\{0, \dots, n-1\}$ com a soma dada pela aritmética módulo $n$, são grupos.
\end{itemize}

Agora iniciaremos a provar algumas propriedades básicas sobre grupos.
\begin{prop}[Unicidade do elemento neutro]
    Seja $(G,\cdot,e)$ um grupo.
    Então, o elemento neutro $e$ é único.
    Isto é, se $h \in G$ é tal que $\forall a \in G$ $h \cdot a = a \cdot h = a$, então $h = e$.
\end{prop}
\begin{proof}
    Note que $h=he$, pois $e$ é elemento neutro.
    Por outro lado, $e=he$, pois $h$ é elemento neutro.
    Assim, $h=he=e$.
\end{proof}

\begin{prop}[Unicidade dos inversos]\label{prop:inverso_unico_grupo}
    Seja $(G,\cdot,e)$ um grupo.
    Então todo $a \in G$ possui um único elemento inverso, ou seja, para todo $g \in G$,  é único.
Isto é $\forall a \in G$ $\exists!\, b \in G$ $a \cdot b = b \cdot a = e$.
\end{prop}
\begin{proof}
    A existência do inverso é garantida pela definição de grupo.
    Para provar a unicidade, suponha que $b, c$ são inversos de $a$, ou seja, $a \cdot b = b \cdot a = e$ e $a \cdot c = c \cdot a = e$.
    Então, temos:
    $$b=be=b(ac)=(ba)c=ec=c.$$
\end{proof}

A unicidade do elemento neutro e dos inversos nos permite definir a notação $a^{-1}$ para o inverso de $a$ em um grupo $(G,\cdot,e)$.
Caso $(G, +, 0)$ seja um grupo Abeliano, a notação $-a$ é frequentemente utilizada para denotar o inverso de $a$, e, nesse caso, $-a$ é chamado de \emph{oposto} de $a$.

Note que assim, ficam definidos operadores unários $(\,)^{-1}:G\rightarrow G$ (ou $-:G\rightarrow G$).
Para o segundo caso, define-se também que $a-b=a+(-b)$.

\begin{prop}[Cancelamento]
    Seja $(G,\cdot,e)$ um grupo.
Então, se $a,b,c \in G$ e $a \cdot b = a \cdot c$, então $b=c$.
    Analogamente, se$b \cdot a = c \cdot a$, então $b=c$.
\end{prop}
\begin{proof}
Provaremos a primeira afirmação.
A segunda é análoga e fica como exercício.
    Suponha que $ba=ca$.
Então $b=be=b(aa^{-1})=(ba)a^{-1}=(ca)a^{-1}=c(aa^{-1})=ce=c$.
Assim, $b=c$.
\end{proof}

\begin{corol}[Cancelamento II]
    Seja $(G,\cdot,e)$ um grupo.
    Para todos $a, b \in G$, se $ab=a$, então $b=e$.
Analogamente, se $ba=a$, então $b=e$.
\end{corol}
\begin{proof}
    Para a primeira afirmação, note que $ab=ae$, logo, pela proposição anterior, $b=e$.
    A segunda afirmação é análoga.
\end{proof}

\begin{prop}[Regras de sinal]\label{prop:regraSinal}
    Seja $G$ um grupo e $a, b \in G$.
    Então:
    \begin{enumerate}[label=\alph*)]
        \item $((a)^{-1})^{-1}=a$ [na notação aditiva, $-(-a)=a$].
\label{prop:regraSinal_A}
        \item $(ab)^{-1}=b^{-1}a^{-1}$ [na notação aditiva, $-(a+b)=(-b)+(-a)]$.\label{prop:regraSinal_B}
        \item $e^{-1}=e$ [na notação aditiva, $-0=0$].\label{prop:regraSinal_C}
    \end{enumerate}
\end{prop}
\begin{proof}
    \ref{prop:regraSinal_A}: Temos que $(a^{-1})^{-1}a^{-1}=e=aa^{-1}$.
    Cancelando $a^{-1}$, segue.
    
    \ref{prop:regraSinal_B}: Temos que $(ab)^{-1}(ab)=e=(b^{-1}a^{-1})ab$.
    Cancelando $ab$, segue que $(ab)^{-1}=b^{-1}a^{-1}$.
    Analogamente, $(ba)^{-1}=a^{-1}b^{-1}$.

    \ref{prop:regraSinal_C}: Temos que $(e^{-1})e=e=ee$.
    Cancelando $e$ à direita, segue.


\end{proof}

\section{Somatórios}

Nessa seção, formalizaremos a noção de somatório.
É desejável que o leitor já possua familiaridade com alguma notação de somatório, mas aqui apresentaremos a notação e as técnicas de ``substituição de variáveis'' que serão utilizadas.

\begin{definition}[Soma de sequência finita]
Seja $G$ um conjunto munido de uma operação $+$ associativa, comutativa e com neutro $0$.
Define-se, recursivamente para $n\geq 0$, o somatório de famílias $(a_i: i \in F)$, onde $F$ é um conjunto de $n$ índices e $a_i \in G$ para todo $i \in F$, como se segue:

\begin{itemize}
    \item \textbf{Notação:} se $a=(a_i)_{i\in F}$ é uma sequência de elementos de $G$, então usamos as notações:
    $$\sum a=\sum(a_i: i\in F)=\sum_{i\in F} a_i.$$
    \item Caso base $n=0$: só existe uma família com $0$ elementos, que é a família vazia $a=()=\emptyset=(a_i:i\in \emptyset)$.
    Definimos: $$\sum a=0$$.
    \item Passo recursivo $n\rightarrow n+1$: considere uma família $(a_i)_{i\in F}$, onde $|F|=n+1$.
    Define-se:
    $$\sum(a_i: i \in F)=\sum(a_i: i \in F\setminus\{j\})+a_j,$$
    onde $j \in I$ é qualquer elemento.
\end{itemize}
\end{definition}
É claro que, para mostrar que a definição acima é consistente, precisamos mostrar que a soma não depende da escolha de $j$.

\begin{lemma}
Qualquer que seja o tamanho (finito) de $F$, $\sum(a_i)_{i\in F}$ está bem definido.
\end{lemma}

\begin{proof}
    Seja $F$ um conjunto finito.
Se $|F|=0$, então $F=\emptyset$, e a soma é $0$.
Se $|F|=1$, então $F=\{j\}$ -- só há uma escolha para $j$, e a soma é $a_j$.
    Se $|F|=n+1$ para $n\geq 1$, tome $j, k \in F$.
    Devemos ver que $\left(\sum_{i\in F\setminus\{j\}} a_i\right)+a_j=\left(\sum_{i\in F\setminus\{k\}} a_i\right)+a_k$.
    Com efeito:

    $$\left(\sum_{i\in F\setminus\{j\}} a_i\right)+a_j=\left(\left(\sum_{i\in F\setminus\{j, k\}} a_i\right)+a_k\right)+a_j=\left(\sum_{i\in F\setminus\{j, k\}} a_i\right)+(a_k+a_j)$$

        $$=\left(\sum_{i\in F\setminus\{j, k\}} a_i\right)+(a_j+a_k)=\left(\left(\sum_{i\in F\setminus\{j, k\}} a_i\right)+a_j\right)+a_k=\left(\sum_{i\in F\setminus\{k\}} a_i\right)+a_k.$$
\end{proof}

\begin{prop}
    Seja $G$ um conjunto munido de uma operação $+$ associativa, comutativa e com neutro $0$.
Seja $(a_i: i \in I)$ uma família finita em $G$ e $\phi:J\rightarrow I$ uma função bijetora.
Então:

    $$\sum_{i \in I}a_i=\sum_{j \in J}a_{\phi(j)}.$$

\end{prop}
\begin{proof}
Novamente, procedemos por indução no tamanho de $|I|=|J|$.
A base de tamanho $0$ é trivial, já que ambos os lados da igualdade são $0$.

Para o passo indutivo em que $|I|=|J|=n+1$, considere $\phi:J\rightarrow I$ como no enunciado.
Fixe $k \in J$ qualquer e sejam $I'=I\setminus\{\phi(k)\}, J'=J\setminus\{k\}$ e $\phi'=\phi|_{J'}:J'\rightarrow I'$, que é bijetora.
Como $|J'|=|I'|=n$, por hipótese indutiva temos que $\sum_{j \in J'}a_{\phi(j)}=\sum_{i \in I'}a_i$.
Segue que:

$$\sum_{j \in J}a_{\phi(j)}=\left(\sum_{j \in J'}a_{\phi(j)}\right)+a_{\phi(k)}=\left(\sum_{i \in I'}a_{i}\right)+a_{\phi(k)}=\sum_{j \in I}a_{i}.$$
\end{proof}
