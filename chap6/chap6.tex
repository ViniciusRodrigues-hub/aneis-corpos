\chapter{Domínios de Integridade}
Neste capítulo, exploraremos com mais detalhes os domínios de integridade e a teoria que nasce deles.

\section{Relações entre corpos e domínios}
Conforme visto, todo corpo é um domínio de integridade, e a recíproca não é verdadeira (sendo $\mathbb Z$ um contra-exemplo).

A seguir, apresentaremos algumas relações entre corpos e domínios.

\begin{prop}
Todo domínio de integridade finito é um corpo.
\end{prop}
\begin{proof}
    Seja $R$ um domínio de integridade finito.
    Fixe $a \in R\setminus\{0\}$.
    Veremos que $a$ é invertível.

    Considere $\phi:R\setminus\{0\}\rightarrow R\setminus \{0\}$ dado por $\phi(x)=ax$.

    Como $R$ é um domínio, para todo $x \in R\setminus \{0\}$, temos $ax \neq 0$, logo, $\phi$ está bem definida.

    $\phi$ é uma função injetora: se $\phi(x)=\phi(y)$, então $ax=ay$.
    Logo, $a(x-y)=0$.
    Como $a\neq 0$ e $R$ é um domínio, segue que $x-y=0$, ou seja, $x=y$.

    Como $R\setminus\{0\}$ é finito e $\phi:R\setminus\{0\}\rightarrow R\setminus \{0\}$ é injetora, segue que $\phi$ é sobrejetora.
    Em particular, existe $x \in X$ tal que $ax=\phi(x)=1$.
    Logo, $a$ é invertível.
\end{proof}

Portanto, restrito aos anéis finitos, o estudo dos corpos e domínios colapsa em um único estudo.

Outra relação importante é a que segue:

\begin{prop}
Seja $R$ um anel comutativo e $I$ um ideal próprio de $R$. São equivalentes:

\begin{enumerate}[label=(\roman*)]
    \item $R/I$ é um corpo;
    \item $I$ é maximal.
\end{enumerate}
\end{prop}

\begin{proof}
    Seja $q:R\rightarrow I$ o mapa quociente.

    (i) $\Rightarrow$ (ii): Suponha que $R/I$ é um corpo.
    

    $I$ é um ideal próprio, caso contrário, teríamos que $R/I$ é o anel trivial, que não é um corpo.

    Agora suponha que $J$ é um ideal que contém $I$ propriamente.
    Veremos que $J=R$.
    Seja $a \in J\setminus I$.
    Como $a\notin I$, temos que $q(a)\neq 0$.
    Como $A/I$ é um corpo, existe $b \in R$ tal que $q(a)q(b)=1$.
    Isso implica que existe $x \in I$ tal que $ab+x=1$.
    Como $a \in J$ e $x \in I\subseteq J$, segue que $1=ab+x\in J$, e, portanto, $J=R$.

    (ii) $\Rightarrow$ (i): Suponha que $I$ é maximal. Vejamos que $R/I$ é um corpo.

    Seja $x \in R\setminus I$ não nulo.
    Tome $a \in R$ tal que $q(a)=x$.
    Temos que $a \notin I$.
    Como $I+\langle a\rangle$ é um ideal que contém $I$ propriamente, segue que $I+\langle a\rangle=R$.
    Logo, existe $b \in R$ e $c \in I$ tais que $c+ba=1$.
    Logo, $q(1)=q(c)+q(ba)=0+q(b)q(a)=q(b)x$.
    Portanto, $x$ é invertível.
\end{proof}

Será que podemos caracterizar, de forma análoga, ser um domínio? A resposta é positiva.

\begin{prop}
    Seja $R$ um anel comutativo e $I$ um ideal próprio de $R$. São equivalentes:
    
    \begin{enumerate}[label=(\roman*)]
        \item $R/I$ é um domínio.
        \item $I$ é primo.
    \end{enumerate}
\end{prop}

\begin{proof}
    Seja $q:R\rightarrow I$ o mapa quociente.

    (i) $\Rightarrow$ (ii): Suponha que $R/I$ é um domínio.
    

    $I$ é um ideal próprio, caso contrário, teríamos que $R/I$ é o anel trivial, que não é um domínio.

    Suponha que $a,b \in R$ tais que $ab \in I$.
    Temos que $q(a)q(b)=q(ab)=0$.
    Como $R/I$ é um domínio, temos que $q(a)=0$ ou $q(b)=0$, ou seja, que $a \in I$ ou $B \in I$.

    Logo, $I$ é primo.

    (ii) $\Rightarrow$ (i): Suponha que $I$ é primo.
    Vejamos que $R/I$ é um domínio.

    Sejam $x, y \in R$ tais que $q(x)q(y)=0$.
    Devemos ver que $q(x)=0$ ou $q(y)=0$.
    Como $q(xy)=q(x)q(y)=0$, segue que $xy\in I$.
    Então, $x\in I$ ou $y\in I$, ou seja, $q(x)=0$ ou $q(y)=0$.
\end{proof}

Como consequência, temos:

\begin{corol}
    Seja $R$ um anel comutativo finito e $I$ um ideal de $R$. Então $I$ é primo se, e somente se $I$ é maximal.
\end{corol}
\begin{proof}
    Temos que $R/I$ é finito, e, portanto, é um corpo se, e somente se for um domínio.
    Portanto:

    \[I \text{ é primo} \Leftrightarrow R/I \text{ é um domínio} \Leftrightarrow R/I \text{ é um corpo} \Leftrightarrow I \text{ é maximal}\]
\end{proof}