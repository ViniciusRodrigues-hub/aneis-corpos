\chapter{Quocientes e Teoremas do Homomorfismo}

Ao estudar o anel dos números inteiros, normalmente são estudadas as relações de congruência e, subsequentemente, os anéis quocientes $\mathbb Z_n=\mathbb Z/n\mathbb Z$.

Neste capítulo, estudaremos quocientes de anéis de forma generalizada, e suas relações com ideais, relações de congruência e homomorfismos de anéis.

\section{Relações de congruência}
As relações de congruência de anéis são relações que generalizam a noção de ``congruência módulo $n$'' do anel dos inteiros.

\begin{definition}
    Seja $A$ um anel. Uma relação de congruência em $A$ é uma relação de equivalência $\sim$ em $A$ que ``preserva operações''.
    Explicitamente, tal que para todos $a, b, c, d \in A$, se $a\sim b$ e $c\sim d$, então $a+c\sim b+d$ e $ac\sim bd$.
\end{definition}

Todo homomorfismo induz naturalmente uma relação de congruência.
Explicitamente:

\begin{prop}
Seja $f: A\rightarrow R$ um homomorfismo de anéis.
Então $\sim_f=\{(a, b) \in A^2: f(a)=f(b)\}$ é uma relação de congruência em $A$.
De outro modo, a relação $\sim_f$ em $A^2$ dada por $a \sim_f b$ se, e somente se $f(a)=f(b)$, é uma relação de congruência em $A$.
\end{prop}

\begin{proof}
    $\sim_f$ é uma relação reflexiva, pois para todo $a \in A$, $f(a)=f(a)$, logo, $a\sim_f a$.

    $\sim_f$ é simétrica, pois se $a\sim_f b$, então $f(a)=f(b)$, e, portanto, $f(b)=f(a)$, o que implica em $b\sim_f a$.

    $\sim_f$ é transitiva, pois se $a\sim_f b$ e $b\sim_f c$, então $f(a)=f(b)$ e $f(b)=f(c)$, logo, $f(a)=f(c)$, o que implica em $a\sim_f c$.

    $\sim_f$ preserva soma, pois se $a\sim_f b$ e $c\sim_f d$, então $f(a)=f(b)$ e $f(c)=f(d)$, logo, $f(a+c)=f(a)+f(c)=f(b)+f(d)=f(b+d)$, o que implica em $a+c\sim_f b+d$.

    $\sim_f$ preserva produto, pois se $a\sim_f b$ e $c\sim_f d$, então $f(a)=f(b)$ e $f(c)=f(d)$, logo, $f(ac)=f(a)f(c)=f(b)f(d)=f(bd)$, o que implica em $ac\sim_f bd$.
\end{proof}

A proposição abaixo classifica todas as relações de congruência a partir dos ideais de um anel.
\begin{prop}[Relações de congruência vs ideais]
    Seja $A$ um anel, $\mathcal R(A)$ o conjunto de todas as relações de congruência em $A$ e $\mathcal I(A)$ o conjunto de todos os ideais de $A$.
    Então, existe uma bijeção entre $\mathcal R(A)$ e $\mathcal I(A)$ dada por
    $\sim \mapsto I_{\sim}=\{a \in A: a\sim 0\}$,
    cuja inversa se dá por $I\mapsto \sim_I=\{(a, b) \in A^2: a-b \in I\}$.
\end{prop}
\begin{proof}
Primeiro, vejamos que se $\sim$ é uma relação de congruência, então $I_\sim$ é um ideal de $A$.

\begin{itemize}
\item $0 \in I_\sim$, pois $0\sim 0$.
\item Se $a, b \in I_\sim$, então $a\sim 0$ e $b\sim 0$, logo $a+b\sim 0+0=0$, portanto, $a+b \in I_\sim$.
\item Se $x \in A$ e $a \in I_\sim$, então $a\sim 0$ e $x\sim 0$, logo $ax\sim a0=0$ e $xa=0a=0$, portanto, $ax, xa \in I_\sim$.
\end{itemize}

Agora, vejamos que se $I$ é um ideal, então $\sim_I$ é uma relação de congruência. De fato, temos que, para todos $a, b, c, d \in A$:
\begin{itemize}
    \item $a\sim_I a$ pois $a-a=0\in I$.
    \item Se $a\sim_I b$, então $a-b \in I$, logo $(-1)(a-b)=b-a\in I$, e, portanto, $b\sim_I a$.
    \item Se $a\sim_I b$ e $b\sim_I c$, então $a-b \in I$ e $b-c \in I$, logo, $(a-b)+(b-c)=a-c \in I$, portanto, $a\sim_I c$.
    \item Se $a\sim_I b$ e $c\sim_I d$, então $a-b \in I$ e $c-d \in I$, logo, $(a-b)+(c-d)=(a+c)-(b+d)\in I$, portanto, $a+c\sim_I b+d$.
    \item Se $a\sim_I b$ e $c\sim_I d$, então $a-b \in I$ e $c-d \in I$, logo, $(a-b)c=ac-bc\in I$ e $b(c-d)=bc-bd\in I$, logo $(ac-bc)+(bc-bd)=ac-bd\in I$, portanto, $ac\sim_I bd$.
    \end{itemize}

Se $I$ é ideal, $I_{\sim_I}=I$, pois, para todo $a\in A$:

$$a\in I_{\sim_I}\Leftrightarrow a\sim_I 0\Leftrightarrow a-0\in I\Leftrightarrow a\in I.$$

Finalmente, se $\sim$ é relação de congruência, $\sim_{I_\sim}=\sim$, pois, para todos $a, b \in A$:

$$a\sim_{I_\sim} b\Leftrightarrow a-b\in I_\sim \Leftrightarrow a-b\sim 0\Leftrightarrow a\sim b.$$

Justificando a última equivalência: se $a-b\sim 0$, como $b\sim b$, temos que $a-b+b\sim b$, ou seja, que $a\sim b$. Reciprocamente, se $a\sim b$, como $(-b)\sim (-b)$, segue que $a+(-b)\sim b+(-b)$, ou seja, que $a-b\sim 0$.
\end{proof}

\begin{exemplo}
Como vimos, $\mathbb Z$ é um domínio de ideais principais.
Assim, todo ideal de $\mathbb Z$ é da forma $n\mathbb Z$.
Como para todo $n$, $n\mathbb Z=(-n)\mathbb Z$, temos que $\{n\mathbb Z: n\geq 0\}$ é a coleção de todos os ideais de $\mathbb Z$.

Quais são todas as relações de congruência em $\mathbb Z$?
Denotemos por $\sim_n$ a relação $\sim_{n\mathbb Z}$.

Temos que $\sim_0$ corresponde à relação de igualdade, pois $a\sim_0 b$ se, e somente se, $a-b=0$, ou seja, $a=b$.
Note que a relação de igualdade sempre é uma relação de congruência, em qualquer anel.

Se $n\geq 1$, $\sim_n$ corresponde à relação de congruência módulo $n$, pois $a\sim_n b$ se, e somente se, $a-b\in n\mathbb Z$, ou seja, $a-b=kn$ para algum $k\in \mathbb Z$.
\end{exemplo}

\section{Quocientes}

Como feito nos inteiros, podemos, ao invés de trabalhar com relações de congruência, encontrar anéis em que a congruência corresponda exatamente à igualdade.

\begin{definition}
Seja $A$ um anel e $\sim$ uma relação de congruência.

Lembremos que o conjunto das classes de equivalência de $\sim$ é denotado por $A/\sim$, e este corresponde, portanto, à $\{[a]_\sim: a \in A\}$, onde $[a]_\sim=\{b\in A: b\sim a\}$ é a classe de equivalência de $a$ com relação a $\sim$.

Define-se que $[a]_\sim+[b]_\sim=[a+b]_\sim$ e que $[a]_\sim[b]_\sim=[ab]_\sim$.
Com essas operações, $(A/\sim, +, \cdot, [0]_\sim, [1]_\sim)$ é chamado de \emph{anel quociente} de $A$ por $\sim$.

Se $I$ é um ideal define-se $A/I=A/\sim_I$, e este é munido das operações anteriores.
Com essas operações, $A/I=A/{\sim_I}$ como descrito acima é chamado de \emph{anel quociente} de $A$ por $I$.

Define-se o \emph{mapa quociente} de $A$ em $A/I$ se dá por $q:A\longrightarrow A/I$ dada por $q(a)=[a]_{\sim_I}$.
\end{definition}

É claro que precisamos mostrar que as operações acima estão bem definidas e torna estes, de fato, anéis.
\begin{lemma}
    As operações dos anéis quocientes estão bem definidas e os tornam anéis.
    Além disso, o mapa quociente é um epimorfismo (homomorfismo sobrejetor).
\end{lemma}

\begin{proof}
    Como as relações de congruência estão em bijeção com os ideais, podemos tratar de um quociente arbitrário da forma $A/\sim$.
    
    Primeiro, vejamos que as operações estão bem definidas, ou seja, que se $a\sim b$ e $c\sim d$, então $[ac]_\sim=[bd]_\sim$ e $[a+b]_\sim=[b+d]_\sim$.
    
    De fato, como $\sim$ é uma relação de congruência e $a\sim b$ e $c\sim d$, temos que $ac\sim bc$ e $a+c\sim b+d$, logo, $[ac]_\sim=[bc]_\sim$ e $[a+c]_\sim=[b+d]_\sim$.
    Note ainda que como $[a]_\sim=q(a)$ e $q(1_A)=[1_A]_\sim$, assim, segue que, caso $A/\sim$ seja anel, $q$ é homomorfismo sobrejetor.

    Agora devemos ver que $A/\sim$ é um anel. Temos que:

    \begin{itemize}
        \item Comutatividade da soma: $q(a)+q(b)=q(a+b)=q(b+a)=q(b)+q(a)$.
        \item Associatividade da soma: $(q(a)+q(b))+q(c)=q(a+b)+q(c)=q((a+b)+c)=q(a+(b+c))=q(a)+q(b+c)=q(a)+(q(b)+q(c))$.
        \item Neutro da soma: $q(0)+q(a)=q(0+a)=q(a)$.
        \item Opostos: $q(a)+q(-a)=q(a+(-a))=q(0)=0$.
        \item Associatividade do produto:$(q(a)q(b))q(c)=q(ab)q(c)=q((ab)c)=q(a(bc))=q(a)q(bc)=q(a)(q(b)q(c))$.
        \item Neutro do produto: $q(1)q(a)=q(1a)=q(a)$, e $q(a)q(1)=q(a1)=q(a)$.
        \item Distributividade: $q(a)(q(b)+q(c))=q(a)q(b+c)=q(a(b+c))=q(ab+ac)=q(ab)+q(ac)=q(a)q(b)+q(a)q(c)$.
        \item Distributividade II: $(q(a)+q(b))q(c)=q(a+b)q(c)=q((a+b)c)=q(ac+bc)=q(ac)+q(bc)=q(a)q(c)+q(b)q(c)$.
    \end{itemize}
        
\end{proof}

Algumas propriedades particulares do quociente:

\begin{lemma}[Propriedades do quociente]
    Na notação acima:
    \begin{enumerate}[label=\alph*)]
        \item $\ker q = I$. \label{lemma:propriedadesQuociente_a}
        \item $q(a)=a+I=\{a+x: x \in I\}$ para todo $a \in A$. \label{lemma:propriedadesQuociente_b}
        \item Se $A$ é anel comutativo, $A/I$ também é. \label{lemma:propriedadesQuociente_c}
    \end{enumerate}
\end{lemma}

\begin{proof}
    \ref{lemma:propriedadesQuociente_a} Temos que $\ker q=\{a \in A: q(a)=q(0)\}=\{a \in A: a\sim_I 0\}=\{a \in A: a\in I\}=I$.

    \ref{lemma:propriedadesQuociente_b} Temos que $q(a)=[a]_{\sim_I}=\{b \in A: b\sim_I a\}=\{b \in A: b-a\in I\}=\{a+x: x \in I\}$ pois se $b-a \in I$ se, e somente se $a-b=x$ para algum $x \in I$.

    \ref{lemma:propriedadesQuociente_c} Se $A$ é comutativo, então $A/I=\ran q$ também é, pois $q$ é homomorfismo de anéis.
\end{proof}

Em particular, temos:

\begin{corol}
    Todo ideal é o núcleo de algum homomorfismo.
\end{corol}

\section{Teoremas do isomorfismo}
Os teoremas do homomorfismo dizem que certos homomorfismos ``fatoram'' para quocientes.
\begin{theorem}[Teorema do homomorfismo]
    Seja $f:A\rightarrow R$ um homomorfismo de anéis e $J$ um ideal tal que $J\subseteq \ker f$. Então, existe um único homomorfismo de anéis $\bar f:A/J\rightarrow R$ tal que $\bar f\circ q=f$, onde $q:A\rightarrow A/J$ é o mapa quociente canônico dado por $q(a)=a+J$.
    \begin{figure}[h]\centering
        \begin{tikzcd}[column sep=1.6cm,row sep=1.2cm]
            A\arrow[d, "q"']\arrow[r, "f"]& R \\
            A/J \arrow[ur, dashed, "\exists!\bar f"']
        \end{tikzcd}
        \caption{Teorema do homomorfismo.}
    \end{figure}
\end{theorem}
\begin{proof}
    Definimos $\bar f:A/J\rightarrow R$ por $\bar f(a+J)=f(a)$.
    Então, $g$ é bem definido, pois se $a+J=b+J$, então $a-b \in J\subseteq \ker f$, logo, $f(a-b)=0_R$, ou seja, $f(a)=f(b)$.

    Agora, vejamos que $\bar f$ é um homomorfismo de anéis.
    De fato, para todo $a', b' \in A/J$, sendo $a'=a+J$ e $b'=b+J$, temos que:
    \begin{itemize}
        \item $\bar f(a'+b')=\bar f((a+J)+(b+J))=\bar f((a+b)+J)=f(a+b)=f(a)+f(b)=\bar f(a+J)+\bar f(b+J)$.
        \item $\bar f(a'b')=\bar f((a+J)(b+J))=\bar f(ab+J)=f(ab)=f(a)f(b)=\bar f(a+J)\bar f(b+J)$.
        \item $\bar f(1_{A/J})=\bar f(1_A+J)=f(1_A)=1_R$.
    \end{itemize}

    Temos que $\bar f\circ q=f$ por definição de $\bar f$.
    Para a unicidade, se $g:A/J\rightarrow R$ é um homomorfismo tal que $g\circ q=f$, fixe $a'\in A/J$.
    Fixe $a \in A$ tal que $a'=q(a)$.
    Então $g(a')=g(q(a))=f(a)=\bar \bar f(q(a))=\bar f(a')$.
    Assim, $g=\bar f$.
\end{proof}

Como consequência, temos o Primeiro Teorema do Isomorfismo:

\begin{theorem}[Primeiro Teorema do Isomorfismo]
    Seja $f:A\rightarrow R$ um homomorfismo de anéis.
    Então, $A/I$ é isomorfo a $\ran f$.
    Mais especificamente, existe um único homomorfismo $\phi:A/\ker f\rightarrow R$ tal que $q\circ \phi=f$, onde $q$ é o mapa quociente, e este homomorfismo é necessariamente um isomorfismo.
    \begin{figure}[H]\centering
        \begin{tikzcd}[column sep=1.6cm,row sep=1.2cm]
            A\arrow[d, "q"']\arrow[r, "f"]& \ran f\\
            A/\ker f \arrow[ur, dashed, "\exists! \phi"'] & 
        \end{tikzcd}
        \caption{Primeiro Teorema do Isomorfismo.}
    \end{figure}
\end{theorem}

\begin{proof}
Pelo Teorema do Homomorfismo, existe um único homomorfismo $\bar \phi:A/\ker f\rightarrow \ran f$ tal que $\phi\circ q=f$, onde $q:A\rightarrow A/\ker f$ é o mapa quociente canônico dado por $q(a)=a+\ker f$.

Temos que $\phi$ é sobrejetor: dado $b \in \ran f$, existe $b \in A$ tal que $f(a)=b$.
Logo, $b=f(a)=\bar \phi(q(a))$, assim, $b \in \ran \phi$.

Agora vejamos que $\phi$ é injetor.
Suponha que $y \in A/\ker f$ é tal que $\phi(y)=0$.
Como $q$ é sobrejetor, tome $a \in A$ tal que $y=q(a)$.
Assim, $0=\phi(y)=\phi\circ q(a)=f(a)$, logo, $a \in \ker f$.
Como $q:A\rightarrow A/\ker f$ é o mapa quociente e $a \in \ker f$, segue que $y=q(a)=0_{A/\ker f}$.
Logo, $\ker\phi=\{0\}$, ou seja, $\phi$ é injetor.
\end{proof}