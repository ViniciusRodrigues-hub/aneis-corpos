
\chapter{Polinômios}

\section{Séries Formais}
Se $R$ é um anel comutativo, intuitivamente uma série formal é um objeto que se escreve na forma:
$$a_0+a_1x+a_2x^2+\dots=\sum_{i=0}^\infty a_ix^i$$
onde $a_i\in R$.

Que propriedades gostaríamos que esse objeto tivesse?

\begin{itemize}
    \item \textbf{Igualdade:} igualdade de objetos desse tipo fosse determinada por uma condição de igualdade entre os coeficientes. Ou seja, que:
    \begin{align*}
      \sum_{i=0}^\infty a_ix^i=\sum_{i=0}^\infty b_ix^i & \Leftrightarrow \forall i\in \mathbb N\, a_i=b_i.
    \end{align*}
    \item \textbf{Soma:} que a soma de dois objetos desse tipo fosse dada por:
    \begin{align*}
        \left(\sum_{i=0}^\infty a_ix^i\right)+\left(\sum_{i=0}^\infty b_ix^i\right)=\left(\sum_{i=0}^\infty (a_i+bi)x^i\right)
    \end{align*}
    \item \textbf{Produto:} que o produto de dois objetos desse tipo fosse dada por:
    \begin{align*}
        \left(\sum_{i=0}^\infty a_ix^i\right)\cdot\left(\sum_{i=0}^\infty b_ix^i\right)=\sum_{i=0}^\infty \left(\sum_{j=0}^i a_{j}b_{i-j}\right)x^i
    \end{align*}
    \item \textbf{Preservação:} que as operações do anel sejam preservadas.
    \item Notação: $R\llbracket x \rrbracket$ é o conjunto de todas as séries formais em $R$.
\end{itemize}

\begin{definition}
Seja $R$ um anel comutativo. Definiremos $R\llbracket x\rrbracket=R^{\mathbb N}$.


Um elemento de $R\llbracket x\rrbracket$ é da forma $p=(p_0, p_1, \dots)=(p_n)_{n \in \mathbb N}=(p(n))_{n \in \mathbb N}=(p(0), p(1), \dots)$ onde $p_i=p(i)\in R$  para todo $i \in \mathbb N$.

O suporte de $p\in R\llbracket x\rrbracket$ é o conjunto $\supp(p)=\{i \in \mathbb N: p_i \neq 0\}$.
O grau de $p \in R[x]$ é o maior elemento de $\supp(p)$, denotado por $\deg(p)$. Se $p=0$, então $\deg(p)=-\infty$

\textbf{Intuição}: $p=(a_0, a_1, \dots)$ corresponderá à $a_0+a_1x+\dots+a_n x^n+\dots$.

\textbf{Operações}:
Se $p, q \in R[x]$, define-se:
$$p+q=(p_0+q_0, q_1+p_1, \dots)=(p_i+q_i)_{i \in \mathbb N}\in R^{\mathbb N}$$
$$p\cdot q=(p_0q_0, p_1q_0+p_0q_1, p2q_0+p_1q_1+p_0q_2, \dots)=\left(\sum_{j=0}^i p_{i-j}q_j\right)_{i \in \mathbb N}$$
$$1_{R\llbracket x\rrbracket}=(1, 0, 0, \dots)$$
$$0_{R\llbracket x\rrbracket}=(0, 0, 0, \dots)$$
\end{definition}

\begin{lemma}[Séries formais formam anéis]
    Se $R$ é um anel comutativo, então $R\llbracket x \rrbracket$ é um anel comutativo.
\end{lemma}

\begin{proof}
    A operação de soma de $\mathbb R\llbracket x \rrbracket$ é a mesma de $\mathbb R^{\mathbb N}$, que já verificamos satisfazer as propriedades de grupo Abeliano. Assim, $R\llbracket x \rrbracket$ é um grupo abeliano sob a soma.
    \begin{itemize}
        \item \textbf{Distributividade:} $$p\cdot (q+r)=p\cdot(q_i+r_i)_{i \in \mathbb N}=\left(\sum_{j=0}^i p_{i-j}(q_j+r_j)\right)_{i \in \mathbb N}$$$$=\left(\sum_{j=0}^i p_{i-j}q_j\right)_{i \in \mathbb N}+\left(\sum_{j=0}^i p_{i-j}r_j\right)_{i \in \mathbb N}=p\cdot q+p\cdot r.$$
        \item \textbf{Elemento Neutro:} $$p\cdot 1=\left(\sum_{j=0}^i p_{i-j}\delta_{0j}\right)_{i\in \mathbb N}=(p_i)_{i \in \mathbb N}.$$
        \item \textbf{Comutatividade:} A $i$-ésima coordenada de $p\cdot q$ é $\sum_{j=0}^i p_{i-j}q_j=\sum(p_{i_j}q_j: j \in A_j)$, onde $A_j=\{0, \dots, i\}$. A função $\phi: A_j:\rightarrow A_j$ dada por $\phi(j)=i-j$ é bijetora, pois é injetora e $A_j$ é finito. Assim:
        $$\sum_{j=0}^i p_{i-j}q_j=\sum_{j=0}^ip_{i-\phi(j)}q_{\phi(j)}=\sum_{j=0}^ip_{j}q_{i-j}=\sum_{j=0}^iq_{i-j}p_{j}$$

        E esta é a $i$-ésima coordenada de $q\cdot p$.

        \item \textbf{Associatividade:} Temos que a $i$-ésima coordenada de $(p\cdot q)\cdot r$ é dada por:

        $$\pi_i((p\cdot q)\cdot r)=\sum_{j=0}^i\pi_{i-j}(p\cdot q)\cdot q_j=\sum_{j=0}^i\left(\sum_{k=0}^{i-j}p_{i-j-k}q_k\right)\cdot q_j$$
        $$=\sum_{j=0}^{i}\sum_{k=0}^{i-j}p_{i-j-k}q_kq_j=\sum\left(p_{i-j-k}q_kr_j:(j, k)\in A\right)$$

        Onde $A=\{(j, k): 0\leq j\leq i, 0\leq k\leq i-j\}$.

        Temos que a $i$-ésima coordenada de $p\cdot (q\cdot r)$ é dada por:

        $$\pi_i(p\cdot (q\cdot r))=\sum_{s=0}^ip_{i-s}\pi_s(q\cdot r)=\sum_{s=0}^ip_{i-s}\left(\sum_{t=0}^sq_{s-t}r_t\right)$$
        $$=\sum_{s=0}^i\sum_{t=0}^sp_{i-s}q_{s-t}r_t=\sum\left(q_{i-s}q_{s-t}r_t:(s, t)\in B\right)$$
        
        onde $B=\{(s, t): 0\leq t\leq s\leq i\}$. A função $\phi: A\rightarrow B$ dada por $\phi(j, k)=(j+k, j)$ é bijetora: é em $B$, pois $0\leq j\leq j+k\leq j+(i-j)=i$. É injetora, pois se $(j+k, j)=(j'+k', j')$ então $j=j'$ e, cancelando, $k=k'$. Finalmente, é sobrejetora, pois se $0\leq t\leq s\leq i$, sendo $j=t$ e $k=s-t$, temos que $0\leq j\leq i$, $0\leq k=s-t\leq i-t=i-j$ e $j+k=s$. Assim, $\phi$ é bijetora. Portanto:

        $$\sum\left(q_{i-s}q_{s-t}r_t:(s, t)\in B\right)=\sum\left(q_{i-(j+k)}q_{(j+k)-j}r_j:(j, k)\in A\right)$$$$=\sum\left(q_{i-j-k}q_{k}r_j:(j, k)\in A\right).$$
    \end{itemize}
\end{proof}

Dado $p \in \mathbb R_x$, utilizamos a notação $$p=p(x)=\sum_{i=0}^\infty p_ix^i.$$

É importante observar que não há, de fato, uma ``soma infinita'' acontecendo aqui. É possível dar sentido à essa soma infinita utilizando a teoria de limites diretos de anéis, mas não faremos isso aqui. Por ora, isso é apenas uma notação especial para tratar desses objetos. Note que, ao menos por enquanto, a letra $x$ é apenas parte da notação, e que não faz sentido, por enquanto, ``substituir $x$'' por algo.
\section{Anéis de Polinômios}
Na subseção anterior, introduzimos o anel das séries formais de um anel comutativo dado. Vimos que tal anel é um anel comutativo.

Deste anel, podemos extrair o anel de polinômios.

\begin{definition}
Seja $R$ um anel comutativo. O anel de polinômios $R[x]$ é o subconjunto de $R\llbracket x \rrbracket$ dado por:
$$R[x]=\{p \in R\llbracket x \rrbracket: \deg(p) < \infty\}$$
\end{definition}

Note que uma série formal tem grau $<\infty$ se, e somente se, todos os coeficientes, a partir de algum ponto, são nulos.

\begin{lemma}
    Seja $R$ um anel comutativo. O anel de polinômios $R[x]$ é um subanel de $R\llbracket x \rrbracket$. Mais especificamente, dados $p(x), q(x) \in R\llbracket x \rrbracket$, temos que, se $\gr(p(x)) < \infty$ e $\gr(q(x)) < \infty$, então:

    \begin{enumerate}[label=\alph*)]
        \item $\gr p(x)q(x)\leq\gr p(x)+\gr q(x)$ caso ambos sejam não nulos, e a igualdade vale se $R$ for um domínio de integridade.
        \item $\gr p(x)+q(x)\leq \max\{\gr p(x),\gr(q(x))\}$.
    \end{enumerate}
\end{lemma}

\begin{proof}
    Para a primeira afirmação, sejam $n, m$ os graus de $p(x)$ e $q(x)$, respectivamente.
    
    Calculemos o coeficiente $n+m$ de $p(x)q(x)$. 

    $$\pi_{n+m}(p(x)q(x))=\sum_{j=0}^{n+m}p_{n+m-j}q_j.$$

    Se $0\leq j< m$ temos que $n+m-j>0$, e $p_{n+m-j}=0$. Se $j>m$, temos que $q_j=0$. Assim, o único termo não nulo da soma é quando $j=m$, e temos que $p_{n+m-m}q_m=p_nq_m$, que é não nulo se $R$ for um domínio. Por outro lado, se $l>n+m$ temos que:

    $$\pi_{l}(p(x)q(x))=\sum_{j=0}^{l}p_{l-j}q_j.$$

    Se $0\leq j\leq m$ temos que $l-j>m+n-m=n$, e $p_{l-j}=0$. Se $j>m$, temos que $q_j=0$. Assim, todos os coeficientes da soma são $0$.


    Para a segunda afirmação, se $l>\max\{\gr p(x), \gr q(x)\}$, temos que o $l$-ésimo coeficiente de $p(x)+q(x)$ é $0$, pois este é $p_l+q_l=0+0$.

    Agora, para a afirmação principal, as duas afirmações itemizadas nos mostram que $R[x]$ é fechado pela soma e produto de $R\llbracket x \rrbracket$. Finalmente, note que o grau da série $1=(1, 0, 0, \dots)$ é $0$, logo, $1\in R[x]$.
\end{proof}

Agora vamos trabalhar um pouco mais nossa notação.

\begin{definition}
    Seja $R$ um anel comutativo. Em $R[x]$, seja $x=(0, 1, 0, 0, \dots)$ e, para cada $r \in R$, seja $\hat r=(\hat r, 0, 0, \dots)$.
\end{definition}

\begin{lemma}
    Na notação anterior, para todo $r \in R$ e $n, i\geq 0$:

    $$\pi_i(\hat r x^n)(i)=\begin{cases}
        0 & \text{se } i\neq n\\
        r & \text{se } i=n.
    \end{cases}$$
    Ou seja, $\hat r x^n=(0, 0, \dots, r, 0, \dots)$ onde o $r$ está na posição $n$.
\end{lemma}

\begin{proof}
    Fixe $r$ Seguimos por indução. Para $n=0$, temos que $\hat rx^0=\hat r=(r, 0, 0, \dots)$ e para $n=1$ temos que $\hat r x^1=x=(0, r, 0, \dots)$.

    Para o passo $n+1$, onde $n\geq 1$, temos que, sendo $i\geq 1$:

    $$\pi_i(\hat rx^{n+1})=\pi_i((\hat rx^n)\cdot x)=\sum_{j=0}^i\pi_{i-j}(\hat r x^n)\cdot \pi_j(x)=\pi_{i-1}^{\hat r x^n}$$

    Assim, se $i=n+1$, temos que a coordenada é $r$, e $0$ caso contrário. Resta apenas verificar que a coordenada $0$ é $0$. Ora, a coordenada $0$ se dá por $\pi_0(\hat r x^n)\pi_0(x)=0$.
\end{proof}

\begin{lemma}
    Na notação anterior, seja $\phi:R\rightarrow R[x]$ dada por $\phi(r)=\hat r$. Então $h$ é um homomorfismo injetor.
\end{lemma}
    
\begin{proof}
    Sejam $r, s \in R$. Então:
    \begin{itemize}
        \item $\phi(r+s)=(r+s, 0, 0, \dots)=\hat r+\hat s=\phi(r)+\phi(s)$
        \item $\phi(rs)=(rs, 0, 0, \dots)=\hat r\cdot \hat s=\phi(r)\cdot \phi(s)$
        \item $\phi(1_R)=\hat 1=(1_R, 0, 0, \dots)=1_{R[x]}=1_{R[x]}$.
    \end{itemize}

    A injetividade é óbvia.
\end{proof}

\begin{prop}
    Na notação anterior, para todo $p(x)\in R[x]$, existem $n\geq 0$ e $r_0, r_1, \dots, r_n\in R$ tais que $p(x)=\sum_{i=0}^n \hat r_ix^i$.
\end{prop}
    
\begin{proof}
    Se $p(x)=0$, seja $n=0$ e $r_0=0$. Caso contrário, seja $n=\gr p(x)$ e $r_i=p_i$ para todo $i\leq n$. Então:

    $$p(x)=(p_0, \dots, p_n, 0, \dots, 0)=(r_0, \dots, r_n, 0, \dots, 0)=\sum_{i=0}^n\hat r_ix^i$$
\end{proof}

\begin{prop}
    Na notação anterior, se $r_1, \dots, r_n$ e $s_1, \dots, s_n$ são elementos de $R$, então:

    $\sum_{i=0}^n \hat r_i x^i=\sum_{i=0}^n \hat s_i x^i$ se, e somente se, $r_i=s_i$ para todo $i\leq n$.
\end{prop}
    
\begin{proof}
A recíproca é imediata. Para a implicação direta, note que a igualdade nos diz que $(r_0, \dots, r_n, 0, 0, \dots)=(s_0, \dots, s_n, 0, 0, \dots)$, ou seja, $r_i=s_i$ para todo $i\leq n$.
\end{proof}

Notação: abandona-se $\hat r$ em favor de $r$, mesmo havendo ambiguidade de notação.

\begin{prop}
    Seja $R$ um anel comutativo.
    $R[x]$ é um anel comutativo que satisfaz a seguinte propriedade:
    Para todo anel $S$, todo homomorfismo $f:R\rightarrow S$ e todo $s \in S$, existe um único homomorfismo $g:R[x]\rightarrow S$ tal que $g\circ \phi=f$ e $g(x)=s$.
\end{prop}

\begin{proof}
    Defina $g:R[x]\rightarrow S$ por $g(p(x))=\sum_{i=0}^n f(r_i)s^i$, onde $p(x)=\sum_{i=0}^n r_ix^i$.
    Note que $g$ é bem definido, pois se $p(x)=\sum_{i=0}^n r_ix^i=\sum_{i=0}^n s_ix^i=q(x)$, então $r_i=s_i$ para todo $i\leq n$.

    $g$ é homomorfismo, pois, dados $p(x)=\sum_{i=0}^n r_ix^i$ e $q(x)=\sum_{i=0}^m s_ix^i$, escrevendo $a_i=0$ para $i>n$ e $b_i=0$ para $i>m$, temos que:
    \begin{align*}
        g(p(x)+q(x))&=g\left(\sum_{i=0}^{\max\{n, m\}}(r_i+s_i)x^i\right)\\
        &=\sum_{i=0}^{\max\{n, m\}}f(r_i+s_i)s^i\\
        &=\sum_{i=0}^{n}f(r_i)s^i+\sum_{i=0}^{m}f(s_i)s^i\\
        &=g(p(x))+g(q(x)).
    \end{align*}
    \begin{align*}
        g(p(x)q(x))&=g\left(\sum_{i=0}^{n+m}\left(\sum_{j=0}^i r_{i-j}s_j\right)x^i\right)\\
        &=\sum_{i=0}^{n+m}\left(\sum_{j=0}^i f(r_{i-j})f(s_j)\right)s^i\\
        &=\sum_{i=0}^{n}f(r_i)s^i\cdot \sum_{j=0}^m f(s_j)s^j\\
        &=g(p(x))g(q(x)).
    \end{align*}
    \begin{align*}
        g(1_{R[x]})
        &=\sum_{i=0}^0 f(1_R)s^i=1_S.
    \end{align*}

    A função $g$ é única, pois se $g'$ é outra tal função, temos que, dado $p(x)=\sum_{i=0}^n r_ix^i$, temos que $g'(p(x))=\sum_{i=0}^nf(r_i)s^i=g(p(x))$.
\end{proof}