
\chapter{Polinômios}
Nesse capítulo, estudaremos os anéis de polinômios no contexto de anéis comutativos.

Apresentaremos uma construção do anel das séries formais, e a partir deste, extrairemos o anel de polinômios.

Após isso, veremos como se dá sua definição abstrata.
\section{Séries Formais}
Ao estudar Análise Real, Análise Funcional, Funções Analíticas ou mesmo Cálculo Diferencial e Integral, é comum se deparar com somas infinitas. Em tais assuntos, essas somas podem convergir ou divergir, e, mesmo quando convergem, não é sempre que podemos manipular essas somas infinitas como gostaríamos.

Para se falar em convergência de tais objetos, é necessária uma noção de convergência, o que pode ser feito por uma noção de métrica, ou, mais geralmente, por uma noção de topologia.
Tal estudo foge do escopo deste texto.

Apesar disso, em anéis comutativos arbitrários, é possível estudar séries de potência como objetos formais, sem nunca de fato computar somar infinitas, ou falar em convergência. É o que faremos nesta seção.


Se $R$ é um anel comutativo, intuitivamente uma série formal é um objeto que se escreve na forma:
$$a_0+a_1x+a_2x^2+\dots=\sum_{i=0}^\infty a_ix^i$$
onde $a_i\in R$.

Antes de definirmos o que, formalmente, é esse objetivo, vamos enunciar algumas propriedades que gostaríamos que esse objeto tivesse.

\begin{itemize}
    \item \textbf{Igualdade:} é conveniente que, no aspecto formal, a igualdade entre séries formais seja determinada pelos seus coeficientes. Ou seja, que:
    \begin{align*}
      \sum_{i=0}^\infty a_ix^i=\sum_{i=0}^\infty b_ix^i & \Leftrightarrow \forall i\in \mathbb N\, a_i=b_i.
    \end{align*}
    \item \textbf{Soma:} intuitivamente, se valem propriedades associativas, comutativas e distributivas, faz sentido que a soma satisfaça a propriedade a seguir, imaginando que as duas séries à esquerda se juntam e se reordenam de modo a obter a da direita.
    \begin{align*}
        \left(\sum_{i=0}^\infty a_ix^i\right)+\left(\sum_{i=0}^\infty b_ix^i\right)=\left(\sum_{i=0}^\infty (a_i+bi)x^i\right)
    \end{align*}
    \item \textbf{Produto:} intuitivamente, se, no lado esquerdo da igualdade, valerem propriedades distributivas ``infinitas'', como podemos definir produto? Ora, o coeficiente $c_i$ da série produto resultante deveria ser obtido agrumando (através de soma) os coeficientes $a_jb_k$ com $j+k=i$. Isso é equivalente à igualdade abaixo:
    \begin{align*}
        \left(\sum_{i=0}^\infty a_ix^i\right)\cdot\left(\sum_{i=0}^\infty b_ix^i\right)=\sum_{i=0}^\infty \left(\sum_{j=0}^i a_{j}b_{i-j}\right)x^i
    \end{align*}
    \item Notação: $R\llbracket x \rrbracket$ é o conjunto de todas as séries formais em $R$.
\end{itemize}
Olhando apenas para as regras acima, parece que as letras $x^i$ parecem não ter nenhum papel a não ser o de demarcar a ``$i$-ésima posição.'' Com isso em mente, definimos:
\begin{definition}
Seja $R$ um anel comutativo.
Definiremos $R\llbracket x\rrbracket$ como o conjunto $R^{\mathbb N}$ munido das operações definidas abaixo.
Nesse contexto, escrevemos, para $(a_i: i \in \mathbb N)\in R^{\mathbb N}$:

\[\sum_{i=0}^\infty a_ix^i=(a_i)_{i \in \mathbb N}.\]

Reforçamos que, neste contexto, não há nenhuma soma infinita ocorrendo, e o lado esquerdo é simplesmente definido como uma notação alternativa para o lado direito.
Note ainda que, com essa definição, a propriedade da igualdade acima vale automaticamente.

Se $p\in \mathbb R\llbracket x\rrbracket$, escrevemos $p(x)=p$ para reforçar o fato de que estamos lidando com uma série formal, e escrevemos $p(x)=\sum_{i=0}^\infty p_ix^i$. Os elementos $p_i$ são chamados de \emph{coeficientes} de $p(x)$, e $p_0$ é chamado de \emph{coeficiente constante}.

\textbf{Operações}:
Se $p(x), q(x) \in R\llbracket x\rrbracket$, define-se:
\[p(x)+q(x)=\sum_{i=0}^\infty p_ix^i+\sum_{i=0}^\infty q_ix^i=\sum_{i=0}^\infty(p_i+q_i)x^i.\]
\[p(x)\cdot q(x)=\sum_{i=0}^\infty p_ix^i\cdot\sum_{i=0}^\infty q_ix^i=\sum_{i=0}^\infty\left(\sum_{j=0}^i p_{i-j}q_{j}\right)x^i.\]
\[1_{R\llbracket x\rrbracket}=\sum_{i=0}^\infty \delta_{i0}x^i=(1, 0, 0, \dots).\]
\[0_{R\llbracket x\rrbracket}=\sum_{i=0}^\infty 0x^i=(0, 0, 0, \dots).\]
\end{definition}

\begin{lemma}[Séries formais formam anéis]
    Se $R$ é um anel comutativo, então $R\llbracket x \rrbracket$ é um anel comutativo.
\end{lemma}

\begin{proof}
    A operação de soma de $\mathbb R\llbracket x \rrbracket$ é a mesma de $\mathbb R^{\mathbb N}$, que já verificamos satisfazer as propriedades de grupo Abeliano. Assim, $R\llbracket x \rrbracket$ é um grupo abeliano sob a soma.

    Para as demais propriedades, fique $p(x), q(x), r(x) \in R\llbracket x \rrbracket$ e $i \in \mathbb N$.
    \begin{itemize}
        \item \textbf{Distributividade:} O $i$-ésimo coeficiente de $p(x)\cdot (q(x)+r(x))$ é:
        
        \[\sum_{j=0}^i p_{i-j}(q_j+r_j)=\sum_{j=0}^i p_{i-j}q_j+\sum_{j=0}^i p_{i-j}r_j.\]

        O que coincide com o $i$-ésimo coeficiente de $p(x)q(x)+p(x)r(x)$.
        \item \textbf{Elemento Neutro:} Temos que:
        \[p(x)\cdot 1=\sum_{i=0}^\infty\left(\sum_{j=0}^i p_{i-j}\delta_{0j}\right)x^i=\sum_{i=0}^\infty p_ix^i=p(x).\]
        \item \textbf{Comutatividade:} A $i$-ésima coordenada de $p(x)\cdot q(x)$ é $\sum_{j=0}^i p_{i-j}q_j=\sum(p_{i-j}q_j: j \in A_i)$, onde $A_i=\{0, \dots, i\}$. A função $\phi: A_i\rightarrow A_i$ dada por $\phi(j)=i-j$ é bijetora, pois é injetora e $A_i$ é finito. Assim:
        \[\sum_{j=0}^i p_{i-j}q_j=\sum_{j=0}^ip_{i-\phi(j)}q_{\phi(j)}=\sum_{j=0}^ip_{j}q_{i-j}=\sum_{j=0}^iq_{i-j}p_{j}.\]

        E esta é a $i$-ésima coordenada de $q(x)\cdot p(x)$.

        \item \textbf{Associatividade:} Temos que a $i$-ésima coordenada de $(p(x)\cdot q(x))\cdot r(x)$ é dada por:

        \[\pi_i((p(x)\cdot q(x))\cdot r(x))=\sum_{j=0}^i\pi_{i-j}(p(x)\cdot q(x))\cdot q_j=\sum_{j=0}^i\left(\sum_{k=0}^{i-j}p_{i-j-k}q_k\right) q_j\]
        \[=\sum_{j=0}^{i}\sum_{k=0}^{i-j}p_{i-j-k}q_kq_j=\sum \left(p_{i-j-k}q_kr_j:(j, k)\in A\right).\]

        Onde $A=\{(j, k): 0\leq j\leq i, 0\leq k\leq i-j\}$.

        Temos que a $i$-ésima coordenada de $p(x)\cdot (q(x)\cdot r(x))$ é dada por:

        $$\pi_i(p(x)\cdot (q(x)\cdot r(x)))=\sum_{s=0}^ip_{i-s}\pi_s(q(x)\cdot r(x))=\sum_{s=0}^ip_{i-s}\left(\sum_{t=0}^sq_{s-t}r_t\right)$$
        $$=\sum_{s=0}^i\sum_{t=0}^sp_{i-s}q_{s-t}r_t=\sum\left(q_{i-s}q_{s-t}r_t:(s, t)\in B\right)$$
        
        onde $B=\{(s, t): 0\leq t\leq s\leq i\}$. A função $\phi: A\rightarrow B$ dada por $\phi(j, k)=(j+k, j)$ é bijetora: é em $B$, pois $0\leq j\leq j+k\leq j+(i-j)=i$. É injetora, pois se $(j+k, j)=(j'+k', j')$ então $j=j'$ e, cancelando, $k=k'$. Finalmente, é sobrejetora, pois se $0\leq t\leq s\leq i$, sendo $j=t$ e $k=s-t$, temos que $0\leq j\leq i$, $0\leq k=s-t\leq i-t=i-j$ e $j+k=s$. Assim, $\phi$ é bijetora. Portanto:

        $$\sum\left(q_{i-s}q_{s-t}r_t:(s, t)\in B\right)=\sum\left(q_{i-(j+k)}q_{(j+k)-j}r_j:(j, k)\in A\right)$$$$=\sum\left(q_{i-j-k}q_{k}r_j:(j, k)\in A\right).$$
    \end{itemize}
\end{proof}

Note que, ao menos por enquanto, a letra $x$ é apenas parte da notação, e que não faz sentido, por enquanto, ``substituir $x$'' por algo.
\section{Anéis de Polinômios}
Na subseção anterior, introduzimos o anel das séries formais de um anel comutativo dado. Vimos que tal anel é um anel comutativo.

Deste anel, podemos extrair o anel de polinômios.

\begin{definition}
Seja $R$ um anel comutativo e $p(x)\in R\llbracket x \rrbracket$.

Define-se o \emph{suporte} de $p(x)$ por:

\[\supp p(x) =\{i \in I: p_i\neq 0\}.\]

Define-se o \emph{grau} de $p(x)$ por:
\[
\gr(p(x))=\begin{cases}
    \infty & \text{se } \supp p(x)  \text{ é infinito}\\
    -\infty  & \text{se } \supp p(x) =\emptyset \,(\text{se } p(x)=0)\\
    \max \supp p(x)  & \text{caso contrário.}
\end{cases}
\]
O \emph{anel de polinômios com coeficientes em $R$}, denotado por $R[x]$, é o subconjunto de $R\llbracket x \rrbracket$ dado por:
$$R[x]=\{p \in R\llbracket x \rrbracket: \gr(p) < \infty\}.$$

Se $p(x)\in R[x]$ é não nulo, define-se o \emph{coeficiente dominante} de $p(x)$ por $a_{\gr(p(x))}$.
Ou seja, o coeficiente dominante de $p(x)$ é seu coeficiente não nulo de mais alta posição.
\end{definition}

Assim, formalmente, o conjunto dos polinômios foi construído como sendo o conjunto de sequências eventualmente nulas de elementos de $R$.

\begin{lemma}
    Seja $R$ um anel comutativo. O anel de polinômios $R[x]$ é um subanel de $R\llbracket x \rrbracket$. Mais especificamente, dados $p(x), q(x) \in R[x]$:

    \begin{enumerate}[label=\alph*)]
        \item $\gr \left(p(x)q(x)\right)\leq\gr p(x)+\gr q(x)$, e a igualdade vale se $R$ for um domínio de integridade.
        \item $\gr\left(p(x)+q(x)\right)\leq \max\{\gr p(x),\gr(q(x))\}$.
        \item $\gr p(x)=\gr (-p(x))$.

    \end{enumerate}
\end{lemma}

\begin{proof}
    Ambas as afirmações são óbvias se $p(x)=0$ ou $q(x)=0$, então suponhamos que $p(x)$ e $q(x)$ são ambos não nulos.
    Sejam $n, m$ os graus de $p(x)$ e $q(x)$, respectivamente.
    
    Calculemos o coeficiente $n+m$ de $p(x)q(x)$.

    $$\pi_{n+m}(p(x)q(x))=\sum_{j=0}^{n+m}p_{n+m-j}q_j.$$

    Se $0\leq j< m$ temos que $n+m-j>0$, e $p_{n+m-j}=0$. Se $j>m$, temos que $q_j=0$. Assim, o único termo possivelmente não nulo da soma é quando $j=m$, que é $p_nq_m$. Este é não nulo $R$ for um domínio.
    
    Por outro lado, se $l>n+m$ temos que:

    $$\pi_{l}(p(x)q(x))=\sum_{j=0}^{l}p_{l-j}q_j.$$

    Se $0\leq j\leq m$ temos que $l-j>m+n-m=n$, e $p_{l-j}=0$. Se $j>m$, temos que $q_j=0$. Assim, todos os coeficientes da soma são $0$.
    Isso conclui que $\gr(p(x)q(x))\leq n+m$, sendo $n+m$ se $R$ for um domínio de integridade.

    Para a segunda afirmação, se $l>\max\{\gr p(x), \gr q(x)\}$, temos que o $l$-ésimo coeficiente de $p(x)+q(x)$ é $0$, pois este é $p_l+q_l=0+0$.

    A terceira afirmação é imediata.

    Agora, para a afirmação principal, as afirmações itemizadas nos mostram que $R[x]$ é fechado pela soma, produto e diferença de $R\llbracket x \rrbracket$. Finalmente, note que o grau da série $1=(1, 0, 0, \dots)$ é $0$, logo, $1\in R[x]$.
\end{proof}

Agora vamos trabalhar um pouco mais nossa notação.
Vamos identificar o elemento $x \in R[x]$, bem como identificar uma cópia de $R$ dentro de $R[x]$.

\begin{definition}
    Seja $R$ um anel comutativo. Em $R[x]$, seja $x=(0, 1, 0, 0, \dots)$ e, para cada $r \in R$, seja $\hat r=(r, 0, 0, \dots)$.
\end{definition}

\begin{lemma}
    Na notação anterior, para todo $r \in R$ e $n, i\geq 0$:

    $$\pi_i(\hat r x^n)(i)=\begin{cases}
        0 & \text{se } i\neq n\\
        r & \text{se } i=n.
    \end{cases}$$
    Ou seja, $\hat r x^n=(0, 0, \dots, r, 0, \dots)$ onde o $r$ está na posição $n$.
\end{lemma}

\begin{proof}
    Fixe $r$ Seguimos por indução. Para $n=0$, temos que $\hat rx^0=\hat r=(r, 0, 0, \dots)$ e para $n=1$ temos que $\hat r x^1=x=(0, r, 0, \dots)$.

    Para o passo $n+1$, onde $n\geq 1$, temos que, sendo $i\geq 1$:

    \[\pi_i(\hat rx^{n+1})=\pi_i((\hat rx^n)\cdot x)=\sum_{j=0}^i\pi_{i-j}(\hat r x^n)\cdot \pi_j(x)=\pi_{i-1}(\hat r x^n).\]

    Assim, se $i=n+1$, temos que a coordenada é $r$, e $0$ caso contrário. Resta apenas verificar que a coordenada $0$ é $0$. Ora, a coordenada $0$ se dá por $\pi_0(\hat r x^n)\pi_0(x)=0$.
\end{proof}

O lema abaixo mostra que os elementos $\hat r$ de fato agem como uma cópia de $R$ dentro de $R[x]$.

\begin{lemma}
    Na notação anterior, seja $\phi:R\rightarrow R[x]$ dada por $\phi(r)=\hat r$. Então $h$ é um homomorfismo injetor.
\end{lemma}
    
\begin{proof}
    Sejam $r, s \in R$. Então:
    \begin{itemize}
        \item $\phi(r+s)=(r+s, 0, 0, \dots)=\hat r+\hat s=\phi(r)+\phi(s)$.
        \item $\phi(rs)=(rs, 0, 0, \dots)=\hat r\cdot \hat s=\phi(r)\cdot \phi(s)$.
        \item $\phi(1_R)=\hat 1=(1_R, 0, 0, \dots)=1_{R[x]}=1_{R[x]}$.
    \end{itemize}

    A injetividade é óbvia.
\end{proof}

Agora veremos que todo elemento de $R[x]$ se escreve como uma soma de elementos de $R$ e potências de $x$, que é o esperado quando pensamos em polinômios.
\begin{prop}
    Na notação anterior, para todo $p(x)\in R[x]$ e $n\geq \gr p(x)$, existem únicos $r_0, r_1, \dots, r_n\in R$ tais que $p(x)=\sum_{i=0}^n \hat r_ix^i$, e estes são os coeficientes de $p(x)$.
\end{prop}
    
\begin{proof}
    Para a unicidade, note que se $p(x)=\sum_{i=0}^n \hat r_ix^i$, então para cada $i\leq n$, o $i$-ésimo coeficiente do lado direito é o $i$-ésimo coeficiente de $\hat r_ix^i$, que é $r_i$.
    Logo, $r_i$ é o $i$-ésimo coeficiente de $p(x)$.

    Para a existência, note que $p(x)=(p_0, p_1, \dots, p_n, 0, 0, \dots)=(p_0, 0, \dots)+(0, p_1, 0, \dots)+\dots+(0, 0, \dots, p_n)=p_0x^0+p_1x^1+\dots+p_nx^n=\sum_{i=0}^n \hat p_ix^i$.
\end{proof}
Note que, diferente do que ocorre na notação inicial sobre séries formais, a notação $\sum_{i=0}^n \hat r_i x^i$ expressa, de fato, uma soma (finita). Além disso, vale a comparação coeficiente-a-coeficiente.
\begin{corol}
    Na notação anterior, se $r_1, \dots, r_n$ e $s_1, \dots, s_n$ são elementos de $R$, então:

    $\sum_{i=0}^n \hat r_i x^i=\sum_{i=0}^n \hat s_i x^i$ se, e somente se, $r_i=s_i$ para todo $i\leq n$.
\end{corol}
Notação: abandona-se $\hat r$ em favor de $r$, mesmo havendo ambiguidade de notação.
\section{A propriedade universal do Anel de Polinômios}

Se $p(x)$ é um polinômio, esperamos poder "substituir" $x$ por um elemento $r$ de $R$, a fim de obter um elemento de $R$.


A proposição abaixo formaliza e generaliza essa ideia.
\begin{prop}[Propriedade universal do anel de polinômios]
    Seja $R$ um anel comutativo. Então
    $R[x]$ é um anel comutativo que satisfaz a seguinte propriedade:

    Para todo anel $S$, todo homomorfismo $f:R\rightarrow S$ e todo $s \in S$, existe um único homomorfismo $g:R[x]\rightarrow S$ tal que $g\circ \phi=f$ e $g(x)=s$.
\end{prop}
\begin{proof}
    Já vimos que $R[x]$ é um anel comutativo.

    Defina $g:R[x]\rightarrow S$ por $g(p(x))=\sum_{i=0}^n f(r_i)s^i$, onde $p(x)=\sum_{i=0}^n r_ix^i$.
    Note que $g$ é bem definido, pois se $p(x)=\sum_{i=0}^n r_ix^i=\sum_{i=0}^n s_ix^i=q(x)$, então $r_i=s_i$ para todo $i\leq n$.

    $g$ é homomorfismo, pois, dados $p(x)=\sum_{i=0}^n r_ix^i$ e $q(x)=\sum_{i=0}^m s_ix^i$, escrevendo $a_i=0$ para $i>n$ e $b_i=0$ para $i>m$, temos que:
    \begin{align*}
        g(p(x)+q(x))&=g\left(\sum_{i=0}^{\max\{n, m\}}(r_i+s_i)x^i\right)\\
        &=\sum_{i=0}^{\max\{n, m\}}f(r_i+s_i)s^i\\
        &=\sum_{i=0}^{n}f(r_i)s^i+\sum_{i=0}^{m}f(s_i)s^i\\
        &=g(p(x))+g(q(x)).
    \end{align*}
    \begin{align*}
        g(p(x)q(x))&=g\left(\sum_{i=0}^{n+m}\left(\sum_{j=0}^i r_{i-j}s_j\right)x^i\right)\\
        &=\sum_{i=0}^{n+m}\left(\sum_{j=0}^i f(r_{i-j})f(s_j)\right)s^i\\
        &=\sum_{i=0}^{n}f(r_i)s^i\cdot \sum_{j=0}^m f(s_j)s^j\\
        &=g(p(x))g(q(x)).
    \end{align*}
    \begin{align*}
        g(1_{R[x]})
        &=\sum_{i=0}^0 f(1_R)s^i=1_S.
    \end{align*}

    A função $g$ é única, pois se $g'$ é outra tal função, temos que, dado $p(x)=\sum_{i=0}^n r_ix^i$, temos que $g'(p(x))=\sum_{i=0}^nf(r_i)s^i=g(p(x))$.
\end{proof}

\begin{definition}
Seja $R$ um anel comutativo e $X$ um conjunto.
Um anel de polinômios sobre $R$ com variáveis em $X$ é uma tripla $(R[X], \phi, \alpha)$, onde $R[X]$ é um anel comutativo e $\phi:R\rightarrow R[X]$ é um homomorfismo e $\alpha:X\rightarrow R[X]$ é uma função que satisfazem a seguinte propriedade:

Para todo anel $S$ e toda função $\beta:X\rightarrow S$, existe um único homomorfismo $g:R[X]\rightarrow S$ tal que $g\circ \phi=f$ e $g\circ \alpha=\beta$.
\end{definition}

\begin{prop}
Seja $R$ um anel comutativo e $X$ um conjunto. Se $(R[X], \phi, \alpha)$ e $(\overline{R[X]}, \bar \phi, \bar \alpha)$ são anéis de polinômios sobre $R$ com variáveis em $X$, então $R[X]$ e $\overline{R[X]}$ são isomorfos por um isomorfismo $f:R[X]\rightarrow \overline{R[X]}$ tal que $f\circ \alpha=\bar \alpha$ e $f\circ \phi=\bar \phi$.
\end{prop}
\begin{proof}
    Aplicando a propriedade de $R[X]$ para o anel $\overline{R[X]}$ e a função $\bar \alpha$, existe um homomorfismo $g:R[X]\rightarrow \overline{R[X]}$ tal que $g\circ \phi=\bar \phi$ e $g\circ \alpha=\bar \alpha$.

    Aplicando a propriedade de $\overline{R[X]}$ para o anel $R[X]$ e a função $\alpha$, existe um homomorfismo $h:\overline{R[X]}\rightarrow R[X]$ tal que $h\circ \bar \phi=\phi$ e $h\circ \bar \alpha=\alpha$.

    Aplicando a propriedade de $R[X]$ para o anel $R[X]$ e a função $ \alpha$, existe um único homomorfismo $u:R[X]\rightarrow R[X]$ tal que $u\circ \phi=\phi$ e $u\circ \alpha=\alpha$.
    É imediato que $\id_{R[X]}$ satisfaz essas propriedades. O mapa $h\circ g$ também satisfaz, pois $h\circ g\circ \phi=h\circ \bar \phi=\phi$ e $h\circ g\circ \alpha=h\circ \bar \alpha=\alpha$.

    Aplicando a propriedade de $\overline{R[X]}$ para o anel $\overline{R[X]}$ e a função $\bar \alpha$, existe um único homomorfismo $v:\overline{R[X]}\rightarrow \overline{R[X]}$ tal que $v\circ \bar \phi=\bar \phi$ e $v\circ \bar \alpha=\bar \alpha$.
    É imediato que $\id_{\overline{R[X]}}$ satisfaz essas propriedades. O mapa $g\circ h$ também satisfaz, pois $g\circ h\circ \bar \phi=g\circ \phi=\bar \phi$ e $g\circ h\circ \bar \alpha=g\circ \bar \alpha=\bar \alpha$.

    Assim, $g\circ h=\id_{\overline{R[X]}}$ e $h\circ g=\id_{R[X]}$. Portanto, $g$ é um isomorfismo de anéis, e $h$ é o inverso de $g$.
\end{proof}

\begin{corol}
    Seja $R$ um anel comutativo e $X$, $Y$ conjuntos de mesmas cardinalidades.

    Se $(R[X], \phi, \alpha)$ e $(R[Y], \bar \phi, \bar \alpha)$ são anéis de polinômios sobre $R$ com variáveis em $X$ e $Y$, respectivamente, então $R[X]$ e $R[Y]$ são isomorfos.
\end{corol}

\begin{proof}
Seja $\theta:Y\rightarrow X$ uma bijeção. Basta ver que $(R[X], \phi, \alpha\circ \theta)$ é um anel de polinômios sobre $R$ com variáveis em $Y$.

Seja $S$ um anel comutativo e $f:R\rightarrow S$ um homomorfismo, e $\beta:Y\rightarrow S$ uma função.
Temos que $\beta'=\beta\circ \theta^{-1}:X\rightarrow S$ é uma função.
Então existe um único homomorfismo $g: R[X]\rightarrow S$ tal que $g\circ \phi=f$ e $g\circ \alpha=\beta'$.
Desta última, $\alpha\circ \theta=\beta$.

Para ver que $g$ é o único tal homomorfismo, se $h: R[Y]\rightarrow S$ é um homomorfismo tal que $h\circ \phi=f$ e $h\circ \alpha\circ \theta=\beta$, então então $h\circ \alpha=\beta'$.
Pela unicidade de $g$, segue que $g=h$.
\end{proof}
\begin{prop}
    Seja $R$ um anel comutativo e $X$ um conjunto. Então $R[x, y]\approx R[x][y]$.
\end{prop}
\begin{proof}
    Seja $X=\{x, y\}$, $\psi:R[x]$ a imersão canônica, $\psi': R[x]\rightarrow R[x][y]$ a imersão, $\phi=\psi'\circ \psi$, $\alpha(x)=x\in R[x][y]$ e $\alpha(y)=y\in R[x][y]$. e $\alpha(x)=\psi'(x)\in R[x]$.
    Então, $(R[x][y], \phi, \alpha)$ é um anel de polinômios sobre $R$ com variáveis em $X$.

    Com efeito, seja $S$ um anel comutativo, $f:R\rightarrow S$ um homomorfismo e $\beta:X\rightarrow S$ uma função.
    Então, existe um único homomorfismo $g:R[x]\rightarrow S$ tal que $g\circ \psi=f$ e $g(x)=\beta(x)$.
    Então, existe um único homomorfismo $h:R[x][y]\rightarrow S$ tal que $h\circ \psi'=g$ e $h(y)=\beta(y)$.
    Assim, $h\circ \phi=h\circ(\psi'\circ \psi)=(h\circ \psi')\circ \psi=g\circ \psi=f$, $h(\alpha(y))=h(y)=\beta(y)$ e $h(\alpha(x))=h\circ \psi'(x)=g(x)=\beta(x)$.

    Para ver a unicidade de $h$, se $\bar h$ é outra tal função, temos que $\bar h\circ \psi=(\bar h\circ \psi')\circ \psi=f$ e $\bar h\circ \alpha(x)=\bar h\circ \psi'(x)=\beta(x)$, logo, $\bar h\circ \psi'=g$.
    Além disso, $\bar h\circ \alpha=\beta$, logo, $\bar h=h$.
\end{proof}

\section{Divisibilidade em anéis de polinômios}
\begin{prop}
Se $R$ é um domínio de integridade, então para todo $p(x), d(x) \in R[x]$ com $d(x)\neq 0$ e o coeficiente dominante de $d$ invertível, existem únicos $q(x), r(x) \in R[x]$ tais que $p(x)=d(x)q(x)+r(x)$, onde $\deg(r(x))<\deg(d(x))$.
Em particular, se $R$ é um corpo, então $R[x]$ é um domínio Euclideano.
\end{prop}
\begin{proof}
    Existência: fixe $d(x)$ e seja $\deg(d(x))=k\geq 0$.
    Provaremos por indução no grau de $p(x)$.
    Se $\deg(p(x))<k$, seja $q(x)=0$ e $r(x)=p(x)$.

    Para o passo indutivo, se a hipótese vale para $\gr(p(x))<n$ e $n+1\geq k$, seja $p_{n+1}$ o coeficiente dominante de $p(x)$ e $d_k$ o de $d(x)$.
    Temos que $p(x)-\frac{p_{n+1}}{d_k}d(x)x^{n+1-k}$ é um polinômio de grau $\leq n$, logo, pela hipótese indutiva, existe $q(x), r(x)$ tais que $p(x)-\frac{p_n}{d_k}d(x)x^{n+1-k}=d(x)q(x)+r(x)$, onde $\deg(r(x))<k$.
    Segue que $p(x)=(\frac{p_{n+1}}{d_k}d(x)x^{n+1-k}+q(x))d(x)+r(x)$, onde $\deg(r(x))<k$.

    Para a unicidade, se existem $q_1(x), r_1(x)$ e $q_2(x), r_2(x)$ tais que $p(x)=d(x)q_1(x)+r_1(x)=d(x)q_2(x)+r_2(x)$, temos que $d(x)(q_1(x)-q_2(x))=r_2(x)-r_1(x)$.
    Se $q_1(x)-q_2(x)=0$ segue a tese.

    Caso contrário, temos $\gr(r_2(x)-r_1(x))<\gr(d(x))\leq\gr(d(x)(q_1(x)-q_2(x)))=\gr(r_2(x)-r_1(x))$, o que é uma contradição.
\end{proof}

\begin{prop}
    Seja $R$ um domínio de integridade e $p(x)\in R[x]$.
    Temos que $(x-a)|p(x)$ se, e somente se, $p(a)=0$.
\end{prop}
\begin{proof}
    Suponha que $p(a)=0$. Existem $q(x)\in R[x]$ e $r \in R$ tais que $p(x)=q(x)(x-a)+r$.
    Como $p(0)=0$, segue que $0=r$, logo, $(x-a)|p(x)$.

    Reciprocamente, se $(x-a)|p(x)$, exite $q(x)$ tal que $p(x)=(x-a)q(x)$. Assim, $p(0)=0$.
\end{proof}

\section{Exercícios}
\begin{exer}
    Seja $K$ um corpo. Demonstre, de acordo com o roteiro abaixo, que o ideal $\langle x-a\rangle$ é maximal no domínio $K[x]$:
    \begin{enumerate}[label=\alph*)]
        \item Considere o homomorfismo avaliação de $K[x]$ em $K$ que avalia $p(x)$ em $a$. Mostre que o núcleo desse homomorfismo é $\langle x-a\rangle$.
        \item Demonstre que $K[x]/\langle x-a\rangle$ é isomorfo a $K$.
        \item Conclua que $\langle x-a\rangle$ é maximal.
    \end{enumerate}
\end{exer}
\begin{exer}
    Considere o anel de polinômios $K[x, y]=K[x][y]$ e  $I=\langle y\rangle$.
    \begin{itemize}
        \item Mostre que $I$ não é um ideal maximal exibindo um ideal próprio que o contém.
        \item Mostre que $I$ não é maximal estudando o quociente $K[x, y]/I$.
    \end{itemize}
\end{exer}
\begin{exer}
Mostre que $\mathbb R[x]/\langle x^2+1\rangle$ é isomorfo à $\mathbb C$.
\end{exer}
\begin{exer}
    Mostre que $\mathbb Z[x]/\langle x^2+1\rangle$ é isomorfo à $\mathbb Z[i]$.
\end{exer}