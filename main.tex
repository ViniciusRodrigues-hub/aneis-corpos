\documentclass{article}
\usepackage[brazil]{babel}
\usepackage{amsthm, amsmath, amssymb}
\usepackage[a4paper]{geometry}
\usepackage{enumitem}
\usepackage{stmaryrd}
\usepackage{tikz-cd}
\usepackage{indentfirst}
\usetikzlibrary{babel}
\usepackage{float}
\theoremstyle{definition}

\newtheorem{definition}{Definição}[section]
\newtheorem{prop}[definition]{Proposição}
\newtheorem{corol}[definition]{Corolário}
\newtheorem{exer}[definition]{Exercício}
\newtheorem{lemma}[definition]{Lema}
\newtheorem{theorem}[definition]{Teorema}

\DeclareMathOperator{\ran}{ran}
\DeclareMathOperator{\supp}{supp}
\DeclareMathOperator{\gr}{gr}
\DeclareMathOperator{\id}{id}
\author{Prof. Vinicius Rodrigues}
\title{Notas da disciplina MAT0264 - Anéis e Corpos}
\begin{document}
\maketitle


\section{Pré-Requisitos Conjuntistas}
Antes de descrevermos grupos, precisamos de algumas definições e resultados básicos envolvendo noções básicas sobre conjuntos e funções. Para isso, utilizaremos a notação usual de conjuntos, como $\mathcal P(X)$ para o conjunto das partes de $X$, $X^n$ para o produto cartesiano de $n$ cópias de $X$, e assim por diante.

Não é objetivo deste texto desenvolver a parte inicial da Teoria dos Conjuntos. Apenas apresentaremos algumas definições, notações e resultados básicos que utilizaremos ao longo do texto.
\subsection{Operações}

\begin{definition}[Operações $n$-árias]
    Se $X$ é um conjunto e $n \in \mathbb N$, uma operação $n$-ária em $X$ é uma função $f:X^n\rightarrow X$.
\end{definition}

Operações $2$-árias e $1$-árias são frequentemente chamadas de \emph{binárias} e \emph{unárias}, respectivamente.

Caso $*$ seja uma operação binária, a notação $x*y$ é frequentemente utilizada para denotar $x*y$.

Caso $*$ seja uma operação unária, a notação $*x$ é frequentemente utilizada para denotar $*(x)$.
\subsection{Produto cartesiano de conjuntos generalizado}

Famílias são funções com notação especial. Tal notação é utilizada quando pensamos em uma função como um ``conjunto indexado de valores'' ao invés de um ``dispositivo de entrada/saída''.

Matemáticamente, funções e famílias podem ser vistas como o mesmo objeto.
\begin{table}[h]
    \centering
    \begin{tabular}{lllll}
        \hline
        \textbf{Conceito} & \textbf{Função} & \textbf{Família} \\ \hline
        Mapa & $u:I\rightarrow A$ & $(u_i)_{i \in I}=(u_i: i \in I)$ \\
        Valor & $u(i)$ & $u_i$ \\
        Imagem & $\ran u$ & $\{u_i: i \in I\}$\\
        Intuição & objeto dinâmico & objeto estático \\
        Inputs & domínio $I$ & conjunto de índices $I$ \\
        \hline
    \end{tabular}
    \caption{Comparativo de família e função}
\end{table}

Exemplo: sequências. Uma sequência é uma família cujo conjunto de índices é $\mathbb N$. Compare a intuição que passa as notações:
\begin{itemize}
\item Considere a sequência $u=(\frac{1}{2^n}))_{n \in \mathbb N}$...
\item Considere a função $u:\mathbb N\rightarrow \mathbb R$ dada por $u(n)=\frac{1}{2^n}$...
\end{itemize}

Exemplo: sequências finitas. Se $n\geq 1$, identificamos $n=\{0, 1, \dots, n-1\}$. Assim:
\begin{itemize}
\item Uma família com $n$ elementos é uma família $(a_i)_{i<n}=(a_i)_{i \in n}=(a_0, \dots, a_{n-1})$.
\end{itemize}

\begin{definition}[Produto cartesiano de conjuntos]
Seja $(A_i)_{i \in I}$ uma família de conjuntos. O produto cartesiano de conjuntos é o conjunto $\prod_{i \in I} A_i$ definido como o conjunto de todas as famílias $(a_i: i \in I)$ tais que para cada $i \in I$, $a_i \in A_i$.
$$\prod_{i \in I} A_i=\{(a_i)_{i \in I}: \forall i \in I\, a_i \in A_i\}.$$
\end{definition}


\begin{definition}[Exponenciação de conjuntos]
    Sejam $A, I$ conjuntos. O conjunto $A^I$ é o conjunto de todas as funções de $I$ em $A$. Ou seja, $A^I=\{f:I\rightarrow A\}$. Note que:

    $$A^I=\prod_{i \in I}A=\{(a_i)_{i \in I}: \forall i \in I\,  a_i\in A\}.$$
    \end{definition}

    Na notação anterior, se $n\geq 1$ $$A^n=\{(a_i)_{i<n}:\forall i<n\, a_i \in A\}=\{(a_0, \dots, a_{n-1}):a_0, \dots, a_{n-1}\in A\}\approx A\times \dots \times A \,(n \text{ vezes}).$$
    \subsection{Produtos de anéis}

    \begin{definition}[Produto Direto de dois anéis]
        Sejam $R, S$ anéis. O produto direto de $R$ e $S$ é o conjunto $R\times S$ munido das operações ``ponto à ponto'': dados $a=(a_1, a_2)\in R\times S$ e $b=(b_1, b_2)\in R\times S$, temos:
        $$a+b=(a_1+b_1, a_2+b_2)$$
        $$a\cdot b=(a_1\cdot b_1, a_2\cdot b_2)$$
        $$0=(0_R, 0_S)$$
        $$1=(1_R, 1_S)$$
    \end{definition}
    
    Exemplo: Seja $R=\mathbb Z_3$ e $S=\mathbb Z_4$. Então $(2, 2)\in R\times S$ e $(1, 2)\in R\times S$. Temos:
    $$(2, 2)+(1, 2)=(2+ 1, 2+ 2)=(0, 0)$$
    $$(2, 2)\cdot (2, 2)=(2\cdot 2, 2\cdot 2)=(1, 0)$$
\section{Noções de Grupos}


\subsection{Definição e Propriedades Básicas}
\begin{definition}
Um grupo é uma quadrupla $(G,\cdot,e)$, tal que $G$ é um conjunto, $\cdot$ é uma operação binária em $G$ e $0 \in G$, e satisfazem:

\begin{itemize}
    \item (\textbf{Propriedade associativa}) $\forall a, b, c \in G$ $(a \cdot b) \cdot c = a \cdot (b \cdot c)$.
    \item (\textbf{Elemento neutro}) $\forall a \in G$  $e \cdot a = a \cdot e = a$.
    \item (\textbf{Elemento inverso}) $\forall a \in G$ $\exists b \in G$ $a \cdot b = b \cdot a = e$.
\end{itemize}
Se, adicionalmente, a seguinte propriedade é satisfeita, o grupo é chamado de \emph{comutativo}, ou, mais comunmente, \emph{abeliano}:
\begin{itemize}
    \item (\textbf{Comutatividade}) $\forall a, b \in G\, a \cdot b = b \cdot a$.
\end{itemize}
\end{definition}
Algumas observações:
\begin{itemize}
\item Ao discursar sobre grupos, é comum omitir a operação e o elemento neutro, referindo-se apenas ao conjunto $G$.
\item Caso o grupo seja abeliano, é comum que sua operação binária seja denotada por $+$ ou outro símbolo similar. Nesse contexto, o elemento neutro é frequentemente denotado por $0$.
\item Caso o grupo não seja abeliano, é comum que sua operação binária seja denotada por $\cdot$ ou outro símbolo similar. Nesse contexto, o elemento neutro é frequentemente denotado por $e$, e a operação é frequentemente omitida, ou seja, $a \cdot b$ é frequentemente escrito como $ab$.
\end{itemize}

Alguns exemplos:

\begin{itemize}
    \item Com a soma usual, $\mathbb{Z, Q, R, C}$ são grupos abelianos.
    \item Com a multiplicação usual, o círculo unitário complexo $\mathbb T=\{x \in \mathbb C: |x|=1\}$ é um grupo abeliano com elemento neutro $1$. De fato, o produto de complexos é comutativo, associativo e tem $1$ como elemento neutro. Note que $1\in \mathbb T$ e $0\notin \mathbb T$. Se $x \in \mathbb T$, o inverso multiplicativo de $x$ é dado por $\frac{\bar x}{|x|^2}$, onde $\bar x$ denota o conjugado de $x$. Como $|\bar x|=|x|=1$, segue que $\mathbb T$ tem todos os inversos de todos seus elementos.
    \item Os inteiros módulo $n$ ($n\geq 1$), dados por $\mathbb Z_n=\{0, \dots, n-1\}$ com a soma dada pela aritmética módulo $n$, são grupos.
\end{itemize}

Agora iniciaremos a provar algumas propriedades básicas sobre grupos.
\begin{prop}[Unicidade do elemento neutro]
    Seja $(G,\cdot,e)$ um grupo. Então, o elemento neutro $e$ é único. Isto é, se $h \in G$ é tal que $\forall a \in G$ $h \cdot a = a \cdot h = a$, então $h = e$.
\end{prop}
\begin{proof}
    Note que $h=he$, pois $e$ é elemento neutro. Por outro lado, $e=he$, pois $h$ é elemento neutro. Assim, $h=he=e$.
\end{proof}

\begin{prop}[Unicidade dos inversos]\label{prop:inverso_unico_grupo}
    Seja $(G,\cdot,e)$ um grupo. Então todo $a \in G$ possui um único elemento inverso, ou seja, para todo $g \in G$,  é único. Isto é $\forall a \in G$ $\exists!\, b \in G$ $a \cdot b = b \cdot a = e$.
\end{prop}
\begin{proof}
    A existência do inverso é garantida pela definição de grupo. Para provar a unicidade, suponha que $b, c$ são inversos de $a$, ou seja, $a \cdot b = b \cdot a = e$ e $a \cdot c = c \cdot a = e$.
    Então, temos:
    $$b=be=b(ac)=(ba)c=ec=c.$$
\end{proof}

A unicidade do elemento neutro e dos inversos nos permite definir a notação $a^{-1}$ para o inverso de $a$ em um grupo $(G,\cdot,e)$. Caso $(G, +, 0)$ seja um grupo abeliano, a notação $-a$ é frequentemente utilizada para denotar o inverso de $a$, e, nesse caso, $-a$ é chamado de \emph{oposto} de $a$.

Note que assim, ficam definidos operadores unários $(\,)^{-1}:G\rightarrow G$ (ou $-:G\rightarrow G$). Para o segundo caso, define-se também que $a-b=a+(-b)$.

\begin{prop}[Cancelamento]
    Seja $(G,\cdot,e)$ um grupo. Então, se $a,b,c \in G$ e $a \cdot b = a \cdot c$, então $b=c$. Analogamente, se$b \cdot a = c \cdot a$, então $b=c$.
\end{prop}
\begin{proof}
Provaremos a primeira afirmação. A segunda é análoga e fica como exercício.
    Suponha que $ba=ca$. Então $b=be=b(aa^{-1})=(ba)a^{-1}=(ca)a^{-1}=c(aa^{-1})=ce=c$. Assim, $b=c$.
\end{proof}

\begin{corol}[Cancelamento II]
    Seja $(G,\cdot,e)$ um grupo. Para todos $a, b \in G$, se $ab=a$, então $b=e$. Analogamente, se $ba=a$, então $b=e$.
\end{corol}
\begin{proof}
    Para a primeira afirmação, note que $ab=ae$, logo, pela proposição anterior, $b=e$. A segunda afirmação é análoga.
\end{proof}

\begin{prop}[Regras de sinal]\label{prop:regraSinal}
    Seja $G$ um grupo e $a, b \in G$. Então:
    \begin{enumerate}[label=\alph*)]
        \item $((a)^{-1})^{-1}=a$ [na notação aditiva, $-(-a)=a$]. \label{prop:regraSinal_A}
        \item $(ab)^{-1}=b^{-1}a^{-1}$ [na notação aditiva, $-(a+b)=(-b)+(-a)]$.\label{prop:regraSinal_B}
        \item $e^{-1}=e$ [na notação aditiva, $-0=0$].\label{prop:regraSinal_C}
    \end{enumerate}
\end{prop}
\begin{proof}
    \ref{prop:regraSinal_A}: Temos que $(a^{-1})^{-1}a^{-1}=e=aa^{-1}$. Cancelando $a^{-1}$, segue.
    
    \ref{prop:regraSinal_B}: Temos que $(ab)^{-1}(ab)=e=(b^{-1}a^{-1})ab$. Cancelando $ab$, segue que $(ab)^{-1}=b^{-1}a^{-1}$. Analogamente, $(ba)^{-1}=a^{-1}b^{-1}$.

    \ref{prop:regraSinal_C}: Temos que $(e^{-1})e=e=ee$. Cancelando $e$ à direita, segue.


\end{proof}

\subsection{Somatórios}

Nessa seção, formalizaremos somatórios.

\begin{definition}[Soma de sequência finita]
Seja $G$ um conjunto munido de uma operação $+$ associativa, comutativa e com neutro $0$. Define-se, recursivamente para $n\geq 0$, o somatório de sequências de $n$ elementos de $G$ como se segue:

\begin{itemize}
\item \textbf{Notação:} se $a=(a_i)_{i<n}$ é uma sequência de elementos de $G$, então usamos as notações:
$$\sum a=\sum(a_i: i<n)=\sum_{i<n} a_i= \sum_{i=0}^{n-1} a_i$$
\item Caso base $n=0$: só existe uma sequência com $0$ elementos, que é a sequência vazia $a=()=\emptyset=(a_i:i<0)$. Definimos: $$\sum a=\sum(a_i: i<0)=\sum_{i<0} a_i=\sum_{i=0}^{-1}a_n=0$$.
\item Passo recursivo $n\rightarrow n+1$: considere uma sequência $(a_i)_{i<n+1}=(a_i)_{i\leq n}$ de elementos de $G$. Define-se:
$$\sum_a=\sum_{i=0}^{n}=\left(\sum_{i=0}^{n-1}\right)+a_n.$$
\end{itemize}

\end{definition}

\textbf{Observação:} Se temos uma sequência unitária $(a_i: i\leq 0)=(a)$, estamos no caso $n=1$, então:

$$\sum_{i=0}^{0}a_i=\sum_{i=0}^{-1}a=a.$$
Muitas pessoas consideram que não é muito natural apresentar a definição acima incluindo o caso $n=0$, tomando como base o caso $1$, definindo-se que a soma unitária é igual ao valor de seu único elemento. A observação acima mostra que a definição apresentada não é inconsistente com essa visão. Como vantagem da notação apresentada, ela possibilita falar em somas vazias, que, em alguns pontos da teoria, são úteis para simplificar a notação (porém nunca imprescindíveis). Como desvantagem, a definição apresentada exige que o conjunto em questão possua o elemento neutro $0$, enquanto a outra, não.

\begin{definition}[Soma de uma família finita]
    Seja $G$ um conjunto munido de uma operação $+$ associativa, comutativa e com neutro $0$. Seja $(a_i: i \in F)$ uma família finita. Define-se:
    
$$\sum_{i \in F} a_i=\sum_{i=0}^{n-1} a_\phi^{-1}(i),$$

onde $\phi:\{0, \dots, n-1\}\rightarrow F$ é uma bijeção.  
\end{definition}

Qual $\phi$ escolher? É indiferente.

\begin{prop}
    Seja $G$ um conjunto munido de uma operação $+$ associativa, comutativa e com neutro $0$. Seja $(a_i: i \in F)$ uma família finita. Sejam $\phi:\{0, \dots, n-1\}\rightarrow F$ e $\psi:\{0, \dots, n-1\}\rightarrow F$ bijeções. Então:

    $$\sum_{i=0}^{n-1} a_\phi^{-1}(i)=\sum_{i=0}^{n-1} a_\psi^{-1}(i).$$
\end{prop}

\section{Anéis}
Nesta seção, começaremos a discutir a noção matemática de anel, uma das principais estruturas que serão estudadas.
\begin{definition}[Anel]
    Um anel é uma $4$-upla $(A, +, \cdot, 0, 1)$ conjunto $A$ com duas operações binárias, adição e multiplicação, denotadas por $+$ e $\cdot$, tais que:
    \begin{itemize}
        \item $(A, +, 0)$ é um grupo abeliano.
        \item (\textbf{Associatividade}) Para todo $a, b \in A$, temos $(a \cdot b)\cdot c = a\cdot(b\cdot c)$.
        \item (\textbf{Elemento identidade}) $\forall a \in A$ $1 \cdot a = a \cdot 1 = a$.
        \item (\textbf{Propriedades distributivas}) Para todos $a, b, c, \in A$, temos:
        \begin{align*}
            a \cdot (b + c) &= a \cdot b + a \cdot c, \text{ e}\\
            (a + b) \cdot c &= a \cdot c + b \cdot c
        \end{align*}
    \end{itemize}
    Se, adicionalmente, a seguinte propriedade é satisfeita, o anel é chamado de \emph{comutativo}.
    \begin{itemize}
        \item (\textbf{Comutatividade}) $\forall a, b \in A$ $a \cdot b = b \cdot a$.
    \end{itemize}
\end{definition}

Algumas observações:
\begin{itemize}
    \item Como em grupos, ao discursar sobre anéis é comum omitir as operações, referindo-se apenas ao conjunto $A$.
    \item Ao discursar sobre anéis, e a exemplo do que foi feito ao enunciar as propriedades dsitributivas, são utilizadas as convenções usuais sobre precedência de operações envolvidas por parênteses. Assim, $a + b \cdot c$ é interpretado como $a + (b \cdot c)$.
    \item Há textos que definem anéis sem incluir o elemento identidade $1$. Nestes textos, a definição acima dá nome ao que chamam de \emph{anéis com identidade}, ou \emph{anéis com 1}. Nesse curso, não usaremos essa convenção, de modo que \textbf{todos nossos anéis possuem identidade}. De modo similar, alguns textos definem anéis como sendo comutativos. Também não adotaremos essa convenção. \textbf{Os nossos anéis podem ser não comutativos}.
    \item A definição de anel não exige que $0=1$.
    \item $0$ é chamado de elemento nulo, e $1$ de elemento identidade.
\end{itemize}

\begin{prop}[Propriedade multiplicativa do $0$]
    Seja $A$ um anel. Então $\forall a \in A$ $0 \cdot a = a \cdot 0 = 0$.
\end{prop}
\begin{proof}
Provaremos a primeira afirmação. A segunda é análoga e fica como exercício.

Temos que $0\cdot a=(0+0)\cdot a=0\cdot a +0\cdot a$. Cancelando, segue que $0=0\cdot a$.
\end{proof}

\begin{prop}[Anel trivial]
    Seja $A={x}$ um conjunto qualquer. Defina $x\dot x=x=x+x=0=1$. Então $(A, +, \cdot, 0, 1)$ é um anel. Um anel dessa forma é chamado de \emph{anel trivial}.
    
    Além disso, se $A$ é um anel tal que $0=1$, então $A$ é um anel trivial.
\end{prop}
\begin{proof}
    A primeira afirmação (de que $A$ como acima é um anel) fica como exercício.

    Para a segunda afirmação, assuma que $A$ é um anel tal que $0=1$. Fixe $a \in A$ qualquer. Então $a=a.1=a.0=0$, ou seja, $a=0$. Assim, $A$ é o conjunto unitário $\{0\}$, que é um anel trivial.
\end{proof}

\begin{prop}[Regras de sinal II]\label{prop:regraSinal2}
    Seja $A$ um anel e $a, b \in A$. Então:
    \begin{enumerate}[label=\alph*)]
        \item $(-a)b=a(-b)=-(ab)$\label{prop:regraSinal2_A}
        \item $(-a)(-b)=ab$.\label{prop:regraSinal2_B}
        \item $(-1)a=-a$.\label{prop:regraSinal2_C}
    \end{enumerate}
\end{prop}
\begin{proof}
    \ref{prop:regraSinal2_A}: Temos que $ab+(-a)b=(-a)b+ab=[-a+a]b=0b=0$. Assim, $(-a)b=-(ab)$. Analogamente, $a(-b)=-(ab)$.
    \ref{prop:regraSinal2_B}: Temos que $(-a)(-b)=-[a(-b)]=-[-(ab)]=ab$ pela regra anterior.
    \ref{prop:regraSinal2_C}: Temos que $(-1)a=-(1a)=-a$.

\end{proof}

\subsection{Elementos invertíveis}
\begin{definition}[Elemento invertível]
    Seja $A$ um anel. Um elemento $a \in A$ é dito \emph{invertível}, ou uma \emph{unidade} se $\exists b \in A$ tal que $a \cdot b = b \cdot a = 1$.
    
    O conjunto de todas das unidades de $A$ é denotado por $A^*$.
\end{definition}

\begin{definition}
    Seja $A$ um anel. Então, se $a \in A^*$, existe um \textbf{único} $b \in A$ tal que $a \cdot b = b \cdot a = 1$. Este elemento é denotado por $a^{-1}$, e é chamado de \emph{inverso} de $a$.
\end{definition}

Observação: para que a definição acima faça sentido, é necessário mostrar que se $a$ é unidade, existe um \textbf{único} $b \in A$ tal que $a \cdot b = b \cdot a = 1$. A existência é garantida pela definição de unidade, e a demonstração da unicidade é análoga à da unicidade do inverso em grupos (Proposição \ref{prop:inverso_unico_grupo}), ficando como exercício.

\begin{prop}
Seja $A$ um anel. Para todos $a, b \in A^*$, temos:
\begin{enumerate}[label=\alph*)]
    \item $ab\in A^U$ e $(ab)^{-1}=b^{-1}a^{-1}$.\label{prop:unidadeProduto_a}
    \item $a^{-1}\in A^U$ e $(a^{-1})^{-1}=a$.\label{prop:unidadeProduto_b}
    \item $1^{-1}=1$.\label{prop:unidadeProduto_c}
\end{enumerate}
Além disso, $A^*$ é, com a restrição da operação de multiplicação do anel, um grupo com identidade $1$. Caso $A$ seja abeliano, $A^*$ é um grupo abeliano.
\end{prop}
\begin{proof}
    \ref{prop:unidadeProduto_a}: Sejam $a, b \in A^*$. Pela associatividade, $(ab)(b^{-1}a^{-1})=1=(b^{-1}a^{-1})(ab)$, logo, pela unicidade do inverso, $(ab)^{-1}=b^{-1}a^{-1}$.

    \ref{prop:unidadeProduto_b}: Seja $a \in A^*$. Temos que $a^{-1}a=1=a(a^{-1})$, logo, pela unicidade do inverso, $(a^{-1})^{-1}=a$.

    \ref{prop:unidadeProduto_c}: Note que $1\cdot 1=1=1\cdot 1$, logo, pela unicidade do inverso, $1^{-1}=1$.

    A última afirmação é imediata e fica como exercício.
\end{proof}

\begin{definition}
Um anel de divisão é um anel não trivial para o qual todo elemento não nulo é invertível. Um corpo é um anel de divisão comutativo.
\end{definition}

\begin{exer}
    Mostre que um anel $A$ é um anel de divisão se, e somente se $A^*=A\setminus\{0\}$.
\end{exer}
\begin{definition}
    Um domínio de integridade é um anel comutativo não trivial $A$ tal que $\forall a, b \in A$, se $ab=0$, então $a=0$ ou $b=0$.
\end{definition}

\begin{prop}
    Seja $K$ um corpo. Então $K$ é um domínio de integridade.
\end{prop}
\begin{proof}
Sabemos que $K$ é um anel comutativo não trivial. Sejam $a, b \in K$ tais que $ab=0$. Se $a=0$, então segue a tese. Caso contrário, como $K$ é um corpo, $a^{-1}$ existe. Assim, temos que $b=(a^{-1}a)b=a^{-1}(ab)=0$, logo, $b=0$.
\end{proof}
\subsection{Ideais}
\begin{definition}[Ideal à esquerda]
    Seja $A$ um anel. Um subconjunto $I \subseteq A$ é dito \emph{ideal à esquerda} se:
    \begin{itemize}
        \item $0 \in I$.
        \item Para todos $a, b \in I$, temos $a+b\in I$.
        \item $\forall a \in A$ e $\forall b \in I$, temos $ab \in I$.
    \end{itemize}
\end{definition}

\begin{definition}[Ideal à direita]
    Seja $A$ um anel. Um subconjunto $I \subseteq A$ é dito \emph{ideal à direita} se:
    \begin{itemize}
        \item $0 \in I$.
        \item Para todos $a, b \in I$, temos $a+b\in I$.
        \item $\forall a \in I$ e $\forall b \in A$, temos $ab \in I$.
    \end{itemize}
\end{definition}

\begin{definition}[Ideal]
    Seja $A$ um anel. Um subconjunto $I \subseteq A$ é dito \emph{ideal} se for um ideal à esquerda e um ideal à direita. Ou seja, $I$ é um ideal se:
    \begin{itemize}
        \item $0 \in I$.
        \item Para todos $a, b \in I$, temos $a+b\in I$.
        \item $\forall a \in A$ e $\forall b \in I$, temos $ab \in I$.
        \item $\forall a \in I$ e $\forall b \in A$, temos $ab \in I$.

    \end{itemize}
\end{definition}

\begin{prop}[Ideal trivial]Seja $A$ um anel. Então $\{0\}$ e $A$ são ideais de $A$. Estes ideais são chamados de \emph{ideais principais}
\end{prop}
\begin{proof}
    Exercício.
\end{proof}

Note que se $A$ é um anel comutativo, então $I$ é um ideal à esquerda se, e somente se, $I$ é um ideal à direita. Assim, em anéis comutativos, a noção de ideal é equivalente à de ideal à esquerda ou à de ideal à direita.

\begin{prop}[Interseção de ideais]
    Seja $A$ um anel e $\mathcal F$ uma coleção não vazia de ideais à esquerda de $A$. Então $\bigcap_{I \in \mathcal F}I=\bigcap \mathcal F$ é um ideal de $A$. O mesmo vale para ideais à direita e ideais.
\end{prop}
\begin{proof}
    Provaremos para ideais à esquerda. A prova para ideais à direita é análoga e fica como exercício.

    Seja $I=\bigcap \mathcal F$. Então $0 \in I$, pois $0 \in I$ para todo $I \in \mathcal F$.

    Sejam $a, b \in I$. Então, para todo $I \in \mathcal F$, temos que $a, b \in I$, logo, $a+b\in I$. Assim, $a+b\in \bigcap \mathcal F$.

    Finalmente, seja $a \in A$ e $b \in I$. Então, para todo $I \in \mathcal F$, temos que $b \in I$, logo, $ab\in I$. Assim, $ab\in \bigcap \mathcal F$.
\end{proof}

\begin{prop}[Ideal gerado]
    Seja $A$ um anel comutativo e $B\subseteq A$ um conjunto não vazio. Então, o conjunto $I=\{a_1b_1+\cdots+a_nb_n: n\geq 1, a_i \in A, b_i \in B\}$ é o menor ideal à esquerda $A$ que contém $B$ (ou seja, além de ser um ideal contendo $B$, se $J$ é qualquer ideal contendo $B$, então $I\subseteq J$). O ideal $I$ é chamado de \emph{ideal gerado por $B$}, e denotado por $\langle B \rangle$.
    
    Se $B=\{x_0, \dots, x_n\}$, então abreviamos $\langle B \rangle$ como $\langle x_0, \dots, x_n \rangle$.
\end{prop}
\begin{proof}
    Primeiro, verificaremos que $I$ é um ideal.

    $0 \in I$, pois $0=0b$ para todo $b \in B$.

    Sejam $x, y \in I$. Então existem $n, m\geq 1$, $a_1, \dots, a_n \in A$, $b_1, \dots, b_n \in B$, $c_1, \dots, c_m \in A$ e $d_1, \dots, d_m \in B$ tais que $x=a_1b_1+\cdots+a_nb_n$ e $y=c_1d_1+\cdots+c_md_m$. Assim, $x+y=(a_1b_1+\cdots+a_nb_n)+(c_1d_1+\cdots+c_md_m)=(a_1b_1+\cdots+a_nb_n)+(c_1d_1+\cdots+c_md_m) \in I$.

    Finalmente, seja $a \in A$ e $b \in I$. Então existem $n\geq 1$, $a_1, \dots, a_n \in A$ e $b_1, \dots, b_n \in B$ tais que $b=a_1b_1+\cdots+a_nb_n$. Assim, $ab=(a_1b_1+\cdots+a_nb_n)a=a_1(b_1a)+\cdots+a_n(b_na) \in I$.

    Agora, seja $J$ um ideal de $A$ que contém $B$. Então, como $J$ é um ideal de $A$, temos que $\forall a_i\in A$, $\forall b_i\in B$, temos que $(a_i b_i)\in J$. Logo, $I\subseteq J$. Portanto, $I$ é o menor ideal de $A$ que contém $B$.
\end{proof}

Observação: note que o menor ideal contendo $B=\emptyset$ é o ideal nulo, $\{0\}$.

\begin{prop}[Ideal principal]
    Seja $A$ um anel. Para todo $x \in A$, o conjunto $xA=\{xa:a \in A\}$ é um ideal à direita de $A$. O ideal $xA$ é chamado de \emph{ideal principal à direita gerado por $x$}.
    Analogamente, o conjunto $Ax=\{ax:a \in A\}$ é um ideal à esquerda de $A$, e é chamado de \emph{ideal principal à esquerda gerado por $x$}.
    Se $A$ é comutativo, o ideal $xA=Ax$ é chamado de \emph{ideal principal gerado por $x$}.
\end{prop}
\begin{proof}
Mostraremos que $xA$ é um ideal à direita. As demais afirmações ficam como exercício.

Note que $0 \in xA$, pois $x0=0$.

Sejam $a, b \in xA$. Então, existem $a_1, a_2 \in A$ tais que $a=xa_1$ e $b=xa_2$. Assim, $a+b=xa_1+xa_2=x(a_1+a_2) \in xA$.

Finalmente, seja $a \in A$ e $b \in xA$. Então, existe $b_1 \in A$ tal que $b=xb_1$. Assim, $ab=(xa)b_1=x(ab_1) \in xA$.
\end{proof}
\begin{definition}[Ideal principal]
    Seja $A$ um anel. Para todo $x \in A$, o conjunto $xA=\{xa:a \in A\}$ é um ideal à esquerda de $A$. O ideal $xA$ é chamado de \emph{ideal principal à esquerda gerado por $x$}.
    Analogamente, o conjunto $Ax=\{ax:a \in A\}$ é um ideal à direita de $A$, e é chamado de \emph{ideal principal à direita gerado por $x$}.
    Se $A$ é comutativo, o ideal $xA=Ax$ é chamado de \emph{ideal principal gerado por $x$}.
\end{definition}

Observação: note que, comparando as definições, se $A$ é um anel comutativo com unidade, $xA=\langle x\rangle$.

Notemos que ideais triviais são principais à esquerda e à direita, pois $0A=\{0\}=A0$ e $A1=A=1A$.

\begin{definition}[Domínio de ideais principais]
    Um domínio de ideais principais (DIP), ou anel principal, é um domínio de integridade $A$ tal que todo ideal de $A$ é principal.
\end{definition}

\begin{prop}[Ideais de um corpo são triviais]
    Todo ideal de um corpo é trivial. Em particular, todo corpo é um DIP. Reciprocamente, se $A$ é um anel comutativo não trivial cujo todo ideal é trivial, então $A$ é um corpo.
\end{prop}
\begin{proof}
Seja $K$ um corpo e $I$ um ideal de $K$. Se $I=\{0\}$, então $I$ é trivial. Se $I\neq \{0\}$, então existe $a \in I$ tal que $a \neq 0$. Daí $1=a^{-1}a=\in I$. Logo, para todo $k \in K$, $k=1k\in I$.

Para a recíproca, seja $A$ um anel comutativo não trivial tal que todo ideal de $A$ é trivial, e fixe $x \in A\setminus \{0\}$. Como $Ax$ é um ideal trivial e $0\neq x \in Ax$, temos que $Ax=A$. Logo, existe $a \in A$ tal que $ax=1$. Assim, $x$ é invertível. Portanto, $A$ é um corpo.
\end{proof}

\begin{prop}[$\mathbb Z$ é um DIP que não é um corpo]
    Seja $I$ um ideal de $\mathbb Z$. Veremos que $I$ é um ideal principal. Se $I=\{0\}$, então $I$ é principal. Caso contrário, $I$ contém ao menos um elemento positivo, já que, sendo $x\in I\setminus\{0\}$, temos que $-x \in I$ e um dos $x, -x$ é positivo.


    Seja $n$ o menor inteiro positivo de $I$. Afirmamos que $I=n\mathbb Z$. De fato, se $x \in I$, então escreva $x=qn+r$, onde $q,r \in \mathbb Z$ e $0\leq r<n$. Como $x \in I$, temos que $r=x-qn \in I$. Assim, $r=0$, ou violaríamos a minimalidade de $n$. Logo, $x=qn\in n\mathbb Z$. Portanto, $I\subseteq n\mathbb Z$. Como $n\mathbb Z=\langle n\rangle$ e $n \in I$, temos que $n\mathbb Z\subseteq I$, o que completa a prova.
\end{prop}
\section{Subanéis}
\begin{definition}[Subanel]
    Seja $A$ um anel. Um subanel de $A$ é um conjunto $B\subseteq A$, com as operações de $A$ restritas à $B$ e com mesmo $0$ e $1$ de $A$, é um anel. Note que, para isso, é necessário e suficiente que:
    
    \begin{itemize}
        \item $1 \in B$
        \item $a-b \in B$ para todo $a, b \in B$.
        \item $ab \in B$ para todo $a, b \in B$.
    \end{itemize}

\end{definition}
\section{Quocientes e homomorfismos}
\subsection{Homomorfismos}
\begin{definition}
Sejam $A$, $R$ aneis. Uma função $f:A\rightarrow R$ é um \emph{homomorfismo} se:
\begin{itemize}
    \item $f(a+b)=f(a)+f(b)$ para todo $a, b \in A$.
    \item $f(-a)=-f(a)$ para todo $a \in A$.
    \item $f(0_A)=0_R$
    \item $f(ab)=f(a)f(b)$ para todo $a, b \in A$.
    \item $f(1_A)=1_R$.
\end{itemize}
\end{definition}

\begin{prop}[Propriedades de homomorfismos]
    Seja $f:A\rightarrow R$ um homomorfismo de anéis. Então:
    \begin{enumerate}[label=\alph*)]
        \item Para todo $a \in A^*$, temos $f(a) \in R^*$ e $f(a^{-1})=f(a)^{-1}$.
        \item $\ker f$
    \end{enumerate}
\end{prop}

\section{Quocientes e homomorfismos}
\subsection{homomorfismos}
\begin{definition}
Sejam $A$, $R$ aneis. Uma função $f:A\rightarrow R$ é um \emph{homomorfismo} se:
\begin{itemize}
    \item $f(a+b)=f(a)+f(b)$ para todo $a, b \in A$.
    \item $f(-a)=-f(a)$ para todo $a \in A$.
    \item $f(0_A)=0_R$
    \item $f(ab)=f(a)f(b)$ para todo $a, b \in A$.
    \item $f(1_A)=1_R$.
\end{itemize}

Caso $f$ seja injetora, dizemos que $f$ é um \emph{monomorfismo}. Caso $f$ seja sobrejetora, dizemos que $f$ é um \emph{epimorfismo}. Caso $f$ seja bijetora, dizemos que $f$ é um \emph{isomorfismo}.
\end{definition}

\begin{prop}[Propriedades de homomorfismos]
    Seja $f:A\rightarrow R$ um homomorfismo de anéis. Então:
    \begin{enumerate}[label=\alph*)]
        \item Para todo $a \in A^*$, temos $f(a)\in  R^*$ e $f(a^{-1})=f(a)^{-1}$. \label{prop:homomorfismo_a}
        \item O núcleo de $f$, definido como $\ker f=f^{-1}(\{0_R\})=\{a \in A: f(a)=0_R\}$, é um ideal de $A$.\label{prop:homomorfismo_b}
        \item A imagem de $f$, $\ran f=\{f(a): a \in A\}$, é um subanel de $R$. Se $A$ é comutativo, $\ran f$ também é.  \label{prop:homomorfismo_c}
        \item Se $f$ é injetora se, e somente se $\ker f=\{0_A\}$. \label{prop:homomorfismo_d}
    \end{enumerate}
\end{prop}
\begin{proof}
\ref{prop:homomorfismo_a} Se $a \in A^*$, então $f(a)f(a^{-1})=f(aa^{-1})=f(1_A)=1_R$ e $f(a^{-1})f(a)=f(aa^{-1})=f(1_A)=1_R$ . Assim, $f(a^{-1})=f(a)^{-1}$ e $f(a)\in R^*$.

\ref{prop:homomorfismo_b} Temos que $0_A \in \ker f$, pois $f(0_A)=0_R$. Sejam $a, b \in \ker f$. Então $f(a)=f(b)=0_R$, logo, $f(a+b)=f(a)+f(b)=0_R+0_R=0_R$. Assim, $a+b \in \ker f$.

Se $a \in \ker f$ e $x \in A$, vejamos que $ax, xa \in \ker f$: $f(ax)=f(a)f(x)=0_Rf(x)=0_R$ e $f(xa)=f(x)f(a)=f(x)0_R=0_R$. Assim, $ax, xa \in \ker f$.

Portanto, $\ker f$ é um ideal de $A$.

\ref{prop:homomorfismo_c} Seja $a, b \in \ran f$. Então existem $x, y \in A$ tais que $a=f(x)$ e $b=f(y)$. Assim, $a-b=f(x)-f(y)=f(x-y)$. Logo, $a-b \in \ran f$. Similarmente, $ab=f(x)f(y)=f(xy)\in \ran f$, e $1_R=f(1_A)\in \ran f$.

Portanto, $\ran f$ é um subanel de $R$. Se $A$ é comutativo, $\ran(f)$ também é comutativo, pois dados $a, b \in \ran f$, existem $x, y \in A$ tais que $a=f(x)$ e $b=f(y)$. Assim, $ab=f(x)f(y)=f(xy)=f(yx)=f(y)f(x)=ba$.

\ref{prop:homomorfismo_d} Se $f$ é injetora, então $f(a)=0_R=f(0_A)$ implica que $a=0_A$, logo, $\ker f=\{0_A\}$. Reciprocamente, se $\ker f=\{0_A\}$, então $f(a)=f(b)$ implica que $f(a-b)=0_R$, logo, $a-b=0_A$, ou seja, $a=b$. Assim, $f$ é injetora.
\end{proof}

\begin{prop}[Critério de homomorfismo]
    Seja $f:A\rightarrow R$ um homomorfismo de anéis. Então, $f$ é um homomorfismo se, e somente se:
    \begin{itemize}
        \item $f(a+b)=f(a)+f(b)$ para todo $a, b \in A$.
        \item $f(ab)=f(a)f(b)$ para todo $a, b \in A$.
        \item $f(1_A)=1_R$.
    \end{itemize}
\end{prop}
\begin{proof}
    Se $f$ é um homomorfismo, então as duas propriedades acima são satisfeitas. Reciprocamente, se as duas propriedades acima são satisfeitas, então:
    \begin{itemize}
    \item $f(0_A)=f(0_A+0_A)=f(0_A)+f(0_A)$. Cancelando, temos $0_R=f(0_A)$.
    \item $f(-a)+f(a)=f(0_A)=0_R$, logo, $f(-a)=-f(a)$.
    \end{itemize} $f(0_A)=f(0_A+0_A)=f(0_A)+f(0_A)=0_R$, e $f(-a)=f(0_A-a)=f(0_A)+f(-a)=0_R-f(a)=-f(a)$ para todo $a \in A$. Assim, $f$ é um homomorfismo.
\end{proof}

\subsection{Quocientes}
\begin{definition}
    Seja $A$ um anel. Uma relação de congruência em $A$ é uma relação de equivalência $\sim$ em $A$ que ``preserva operações''. Explcitamente, tal que para todos $a, b, c, d \in A$, se Se $a\sim b$ e $c\sim d$, então $a+c\sim b+d$ e $ac\sim bd$.
\end{definition}

Quais são todas as relações de congruência em $A$? A proposição abaixo classifica-as a partir dos ideais de $A$.
\begin{prop}[Relações de congruência vs ideais]
    Seja $A$ um anel, $\mathcal R(A)$ o conjunto de todas as relações de congruência em $A$ e $\mathcal I(A)$ o conjunto de todos os ideais de $A$. Então, existe uma bijeção entre $\mathcal R(A)$ e $\mathcal I(A)$ dada por
    $\sim \mapsto I_{\sim}=\{a \in A: a\sim 0\}$,
    cuja inversa se dá por $I\mapsto \sim_I=\{(a, b) \in A^2: a-b \in I\}$.
\end{prop}
\begin{proof}
Primeiro, vejamos que se $\sim$ é uma relação de congruência, então $I_\sim$ é um ideal de $A$.

\begin{itemize}
\item $0 \in I_\sim$, pois $0\sim 0$.
\item Se $a, b \in I_\sim$, então $a\sim 0$ e $b\sim 0$, logo $a+b\sim 0+0=0$, portanto, $a+b \in I_\sim$.
\item Se $x \in A$ e $a \in I_\sim$, então $a\sim 0$ e $x\sim 0$, logo $ax\sim a0=0$ e $xa=0a=0$, portanto, $ax, xa \in I_\sim$.
\end{itemize}

Agora, vejamos que se $I$ é um ideal, então $\sim_I$ é uma relação de congruência. De fato, temos que, para todos $a, b, c, d \in A$:
\begin{itemize}
    \item $a\sim_I a$ pois $a-a=0\in I$.
    \item Se $a\sim_I b$, então $a-b \in I$, logo $(-1)(a-b)=b-a\in I$, e, portanto, $b\sim_I a$.
    \item Se $a\sim_I b$ e $b\sim_I c$, então $a-b \in I$ e $b-c \in I$, logo, $(a-b)+(b-c)=a-c \in I$, portanto, $a\sim_I c$.
    \item Se $a\sim_I b$ e $c\sim_I d$, então $a-b \in I$ e $c-d \in I$, logo, $(a-b)+(c-d)=(a+c)-(b+d)\in I$, portanto, $a+c\sim_I b+d$.
    \item Se $a\sim_I b$ e $c\sim_I d$, então $a-b \in I$ e $c-d \in I$, logo, $(a-b)c=ac-bc\in I$ e $b(c-d)=bc-bd\in I$, logo $(ac-bc)+(bc-bd)=ac-bd\in I$, portanto, $ac\sim_I bd$.
    \end{itemize}

Se $I$ é ideal, $I_{\sim_I}=I$, pois, para todo $a\in A$:

$$a\in I_{\sim_I}\Leftrightarrow a\sim_I 0\Leftrightarrow a-0\in I\Leftrightarrow a\in I.$$

Finalmente, se $\sim$ é relação de congruência, $\sim_{I_\sim}=\sim$, pois, para todos $a, b \in A$:

$$a\sim_{I_\sim} b\Leftrightarrow a-b\in I_\sim \Leftrightarrow a-b\sim 0\Leftrightarrow a\sim b.$$

Justificando a última equivalência: se $a-b\sim 0$, como $b\sim b$, temos que $a-b+b\sim b$, ou seja, que $a\sim b$. Reciprocamente, se $a\sim b$, como $(-b)\sim (-b)$, segue que $a+(-b)\sim b+(-b)$, ou seja, que $a-b\sim 0$.
\end{proof}

Como feito nos inteiros, podemos, ao invés de trabalhar com relações de congruência, encontrar anéis em que a congruência corresponda exatamente à igualdade.

\begin{definition}
Seja $A$ um anel e $\sim$ uma relação de congruência. Define-se que $A/\sim$ é $A/\sim=\{[a]_\sim: a \in A\}$, onde $[a]_\sim=\{b\in A: b\sim a\}$ é a classe de equivalência de $a$ com relação a $\sim$.

Define-se que $[a]_\sim+[b]_\sim=[a+b]_\sim$ e que $[a]_\sim[b]_\sim=[ab]\sim$.

Se $I$ é um ideal, $A/I=A/\sim_I$, e o mapa quociente de $A$ em $A/I$ se dá por $q:A\longrightarrow A/I$ dada por $q(a)=[a]_{\sim_I}$.
\end{definition}

Pelas propriedades das relações de congruência, a soma e produto de $A/\sim$ (ou $A/I$) estão bem definidas. Além disso:

\begin{lemma}[Propriedades do quociente]
    Na notação acima:
    \begin{enumerate}[label=\alph*)]
        \item $q$ é epimorfismo de anéis. \label{lemma:propriedadesQuociente_a}
        \item $\ker q = I$. \label{lemma:propriedadesQuociente_b}
        \item $q(a)=a+I=\{a+x: x \in I\}$ para todo $a \in A$. \label{lemma:propriedadesQuociente_c}
        \item Se $A$ é anel comutativo, $A/I$ também é. \label{lemma:propriedadesQuociente_d}
    \end{enumerate}
\end{lemma}

\begin{proof}
    \ref{lemma:propriedadesQuociente_a} Seja $a, b, c, d \in A$. Temos que $q(a+b)=q(a)+q(b)$ e $q(ab)=q(a)q(b)$ por definição da soma em $A/I$, e $q$ é sobrejetora pela definição de $q$. Finalmente, $q(1_A)$ é identidade pois para todo $a \in A$, $q(1_A)q(a)=q(1_Aa)=q(a)$ e $q(a)q(1_A)=q(a1_A)=q(a)$, logo, $q(1_A)=1_{A/I}$.

    \ref{lemma:propriedadesQuociente_b} Temos que $\ker q=\{a \in A: q(a)=q(0)\}=\{a \in A: a\sim_I 0\}=\{a \in A: a\in I\}=I$.

    \ref{lemma:propriedadesQuociente_c} Temos que $q(a)=[a]_{\sim_I}=\{b \in A: b\sim_I a\}=\{b \in A: b-a\in I\}=\{a+x: x \in I\}$ pois se $b-a \in I$ se, e somente se $a-b=x$ para algum $x \in I$.

    \ref{lemma:propriedadesQuociente_d} Se $A$ é comutativo, então $A/I=\ran q$ também é, pois $q$ é homomorfismo de anéis.
\end{proof}

\subsection{Teoremas do isomorfismo}
\begin{theorem}[Teorema do homomorfismo]
    Seja $f:A\rightarrow R$ um homomorfismo de anéis e $J$ um ideal tal que $J\subseteq \ker f$. Então, existe um único homomorfismo de anéis $g:A/J\rightarrow R$ tal que $g\circ q=f$, onde $q:A\rightarrow A/J$ é o mapa quociente canônico dado por $q(a)=a+J$.
\end{theorem}
\begin{proof}
    Definimos $g:A/J\rightarrow R$ por $g(a+J)=f(a)$. Então, $g$ é bem definido, pois se $a+J=b+J$, então $a-b \in J\subseteq \ker f$, logo, $f(a-b)=0_R$, ou seja, $f(a)=f(b)$.

    Agora, vejamos que $g$ é um homomorfismo de anéis. De fato, para todo $a', b' \in A/J$, sendo $a'=a+J$ e $b'=b+J$, temos que:
    \begin{itemize}
        \item $g(a'+b')=g((a+J)+(b+J))=g((a+b)+J)=f(a+b)=f(a)+f(b)=g(a+J)+g(b+J)$.
        \item $g(a'b')=g((a+J)(b+J))=g(ab+J)=f(ab)=f(a)f(b)=g(a+J)g(b+J)$.
        \item $g(1_{A/J})=g(1_A+J)=f(1_A)=1_R$.
    \end{itemize}
\end{proof}

\begin{theorem}[Primeiro Teorema do Isomorfismo]
    Seja $f:A\rightarrow R$ um homomorfismo de anéis. Então, existe um único homomorfismo de anéis $g:A/\ker f\rightarrow R$ tal que $g\circ q=f$, onde $q:A\rightarrow A/J$ é o mapa quociente canônico dado por $q(a)=a+J$, e $g:A\ker f\rightarrow \ran f$ é isomorfismo.
\end{theorem}
\begin{proof}
    Pelo Teorema do homomorfismo com $J=\ker f$, existe um único homomorfismo de anéis $g:A/\ker f\rightarrow \ran f$ tal que $g\circ q=f$. Como $g\circ q=f$ e $q$ é sobrejetora, então $\ran g=\ran (g\circ q)=\ran f$, logo, $q$ é sobre $\ran f$.
    
    Resta ver que $g$ é injetora. De fato, seja $q(a)\in A/\ker f$ tal que $g(q(a))=0_R$. Então, $f(a)=0_R$, logo, $a\in \ker f=J$. ou seja, $a\sim_I 0$, logo $q(a)=q(0)=0_{A/\ker f}$. Assim, $g$ é injetora.
\end{proof}
\section{Produtos de anéis}


\begin{definition}[Produtos de anéis]
    Seja $(R_i)_{i \in I}$ uma família de anéis, onde cada $R_i$ tem as operações $+_i, \cdot_i$ e constantes $0_i, 1_i$.
    
    O produto (direto) de $(R_i)_{i \in I}$ é o conjunto $\prod_{i \in I} R_i$ munido das operações ``ponto à ponto'': dados $a=(a_i: i \in I), b=(b_i: i \in I)$ em $\prod_{i \in I}R_i$:

    $$a+b=(a_i: i \in I)+(b_i: i \in I)=(a_i+_i b_i: i \in I)=(a_i+_ib_i)_{i \in I}$$
    $$a\cdot b=(a_i: i \in I)\cdot (b_i: i \in I)=(a_i\cdot _i b_i: i \in I)=(a_i\cdot _ib_i)_{i \in I}$$

\end{definition}

\begin{lemma}[O produto de anéis está bem definido]
    Seja $(R_i)_{i \in I}$ uma família de anéis. Então seu produto direto $\prod_{i \in I}R_i$ é um anel com $0=(0_i: i \in I)$ e $1=(1_i: i \in I)$.
\end{lemma}

\begin{proof}
    Sejam $a=(a_i: i \in I), b=(b_i: i \in I)$ e $c=(c_i: i \in I)$ em $\prod_{i \in I}R_i$.
    \begin{itemize}
        \item \textbf{Associatividade da soma:} $(a+b)+c=(a_i+_i b_i)_{i \in I}+c=((a_i+_i b_i)+_ic_i)_{i \in I}=(a_i+_i (b_i+_i c_i))_{i \in I}=a+(b+c)$
        \item \textbf{Associatividade do produto:} Análogo.
        \item \textbf{Comutatividade da soma:} $a+b=(a_i+_i b_i)_{i \in I}=(b_i+_i a_i)_{i \in I}=b+a$
        \item \textbf{Neutro da soma:} $a+0=(a_i+_i 0_i)_{i \in I}=(a_i)_{i \in I}=a$
        \item \textbf{Inverso da soma:} Dado $a=(a_i)_{i \in I}$, considere $-a=(-a_i)_{i \in I}$. Então $a+(-a)=(a_i+_i (-a_i))_{i \in I}=(0_i)_{i \in I}=0$.
        \item \textbf{Distributividade:} $a\cdot (b+c)=(a_i\cdot _i (b_i+c_i))_{i \in I}=(a_i\cdot _i b_i+a_i\cdot _i c_i)_{i \in I}=a\cdot b+a\cdot c$.
        \item \textbf{Distributividade II:} $(a+b)\cdot c=((a_i+b_i)\cdot _i c_i)_{i \in I}=(a_i\cdot _i c_i+b_i\cdot _i c_i)_{i \in I}=a\cdot c+b\cdot c$.
        \item \textbf{Neutro do produto:} $a\cdot 1=(a_i\cdot _i 1_i)_{i \in I}=(a_i)_{i \in I}=a$ e $1\cdot a=(1_i\cdot _i a_i)_{i \in I}=(a_i)_{i \in I}=a$.
    \end{itemize}
\end{proof}
\begin{definition}[Os mapas de projeção]
    Seja $(R_i)_{i \in I}$ uma família de anéis e seja $R=\prod_{i \in I}R_i$. Para cada $i \in I$, o mapa de projeção $\pi_i:R\rightarrow R_i$ é dado por $\pi_i(a)=a_i$.

    Escrevendo de outra forma, $\pi_i((a_j: j \in I))=a_i$.
\end{definition}

\begin{lemma}[Os mapas de projeção são homomorfismos]
    Seja $(R_i)_{i \in I}$ uma família de anéis e seja $R=\prod_{i \in I}R_i$. Para cada $i \in I$, o mapa de projeção $\pi_i:R\rightarrow R_i$ é um homomorfismo de anéis.
\end{lemma}
\begin{proof}
    Sejam $a=(a_j: j \in I), b=(b_j: j \in I)$ em $R$. Então:
    \begin{itemize}
        \item $\pi_i(a+b)=\pi_i((a_j+b_j)_{j \in I})=a_i+b_i=\pi_i(a)+\pi_i(b)$
        \item $\pi_i(a\cdot b)=\pi_i((a_j\cdot b_j)_{j \in I})=a_i\cdot b_i=\pi_i(a)\cdot \pi_i(b)$
        \item $\pi_i(1_R)=\pi_i((1_j)_{j \in I})=1_{i}$
    \end{itemize}
\end{proof}
Notação: se $R, S$ são anéis, o produto direto de $(R, S)$ é denotado também como $R\times S$. Assim, se $(r, s), (r', s')\in R\times S$ e 

\subsection{A propriedade universal do produto direto de anéis}
\begin{theorem}[Propriedade universal do produto direto de anéis]
    Seja $(R_i)_{i \in I}$ uma família de anéis e seja $R=\prod_{i \in I}R_i$ seu produto direto. Então, para cada anel $S$ e cada família de homomorfismos de anéis $f_i:R_i\rightarrow S$, existe um único homomorfismo de anéis $f:R\rightarrow S$ tal que $\pi_i\circ f=f_i$ para todo $i \in I$.
    \begin{figure}[H]
        \centering
    \begin{tikzcd}[column sep=1.5cm,row sep=1.2cm]
        & S\arrow[ld, "f_i"']\arrow[d, dashed, "\exists! f"]\\
        R_i  & \arrow[l, "\pi_i"]R\\
    \end{tikzcd}
    \end{figure}

    Além disso, se $R'$ e $(p_i:R'\rightarrow R)_{i \in I}$ é um anel e uma família de homomorfismos de anéis, respectivamente, tal que para cada anel $S$ e cada família de homomorfismos de anéis $f_i:R_i\rightarrow S$, existe um único homomorfismo de anéis $f:R'\rightarrow S$ tal que $p_i\circ f=f_i$ para todo $i \in I$., então existe um único isomorfismo de anéis $\phi: R\rightarrow R'$ tal que $p_i\circ \phi=\pi_i$ para todo $i \in I$.
\end{theorem}

\begin{proof}
    Seja $R=\prod_{i \in I}R_i$ e seja $S$ um anel comutativo. Para cada $i \in I$, considere $f_i:S\rightarrow R_i$ um homomorfismo de anéis. Defina $f:S\rightarrow R$ tal que, dado $s \in S$:
    $$f(s)=(f_i(s))_{i \in I}.$$

    Então, para cada $i \in I$, $\pi_i\circ f(s)=\pi_i(f_j(s): j \in I)=f_i(s)$, ou seja, $\pi_i\circ f=f_i$.
    Vejamos que $f$ é homomorfismo de anéis. Dados $s, t \in S$, temos:
    \begin{itemize}
        \item $f(s+t)=(f_i(s+t))_{i \in I}=(f_i(s)+f_i(t))_{i \in I}=(f_i(s))_{i \in I}+(f_i(t))_{i \in I}=f(s)+f(t)$.
        \item $f(s\cdot t)=(f_i(s\cdot t))_{i \in I}=(f_i(s)\cdot f_i(t))_{i \in I}=(f_i(s))_{i \in I}\cdot (f_i(t))_{i \in I}=f(s)\cdot f(t)$.
        \item $f(1_S)=(f_i(1_S))_{i \in I}=(1_i)_{i \in I}=1_R$.  
    \end{itemize}
    Vejamos que $f$ é único. Se $g:R\rightarrow S$ é um homomorfismo de anéis tal que $\pi_i\circ g=f_i$, então, para cada $s \in S$, temos, que, para cada $i \in I$:
    $$\pi_i(g(s))=f_i(s).$$

    Assim: $$g(s)=(\pi_i(g(s)): i \in I)=(f_i(s): i \in I)=f(s).$$
    Portanto, $g=f$.

    Agora suponha que $R'$ e $(p_i:R'\rightarrow R)_{i \in I}$ são como no enunciado.

    Aplicando a propriedade de $R'$ para $(\pi_i: i \in I)$, existe um homomorfismo de anéis $\phi: R'\rightarrow R$ tal que $p_i\circ \phi=f_i$ para todo $i \in I$.

    \begin{figure}[H]
        \centering
    \begin{tikzcd}[column sep=1.5cm,row sep=1.2cm]
        & R\arrow[ld, "\pi_i"']\arrow[d, dashed, "\exists! \phi"]\\
        R_i  & \arrow[l, "p_i"]R'\\
    \end{tikzcd}
    \end{figure}

    Nosso objetivo é mostrar que $\phi$ é isomorfismo. Construiremos uma inversa.

    Aplicando a propriedade de $R$ para $(\pi_i: i \in I)$, existe um homomorfismo de anéis $\psi: R'\rightarrow R$ tal que $\pi_i\circ \psi=p_i$ para todo $i \in I$.  

    \begin{figure}[H]
        \centering
    \begin{tikzcd}[column sep=1.5cm,row sep=1.2cm]
        & R'\arrow[ld, "p_i"']\arrow[d, dashed, "\exists! \psi"]\\
        R_i  & \arrow[l, "\pi_i"]R\\
    \end{tikzcd}
    \end{figure}

    Tanto os mapas $\phi\circ \psi$ quanto a identidade $\id_{R'}:R'\rightarrow R'$ são homomorfismos de anéis que satisfazem o seguinte diagrama:

    \begin{figure}[H]
        \centering
    \begin{tikzcd}[column sep=1.5cm,row sep=1.2cm]
        & R'\arrow[ld, "p_i"']\arrow[d, "\id_{R'}",  "\phi\circ\psi"']\\
        R_i  & \arrow[l, "p_i"]R'\\
    \end{tikzcd}
    \end{figure}

    Pois $p_i\circ \id_{R'}=p_i$ e $p_i\circ \phi\circ\psi=\pi_i\circ \psi=p_i$.
    Assim, pela unicidade do homomorfismo de anéis, $\phi\circ \psi=\id_{R'}$.

    Analogamente, tanto os mapas $\psi\circ \phi$ quanto a identidade $\id_{R}:R\rightarrow R$ são homomorfismos de anéis que satisfazem o seguinte diagrama:
    \begin{figure}[H]
        \centering
    \begin{tikzcd}[column sep=1.5cm,row sep=1.2cm]
        & R\arrow[ld, "\pi_i"']\arrow[d, "\id_{R}",  "\psi\circ\phi"']\\
        R_i  & \arrow[l, "\pi_i"]R\\
    \end{tikzcd}
    \end{figure}

    Pois $\pi_i\circ \id_{R}=\pi_i$ e $\pi_i\circ \psi\circ\phi=p_i\circ \phi=\pi$.
    Assim, pela unicidade do homomorfismo de anéis, $\psi\circ \phi=\id_{R}$. Assim, $\psi$ e $\phi$ são isomorfismos inversos.

    A unicidade de $\phi$ como isomorfismo vem de sua unicidade como homomorfismo.
\end{proof}

\section{Polinômios}

\subsection{Séries Formais}
Se $R$ é um anel comutativo, intuitivamente uma série formal é um objeto que se escreve na forma:
$$a_0+a_1x+a_2x^2+\dots=\sum_{i=0}^\infty a_ix^i$$
onde $a_i\in R$.

Que propriedades gostaríamos que esse objeto tivesse?

\begin{itemize}
    \item \textbf{Igualdade:} igualdade de objetos desse tipo fosse determinada por uma condição de igualdade entre os coeficientes. Ou seja, que:
    \begin{align*}
      \sum_{i=0}^\infty a_ix^i=\sum_{i=0}^\infty b_ix^i & \Leftrightarrow \forall i\in \mathbb N\, a_i=b_i.
    \end{align*}
    \item \textbf{Soma:} que a soma de dois objetos desse tipo fosse dada por:
    \begin{align*}
        \left(\sum_{i=0}^\infty a_ix^i\right)+\left(\sum_{i=0}^\infty b_ix^i\right)=\left(\sum_{i=0}^\infty (a_i+bi)x^i\right)
    \end{align*}
    \item \textbf{Produto:} que o produto de dois objetos desse tipo fosse dada por:
    \begin{align*}
        \left(\sum_{i=0}^\infty a_ix^i\right)\cdot\left(\sum_{i=0}^\infty b_ix^i\right)=\sum_{i=0}^\infty \left(\sum_{j=0}^i a_{j}b_{i-j}\right)x^i
    \end{align*}
    \item \textbf{Preservação:} que as operações do anel sejam preservadas.
    \item Notação: $R\llbracket x \rrbracket$ é o conjunto de todas as séries formais em $R$.
\end{itemize}

\begin{definition}
Seja $R$ um anel comutativo. Definiremos $R\llbracket x\rrbracket=R^{\mathbb N}$.


Um elemento de $R\llbracket x\rrbracket$ é da forma $p=(p_0, p_1, \dots)=(p_n)_{n \in \mathbb N}=(p(n))_{n \in \mathbb N}=(p(0), p(1), \dots)$ onde $p_i=p(i)\in R$  para todo $i \in \mathbb N$.

O suporte de $p\in R\llbracket x\rrbracket$ é o conjunto $\supp(p)=\{i \in \mathbb N: p_i \neq 0\}$.
O grau de $p \in R[x]$ é o maior elemento de $\supp(p)$, denotado por $\deg(p)$. Se $p=0$, então $\deg(p)=-\infty$

\textbf{Intuição}: $p=(a_0, a_1, \dots)$ corresponderá à $a_0+a_1x+\dots+a_n x^n+\dots$.

\textbf{Operações}:
Se $p, q \in R[x]$, define-se:
$$p+q=(p_0+q_0, q_1+p_1, \dots)=(p_i+q_i)_{i \in \mathbb N}\in R^{\mathbb N}$$
$$p\cdot q=(p_0q_0, p_1q_0+p_0q_1, p2q_0+p_1q_1+p_0q_2, \dots)=\left(\sum_{j=0}^i p_{i-j}q_j\right)_{i \in \mathbb N}$$
$$1_{R\llbracket x\rrbracket}=(1, 0, 0, \dots)$$
$$0_{R\llbracket x\rrbracket}=(0, 0, 0, \dots)$$
\end{definition}

\begin{lemma}[Séries formais formam anéis]
    Se $R$ é um anel comutativo, então $R\llbracket x \rrbracket$ é um anel comutativo.
\end{lemma}

\begin{proof}
    A operação de soma de $\mathbb R\llbracket x \rrbracket$ é a mesma de $\mathbb R^{\mathbb N}$, que já verificamos satisfazer as propriedades de grupo Abeliano. Assim, $R\llbracket x \rrbracket$ é um grupo abeliano sob a soma.
    \begin{itemize}
        \item \textbf{Distributividade:} $$p\cdot (q+r)=p\cdot(q_i+r_i)_{i \in \mathbb N}=\left(\sum_{j=0}^i p_{i-j}(q_j+r_j)\right)_{i \in \mathbb N}$$$$=\left(\sum_{j=0}^i p_{i-j}q_j\right)_{i \in \mathbb N}+\left(\sum_{j=0}^i p_{i-j}r_j\right)_{i \in \mathbb N}=p\cdot q+p\cdot r.$$
        \item \textbf{Elemento Neutro:} $$p\cdot 1=\left(\sum_{j=0}^i p_{i-j}\delta_{0j}\right)_{i\in \mathbb N}=(p_i)_{i \in \mathbb N}.$$
        \item \textbf{Comutatividade:} A $i$-ésima coordenada de $p\cdot q$ é $\sum_{j=0}^i p_{i-j}q_j=\sum(p_{i_j}q_j: j \in A_j)$, onde $A_j=\{0, \dots, i\}$. A função $\phi: A_j:\rightarrow A_j$ dada por $\phi(j)=i-j$ é bijetora, pois é injetora e $A_j$ é finito. Assim:
        $$\sum_{j=0}^i p_{i-j}q_j=\sum_{j=0}^ip_{i-\phi(j)}q_{\phi(j)}=\sum_{j=0}^ip_{j}q_{i-j}=\sum_{j=0}^iq_{i-j}p_{j}$$

        E esta é a $i$-ésima coordenada de $q\cdot p$.

        \item \textbf{Associatividade:} Temos que a $i$-ésima coordenada de $(p\cdot q)\cdot r$ é dada por:

        $$\pi_i((p\cdot q)\cdot r)=\sum_{j=0}^i\pi_{i-j}(p\cdot q)\cdot q_j=\sum_{j=0}^i\left(\sum_{k=0}^{i-j}p_{i-j-k}q_k\right)\cdot q_j$$
        $$=\sum_{j=0}^{i}\sum_{k=0}^{i-j}p_{i-j-k}q_kq_j=\sum\left(p_{i-j-k}q_kr_j:(j, k)\in A\right)$$

        Onde $A=\{(j, k): 0\leq j\leq i, 0\leq k\leq i-j\}$.

        Temos que a $i$-ésima coordenada de $p\cdot (q\cdot r)$ é dada por:

        $$\pi_i(p\cdot (q\cdot r))=\sum_{s=0}^ip_{i-s}\pi_s(q\cdot r)=\sum_{s=0}^ip_{i-s}\left(\sum_{t=0}^sq_{s-t}r_t\right)$$
        $$=\sum_{s=0}^i\sum_{t=0}^sp_{i-s}q_{s-t}r_t=\sum\left(q_{i-s}q_{s-t}r_t:(s, t)\in B\right)$$
        
        onde $B=\{(s, t): 0\leq t\leq s\leq i\}$. A função $\phi: A\rightarrow B$ dada por $\phi(j, k)=(j+k, j)$ é bijetora: é em $B$, pois $0\leq j\leq j+k\leq j+(i-j)=i$. É injetora, pois se $(j+k, j)=(j'+k', j')$ então $j=j'$ e, cancelando, $k=k'$. Finalmente, é sobrejetora, pois se $0\leq t\leq s\leq i$, sendo $j=t$ e $k=s-t$, temos que $0\leq j\leq i$, $0\leq k=s-t\leq i-t=i-j$ e $j+k=s$. Assim, $\phi$ é bijetora. Portanto:

        $$\sum\left(q_{i-s}q_{s-t}r_t:(s, t)\in B\right)=\sum\left(q_{i-(j+k)}q_{(j+k)-j}r_j:(j, k)\in A\right)$$$$=\sum\left(q_{i-j-k}q_{k}r_j:(j, k)\in A\right).$$
    \end{itemize}
\end{proof}

Dado $p \in \mathbb R_x$, utilizamos a notação $$p=p(x)=\sum_{i=0}^\infty p_ix^i.$$

É importante observar que não há, de fato, uma ``soma infinita'' acontecendo aqui. É possível dar sentido à essa soma infinita utilizando a teoria de limites diretos de anéis, mas não faremos isso aqui. Por ora, isso é apenas uma notação especial para tratar desses objetos. Note que, ao menos por enquanto, a letra $x$ é apenas parte da notação, e que não faz sentido, por enquanto, ``substituir $x$'' por algo.
\subsection{Anéis de Polinômios}
Na subseção anterior, introduzimos o anel das séries formais de um anel comutativo dado. Vimos que tal anel é um anel comutativo.

Deste anel, podemos extrair o anel de polinômios.

\begin{definition}
Seja $R$ um anel comutativo. O anel de polinômios $R[x]$ é o subconjunto de $R\llbracket x \rrbracket$ dado por:
$$R[x]=\{p \in R\llbracket x \rrbracket: \deg(p) < \infty\}$$
\end{definition}

Note que uma série formal tem grau $<\infty$ se, e somente se, todos os coeficientes, a partir de algum ponto, são nulos.

\begin{lemma}
    Seja $R$ um anel comutativo. O anel de polinômios $R[x]$ é um subanel de $R\llbracket x \rrbracket$. Mais especificamente, dados $p(x), q(x) \in R\llbracket x \rrbracket$, temos que, se $\gr(p(x)) < \infty$ e $\gr(q(x)) < \infty$, então:

    \begin{enumerate}[label=\alph*)]
        \item $\gr p(x)q(x)\leq\gr p(x)+\gr q(x)$ caso ambos sejam não nulos, e a igualdade vale se $R$ for um domínio de integridade.
        \item $\gr p(x)+q(x)\leq \max\{\gr p(x),\gr(q(x))\}$.
    \end{enumerate}
\end{lemma}

\begin{proof}
    Para a primeira afirmação, sejam $n, m$ os graus de $p(x)$ e $q(x)$, respectivamente.
    
    Calculemos o coeficiente $n+m$ de $p(x)q(x)$. 

    $$\pi_{n+m}(p(x)q(x))=\sum_{j=0}^{n+m}p_{n+m-j}q_j.$$

    Se $0\leq j< m$ temos que $n+m-j>0$, e $p_{n+m-j}=0$. Se $j>m$, temos que $q_j=0$. Assim, o único termo não nulo da soma é quando $j=m$, e temos que $p_{n+m-m}q_m=p_nq_m$, que é não nulo se $R$ for um domínio. Por outro lado, se $l>n+m$ temos que:

    $$\pi_{l}(p(x)q(x))=\sum_{j=0}^{l}p_{l-j}q_j.$$

    Se $0\leq j\leq m$ temos que $l-j>m+n-m=n$, e $p_{l-j}=0$. Se $j>m$, temos que $q_j=0$. Assim, todos os coeficientes da soma são $0$.


    Para a segunda afirmação, se $l>\max\{\gr p(x), \gr q(x)\}$, temos que o $l$-ésimo coeficiente de $p(x)+q(x)$ é $0$, pois este é $p_l+q_l=0+0$.

    Agora, para a afirmação principal, as duas afirmações itemizadas nos mostram que $R[x]$ é fechado pela soma e produto de $R\llbracket x \rrbracket$. Finalmente, note que o grau da série $1=(1, 0, 0, \dots)$ é $0$, logo, $1\in R[x]$.
\end{proof}

Agora vamos trabalhar um pouco mais nossa notação.

\begin{definition}
    Seja $R$ um anel comutativo. Em $R[x]$, seja $x=(0, 1, 0, 0, \dots)$ e, para cada $r \in R$, seja $\hat r=(\hat r, 0, 0, \dots)$.
\end{definition}

\begin{lemma}
    Na notação anterior, para todo $r \in R$ e $n, i\geq 0$:

    $$\pi_i(\hat r x^n)(i)=\begin{cases}
        0 & \text{se } i\neq n\\
        r & \text{se } i=n.
    \end{cases}$$
    Ou seja, $\hat r x^n=(0, 0, \dots, r, 0, \dots)$ onde o $r$ está na posição $n$.
\end{lemma}

\begin{proof}
    Fixe $r$ Seguimos por indução. Para $n=0$, temos que $\hat rx^0=\hat r=(r, 0, 0, \dots)$ e para $n=1$ temos que $\hat r x^1=x=(0, r, 0, \dots)$.

    Para o passo $n+1$, onde $n\geq 1$, temos que, sendo $i\geq 1$:

    $$\pi_i(\hat rx^{n+1})=\pi_i((\hat rx^n)\cdot x)=\sum_{j=0}^i\pi_{i-j}(\hat r x^n)\cdot \pi_j(x)=\pi_{i-1}^{\hat r x^n}$$

    Assim, se $i=n+1$, temos que a coordenada é $r$, e $0$ caso contrário. Resta apenas verificar que a coordenada $0$ é $0$. Ora, a coordenada $0$ se dá por $\pi_0(\hat r x^n)\pi_0(x)=0$.
\end{proof}

\begin{lemma}
    Na notação anterior, seja $h:R\rightarrow R[x]$ dada por $h(r)=\hat r$. Então $h$ é um homomorfismo injetor.
\end{lemma}
    
\begin{proof}
    Sejam $r, s \in R$. Então:
    \begin{itemize}
        \item $h(r+s)=(r+s, 0, 0, \dots)=\hat r+\hat s=h(r)+h(s)$
        \item $h(rs)=(rs, 0, 0, \dots)=\hat r\cdot \hat s=h(r)\cdot h(s)$
        \item $h(1_R)=\hat 1=(1_R, 0, 0, \dots)=1_{R[x]}=1_{R[x]}$.
    \end{itemize}

    A injetividade é óbvia.
\end{proof}

\begin{prop}
    Na notação anterior, para todo $p(x)\in R[x]$, existem $n\geq 0$ e $r_0, r_1, \dots, r_n\in R$ tais que $p(x)=\sum_{i=0}^n \hat r_ix^i$.
\end{prop}
    
\begin{proof}
    Se $p(x)=0$, seja $n=0$ e $r_0=0$. Caso contrário, seja $n=\gr p(x)$ e $r_i=p_i$ para todo $i\leq n$. Então:

    $$p(x)=(p_0, \dots, p_n, 0, \dots, 0)=(r_0, \dots, r_n, 0, \dots, 0)=\sum_{i=0}^n\hat r_ix^i$$
\end{proof}

\begin{prop}
    Na notação anterior, se $r_1, \dots, r_n$ e $s_1, \dots, s_n$ são elementos de $R$, então:

    $\sum_{i=0}^n \hat r_i x^i=\sum_{i=0}^n \hat s_i x^i$ se, e somente se, $r_i=s_i$ para todo $i\leq n$.
\end{prop}
    
\begin{proof}
A recíproca é imediata. Para a implicação direta, note que a igualdade nos diz que $(r_0, \dots, r_n, 0, 0, \dots)=(s_0, \dots, s_n, 0, 0, \dots)$, ou seja, $r_i=s_i$ para todo $i\leq n$.
\end{proof}

Notação: abandona-se $\hat r$ em favor de $r$, mesmo havendo ambiguidade de notação.
\end{document} 