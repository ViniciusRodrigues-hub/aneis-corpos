\documentclass{book}
\usepackage[brazil]{babel}
\usepackage{amsthm, amsmath, amssymb}
\usepackage[a4paper]{geometry}
\usepackage{enumitem}
\usepackage{stmaryrd}
\usepackage{tikz-cd}
\usepackage{indentfirst}
\usepackage{hyperref}
\usetikzlibrary{babel}
\usepackage{float}
\theoremstyle{definition}

\newtheorem{definition}{Definição}[section]
\AtEndEnvironment{definition}{\null\hfill\qedsymbol}%

\newtheorem{prop}[definition]{Proposição}
\newtheorem{corol}[definition]{Corolário}
\newtheorem{exer}[definition]{Exercício}
\newtheorem{lemma}[definition]{Lema}
\newtheorem{theorem}[definition]{Teorema}
\newtheorem{exemplo}[definition]{Exemplo}
\AtEndEnvironment{eexemplo}{\null\hfill\qedsymbol}%


\DeclareMathOperator{\ran}{ran}
\DeclareMathOperator{\supp}{supp}
\DeclareMathOperator{\gr}{gr}
\DeclareMathOperator{\id}{id}
\author{Prof. Vinicius Rodrigues}
\title{Notas da disciplina MAT0264 - Anéis e Corpos}
\begin{document}
\frontmatter
\begin{titlepage}
    \maketitle
\end{titlepage}
\tableofcontents
\chapter{Agradecimentos}

O autor agradece às seguintes pessoas que contribuíram com a elaboração deste material:
\begin{itemize}
\item \textbf{Gabriel Alves Andretta} -- Aluno do Bacharelado em Matemática do IME-USP. Foi monitor da disciplina ``Anéis e Corpos'' em 2025 e avisou sobre erros de digitação no texto.
\item \textbf{Renan Ribeiro Marcelino} -- Aluno do Bacharelado em Matemática do IME-USP. Colaborou com algumas correções de erros de digitação no texto a partir do GitHub.
\item \textbf{Ugo Bruzzo} -- Professor do Departamento de Matemática do IME-USP. Lecionou o primeiro terço dessa disciplina em 2025, e indicou uma porção considerável dos exercícios aqui expostos.
\end{itemize}

\chapter{Prefácio}

Estas notas começaram a ser escritas durante o primeiro semestre de 2025, enquanto lecionava a disciplina MAT0264 - Anéis e Corpos, no Instituto de Matemática e Estatística da Universidade de São Paulo (IME-USP).
No presente estado, elas estão em um formato de rascunho, e não são um material completo, nem revisado. O objetivo é que, ao longo do semestre, as notas sejam revisadas e completadas, de modo a se tornarem um material didático mais completo e acessível aos alunos da disciplina.

É assumido que o estudante já tem algum traquejo ao lidar com números inteiros e aritmética modular, tendo já estudado, formalmente, divisibilidade de inteiros, congruência módulo $n$ e os anéis $\mathbb Z_n$.
Será assumida a existência do anel dos números inteiros.
Ao longo do texto, apresentaremos as construções de todos os outros anéis relevantes, porém alguns outros anéis importantes e conhecidos, como $\mathbb Q$, $\mathbb R$ e $\mathbb C$, com o qual se espera que o estudante já possua alguma familiaridade, serão utilizados em exemplos desde seu início, mesmo antes que construções formais sejam apresentadas.

Ao final de cada seção serão apresentados exercícios. Recomenda-se que o estudante resolva-os para fixar o conteúdo apresentado.
\mainmatter

\chapter{Pré-Requisitos Conjuntistas}
Durante o texto, precisamos de algumas definições e resultados envolvendo noções básicas sobre conjuntos e funções.

Não é objetivo deste texto desenvolver a parte inicial da Teoria dos Conjuntos. Também não é o objetivo desta seção explicar toda a notação de conjuntos utilizada. Assumimos familiaridade do leitor com funções e com manipulação de conjuntos a nível básico. Apenas apresentaremos algumas definições, notações e resultados básicos que utilizaremos ao longo do texto.

\section{Produtos cartesianos de conjuntos}

Famílias são funções com notação especial. Muitas vezes, ao pensar em funções, pensamos em um ``dispositivo de entrada/saída''. Quando, ao invés disso, estamos pensando apenas em um  ``conjunto indexado de valores'', a notação de família pode ser mais conveniente.

No quadro abaixo, apresentamos uma comparação entre as duas notações. Enfatizamos que, matemáticamente, funções e famílias podem ser vistas como o mesmo objeto.
\begin{table}[h]
    \centering
    \begin{tabular}{lllll}
        \hline
        \textbf{Conceito} & \textbf{Função} & \textbf{Família} \\ \hline
        Mapa & $u:I\rightarrow A$ & $(u_i)_{i \in I}=(u_i: i \in I)$ \\
        Valor & $u(i)$ & $u_i$ \\
        Imagem & $\ran u$ & $\{u_i: i \in I\}$\\
        Intuição & objeto dinâmico & objeto estático \\
        Inputs & domínio $I$ & conjunto de índices $I$ \\
        \hline
    \end{tabular}
    \caption{Comparativo de família e função}
\end{table}

Exemplo: sequências. Uma sequência é uma família cujo conjunto de índices é $\mathbb N$. Compare a intuição que passa as notações:
\begin{itemize}
\item Considere a sequência $u=(\frac{1}{2^n}))_{n \in \mathbb N}$...
\item Considere a função $u:\mathbb N\rightarrow \mathbb R$ dada por $u(n)=\frac{1}{2^n}$...
\end{itemize}

Exemplo: sequências finitas. Se $n\geq 1$, identificamos $n=\{0, 1, \dots, n-1\}$. Assim:
\begin{itemize}
\item Uma família com $n$ elementos é uma família $(a_i)_{i<n}=(a_i)_{i \in n}=(a_0, \dots, a_{n-1})$.
\end{itemize}

\begin{definition}[Produto cartesiano de conjuntos]
Seja $(A_i)_{i \in I}$ uma família de conjuntos. O produto cartesiano de conjuntos é o conjunto $\prod_{i \in I} A_i$ definido como o conjunto de todas as famílias $(a_i: i \in I)$ tais que para cada $i \in I$, $a_i \in A_i$.
$$\prod_{i \in I} A_i=\{(a_i)_{i \in I}: \forall i \in I\, a_i \in A_i\}.$$
\end{definition}


\begin{definition}[Exponenciação de conjuntos]
    Sejam $A, I$ conjuntos. O conjunto $A^I$ é o conjunto de todas as funções de $I$ em $A$. Ou seja, $A^I=\{f:I\rightarrow A\}$. Note que:

    $$A^I=\prod_{i \in I}A=\{(a_i)_{i \in I}: \forall i \in I\,  a_i\in A\}.$$
    \end{definition}

    Na notação anterior, se $n\geq 1$ $$A^n=\{(a_i)_{i<n}:\forall i<n\, a_i \in A\}=\{(a_0, \dots, a_{n-1}):a_0, \dots, a_{n-1}\in A\}\approx A\times \dots \times A \,(n \text{ vezes}).$$

    \section{Operações}

\begin{definition}[Operações $n$-árias]
    Se $X$ é um conjunto e $n \in \mathbb N$, uma operação $n$-ária em $X$ é uma função $f:X^n\rightarrow X$.
\end{definition}

Operações $2$-árias e $1$-árias são frequentemente chamadas de \emph{binárias} e \emph{unárias}, respectivamente.

Caso $*$ seja uma operação binária, a notação $x*y$ é frequentemente utilizada para denotar $x*y$.

Caso $*$ seja uma operação unária, a notação $*x$ é frequentemente utilizada para denotar $*(x)$.

\chapter{Noções de Grupos}


\section{Definição e Propriedades Básicas}

Grupos são estruturas matemáticas munidas de uma operação binária com algumas propriedades espeiciais.
O principal objetivo deste texto é servir como texto para um estudo introdutório sobre anéis e corpos, que são estruturas matemáticas que possuem duas operações binárias com propriedades especiais.
Conforme veremos no Capítulo 3, todo anel e todo corpo, com uma dessas operações, forma um grupo.
Assim, é útil, para o estudo de anéis e corpos, o conhecimento de noções básicas sobre grupos.

Apesar das noções de anel e de corpo serem, a nível de definição, noções mais complexas que a de grupo, a noção de grupo, em parte por ser menos restritiva, necessita o desenvolvimento de ferramentas específicas para seu estudo completo.
A área do conhecimento matemático resultante do desenvolvimento dessa teoria é extremamente rica, e chamada de \emph{Teoria dos Grupos}.
Nosso objetivo, por outro lado, é focar no estudo inicial das teorias de anéis e corpos, e, portanto, não mergulharemos nesta importante área.

Assim, não é objetivo deste capítulo apresentar uma introdução ao estudo de grupos, mas sim apenas introduzir as noções e resultados básicos próprios de grupos que são estritamente necessários para os resultados envolvendo anéis e corpos descritos no restante do texto.

\begin{definition}
Um grupo é uma quadrupla $(G,\cdot,e)$, tal que $G$ é um conjunto, $\cdot$ é uma operação binária em $G$ e $0 \in G$, e satisfazem:

\begin{itemize}
    \item (\textbf{Propriedade associativa}) $\forall a, b, c \in G$ $(a \cdot b) \cdot c = a \cdot (b \cdot c)$.
    \item (\textbf{Elemento neutro}) $\forall a \in G$  $e \cdot a = a \cdot e = a$.
    \item (\textbf{Elemento inverso}) $\forall a \in G$ $\exists b \in G$ $a \cdot b = b \cdot a = e$.
\end{itemize}
Se, adicionalmente, a seguinte propriedade é satisfeita, o grupo é chamado de \emph{comutativo}, ou, mais comunmente, \emph{Abeliano}:
\begin{itemize}
    \item (\textbf{Comutatividade}) $\forall a, b \in G\, a \cdot b = b \cdot a$.
\end{itemize}
\end{definition}

Alguns exemplos:

\begin{exemplo}Abaixo, exemplificamos alguns grupos importantes.
    \begin{enumerate}[label=\alph*)]
        \item Com a soma usual e $0$, $\mathbb{Z, Q, R, C}$ são grupos Abelianos.
        \item Com a multiplicação usual, o círculo unitário complexo $\mathbb T=\{x \in \mathbb C: |x|=1\}$ é um grupo Abeliano com elemento neutro $1$.
        De fato, o produto de complexos é comutativo, associativo e tem $1$ como elemento neutro.
        Note que $1\in \mathbb T$ e $0\notin \mathbb T$.
        Se $x \in \mathbb T$, o inverso multiplicativo de $x$ é dado por $\frac{\bar x}{|x|^2}$, onde $\bar x$ denota o conjugado de $x$.
        Como $|\bar x|=|x|=1$, segue que $\mathbb T$ tem todos os inversos de todos seus elementos.
        \item Os inteiros módulo $n$ ($n\geq 1$), dados por $\mathbb Z_n=\{0, \dots, n-1\}$ com a soma dada pela aritmética módulo $n$, são grupos.
        \item Se $X$ é um conjunto qualquer, o conjunto das bijeções de $X$ em $X$ é, com a composição usual de funções e a identidade, um grupo, cuja operação inversa é a inversão usual de funções.
        Tal grupo é denominado \emph{grupo de permutações de $X$}.
        
        Caso $X$ tenha ao menos $3$ elementos, ele não é abeliano: sendo $a, b, c$ três elementos distintos de $X$, sendo $f$ a função que permuta $a, b$ e fixa os demais elementos, $g$ a que permuta $b, c$ temos que $f\circ g(a)=f(a)=b$, mas $g\circ f(a)=g(b)=c$, logo, $f\circ g\neq g\circ f$.
    \end{enumerate}
\end{exemplo}

Algumas observações importantes sobre a notação utilizada no estudo de grupos:
\begin{itemize}
\item Ao discursar sobre grupos, é comum omitir a operação e o elemento neutro, referindo-se apenas ao conjunto $G$, conforme fizemos acima ao mencionar que $\mathbb Z$ é um grupo.
O mais formal, porém muito menos usual, feito principalmente em situações em há chance de confusão, é escrever que, por exemplo, $(\mathbb Z, +, 0)$ é um grupo.
\item Como também ocorre com $\mathbb Z$, caso o grupo seja Abeliano, é comum que sua operação binária seja denotada por $+$ ou outro símbolo similar.
Nesse contexto, o elemento neutro é frequentemente denotado por $0$.
\item Caso o grupo em discurso não seja necessariamente Abeliano, é comum que sua operação binária seja denotada por $\cdot$ ou outro símbolo similar.
Nesse contexto, o elemento neutro é frequentemente denotado por $e$, e a operação é frequentemente omitida, ou seja, $a \cdot b$ é frequentemente escrito como $ab$.

Porém, há grupos Abelianos cujas operações também são denotadas por $\cdot$, como no caso o grupo $\mathbb T$ mencionado acima.
\end{itemize}


Agora iniciaremos a provar algumas propriedades básicas sobre grupos.
\begin{prop}[Unicidade do elemento neutro]\label{prop:group_uniqueNeutral}
    Seja $(G,\cdot,e)$ um grupo.
    Então, o elemento neutro $e$ é único.
    Isto é, se $h \in G$ é tal que $\forall a \in G$ $h \cdot a = a \cdot h = a$, então $h = e$.
\end{prop}
\begin{proof}
    Note que $h=he$, pois $e$ é elemento neutro.
    Por outro lado, $e=he$, pois $h$ é elemento neutro.
    Assim, $h=he=e$.
\end{proof}

\begin{prop}[Unicidade dos inversos]\label{prop:group_uniqueInverse}
    Seja $(G,\cdot,e)$ um grupo.
    Então todo $a \in G$ possui um único elemento inverso.
Isto é $\forall a \in G$ $\exists!\, b \in G$ $a \cdot b = b \cdot a = e$.
\end{prop}
\begin{proof}
    A existência do inverso é garantida pela definição de grupo.
    
    Para provar a unicidade, suponha que $b, c$ são inversos de $a$, ou seja, que $a \cdot b = b \cdot a = e$ e $a \cdot c = c \cdot a = e$.
    Segue que:
    $$b=be=b(ac)=(ba)c=ec=c.$$
\end{proof}

A unicidade dos inversos nos permite definir a notação $a^{-1}$ para o inverso de $a$ em um grupo $(G,\cdot,e)$.
Caso $(G, +, 0)$ seja um grupo Abeliano, a notação $-a$ é frequentemente utilizada para denotar o inverso de $a$, e, nesse caso, $-a$ é chamado de \emph{oposto} de $a$.

Note que assim, ficam definidos operadores unários $(\,)^{-1}:G\rightarrow G$ (ou $-:G\rightarrow G$).
Para o segundo caso, define-se também que $a-b=a+(-b)$.

\begin{prop}[Cancelamento]\label{prop:group_cancel}
    Seja $(G,\cdot,e)$ um grupo e $a,b,c \in G$.
    Se $a \cdot b = a \cdot c$, então $b=c$.
    Analogamente, se $b \cdot a = c \cdot a$, então $b=c$.
\end{prop}
\begin{proof}
    Provaremos a primeira afirmação.
    A segunda é análoga e fica como exercício.
    Suponha que $ba=ca$.
    Segue que $(ba)a^{-1}=(ca)a^{-1}$.
    
    Pela propriedade associativa, $b(aa^{-1})=c(aa^{-1})$.

    Pela definição de inverso, segue que $be=ce$.

    Pela neutralidade de $e$, segue que $b=c$.
\end{proof}

\begin{corol}[Cancelamento II]
    Seja $(G,\cdot,e)$ um grupo.
    Para todos $a, b \in G$, se $ab=a$, então $b=e$.
Analogamente, se $ba=a$, então $b=e$.
\end{corol}
\begin{proof}
    Para a primeira afirmação, note que $ab=ae$, logo, pela proposição anterior, $b=e$.
    A segunda afirmação é análoga.
\end{proof}

\begin{prop}[Regras de sinal]\label{prop:regraSinal}
    Seja $G$ um grupo e $a, b \in G$.
    Então:
    \begin{enumerate}[label=\alph*)]
        \item $((a)^{-1})^{-1}=a$ [na notação aditiva, $-(-a)=a$].
\label{prop:regraSinal_A}
        \item $(ab)^{-1}=b^{-1}a^{-1}$ [na notação aditiva, $-(a+b)=(-b)+(-a)]$.\label{prop:regraSinal_B}
        \item $e^{-1}=e$ [na notação aditiva, $-0=0$].\label{prop:regraSinal_C}
    \end{enumerate}
\end{prop}
\begin{proof}
    \ref{prop:regraSinal_A}: Temos que $(a^{-1})^{-1}a^{-1}=e=aa^{-1}$.
    Cancelando $a^{-1}$, segue.
    
    \ref{prop:regraSinal_B}: Temos que $(ab)^{-1}(ab)=e=(b^{-1}a^{-1})ab$.
    Cancelando $ab$, segue que $(ab)^{-1}=b^{-1}a^{-1}$.
    Analogamente, $(ba)^{-1}=a^{-1}b^{-1}$.

    \ref{prop:regraSinal_C}: Temos que $(e^{-1})e=e=ee$.
    Cancelando $e$ à direita, segue.


\end{proof}

\section{Somatórios}

Nessa seção, formalizaremos a noção de somatório.
É desejável que o leitor já possua familiaridade com alguma notação de somatório, mas aqui apresentaremos a notação e as técnicas de ``substituição de variáveis'' que serão utilizadas.

\begin{definition}[Soma de sequência finita]
Seja $G$ um conjunto munido de uma operação $+$ associativa, comutativa e com neutro $0$.
Define-se, recursivamente para $n\geq 0$, o somatório de famílias $(a_i: i \in F)$, onde $F$ é um conjunto de $n$ índices e $a_i \in G$ para todo $i \in F$, como se segue:

\begin{itemize}
    \item \textbf{Notação:} se $a=(a_i)_{i\in F}$ é uma sequência de elementos de $G$, então usamos as notações:
    \[\sum a=\sum(a_i: i\in F)=\sum_{i\in F} a_i.\]
    \item Caso base $n=0$ (soma vazia): só existe uma família com $0$ elementos, que é a família vazia $a=()=\emptyset=(a_i:i\in \emptyset)$.
    Definimos: \[\sum a=\sum_{i \in \emptyset}a_i=0\].
    \item Passo recursivo $n\rightarrow n+1$: considere uma família $(a_i)_{i\in F}$, onde $|F|=n+1$.
    Define-se:
    \[\sum(a_i: i \in F)=\sum(a_i: i \in F\setminus\{j\})+a_j,\]
    onde $j \in I$ é qualquer elemento.
\end{itemize}
\end{definition}
É claro que, para mostrar que a definição acima é consistente, precisamos mostrar que a soma não depende da escolha de $j$.

\begin{lemma}
Qualquer que seja o tamanho (finito) de $F$, $\sum(a_i)_{i\in F}$ está bem definido.
\end{lemma}

\begin{proof}
    Seja $F$ um conjunto finito.
Se $|F|=0$, então $F=\emptyset$, e a soma é $0$.
Se $|F|=1$, então $F=\{j\}$ -- só há uma escolha para $j$, e a soma é $a_j$.
    Se $|F|=n+1$ para $n\geq 1$, tome $j, k \in F$.
    Devemos ver que $\left(\sum_{i\in F\setminus\{j\}} a_i\right)+a_j=\left(\sum_{i\in F\setminus\{k\}} a_i\right)+a_k$.
    Com efeito:

    \[\left(\sum_{i\in F\setminus\{j\}} a_i\right)+a_j=\left(\left(\sum_{i\in F\setminus\{j, k\}} a_i\right)+a_k\right)+a_j=\left(\sum_{i\in F\setminus\{j, k\}} a_i\right)+(a_k+a_j)\]

    \[=\left(\sum_{i\in F\setminus\{j, k\}} a_i\right)+(a_j+a_k)=\left(\left(\sum_{i\in F\setminus\{j, k\}} a_i\right)+a_j\right)+a_k=\left(\sum_{i\in F\setminus\{k\}} a_i\right)+a_k.\]
\end{proof}

\begin{prop}
    Seja $G$ um conjunto munido de uma operação $+$ associativa, comutativa e com neutro $0$.
Seja $(a_i: i \in I)$ uma família finita em $G$ e $\phi:J\rightarrow I$ uma função bijetora.
Então:

    \[\sum_{i \in I}a_i=\sum_{j \in J}a_{\phi(j)}.\]

\end{prop}
\begin{proof}
Novamente, procedemos por indução no tamanho de $n=|I|$.
A base de tamanho $0$ é trivial, já que ambos os lados da igualdade são $0$.

Para o passo indutivo em que $|I|=|J|=n+1$, considere $\phi:J\rightarrow I$ como no enunciado.
Fixe $k \in J$ qualquer e sejam $I'=I\setminus\{\phi(k)\}, J'=J\setminus\{k\}$ e $\phi'=\phi|_{J'}:J'\rightarrow I'$, que é bijetora.
Como $|J'|=|I'|=n$, por hipótese indutiva temos que $\sum_{j \in J'}a_{\phi(j)}=\sum_{i \in I'}a_i$.
Segue que:

\[\sum_{j \in J}a_{\phi(j)}=\left(\sum_{j \in J'}a_{\phi(j)}\right)+a_{\phi(k)}=\left(\sum_{i \in I'}a_{i}\right)+a_{\phi(k)}=\sum_{j \in I}a_{i}.\]
\end{proof}

\section{Exercícios}
\begin{exer}
    Suponha que $a$, $b$ e $c$ sejam elementos de um anel $A$, e que $a$ não é divisor de $0$.
    
    Mostre que se $ab = ac$, então ou $a = 0$ ou $b = c$ (isto é, se $a\neq 0$, podemos cancela-lo).
\end{exer}
\chapter{Anéis e subanéis}
Nesta seção, iniciaremos o estudo dos anéis e de estruturas relacionadas. Apresentaremos as definições dessas estruturas e suas propriedades mais elementares.

\section{A definição de anel}
No Capítulo 2, conhecemos, por alto, a definição de grupo.
Um grupo é um conjunto munido de uma operação binária que satisfaz algumas propriedades.
Grupos pode ser Abelianos ou não Abelianos, e, quando é Abeliano, lembra-nos da adição de inteiros.

Porém, estuturas como as dos números inteiros, racionais e reais parecem não ter sua estrutura algébrica completamente capturada pela noção de grupo Abeliano, pois possuem também outra operação binária -- a multiplicação.
Esta operação se relaciona com a soma através das propriedades distributivas. A noção de anel surge para capturar parte destas ideias, generalizando o estudo das estruturas mencionadas.
\begin{definition}[Anel]
    Um anel é uma $5$-upla $(A, +, \cdot, 0, 1)$ conjunto $A$ com duas operações binárias, adição e multiplicação, denotadas por $+$ e $\cdot$, tais que:
    \begin{itemize}
        \item $(A, +, 0)$ é um grupo abeliano.
        \item (\textbf{Associatividade}) Para todo $a, b \in A$, temos $(a \cdot b)\cdot c = a\cdot(b\cdot c)$.
        \item (\textbf{Elemento identidade}) $\forall a \in A$ $1 \cdot a = a \cdot 1 = a$.
        \item (\textbf{Propriedades distributivas}) Para todos $a, b, c, \in A$, temos:
        \begin{align*}
            a \cdot (b + c) &= a \cdot b + a \cdot c, \text{ e}\\
            (a + b) \cdot c &= a \cdot c + b \cdot c
        \end{align*}
    \end{itemize}
    Se, adicionalmente, a seguinte propriedade é satisfeita, o anel é chamado de \emph{comutativo}.
    \begin{itemize}
        \item (\textbf{Comutatividade}) $\forall a, b \in A$ $a \cdot b = b \cdot a$.
    \end{itemize}
\end{definition}

Algumas observações:
\begin{itemize}
    \item Como em grupos, ao discursar sobre anéis é comum omitir as operações, referindo-se apenas ao conjunto $A$.
    \item Ao discursar sobre anéis, e a exemplo do que foi feito ao enunciar as propriedades distributivas, são utilizadas as convenções usuais sobre precedência de operações envolvidas por parênteses.
    Assim, $a + b \cdot c$ é interpretado como $a + (b \cdot c)$.
    \item Há textos que definem anéis sem incluir o elemento identidade $1$.
    Nestes textos, a definição acima dá nome ao que chamam de \emph{anéis com identidade}, ou \emph{anéis com 1}.
    Nesse curso, não usaremos essa convenção, de modo que \textbf{todos nossos anéis possuem identidade}.
    De modo similar, alguns textos definem anéis como sendo comutativos. Também não adotaremos essa convenção.
    \textbf{Os nossos anéis podem ser não comutativos}.
    \item A definição de anel não exige que $0=1$.
    \item $0$ é chamado de elemento nulo, e $1$ de elemento identidade.
\end{itemize}

\begin{prop}[Propriedade multiplicativa do $0$]
    Seja $A$ um anel.
    Então $\forall a \in A$ $0 \cdot a = a \cdot 0 = 0$.
\end{prop}
\begin{proof}
Provaremos a primeira afirmação.
A segunda é análoga e fica como exercício.

Temos que $0\cdot a=(0+0)\cdot a=0\cdot a +0\cdot a$.
Cancelando, segue que $0=0\cdot a$.
\end{proof}

\begin{prop}[Anel trivial]
    Seja $A={x}$ um conjunto qualquer.
    Defina $x\cdot x=x=x+x=0=1$.
    Então $(A, +, \cdot, 0, 1)$ é um anel.
    Um anel dessa forma é chamado de \emph{anel trivial}.
    
    Além disso, se $A$ é um anel tal que $0=1$, então $A$ é um anel trivial.
\end{prop}
\begin{proof}
    A primeira afirmação (de que $A$ como acima é um anel) fica como exercício.

    Para a segunda afirmação, assuma que $A$ é um anel tal que $0=1$.
    Fixe $a \in A$ qualquer.
    Então $a=a\cdot1=a\cdot0=0$, ou seja, $a=0$.
    Assim, $A$ é o conjunto unitário $\{0\}$, que é um anel trivial.
\end{proof}

Todo anel satisfaz as conhecidas regras de sinais referentes à multiplicação e adição, como:
\begin{prop}[Regras de sinal II]\label{prop:regraSinal2}
    Seja $A$ um anel e $a, b \in A$. Então:
    \begin{enumerate}[label=\alph*)]
        \item $(-a)b=a(-b)=-(ab)$\label{prop:regraSinal2_A}
        \item $(-a)(-b)=ab$.\label{prop:regraSinal2_B}
        \item $(-1)a=-a$.\label{prop:regraSinal2_C}
    \end{enumerate}
\end{prop}
\begin{proof}
    \ref{prop:regraSinal2_A}: Temos que $ab+(-a)b=(-a)b+ab=[-a+a]b=0b=0$. Assim, $(-a)b=-(ab)$. Analogamente, $a(-b)=-(ab)$.

    \ref{prop:regraSinal2_B}: Temos que $(-a)(-b)=-[a(-b)]=-[-(ab)]=ab$ pela regra anterior.

    \ref{prop:regraSinal2_C}: Temos que $(-1)a=-(1a)=-a$.
\end{proof}
\section{Anéis de Matrizes}
Matrizes são objetos muito importantes na matemática, sendo amplamente utilizadas na Álgebra Linear.

Nesta seção, construiremos os anéis de matrizes com coeficientes em um anel arbitrário.

\begin{definition}
    Seja $A$ um anel e $n, m$ inteiros positivos. O conjunto $M_{n\times m}(A)$ é o conjunto de matrizes $n\times m$ cujos coeficientes estão em $A$. Formalmente, $M_{n\times m}$ é o conjunto de todas as famílias $(a_{ij})_{i,j}=(a_{ij}: (i, j) \in \{1, \dots, n\}\times \{1, \dots, m\})$. Quando conveniente, representamos a tal matriz de qualquer uma das duas formas a seguir:
\begin{multicols}{2}
    \centering
    \begin{equation*}
        \begin{pmatrix}
            a_{11} & \cdots & a_{1m}\\
            \vdots & \ddots & \vdots\\
            a_{n1} & \cdots & a_{nm}
        \end{pmatrix}
    \end{equation*}

    \begin{equation*}
        \begin{bmatrix}
            a_{11} & \cdots & a_{1m}\\
            \vdots & \ddots & \vdots\\
            a_{n1} & \cdots & a_{nm}
        \end{bmatrix}
    \end{equation*}
\end{multicols}

Se $(a_{i, j})_{i,j}$ e $(b_{i, j})_{i,j}$ são matrizes $n\times m$ em $M_{n\times m}(A)$, definimos sua \emph{soma} como $(a_{i, j})+(b_{i, j})_{i,j}=(a_{i, j}+b_{i, j})_{i,j}$.
Em outra notação:

\begin{equation*}
    \begin{pmatrix}
        a_{11} & \cdots & a_{1m}\\
        \vdots & \ddots & \vdots\\
        a_{n1} & \cdots & a_{nm}
    \end{pmatrix}
    +
    \begin{pmatrix}
        b_{11} & \cdots & b_{1m}\\
        \vdots & \ddots & \vdots\\
        b_{n1} & \cdots & b_{nm}
    \end{pmatrix}
    =
    \begin{pmatrix}
        a_{11}+b_{11} & \cdots & a_{1m}+b_{1m}\\
        \vdots & \ddots & \vdots\\
        a_{n1}+b_{n1} & \cdots & a_{nm}+b_{nm}
    \end{pmatrix}
\end{equation*}

Se $(a_{ij})_{i, j}\in M_{n\times m}(A)$ e $(b_{ij})_{i, j}\in M_{m\times p}(A)$, definimos o produto de matrizes como $(a_{ij})_{i,j}\cdot(b_{ij})_{i,j}=(c_{ij})_{i,j}$, onde $c_{ij}=\sum_{k=1}^m a_{ik}b_{kj}$.
Em outra notação:
\begin{equation*}
    \begin{pmatrix}
        a_{11} & \cdots & a_{1m}\\
        \vdots & \ddots & \vdots\\
        a_{n1} & \cdots & a_{nm}
    \end{pmatrix}
    \cdot
    \begin{pmatrix}
        b_{11} & \cdots & b_{1p}\\
        \vdots & \ddots & \vdots\\
        b_{m1} & \cdots & b_{mp}
    \end{pmatrix}
    =
    \begin{pmatrix}
        c_{11} & \cdots & c_{1p}\\
        \vdots & \ddots & \vdots\\
        c_{n1} & \cdots & c_{np}
    \end{pmatrix}
\end{equation*}

A matriz nula de $M_{n\times m}(A)$ é a matriz cuja todas as entradas são $0\in A$, e é denotada por $0_{n\times m}$, ou, simplesmente, $0$.

Caso $n=m$, abreviamos $M_{n\times n}(A)$ como $M_n(A)$.
\end{definition}

Sobre a aditividade, independente de $m, n$, sempre temos um grupo Abeliano:

\begin{prop}
Seja $A$ um anel e $n, m \in \mathbb{N}$. Então, o conjunto $M_{n \times m}(A)$, munido da operação de soma de matrizes, é um grupo abeliano.
\end{prop}

\begin{proof}
    Sejam $(a_{ij})_{i,j}, (b_{ij})_{i,j}, (c_{ij})_{i,j} \in M_{n \times m}(A)$. Mostraremos que $(M_{n \times m}(A), +)$ satisfaz as propriedades de um grupo abeliano:
    
    \begin{enumerate}
        \item \textbf{Fechamento:} Para todos $(a_{ij})_{i,j}, (b_{ij})_{i,j} \in M_{n \times m}(A)$, temos que $(a_{ij} + b_{ij})_{i,j} \in M_{n \times m}(A)$, pois $A$ é fechado sob adição.
    
        \item \textbf{Associatividade:} para todos $(a_{ij})_{i,j}, (b_{ij})_{i,j}, (c_{ij})_{i,j} \in M_{n \times m}(A)$, temos:
        \[
        \big((a_{ij}) + (b_{ij})\big) + (c_{ij}) = (a_{ij} + b_{ij}) + c_{ij} = a_{ij} + (b_{ij} + c_{ij}) = (a_{ij}) + \big((b_{ij}) + (c_{ij})\big).
        \]
    
        \item \textbf{Elemento neutro:} A matriz nula é o elemento neutro.
        Com efeito, dado $(a_{ij})_{i,j} \in M_{n \times m}(A)$, temos:
        \[
        (a_{ij}) + 0_{m\times n} = (a_{ij} + 0) = (a_{ij}).
        \]
    
        \item \textbf{Elemento inverso:} Para cada $(a_{ij})_{i,j} \in M_{n \times m}(A)$, a matriz $(-a_{ij})_{i,j}$, é oposto aditivo, pois:
        \[
        (a_{ij}) + (-a_{ij}) = (a_{ij} +(-a_{ij})) = 0
        \]
    
        \item \textbf{Comutatividade:} A soma de matrizes é comutativa, pois, para todos $(a_{ij})_{i,j}, (b_{ij})_{i,j} \in M_{n \times m}(A)$, temos:
        \[
        (a_{ij}) + (b_{ij}) = (a_{ij} + b_{ij}) = (b_{ij} + a_{ij}) = (b_{ij}) + (a_{ij}).
        \]
    \end{enumerate}
    
    Portanto, $(M_{n \times m}(A), +)$ é um grupo abeliano.
\end{proof}

A multiplicação de matrizes é associativa e distributiva sobre a soma. Formalmente:

\begin{prop}
Seja $A$ um anel e $n, m, p, q \geq 1$. Então:
\begin{enumerate}[label=\alph*)]
\item (\textbf{Associatividade}) Para todos $(a_{ij})_{i,j} \in M_{n \times m}(A)$, $(b_{jk})_{j,k} \in M_{m \times p}(A)$ e $(c_{kl})_{k,l} \in M_{p \times q}(A)$, temos:
\[
\big((a_{ij}) \cdot (b_{jk})\big) \cdot (c_{kl}) = (a_{ij}) \cdot \big((b_{jk}) \cdot (c_{kl})\big).
\]

\item (\textbf{Distributividade}) Para todos $(a_{ij})_{i,j} \in M_{n \times m}(A)$, $(b_{jk})_{j,k}, (c_{jk})_{j,k} \in M_{m \times p}(A)$, temos:
\[
(a_{ij}) \cdot\left((b_{jk}) + (c_{jk})\right)
= (a_{ij}) \cdot (b_{jk}) + (a_{ij}) \cdot (c_{jk}).
\]
E, para todos $(a_{ij})_{i,j}, (b_{ij})_{i,j} \in M_{n \times m}(A)$ e $(c_{jk})_{j,k} \in M_{m \times p}(A)$, temos:
\[
\left((a_{ij}) + (b_{ij})\right) \cdot (c_{jk})
= (a_{ij}) \cdot (c_{jk}) + (b_{ij}) \cdot (c_{jk}).
\]
\end{enumerate}
\end{prop}
\begin{proof}
    \textbf{a)} Sejam $(a_{ij})_{i,j} \in M_{n \times m}(A)$, $(b_{jk})_{j,k} \in M_{m \times p}(A)$ e $(c_{kl})_{k,l} \in M_{p \times q}(A)$. Considere o elemento $(i,l)$ da matriz resultante de $\big((a_{ij}) \cdot (b_{jk})\big) \cdot (c_{kl})$.
    Pela propriedade distributiva, temos:
    \[
    \sum_{k=1}^p \left( \sum_{j=1}^m a_{ij} b_{jk} \right) c_{kl}
    =\sum_{k=1}^p \left( \sum_{j=1}^m a_{ij} b_{jk}c_{kl} \right) .
    \]
    Comutando os somatórios e novamente pela propriedade distributiva, isso é:
    \[
    \sum_{j=1}^m\left( \sum_{k=1}^p  a_{ij} b_{jk} c_{kl} \right),
    =\sum_{j=1}^m a_{ij} \left( \sum_{k=1}^p b_{jk} c_{kl} \right),
    \]
    que é exatamente o elemento $(i,l)$ da matriz $(a_{ij}) \cdot \big((b_{jk}) \cdot (c_{kl})\big)$. Assim, a associatividade é satisfeita.

    \textbf{b)} Para a distributividade, considere $(a_{ij})_{i,j} \in M_{n \times m}(A)$, $(b_{jk})_{j,k}, (c_{jk})_{j,k} \in M_{m \times p}(A)$. O elemento $(i,k)$ da matriz resultante de $(a_{ij}) \cdot \big((b_{jk}) + (c_{jk})\big)$ é dado por:
    \[
    \sum_{j=1}^m a_{ij} (b_{jk} + c_{jk})
    =\sum_{j=1}^m (a_{ij}b_{jk} + a_{ij}c_{jk})
    =\sum_{j=1}^m a_{ij}b_{jk} + \sum_{j=1}^ma_{ij}c_{jk}
    \]

    Isso corresponde ao elemento $(i,k)$ da matriz $(a_{ij}) \cdot (b_{jk}) + (a_{ij}) \cdot (c_{jk})$. A outra distributividade é provada de forma análoga.
\end{proof}

Como o produto de uma matriz de $M_{n\times m}(A)$ com uma matriz de $M_{m\times p}(A)$ é uma matriz de $M_{n\times p}(A)$, em geral, não há uma propriedade de fechamento para o produto de matrizes.

Lembremos que a matriz identidade de $M_{n\times n}(A)$ é a matriz cujos elementos da diagonal principal são $1$ e os demais são $0$.
Utilizando a notação do delta de Kronecker, em que $\delta_{ij}$ é $1$ caso $i=j$ e $0$ caso contrário, a matriz identidade é a matriz $I_n=(\delta_{ij})_{i,j}\in M_{n}(A)$.

Porém, tal fato acontece para matrizes quadradas. De fato, temos:

\begin{prop}[Anéis de matrizes]
    Seja $A$ um anel e $n\geq 1$. Com as operações de soma e multiplicação definidas acima, e com a identidade $I_n$ como a matriz identidade de $M_{n}(A)$, o conjunto $M_{n}(A)$ é um anel, denominado \emph{anel das matrizes $n\times n$ de $A$}.

    Se $n\geq 2$ e $A$ é um anel não trivial, $M_n(A)$ não é comutativo.
\end{prop}

\begin{proof}
Para a verificação das propriedades de anel, resta apenas ver que a matriz identidade $I_n$ é uma identidade multiplicativa.
Com efeito, dado $(a_{ij})_{i,j} \in M_n(A)$, temos:
\begin{align*}
    (a_{ij}) \cdot I_n &= \left( \sum_{k=1}^n a_{ik} \delta_{kj} \right)_{i,j} \\
    &= \left( a_{ij} \right)_{i,j},
\end{align*}

e:

\begin{align*}
    I_n \cdot (a_{ij}) &= \left( \sum_{k=1}^n \delta_{ik} a_{kj} \right)_{i,j} \\
    &= \left( a_{ij} \right)_{i,j}.
\end{align*}

Para a última afirmação, considere $(a_{ij})_{i,j}, (b_{ij})_{i,j} \in M_n(A)$ definidos por:
\begin{multicols}{2}
\centering
\[a_{ij}=\begin{cases}
    1 & \text{se } i=j=1\\
    0 & \text{caso contrário}
\end{cases}\]

\[b_{ij}=\begin{cases}
    1 & \text{se } i=1, j=n\\
    0 & \text{caso contrário}
\end{cases}\]
\end{multicols}

Temos que o elemento $(1,n)$ da matriz $(a_{ij})(b_{ij})$ é dado por $\sum_{k=1}^n a_{1k}b_{kn}=1$, enquanto o elemento $(1,n)$ da matriz $(b_{ij})(a_{ij})$ é dado por $\sum_{k=1}^n b_{1k}a_{kn}=1$.
\end{proof}

Assim, os anéis de matrizes nos dão uma ampla gama de anéis não comutativos.
\section{Domínios de integridade e divisores de zero}
O anel dos números inteiros, bem como o anel dos racionais reais, possuem a seguinte importante propriedade:
\begin{definition}
Seja $A$ um anel comutativo. Dizemos que $A$ é um \emph{domínio de integridade} se, e somente se, $\forall a, b \in A$, se $ab=0$, então $a=0$ ou $b=0$.
\end{definition}

Nem todos os anéis comutativos são domínios de integridade. Por exemplo, no anel dos inteiros módulo $4$, $\mathbb Z_4$, temos que $2\cdot 2=4=0$, e $2\neq 0$.

Divisores de zero são elementos não nulos que, multiplicados entre si, resultam em zero.

\begin{definition}
Sejam $A$ um anel. Um divisor de zero de $A$ é um elemento $a \in A$ não nulo para o qual exista $b \in A$ não nulo tal que $ab=0$ ou $ba=0$.
\end{definition}

Note que um domínio de integridade é um anel comutativo sem divisores de zero.

Divisores de zero são patológicos ao estudar a teoria de divisibilidade em anéis, assim, muitas vezes, eles são excluídos de tal estudo.

\section{Elementos invertíveis}
Um anel, com sua soma, é um grupo Abeliano, e, portanto, possui opostos aditivos. Porém, não necessita possuir opostos multiplicativos. Os elementos de um anel que possuem inversos no anel são os chamados \emph{elementos invertíveis} ou \emph{unidades}.
\begin{definition}[Elemento invertível]
    Seja $A$ um anel.
    Um elemento $a \in A$ é dito \emph{invertível}, ou uma \emph{unidade} se $\exists b \in A$ tal que $a \cdot b = b \cdot a = 1$.
    
    O conjunto de todas das unidades de $A$ é denotado por $A^*$.
\end{definition}

\begin{definition}
    Seja $A$ um anel.
    Então, se $a \in A^*$, existe um \textbf{único} $b \in A$ tal que $a \cdot b = b \cdot a = 1$. Este elemento é denotado por $a^{-1}$, e é chamado de \emph{inverso} de $a$.
\end{definition}

Observação: para que a definição acima faça sentido, é necessário mostrar que se $a$ é unidade, existe um \textbf{único} $b \in A$ tal que $a \cdot b = b \cdot a = 1$.
A existência é garantida pela definição de unidade, e a demonstração da unicidade é análoga à da unicidade do inverso em grupos (Proposição \ref{prop:group_uniqueInverse}, ficando como exercício.

\begin{prop}
Seja $A$ um anel. Para todos $a, b \in A^*$, temos:
\begin{enumerate}[label=\alph*)]
    \item $ab\in A^U$ e $(ab)^{-1}=b^{-1}a^{-1}$.\label{prop:unidadeProduto_a}
    \item $a^{-1}\in A^U$ e $(a^{-1})^{-1}=a$.\label{prop:unidadeProduto_b}
    \item $1^{-1}=1$.\label{prop:unidadeProduto_c}
\end{enumerate}
Além disso, $A^*$ é, com a restrição da operação de multiplicação do anel, um grupo com identidade $1$. Caso $A$ é um anel comutativo, $A^*$ é um grupo abeliano.
\end{prop}
\begin{proof}
    \ref{prop:unidadeProduto_a}: Sejam $a, b \in A^*$. Pela associatividade, $(ab)(b^{-1}a^{-1})=1=(b^{-1}a^{-1})(ab)$, logo, pela unicidade do inverso, $(ab)^{-1}=b^{-1}a^{-1}$.

    \ref{prop:unidadeProduto_b}: Seja $a \in A^*$. Temos que $a^{-1}a=1=a(a^{-1})$, logo, pela unicidade do inverso, $(a^{-1})^{-1}=a$.

    \ref{prop:unidadeProduto_c}: Note que $1\cdot 1=1=1\cdot 1$, logo, pela unicidade do inverso, $1^{-1}=1$.

    Se $A$ é um anel comutativo, então $A^*$ é um grupo abeliano, pois para todo $a, b \in A^*$, temos que $ab=ba$, logo $(ab)^{-1}=b^{-1}a^{-1}=a^{-1}b^{-1}$.
\end{proof}

\section{O anel dos números inteiros}
Espera-se que o estudante já possua traquejo com o anel dos números inteiros, incluindo contato com a noção formal de divisibilidade, o teorema fundamental da aritmética e a noção de congruência módulo $n$.

Primeiramente, reconheçamos que $\mathbb Z$ possui, além da estrutura de domínio de integridade, uma estrutura de ordem.

\begin{definition}
    Um anel ordenado é uma tupla $(A, +, \cdot, 0, 1, \leq)$ tal que $(A, +, \cdot, 0, 1)$ é um anel comutativo tal que $\leq$ é uma relação de ordem total (também chamada de ordem linear) em $A$, ou seja, que satisfaça:

    \begin{itemize}
        \item (Propriedade reflexiva) $\forall a \in A$, $a \leq a$.
        \item (Propriedade antissimétrica) $\forall a, b \in A$, se $a \leq b$ e $b \leq a$, então $a=b$.
        \item (Propriedade transitiva) $\forall a, b, c \in A$, se $a \leq b$ e $b \leq c$, então $a \leq c$.
        \item (Linearidade) $\forall a, b \in A$, $a \leq b$ ou $b \leq a$.
    \end{itemize}

    e tal que:

    \begin{itemize}
        \item (Compatibilidade da soma) $\forall a, b, c \in A$, se $a \leq b$, então $a+c \leq b+c$ e $ac \leq bc$.
        \item (Compatibilidade da multiplicação) $\forall a, b, c \in A$, se $a \leq b$ e $0 \leq c$, então $ac \leq bc$.
    \end{itemize}

    Nesse caso, dizemos que $a<b$ se $a\leq b$ e $a\neq b$.
    
    Os elementos positivos de $A$ são os elementos maiores do que $0$.
    
    Os negativos são os menores do que $0$.
\end{definition}

Assumiremos, sem demonstração (por fugir do escopo do texto), que existe uma estrutura $\mathbb Z=(\mathbb Z,+,\cdot,0,1, \leq)$ como abaixo:

\begin{definition}[Inteiros, anel ordenado]
$\mathbb Z=(\mathbb Z,+,\cdot,0,1, \leq)$ é um domínio de integridade ordenado cujos elementos positivos possuem a propriedade da boa ordenação:

Qualquer subconjunto não vazio de inteiros positivos possui elemento mínimo.
\end{definition}

Assumiremos todos os fatos elementares sobre $\mathbb Z$ que não foram provados, inclusive o fato de que $\mathbb Z=\{\dots, -2, -1, 0, 1, 2, \dots, \}$.
\section{Corpos e anéis de divisão}

Abaixo, segue a definição de anel de divisão e corpo.
A noção de corpo será uma das noções mais importantes deste texto.
\begin{definition}[Corpo e Anel de Divisão]
Um \emph{anel de divisão} é um anel não trivial para o qual todo elemento não nulo é invertível.
Um \emph{corpo} é um anel de divisão comutativo.
\end{definition}

Todo corpo é um domínio de integridade.
De fato:
\begin{prop}
    Seja $K$ um corpo.
    Então $K$ é um domínio de integridade.
\end{prop}
\begin{proof}
Sabemos que $K$ é um anel comutativo não trivial.
Sejam $a, b \in K$ tais que $ab=0$.
Se $a=0$, então segue a tese.
Caso contrário, como $K$ é um corpo, $a^{-1}$ existe.
Assim, temos que $b=(a^{-1}a)b=a^{-1}(ab)=0$, logo, $b=0$.
\end{proof}

Porém, nem todo domínio de integridade é um corpo: por exemplo, $\mathbb Z$ é um domínio de integridade que não é um corpo, pois $2$ não possui inverso multiplicativo em $\mathbb Z$.

\section{O corpo dos números reais}
Assim como fizemos com $\mathbb Z$, assumiremos a existência do corpo dos números reais, que é um corpo ordenado que satisfaz a propriedade de ser Dedekind-completo. Formalmente:

\begin{prop}
    O corpo dos números reais $\mathbb R$ é um corpo ordenado, e satisfaz a propriedade de ser Dedekind-completo.
    Ou seja, tal que para todo $A\subseteq \mathbb R$ não vazio, se $A$ é limitado superiormente (ou seja, se existe $a \in \mathbb R$ tal que $\forall x \in A, x\leq a$), então $A$ admite um supremo (um menor limitante superior, ou seja, existe $b \in \mathbb R$ tal que $\forall x \in A, x\leq b$ e $\forall c \in \mathbb R$, se $x\leq c$ para todo $x\in A$, então $b\leq c$).
\end{prop}

O estudo das propriedades dos números reais é um assunto central de um curso básico de Análise Real.

Nesse texto, detalharemos tais propriedades somente de acordo com nossa necessidade.

\section{O corpo dos números complexos}
A história dos números complexos remete à representar uma solução para a equação $x^2+1=0$, que não possui solução real.

A ideia é que adiciona-se em $\mathbb R$ um novo elemento, $i$, para o qual vale $i^2=-1$ e para o qual as demais propriedades operacionais de números reais são preservadas. Nesse anel, todo elemento se escreverá de forma única como $a+bi$, onde $a,bin \mathbb R$.

Apresentaremos uma construção a seguir.

\begin{definition}[Quaternions]
    Definimos $\mathbb C=\mathbb R^2$.
    
    Se $a \in \mathbb R$, identifique $a=(a, 0)$ e $i=(0, 1)$.

    Segue que, utilizando a linguagem de produto por escalar oriunda da álgebra linear, que para todo $x \in \mathbb H$, exisem únicos $a, b\in \mathbb R$ tais que $x=a+bi$.

    Em $\mathbb C$, definimos a soma coordenada-a-coordenada. Da Álgebra Linear, sabemos que isso nos dá um grupo Abeliano.

    Define-se também a multiplicação, inspirada pela discussão acima, como se segue: para $a, b, c, d \in \mathbb R$:

    \begin{equation*}
        (a, b)(u, v)=(au-bv, bu+av).
    \end{equation*}

    Ou, em outra notação:

    \begin{align*}
    (a+bi)(c+di)& \\
    &=(ac-bd)+(ad+bc)i
    \end{align*}
\end{definition}

\begin{prop}
    $\mathbb C$ é um corpo.
\end{prop}
\begin{proof}
    \begin{prop}
        $\mathbb H$ é um domínio de integridade.
    \end{prop}
    \begin{proof}
        $1$ é neutro multiplicativo: dado $a+bi=(a, b)\in \mathbb C$, pela definição, temos que $(1, 0)(a, b)=(a, b)$, pois as demais parcelas zeram. Analogamente, $(a, b)(1, 0)=(a, b)$.
        
        A multiplicação é associativa:
        Para $x, y, z \in \mathbb H$, tome $a, b, u, v, p, q \in \mathbb R$ com e $x=(a, b)$, $y=(u, v)$ e $z=(p, q)$.
        Temos que:
        \begin{align*}
            (xy)z &=(au-bv, bu+av)(p, q)\\
            &=((au-bv)p-(bu+av)q, (bu+av)p+(au-bv)q)\\
            &=(aup-bvp-buq+avq, bup+avp+auq-bvq)
        \end{align*}
        e $x(yz)$ é dado por:
        \begin{align*}
        x(yz) &=(a, b)(up-pv,uq+vp)\\
        &=(a(up-bv)-b(uq+vp), b(up-bv)+a(uq+vp))\\
        &=(aup-bvq-bup+avq, bup+avp+auq-bvq)
        \end{align*}
        Comparando, segue.

        A multiplicação é comutativa: Para $x, y \in \mathbb H$, temos que $x=(a, b)$ e $y=(u, v)$. Temos que:
        \begin{align*}
            xy &=(a, b)(u, v)
            =(au-bv, bu+av)\\
            &= (ua-vb, va+ub)\\
            &= (u, v)(a, b)\\
            &=yx.
        \end{align*}    
        A propriedade distributiva também é válida:

        Para $x, y, z \in \mathbb H$, temos que $x=(a, b)$, $y=(u, v)$ e $z=(p, q)$. Temos que:
        \begin{align*}
            x(y+z) &=(a, b)((u, v)+(p, q))\\
            &=(a, b)(u+p, v+q)\\
            &=(a(u+p)-b(v+q), b(u+p)+a(v+q))\\
            &=(au-bv, bu+av)+(ap-bq, bp+aq)\\
            &=xy+xz
        \end{align*}

        Finalmente, todo elemento distindo de $(0, 0)$ é invertível: seja $x=(a, b)\in \mathbb C$ tal que $x\neq 0$.
        Então, $a^2+b^2\neq 0$.
        Considere $y=(\frac{a}{a^2+b^2}, \frac{-b}{a^2+b^2})$.
        Calculemos $xy$:
        \begin{align*}
            xy &=(a, b)(\frac{a}{a^2+b^2}, \frac{-b}{a^2+b^2})\\
            &=(\frac{a^2}{a^2+b^2}+\frac{b^2}{a^2+b^2}, \frac{-ab}{a^2+b^2}+\frac{ab}{a^2+b^2})\\
            &=\left(\frac{a^2+b^2}{a^2+b^2}, 0\right)\\
            &=1.
        \end{align*}
    \end{proof}
\end{proof}
\section{O Anel dos Quaternions}
Discutimos as noções de corpo e de anel de divisão.
Por definição, todo corpo é um anel de divisão.
Um dos primeiros exemplos de um anel de divisão que não é um corpo é o anel dos quaternions $\mathbb H$, que descreveremos abaixo.

A ideia é que adiciona-se em $\mathbb R$ três elementos distintos: $i, j, k$, para os quais valem as propriedades de que $i^2=j^2=k^2=-1$, e $ij=k$, $jk=i$ e $ki=j$, e para o qual as demais propriedades operacionais de números reais são preservadas. Nesse anel, todo elemento se escreverá de forma única como $a+bi+cj+dk$, onde $a,b,c,d\in \mathbb R$.

Apresentaremos uma construção a seguir. Antes disso, note que, como $k=ij$, multiplicando ambos os lados por $i$ à esquerda, supondo que a propriedade associativa ainda valha, temos que $ik=-j$.

Multiplicando por $j$ à direita, temos que $kj=-1$.

Além disso, multiplicando por $i=jk$ à esquerda por $j$, temos que $ji=-1$. Assim, temos que $ij=k$, $jk=i$, $ki=j$, $ji=-k$, $kj=-i$ e $ik=-j$.

Assumindo que $-i\neq i$, $-j\neq j$ e $-k\neq k$, temos que $i,j,k$ vêmos que a nossa estrutura deverá ser não comutativa.

\begin{definition}[Quaternions]
    Definimos $\mathbb H=\mathbb R^4$.
    
    Se $a \in \mathbb R$, seja $a=(a, 0, 0, 0)$, $i=(0, 1, 0, 0)$, $j=(0, 0, 1, 0)$ e $k=(0, 0, 0, 1)$.

    Segue que, utilizando a linguagem de produto por escalar oriunda da álgebra linear, que para todo $x \in \mathbb H$, exisem únicos $a, b, c, d \in \mathbb R$ tais que $x=a+bi+cj+dk$.

    Em $\mathbb H$, definimos a soma coordenada-a-coordenada. Da Álgebra Linear, sabemos que isso nos dá um grupo Abeliano.

    Define-se também a multiplicação, inspirada pela discussão acima, como se segue: para $a, b, c, d, u, v, z, w \in \mathbb R$:

    \begin{equation*}
        (a, b, c, d)(u, v, z, w)=(au-bv-cz-dw, av+bu+cw-dz, az+bw-cu+dv, aw+bz+cv-du).
    \end{equation*}

    Ou, em outra notação:

    \begin{align*}
    (a+bi+cj+dk)(u+vi+zj+kw)& \\
    &=(au-bv-cz-dw)+(av+bu+cw-dz)i\\
    &+(az+bw-cu+dv)j+(aw+bz+cv-du)k.
    \end{align*}
\end{definition}

Note que, com isso, temos $i^2=j^2=k^2=-1$, $ij=k$, $jk=i$ e $ki=j$, além de $i\neq -i$, $j\neq -j$ e $k\neq -k$.

Porém, $\mathbb H$ é um anel de divisão. Primeiro, provaremos que:

\begin{prop}
    $\mathbb H$ é um domínio de integridade.
\end{prop}
\begin{proof}
    $1$ é neutro multiplicativo: dado $a+bi+cj+dk=(a, b, c, d)\in \mathbb H$, pela definição, temos que $(1, 0, 0, 0)(a, b, c, d)=(a, b, c, d)$, pois as demais parcelas zeram. Analogamente, $(a, b, c, d)(1, 0, 0, 0)=(a, b, c, d)$.
    
    A multiplicação é associativa:
    Para $x, y, z \in \mathbb H$, temos que $x=(a, b, c, d)$, $y=(u, v, z, w)$ e $z=(p, q, r, s)$. Temos que:
    \begin{align*}
        (xy)z &=(au-bv-cz-dw, av+bu+cw-dz, az+bw-cu+dv, aw+bz+cv-du)(p,q,r,s)
    \end{align*}
    e $x(yz)$ é dado por:
    \begin{align*}
    x(yz) &=(a, b, c, d)(up-vq-zr-sw, uq+vp+zs-tw, ur+vq-pw+zt, us+vq+pw-qt)\\
    \end{align*}
    Expandindo os últimos produtos e comparando-os, vê-se que são iguais. Os detalhes ficam a cargo do leitor.

    De maneira igualmente trabalhosa, porém mecânica, verifica-se às duas propriedades distributivas.
\end{proof}

Mais interessante é demonstrar que $\mathbb H$ é um anel de divisão. Para isso, precisamos mostrar que todo elemento não nulo de $\mathbb H$ é invertível.

\begin{prop}
    $\mathbb H$ é um anel de divisão.
\end{prop}
\begin{proof}
    Fica a cargo do leitor. Para um guia, ver o Exercício~\ref{exer:quaternion}
\end{proof}
\section{Subanéis}
Em Matemática, é comum que as estruturas estudadas possuam uma noção de subestrutura.
Em geral, uma subestrutura de uma estrutura data é um subconjunto desta que seja, de forma natural, uma estrutura da mesma natureza daquela.

Veremos que, quando tratamos de anéis, nem todo subconjunto pode ser visto como uma subestrutura.

\begin{definition}[Subanel]
    Seja $A$ um anel e $B \subseteq A$. Dizemos que $B$ é subanel de $A$ se, e somente se $(B, +|_{B^2}, \cdot|_{B^2}, 0_A, 1_A)$ é um anel, onde $+|_{B^2}:B^2\rightarrow B$ e $\cdot|_{B^2}:B^2\rightarrow B$ são as restrições das operações de $A$ à $B^2$.
\end{definition}

Na definição acima, estamos pedindo que $B$ seja um subconjunto de $A$ que possua as mesmas operações que $A$, e que essas operações sejam restritas a $B$ e satisfaçam todas as cláusulas da definição de anel. Aparentemente, na prática, provar que um dado subconjunto de $A$ é um subanel pode parecer uma tarefa longa. Porém, a seguinte proposição encurta esta tarefa significativamente: 

\begin{prop}[Subanel]
    Seja $A$ um anel e $B\subseteq A$. Então $B$ é um subanel de $A$ se, e somente se:
    \begin{itemize}
        \item $1_A \in B$
        \item Para todos $a, b \in B$, $a-b \in B$.
        \item Para todos $a, b \in B$, $ab\in B$
    \end{itemize}

    Além disso, caso $B$ seja um subanel de $A$, os opostos aditivos de $B$ são os mesmos que os de $A$, ou seja, que $-b \in B$ para todo $B \in B$.
\end{prop}

\begin{proof}
    Primeiro, notemos suponhamos que $B$ seja um subanel de $A$. Então $B$ é fechado por $+, \cdot$ e $1_A\in B$. Resta apenas ver que para todos $a, b \in B$, $a-b \in B$.
    Como $B$ é fechado por soma, basta provar a última afirmação: que para todo $b \in B$, $-b \in B$.
    Fixe $b \in B$. Como $(B, +|_B^2, 0_A)$ é um grupo abeliano, existe $x \in B$ tal que $b+x=0_B$. Então, em $a$, segue que $b+x=x+b=0_A$. Pela unicidade dos opostos em $A$, segue que $-b=x\in B$.

    Reciprocamente, provaremos que se $B$ possui $1_B$ como elemento e é fechado por diferença e por produto, então $B$ é um subanel de $A$. Iniciaremos verificando que $B$ é fechado por soma, por opostos e que tem $0_A$ como elemento.

    Como $1_A$ é elemento de $B$, temos que $0_A=1_A-1_A\in B$. Assim, $B$ possui $0_A$ como elemento. Agora, dado $b \in B$, $0_A-b=-b \in B$, o que mostra que $B$ é fechado por opostos. Finalmente, dados $a, b \in B$, $a-(-b)=a+b\in B$, o que mostra que $B$ é fechado para soma.

    As propriedades associativas, comutativas, distributivas e de identidade valem em $B$, pois valem em $A$ e as operações de $B$ são as mesmas de $A$, restritas. Para finalizar, basta observar que dado $a \in B$, $(-a)\in B$, como já mostrado, e que $a+(-a)=(-a)+a=0_A$, o que mostra que $B$ possui opostos aditivos.
\end{proof}

\begin{exemplo}
$\mathbb N$ não é um subanel de $\mathbb Z$, pois $-1 \notin \mathbb Z$.
Porém, note que $\mathbb N$ tem $1$ e é fechado por soma e produto, o que mostra que na proposição anterior, a expressão $a-b$ não pode ser substituída por $a+b$.
\end{exemplo}

\begin{exemplo}[Subanel trivial]
    Para todo $A$, temos que $A$ é subanel de si mesmo.
\end{exemplo}

\begin{exemplo}
O único subanel de $\mathbb Z$ é $\mathbb Z$: se $B$ é um subanel de $\mathbb Z$, então $0, 1 \in B$.
Por indução, para todo $n\geq 1$ temos que $n \in \mathbb B$: com efeito, $1\in B$, e, se $n \in B$, $n+1\in B$, logo vale o passo indutivo. Finalmente, $-n\in B$ para todo $n\geq 1$. Como $\mathbb Z=\{0\}\cup\{n \in \mathbb Z: n\geq 1\}\cup \{-n \in \mathbb Z: n\geq 1\}$, temos que $B=\mathbb Z$.
\end{exemplo}

Como as operações de um subanel são as mesmas de um anel, um subanel de um anel comutativo é comutativo.

\begin{prop}
    Subanéis de aneis comutativos são comutativos.
\end{prop}
\begin{proof}
Seja $A$ um anel comutativo e $B$ um subanel de $A$. Para todos $a, b \in B$, temos que o produto $a\cdot b$ em $B$ é dado pelo produto (comutativo) $a\cdot b$ em $A$, logo $a\cdot b=b\cdot a$.
\end{proof}

\section{O centro de um anel}
Apesar de nem todo anel ser comutativo, todos os anéis possuem elementos que comutam com qualquer outro elemento -- ao menos o elemento $1$.

O centro do anel é o conjunto de tais elementos.

\begin{definition}[Centro de um anel]
    Seja $A$ um anel.

    O \emph{centro} de $A$, denotado por $Z(A)$, é o conjunto dos elementos de $A$ que comutam com todos os outros elementos de $A$.

    Formalmente, $Z(A)=\{a \in A: \forall b \in A, ab=ba\}$.
\end{definition}

O centro de um anel sempre é um subanel.

\begin{prop}
    Para todo anel $A$, o conjunto $Z(A)$ é um subanel de $A$.
\end{prop}

\begin{proof}
    Temos que $1 \in Z(A)$ pois para todo $b \in A$, $1a=a1=a$.

    Se $a, a' \in A$, temos que $aa' \in Z(A)$ pois para todo $b \in A$, $(aa')b=a(a'b)=a(ba')=(ab)a'=(ba)a'=b(a'a)$.

    Finalmente, se $a, a' \in A$, temos que $a-a' \in Z(A)$, pois para todo $b \in A$, $(a-a')b=ab-a'b=ba-ba'=b(a-a')$.
\end{proof}



\section{Exercícios}

\begin{exer} Seja $R$ um anel com identidade e seja $S$ um subanel de $R$ que contém a identidade de $R$.
Prove que se $u$ é uma unidade em $S$, então $u$ é uma unidade em $R$.
Apresente um exemplo que demonstre que a recíproca é falsa.

\end{exer} 
\begin{exer}
    Seja $A$ um anel. Mostre que um anel $A$ é um anel de divisão se, e somente se $A^*=A\setminus\{0\}$.
\end{exer}

\begin{exer}
    No anel dos quaternions $\mathbb H$, identifique $x \in \mathbb R$ com $(x, 0, 0, 0)=x+0i+0j+0k$.

    Mostre que $\mathbb R=Z(\mathbb H)$.

    (Dica: após mostrar que $\mathbb R\subseteq Z(\mathbb H)$, tome um elemento arbitrário de $Z(\mathbb H)$ e estude sua multiplicação por $i$, $j$ e $k$.)
\end{exer}
\begin{exer}\label{exer:quaternion}
    No anel dos quaternions, dado $q \in \mathbb H$, seu conjugado é definido como $\bar q = a + bi + cj + dk$.

    \begin{enumerate}[label=\alph*)]
        \item Calcule  $q\bar q$ e $q\bar q$.
        \item Prove que, se $q\neq 0$,  $\bar q(q\bar q)^{-1}$ é inverso multiplicativo de $q$.
        Conclua que $\mathbb H$ é anel de divisão.
    \end{enumerate}
\end{exer}
\begin{exer}
    Seja $A$ um anel. Prove que se $q \in Z(A)$ e $q$ é uma unidade, então $q^{-1} \in Z(A)$.
    Utilize esse fato para provar que o centro de qualquer anel de divisão é um corpo.
\end{exer}

\begin{exer}
    Seja $\mathbb Z[i] = \{m+in: m, n \in \mathbb Z\}\subseteq \mathbb C$ (o conjunto dos inteiros de Gauss).
    Mostre que $\mathbb Z[i]$ é um subanel de $\mathbb C$, e que é um domínio de integridade.
\end{exer}

\begin{exer}
    Seja $A$ um anel e $Z=Z[A]$ seu centro. Seja $a \in A$ qualquer.

    \begin{enumerate}[label=\alph*)]
        \item Mostre que $\left\{\sum_{i=0}^n a_i s^i:\,n\geq 0,\,a_0, \dots, a_n \in Z\right\}$ é um subanel de $A$.
        \item Mostre que tal subanel é comutativo.
    \end{enumerate}
\end{exer}

\chapter{Homomorfismos e Ideais}
Em matemática, boa parte das coleções de estruturas estudadas possui uma classe de funções que preservam, em algum sentido, suas propriedades.
O estudo generalizado destas estruturas é o que chamamos de \emph{teoria de categorias}, tema que não será tratado neste texto.
Na classe dos anéis, estas funções são o que chamamos de \emph{homomorfismos}.
\section{Definição de homomorfismo}
Homomorfismos são funções que preservam a estrutura de anéis.
Formalmente:
\begin{definition}
Sejam $A$, $R$ aneis.
Uma função $f:A\rightarrow R$ é um \emph{homomorfismo} se:
\begin{itemize}
    \item $f(a+b)=f(a)+f(b)$ para todo $a, b \in A$.
    \item $f(-a)=-f(a)$ para todo $a \in A$.
    \item $f(0_A)=0_R$
    \item $f(ab)=f(a)f(b)$ para todo $a, b \in A$.
    \item $f(1_A)=1_R$.
\end{itemize}

Caso $f$ seja injetora, dizemos que $f$ é um \emph{monomorfismo}.
Caso $f$ seja sobrejetora, dizemos que $f$ é um \emph{epimorfismo}.
Caso $f$ seja bijetora, dizemos que $f$ é um \emph{isomorfismo}.
\end{definition}

A noção de isomorfismo é extremamente importante na Teoria de Anéis. Muitas vezes, temos dois anéis que ``deveriam ser a mesma coisa'', mas, como objetos matemáticos, não são iguais. A noção de isomorfismo entra em campo para dizer que, mesmo que dois anéis não sejam o mesmo objeto, eles possuem exatamente as mesmas propriedades algébricas e operacionais. Para darmos um exemplo concreto:

\begin{exemplo}
Seja $A=\{0, 1\}$ e $R=\{Z, U\}$, onde $Z, U$ são objetos diferentes, e diferentes de $0, 1$. Defina em $A$ as operações $\cdot$ e $+$ dadas pelas seguintes tabelas:

Em $A$:
\begin{multicols}{2}\centering
    \begin{tabular}{c|cc}
        $+$ & 0 & 1 \\ \hline
        0 & 0 & 1 \\
        1 & 1 & 0 \\
    \end{tabular}

    \begin{tabular}{c|cc}
        $\cdot$ & 0 & 1 \\ \hline
        0 & 0 & 0 \\
        1 & 0 & 1 \\
    \end{tabular}
\end{multicols}

Em $R$:
\begin{multicols}{2}\centering
    \begin{tabular}{c|cc}
        $+$ & Z & U \\ \hline
        Z & Z & U \\
        U & U & Z \\
    \end{tabular}

    \begin{tabular}{c|cc}
        $\cdot$ & Z & U \\ \hline
        Z & Z & Z \\
        U & Z & U \\
    \end{tabular}
\end{multicols}

Intuitivamente, $A$ e $R$ correspondem a duas apresentações de uma mesma estrutura algébrica, porém, como $A\cap R=\emptyset$, estes dois anéis não são o mesmo anel.
Como formalizar este fato?
Ora, há uma relação biunívoca (uma bijeção) entre $A$ e $R$ que preserva suas operações, e ela é dada por $\phi(0)=Z$ e $\phi(1)=U$.
Tal $\phi$ é um isomorfismo.
\end{exemplo}

Para todos os fins que interessam à Álgebra, anéis isomorfos tem exatamente as mesmas propriedades, e, assim, são considerados como sendo, em algum sentido, a mesma estrutura.

A definição de homomorfismo, por possuir várias cláusulas, pode parecer de longa verificação.
A proposição abaixo encurta esta verificação substancialmente.

\begin{prop}
    Sejam $A, R$ anéis e $f:A\rightarrow R$ uma função.
    Então $f$ é um homomorfismo se, e somente se:
    \begin{itemize}
        \item $f(a+b)=f(a)+f(b)$ para todo $a, b \in A$.
        \item $f(ab)=f(a)f(b)$ para todo $a, b \in A$.
        \item $f(1_A)=1_R$.
    \end{itemize}
\end{prop}
\begin{proof}
    Provaremos o lado que não é imediatamente trivial.
    Começaremos mostrando que $f(0_A)=0_R$.
    Temos que $f(0_A)=f(0_A+0_A)=f(0_A)+f(0_A)$, logo, cancelando, $f(0_A)=0_R$.

    Agora, vejamos que $f(-a)=-f(a)$ para todo $a \in A$.
    Temos que $f(a)+f(-a)=f(a+(-a))=f(0_A)=0_R$, logo, $f(-a)=-f(a)$.

    Assim, $f$ é um homomorfismo.
\end{proof}

\section{Propriedades elementares}
\begin{lemma}
    Sejam $f:A\rightarrow R$ e $g:R\rightarrow S$ homomorfismos de anéis.
    Então a composição $g\circ f:A\rightarrow S$ é um homomorfismo de anéis.
\end{lemma}

\begin{proof}
    Sejam $a, b \in A$. Então:
    \begin{itemize}
        \item $g\circ f(a+b)=g(f(a+b))=g(f(a)+f(b))=g(f(a))+g(f(b))=(g\circ f)(a)+(g\circ f)(b)$.
        \item $g\circ f(ab)=g(f(ab))=g(f(a)f(b))=g(f(a))g(f(b))=(g\circ f)(a)(g\circ f)(b)$.
        \item $g\circ f(1_A)=g(f(1_A))=g(1_R)=1_S$.
    \end{itemize}
    Assim, $g\circ f$ é um homomorfismo de anéis.
\end{proof}

\begin{prop}[Propriedades de homomorfismos]
    Seja $f:A\rightarrow R$ um homomorfismo de anéis. Então:
    \begin{enumerate}[label=\alph*)]
        \item Para todo $a \in A^*$, temos $f(a)\in  R^*$ e $f(a^{-1})=f(a)^{-1}$. \label{prop:homomorfismo_a}
        \item A imagem de $f$, $\ran f=\{f(a): a \in A\}$, é um subanel de $R$. Se $A$ é comutativo, $\ran f$ também é.  \label{prop:homomorfismo_b}
        \item Se $f$ é injetora, a imagem de $f$ é um subanel de $R$ isomorfo a $A$. \label{prop:homomorfismo_c}
    \end{enumerate}
\end{prop}
\begin{proof}
\ref{prop:homomorfismo_a} Se $a \in A^*$, então $f(a)f(a^{-1})=f(aa^{-1})=f(1_A)=1_R$ e $f(a^{-1})f(a)=f(aa^{-1})=f(1_A)=1_R$. Assim, $f(a^{-1})=f(a)^{-1}$ e $f(a)\in R^*$.

\ref{prop:homomorfismo_b} Seja $a, b \in \ran f$. Então existem $x, y \in A$ tais que $a=f(x)$ e $b=f(y)$.
Assim, $a-b=f(x)-f(y)=f(x-y)$. Logo, $a-b \in \ran f$.
Similarmente, $ab=f(x)f(y)=f(xy)\in \ran f$, e $1_R=f(1_A)\in \ran f$.

Portanto, $\ran f$ é um subanel de $R$.
Se $A$ é comutativo, $\ran(f)$ também é comutativo, pois dados $a, b \in \ran f$, existem $x, y \in A$ tais que $a=f(x)$ e $b=f(y)$.
Assim, $ab=f(x)f(y)=f(xy)=f(yx)=f(y)f(x)=ba$.

\ref{prop:homomorfismo_c} Se $f$ é injetora, então $f$ é bijetora entre $A$ e $\ran f$. Assim, $f$ é um isomorfismo entre $A$ e $\ran f$, dado que é um homomorfismo.
\end{proof}

A noção de isomorfismo é uma relação de equivalência na classe dos anéis.

\begin{prop}[Propriedades de isomorfismo]
    Sejam $A, R, S$ anéis e $f:A\rightarrow R$ e $g:R\rightarrow S$ isomorfismos de anéis.
    Então:
    \begin{enumerate}[label=\alph*)]
        \item $g\circ f$ é um isomorfismo de anéis. \label{prop:isomorfismo_a}
        \item $f^{-1}:R\rightarrow A$ é um isomorfismo de anéis. \label{prop:isomorfismo_b}
        \item $\id_A:A\rightarrow A$ é um isomorfismo de anéis. \label{prop:isomorfismo_c}
    \end{enumerate}
\end{prop}

\begin{proof}
\ref{prop:isomorfismo_a} A composição de funções bijetoras é bijetora, e a composição de homomorfismos é homomorfismo.
Como um isomorfismo é um homomorfismo bijetor, segue que a composição de dois isomorfismos é um isomorfismo.

\ref{prop:isomorfismo_b} Como $f$ é um isomorfismo, $f$ é bijetora, assim, $f^{-1}:R\rightarrow A$ está bem definida e é bijetora. Verificaremos que $f^{-1}$ é um homomorfismo. Dados $r, s \in R$, sejam $a, b \in A$ tais que $f(a)=r$ e $f(b)=s$. Temos que:
\begin{itemize}
    \item $f^{-1}(r+s)=f^{-1}(f(a)+f(b))=f^{-1}(f(a+b))=a+b=f^{-1}(r)+f^{-1}(s)$.
    \item $f^{-1}(rs)=f^{-1}(f(a)f(b))=f^{-1}(f(ab))=a\cdot b=f^{-1}(r)f^{-1}(s)$.
    \item $f^{-1}(1_R)=f^{-1}(f(1_A))=1_A$.
\end{itemize}

\ref{prop:isomorfismo_c} A função identidade $\id_A$ é claramente bijetora, e é um homomorfismo, pois, para todos $a, b \in A$:
    \begin{itemize}
        \item $\id_A(a+b)=a+b=\id_A(a)+\id_A(b)$.
        \item $\id_A(ab)=ab=\id_A(a)\id_A(b)$.
        \item $\id_A(1_A)=1_A$.
    \end{itemize}
\end{proof}

Agora introduziremos o núcleo de um homomorfismo.
\begin{definition}
    Seja $f: A\rightarrow R$ um homomorfismo de anéis.
    Definimos o \emph{núcleo} de $f$, também chamado de $\emph{kernel}$ de $f$, como sendo o conjunto dos zeros de $f$.
    Em símbolos:

    \[\ker f=\{a \in A: f(a)=0_R\}.\]
\end{definition}
Uma importante relação entre o homomorfismo e seu núcleo é dado como se segue:

\begin{prop}
Sejam $A, R$ anéis e $f:A\rightarrow R$ um homomorfismo. Então $f:A\rightarrow R$ é injetor (um monomorfismo) se, e somente se $\ker f = \{0_A\}$.
\end{prop}

\begin{proof}
    Primeiro, suponha que $f$ é um monomorfismo.
    Sabemos que $f(0_A)=0_R$, pois $f$ é homomorfismo, e, portanto, $\{0_A\}\subseteq \ker f$.
    Reciprocamente, seja $a \in \ker f$.
    Temos que $f(a)=0_R=f(0_A)$. Pela injetividade de $f$ segue que $a=0_A\in \{0_A\}$.

    Agora suponha que $\ker f=\{0_A\}$.
    Veremos que $f$ é injetora.
    Para tanto, sejam $a, b \in A$ e suponha que $f(a)=f(b)$.
    Temos que $f(a-b)=f(a)-f(b)=0_R$, assim, $a-b\in \ker_f = \{0_A\}$, o que implica em $a-b=0_A$, e, portanto, $a=b$.
\end{proof}
\section{Ideais}
Ideais são as estruturas responsáveis pela noção de quociente em anéis, assunto que será estudado no próximo capítulo.
Introduziremos a noção de ideal neste capítulo pois ela tem interações fundamentais com a noção de homomorfismo, porém, apenas no próximo capítulo ficará clara a sua enorme importância para esta teoria.
Nesta seção, motivaremos, nesta seção, a noção de ideal, a partir do núcleo de homomorfismos.

Para começar, notemos algumas propriedades do núcleo.

\begin{prop}
Seja $f:A\rightarrow R$ um homomorfismo de anéis. Seja $I=\ker f$. Então:

\begin{enumerate}[label=\alph*)]
    \item $0_A \in I$.
    \item Para todos $a, b \in I$, $a+b \in I$.
    \item Para todos $a \in I$ e $x \in A$, $ax \in I$.
    \item Para todos $a \in I$ e $x \in A$, $xa \in I$.
\end{enumerate}
\end{prop}

\begin{proof}
    \begin{enumerate}[label=\alph*)]
        \item $0_A \in I$ pois $f(0_A)=0_R$.
        \item Se $a, b \in I$, então $f(a)=0_R$ e $f(b)=0_R$. Assim, $f(a+b)=f(a)+f(b)=0_R+0_R=0_R$, logo, $a+b\in I$.
        \item Se $a \in I$ e $x \in A$, então $f(a)=0_R$. Assim, $f(ax)=f(a)f(x)=0_Rf(x)=0_R$, logo, $ax\in I$.
        \item Se $a \in I$ e $x \in A$, então $f(a)=0_R$. Assim, $f(xa)=f(x)f(a)=f(x)0_R=0_R$, logo, $xa\in I$.
    \end{enumerate}
\end{proof}

É possível indagar se $\ker f$ é um subanel de $A$. Observemos que as propriedades c) e d) são mais fortes do que a propriedade exigida para produto para ser um subanel. Além disso, $\ker f$ é fechado por diferenças, pois se $a, b \in \ker f$, pela propriedade d), $(-1)b=-b\in \ker f$, e, portanto, $a-b \in \ker f$. Porém, $1_A$ raramente está em $\ker f$, como vemos a seguir:

\begin{prop}
    Seja $f:A\rightarrow R$ um homomorfismo de anéis. Se $1_A \in \ker f$, então $R$ é o anel trivial, ou seja, $R=\{0_R\}$.
\end{prop}

\begin{proof}
    Se $1_A \in \ker f$, então $f(1_A)=0_R$.
    Como $f$ é um homomorfismo, temos que $f(1_A)=f(1_A\cdot 1_A)=f(1_A)f(1_A)=0_R\cdot 0_R=0_R$.
    Como $1_R=0_R$, segue que $R=\{0_R\}$, pois dado $x \in R$ temos $x=x\cdot 1_R=x\cdot 0_R=0_R$.
\end{proof}

Como recíproca, notemos que um homomorfismo acima existe para qualquer anel $A$:

\begin{prop}
    Seja $A$ um anel e $R=\{0_R\}$ um anel trivial.
    
    Então $f:A\rightarrow R$ dado por $f(x)=0_R$ para todo $x \in A$ é um homomorfismo de anéis, e $\ker f=A$.
\end{prop}

\begin{proof}
    Temos que $f$ é um homomorfismo de anéis, já que dados $a, b \in R$, temos $f(a+b)=0_R=0_R+0_R=f(a)+f(b)$, $f(ab)=0_R=0_R\cdot 0_R=f(a)f(b)$, $f(1_A)=0_R=1_R$.
    Como $f$ é a função nula, $\ker f=A$.
\end{proof}

Podemos ver $\ker f$, em algum sentido, como uma medida do quão longe um homomorfismo $f$ está de ser injetor: temos que $\{0\}\ker f\subseteq A$.
Como vimos, $f$ ser injetor é equivalente à $f=\{0\}$.
No outro extremo, $f$ ser constante significa que $\ker f = A$.

Vimos ainda que $\ker f$ não é um subanel, mas que possui propriedades especiais. Tais propriedades são a definição de ideal.

\begin{definition}[Ideal]
    Seja $A$ um anel.
    Um subconjunto $I \subseteq A$ é dito \emph{ideal}, ou um \emph{ideal bilateral} se:

    \begin{enumerate}[label=\alph*)]
        \item $0_A \in I$.
        \item Para todos $a, b \in I$, $a+b \in I$.
        \item Para todos $a \in I$ e $x \in A$, $ax \in I$.
        \item Para todos $a \in I$ e $x \in A$, $xa \in I$.
    \end{enumerate}

    Caso $I$ satisfaça todas as propriedades menos d), $I$ é dito um ideal à direita.
    De forma similar, caso $I$ satisfaça todas as propriedades menos c), $I$ é dito um ideal à esquerda.
\end{definition}

Note que se $A$ é um anel comutativo, então $I$ é um ideal à esquerda se, e somente se, $I$ é um ideal à direita.
Assim, em anéis comutativos, a noção de ideal é equivalente à de ideal à esquerda ou à de ideal à direita.
Por simplicidade, neste texto, focaremos nosso estudo em ideais bilaterais.
Porém, muitos resultados aqui expressados possuem versões para ideais à esquerda e à direita.

Da discussão anterior, temos:

\begin{corol}
    Seja $f:A\rightarrow R$ um homomorfismo de anéis. Então $\ker f$ é um ideal de $A$.
\end{corol}

Então, todo núcleo é um ideal.
No próximo capítulo, veremos que vale uma recíproca: todo ideal é um núcleo de algum homomorfismo.

Todo anel possui ao menos os ideais abaixos, chamados de ideais triviais:

\begin{prop}[Ideal trivial]Seja $A$ um anel. Então $\{0\}$ e $A$ são ideais de $A$. Estes ideais são chamados de \emph{ideais principais}
\end{prop}
\begin{proof}
    Exercício.
\end{proof}

\begin{prop}[Interseção de ideais]
    Seja $A$ um anel e $\mathcal F$ uma coleção não vazia de ideais de $A$. Então $\bigcap_{I \in \mathcal F}I=\bigcap \mathcal F$ é um ideal de $A$.
\end{prop}

Ideais também são preservados por imagens inversas.

\begin{prop}
    $f:A\rightarrow R$ um homomorfismo de anéis e $J$ um ideal de $R$.
    Então $f^{-1}[J]=\{a \in A: f(a) \in J\}$ é um ideal de $A$.
\end{prop}
\begin{proof}
    Seja $I=f^{-1}[J]$.
    Temos que $J\neq \emptyset$ já que $0 \in \ker f\subseteq I$.

    Sejam $a, b \in I$.
    Então $f(a), f(b) \in J$, logo, $f(a+b)=f(a)+f(b) \in J$, o que implica $a+b \in I$.

    Agora seja $a \in A$ e $b \in I$.
    Temos que $f(ab)=f(a)f(b)\in J$ e $f(ba)=f(b)f(a)\in J$, pois $f(b)\in J$.
    Assim, $ab, ba \in J$.
\end{proof}
\begin{proof}

    Seja $I=\bigcap \mathcal F$.

    Então $0 \in I$, pois $0 \in I$ para todo $I \in \mathcal F$.

    Sejam $a, b \in I$.
    Então, para todo $I \in \mathcal F$, temos que $a, b \in I$, logo, $a+b\in I$.
    Assim, $a+b\in \bigcap \mathcal F$.

    Seja $a \in A$ e $b \in I$.
    Então, para todo $I \in \mathcal F$, temos que $b \in I$, logo, $ab\in I$.
    Assim, $ab\in \bigcap \mathcal F$.

    Analogamente, se $a \in I$ e $b \in A$, então $ba\in I$.
\end{proof}

\begin{prop}[Ideal gerado]
    Seja $A$ um anel e $B\subseteq A$ um conjunto não vazio.
    Então, o conjunto $I=\{a_1b_1c_1+\cdots+a_nb_nc_n: n\geq 1, a_i, c_i \in A, b_i \in B\}$ é o menor ideal $A$ que contém $B$ (ou seja, além de ser um ideal contendo $B$, se $J$ é qualquer ideal contendo $B$, então $I\subseteq J$).

    Além disso, se $B\subseteq Z(R)$, onde $Z(R)$ denota o centro de $R$, então $I=\{a_1b_1+\cdots+a_nb_n: n\geq 1, a_i \in A, b_i \in B\}$.
\end{prop}

\begin{proof}
    Primeiro, verificaremos que $I$ é um ideal.

    $0 \in I$, pois $0=0b0$ para todo $b \in B$.

    Considere $x, y \in I$.
    Então existem $n, m\geq 1$, $a_1, \dots, a_n, c_1, \dots, c_n \in A$, $b_1, \dots, b_n \in B$, $a_1', \dots, a_m', c_1', \dots, c_m' \in A$ e $b_1', \dots, b_m' \in B$ tais que $x=a_1b_1c_1+\cdots+a_nb_nc_n$ e $y=a_1'b_1c_1'+\cdots+a_m'b_m'd_m'$.
    Assim, $x+y=(a_1b_1+\cdots+a_nb_n)+(a_1'b_1c_1+\cdots+c_md_m)=(a_1b_1c_1+\cdots+a_nb_nc_n)+(a_1'b_1'c_1'+\cdots+a_m'b_m'd_m') \in I$.
    Concatenando as sequências, vemos que $x+y\in I$.

    Seja $x \in A$ e $b \in I$.
    Então existem $n\geq 1$, $a_1, \dots, a_n, c_1, \dots, c_n \in A$ e $b_1, \dots, b_n \in B$ tais que $b=a_1b_1c_n+\cdots+a_nb_nc_n$. Assim, $xb=(xa_1)b_1c_1+\cdots+(xa_n)b_nc_n\in I$.
    Analogamente, $bx \in I$.

    Agora, seja $J$ um ideal de $A$ que contém $B$.
    Fixe $x \in I$.
    Existem $n\geq 1$, $a_1, \dots, a_n, c_1, \dots, c_n\in A$ e $b_1, \dots, b_n \in B$ tais que $x=a_1b1c_1+\dots+a_nb_nc_n$.
    Como $J$ é um ideal de $A$ e $B\subseteq A$, para cada $i \in \{1, \dots, n\}$ temos que $a_ib_ic_i \in J$.
    Somando, segue que $x \in J$.

    Finalmente, provaremos a afirmação final para quando $B\subseteq Z(R)$. Seja $I'=\{a_1b_1+\cdots+a_nb_n: n\geq 1, a_i \in A, b_i \in B\}$.
    Veremos que $I=I'$.
    Pondo $c_1=\cdots=c_n=1$, vemos que que $I'\subseteq I$.

    Reciprocamente, se $x=a_1b_1c_1+\cdots+a_nb_nc_n \in I$ com $n\geq 1$, $a_1,\dots, a_n, c_1, \dots, c_n \in A$ e $b_1, \dots, b_n \in B\subseteq Z(A)$, temos que $x=(a_1c_1)b_1+\dots+(a_nc_n)b_n\in I'$.
\end{proof}


\begin{definition}
    Na notação da proposição acima, $I$ é chamado de \emph{ideal gerado por $B$} e denotamos por $\langle B \rangle$. 
    
    Caso $B=\{x_1, \dots, x_n\}$, denotamos o ideal gerado por $B$ como $\langle x_1, \dots, x_n \rangle$.
    Em particular, se $B=\{x\}$, denotamos o ideal gerado por $B$ como $\langle x \rangle$.
    
    Caso $B$ seja a imagem de uma família $(x_i: i \in Z)$, denotamos o ideal gerado por $B$ como $\langle x_i: i \in Z \rangle$. 

    Em qualquer um desses casos, $B$ é dito um gerador do ideal.
\end{definition}

Observação: note que o menor ideal contendo $B=\emptyset$ é o ideal nulo, $\{0\}$.
Escrevemos $\langle \emptyset\rangle=\{0\}$.

\begin{definition}[Ideal principal]
    Um \em{ideal principal} é um ideal gerado por um único elemento.
\end{definition}

Notemos que ideais triviais são principais à esquerda e à direita, pois $0A=\{0\}=A0$ e $A1=A=1A$.

\begin{definition}[Domínio de ideais principais]
    Um domínio de ideais principais (DIP), ou anel principal, é um domínio de integridade $A$ tal que todo ideal de $A$ é principal.
\end{definition}

Em um anel comutativo $A$, como um domínio de integridade, pelo exposto acima, para todo $x \in A$, o conjunto $xA=\{xa: a \in A\}$ é o conjunto $\langle x\rangle$.
Assim, um domínio de ideais principais é um domínio cujos ideais são exatamente os conjuntos da forma $xA$ para algum $x \in A$. Note que os ideais principais são sempre triviais, pois $\langle 0\rangle=\{0\}$ e $\langle 1\rangle = A$.

Quais são exemplos de DIPs? Para começar, qualquer corpo é um DIP. Mais especificamente:

\begin{prop}[Ideais de um corpo são triviais]
    Os únicos ideais de qualquer corpo são os triviais.
    Em particular, todo corpo é um DIP.
    Reciprocamente, se $A$ é um anel comutativo não trivial cujo todo ideal é trivial, então $A$ é um corpo.
\end{prop}
\begin{proof}
    Seja $K$ um corpo e $I$ um ideal de $K$.
    Se $I=\{0\}$, então $I$ é trivial.
    Se $I\neq \{0\}$, então existe $a \in I$ tal que $a \neq 0$. Daí $1=a^{-1}a=\in I$.
    Logo, para todo $k \in K$, $k=1k\in I$.

    Para a recíproca, seja $A$ um anel comutativo não trivial tal que todo ideal de $A$ é trivial, e fixe $x \in A\setminus \{0\}$.
    Como $Ax$ é um ideal trivial e $0\neq x \in Ax$, temos que $Ax=A$.
    Logo, existe $a \in A$ tal que $ax=1$. Assim, $x$ é invertível.
    Portanto, $A$ é um corpo.
\end{proof}

Porém, nem todo DIP é um corpo, como exemplificado pelo anel dos números inteiros.

\begin{prop}[Um DIP que não é um corpo] O anel dos inteiros $\mathbb Z$ é um domínio de ideais principais que não é um corpo.
\end{prop}
\begin{proof}
    Seja $I$ um ideal de $\mathbb Z$.
    Veremos que $I$ é um ideal principal.
    Se $I=\{0\}$, então $I$ é principal.
    Caso contrário, $I$ contém ao menos um elemento positivo, já que, sendo $x\in I\setminus\{0\}$, temos que $-x \in I$ e um dos $x, -x$ é positivo.

    Seja $n$ o menor inteiro positivo de $I$.
    Afirmamos que $I=n\mathbb Z$.
    De fato, se $x \in I$, então escreva $x=qn+r$, onde $q,r \in \mathbb Z$ e $0\leq r<n$.
    Como $x \in I$, temos que $r=x-qn \in I$. Assim, $r=0$, ou violaríamos a minimalidade de $n$.
    Logo, $x=qn\in n\mathbb Z$.
    Portanto, $I\subseteq n\mathbb Z$.
    Como $n\mathbb Z=\langle n\rangle$ e $n \in I$, temos que $n\mathbb Z\subseteq I$, o que completa a prova.
\end{proof}

\section{Exercícios}
\begin{exer}
Lembremos que, da Álgebra Linear, um espaço vetorial $V$ sobre um corpo $K$ é uma quadrupla $(V, +, 0, \cdot)$, onde $(V, +, 0)$ é um grupo Abeliano e $\cdot:K\times V\rightarrow V$ é uma operação que satisfaz:

\begin{itemize}
    \item Associatividade: para todos $\alpha, \beta \in K$ e para todo $v \in V$, $(\alpha\beta)v=\alpha(\beta v)$.
    \item Distributividade: para todo $x, y \in K$ e para todo $v \in V$, $(x+y)v=xv+yv$.
    \item Distributividade II: para todo $x \in K$ e para todo $u, v \in V$, $x(u+v)=xu+xv$.
    \item Identidade: $1v=v$ para todo $v \in V$.
\end{itemize}

Uma transformação linear $T:V\rightarrow W$ entre dois espaços vetoriais $V$ e $W$ sobre um mesmo corpo $K$ é uma função que preserva a estrutura de espaço vetorial, ou seja, satisfaz:

\begin{itemize}
    \item $T(v+u)=T(v)+T(u)$ para todo $v, u \in V$.
    \item $T(\alpha v)=\alpha T(v)$ para todo $\alpha \in K$ e para todo $v \in V$.
\end{itemize}

Dado um espaço vetorial $V$, o conjunto de todas as transformações lineares de $V$ em $V$, também chamadas de endomorfismos de $V$, é denotado por $\End(V)$.
A função identidade $\id_V:V\rightarrow V$ é um endomorfismo, bem como a função nula.

Assumindo todo o exposto acima, mostre que, com a soma usual de transformações lineares (que é efetuada ponto-a-ponto) e com operação de composição como produto, $\End(V)$ é um anel.

Mostre com um exemplo que $\End(V)$ pode não ser comutativo.
\end{exer}

\begin{exer}
Seja $V$ um espaco vetorial sobre um corpo $K$.
Defina $\rho:K\rightarrow V^V$ da seguinte forma: 

Para cada $\alpha \in K$, o mapa $\rho(\alpha):V\rightarrow V$ é dado por $\rho(\alpha)(v)=\alpha v$ para todo $v \in V$.

Mostre que $\rho$ é um homomorfismo de anéis, onde $V^V$ é o anel dos endomorfismos de $V$.

(Dica: não se esqueça de verificar que $\rho$ possui o contradomínio correto.)
\end{exer}

\chapter{Quocientes e homomorfismos}
\section{Ideais}
\begin{definition}[Ideal à esquerda]
    Seja $A$ um anel. Um subconjunto $I \subseteq A$ é dito \emph{ideal à esquerda} se:
    \begin{itemize}
        \item $0 \in I$.
        \item Para todos $a, b \in I$, temos $a+b\in I$.
        \item $\forall a \in A$ e $\forall b \in I$, temos $ab \in I$.
    \end{itemize}
\end{definition}

\begin{definition}[Ideal à direita]
    Seja $A$ um anel. Um subconjunto $I \subseteq A$ é dito \emph{ideal à direita} se:
    \begin{itemize}
        \item $0 \in I$.
        \item Para todos $a, b \in I$, temos $a+b\in I$.
        \item $\forall a \in I$ e $\forall b \in A$, temos $ab \in I$.
    \end{itemize}
\end{definition}

\begin{definition}[Ideal]
    Seja $A$ um anel. Um subconjunto $I \subseteq A$ é dito \emph{ideal} se for um ideal à esquerda e um ideal à direita. Ou seja, $I$ é um ideal se:
    \begin{itemize}
        \item $0 \in I$.
        \item Para todos $a, b \in I$, temos $a+b\in I$.
        \item $\forall a \in A$ e $\forall b \in I$, temos $ab \in I$.
        \item $\forall a \in I$ e $\forall b \in A$, temos $ab \in I$.
    \end{itemize}
\end{definition}

\begin{prop}[Ideal trivial]Seja $A$ um anel. Então $\{0\}$ e $A$ são ideais de $A$. Estes ideais são chamados de \emph{ideais principais}
\end{prop}
\begin{proof}
    Exercício.
\end{proof}

Note que se $A$ é um anel comutativo, então $I$ é um ideal à esquerda se, e somente se, $I$ é um ideal à direita. Assim, em anéis comutativos, a noção de ideal é equivalente à de ideal à esquerda ou à de ideal à direita.

\begin{prop}[Interseção de ideais]
    Seja $A$ um anel e $\mathcal F$ uma coleção não vazia de ideais à esquerda de $A$. Então $\bigcap_{I \in \mathcal F}I=\bigcap \mathcal F$ é um ideal de $A$. O mesmo vale para ideais à direita e ideais.
\end{prop}
\begin{proof}
    Provaremos para ideais à esquerda. A prova para ideais à direita é análoga e fica como exercício.

    Seja $I=\bigcap \mathcal F$. Então $0 \in I$, pois $0 \in I$ para todo $I \in \mathcal F$.

    Sejam $a, b \in I$. Então, para todo $I \in \mathcal F$, temos que $a, b \in I$, logo, $a+b\in I$. Assim, $a+b\in \bigcap \mathcal F$.

    Finalmente, seja $a \in A$ e $b \in I$. Então, para todo $I \in \mathcal F$, temos que $b \in I$, logo, $ab\in I$. Assim, $ab\in \bigcap \mathcal F$.
\end{proof}

\begin{prop}[Ideal gerado]
    Seja $A$ um anel comutativo e $B\subseteq A$ um conjunto não vazio. Então, o conjunto $I=\{a_1b_1+\cdots+a_nb_n: n\geq 1, a_i \in A, b_i \in B\}$ é o menor ideal à esquerda $A$ que contém $B$ (ou seja, além de ser um ideal contendo $B$, se $J$ é qualquer ideal contendo $B$, então $I\subseteq J$). O ideal $I$ é chamado de \emph{ideal gerado por $B$}, e denotado por $\langle B \rangle$.
    
    Se $B=\{x_0, \dots, x_n\}$, então abreviamos $\langle B \rangle$ como $\langle x_0, \dots, x_n \rangle$.
\end{prop}
\begin{proof}
    Primeiro, verificaremos que $I$ é um ideal.

    $0 \in I$, pois $0=0b$ para todo $b \in B$.

    Sejam $x, y \in I$. Então existem $n, m\geq 1$, $a_1, \dots, a_n \in A$, $b_1, \dots, b_n \in B$, $c_1, \dots, c_m \in A$ e $d_1, \dots, d_m \in B$ tais que $x=a_1b_1+\cdots+a_nb_n$ e $y=c_1d_1+\cdots+c_md_m$. Assim, $x+y=(a_1b_1+\cdots+a_nb_n)+(c_1d_1+\cdots+c_md_m)=(a_1b_1+\cdots+a_nb_n)+(c_1d_1+\cdots+c_md_m) \in I$.

    Finalmente, seja $a \in A$ e $b \in I$. Então existem $n\geq 1$, $a_1, \dots, a_n \in A$ e $b_1, \dots, b_n \in B$ tais que $b=a_1b_1+\cdots+a_nb_n$. Assim, $ab=(a_1b_1+\cdots+a_nb_n)a=a_1(b_1a)+\cdots+a_n(b_na) \in I$.

    Agora, seja $J$ um ideal de $A$ que contém $B$. Então, como $J$ é um ideal de $A$, temos que $\forall a_i\in A$, $\forall b_i\in B$, temos que $(a_i b_i)\in J$. Logo, $I\subseteq J$. Portanto, $I$ é o menor ideal de $A$ que contém $B$.
\end{proof}

Observação: note que o menor ideal contendo $B=\emptyset$ é o ideal nulo, $\{0\}$.

\begin{definition}[Ideal principal]
    Seja $A$ um anel. Para todo $x \in A$, o conjunto $xA=\{xa:a \in A\}$ é um ideal à direita de $A$. O ideal $xA$ é chamado de \emph{ideal principal à direita gerado por $x$}.
    Analogamente, o conjunto $Ax=\{ax:a \in A\}$ é um ideal à esquerda de $A$, e é chamado de \emph{ideal principal à esquerda gerado por $x$}.
\end{definition}
\begin{proof}
Mostraremos que $xA$ é um ideal à direita. As demais afirmações ficam como exercício.

Note que $0 \in xA$, pois $x0=0$.

Sejam $a, b \in xA$. Então, existem $a_1, a_2 \in A$ tais que $a=xa_1$ e $b=xa_2$. Assim, $a+b=xa_1+xa_2=x(a_1+a_2) \in xA$.

Finalmente, seja $a \in A$ e $b \in xA$. Então, existe $b_1 \in A$ tal que $b=xb_1$. Assim, $ab=(xa)b_1=x(ab_1) \in xA$.
\end{proof}
\begin{definition}[Ideal principal]
    Seja $A$ um anel. Para todo $x \in A$, o conjunto $xA=\{xa:a \in A\}$ é um ideal à esquerda de $A$. O ideal $xA$ é chamado de \emph{ideal principal à esquerda gerado por $x$}.
    Analogamente, o conjunto $Ax=\{ax:a \in A\}$ é um ideal à direita de $A$, e é chamado de \emph{ideal principal à direita gerado por $x$}.
    Se $A$ é comutativo, o ideal $xA=Ax$ é chamado de \emph{ideal principal gerado por $x$}.
\end{definition}

Observação: note que, comparando as definições, se $A$ é um anel comutativo com unidade, $xA=\langle x\rangle$.

Notemos que ideais triviais são principais à esquerda e à direita, pois $0A=\{0\}=A0$ e $A1=A=1A$.

\begin{definition}[Domínio de ideais principais]
    Um domínio de ideais principais (DIP), ou anel principal, é um domínio de integridade $A$ tal que todo ideal de $A$ é principal.
\end{definition}

\begin{prop}[Ideais de um corpo são triviais]
    Todo ideal de um corpo é trivial. Em particular, todo corpo é um DIP. Reciprocamente, se $A$ é um anel comutativo não trivial cujo todo ideal é trivial, então $A$ é um corpo.
\end{prop}
\begin{proof}
Seja $K$ um corpo e $I$ um ideal de $K$. Se $I=\{0\}$, então $I$ é trivial. Se $I\neq \{0\}$, então existe $a \in I$ tal que $a \neq 0$. Daí $1=a^{-1}a=\in I$. Logo, para todo $k \in K$, $k=1k\in I$.

Para a recíproca, seja $A$ um anel comutativo não trivial tal que todo ideal de $A$ é trivial, e fixe $x \in A\setminus \{0\}$. Como $Ax$ é um ideal trivial e $0\neq x \in Ax$, temos que $Ax=A$. Logo, existe $a \in A$ tal que $ax=1$. Assim, $x$ é invertível. Portanto, $A$ é um corpo.
\end{proof}

\begin{prop}[Um DIP que não é um corpo] O anel dos inteiros $\mathbb Z$ é um domínio de ideais principais que não é um corpo.
\end{prop}
\begin{proof}
    Seja $I$ um ideal de $\mathbb Z$.
    Veremos que $I$ é um ideal principal.
    Se $I=\{0\}$, então $I$ é principal.
    Caso contrário, $I$ contém ao menos um elemento positivo, já que, sendo $x\in I\setminus\{0\}$, temos que $-x \in I$ e um dos $x, -x$ é positivo.

    Seja $n$ o menor inteiro positivo de $I$.
    Afirmamos que $I=n\mathbb Z$.
    De fato, se $x \in I$, então escreva $x=qn+r$, onde $q,r \in \mathbb Z$ e $0\leq r<n$.
    Como $x \in I$, temos que $r=x-qn \in I$. Assim, $r=0$, ou violaríamos a minimalidade de $n$.
    Logo, $x=qn\in n\mathbb Z$.
    Portanto, $I\subseteq n\mathbb Z$.
    Como $n\mathbb Z=\langle n\rangle$ e $n \in I$, temos que $n\mathbb Z\subseteq I$, o que completa a prova.
\end{proof}

\section{Quocientes}
\begin{definition}
    Seja $A$ um anel. Uma relação de congruência em $A$ é uma relação de equivalência $\sim$ em $A$ que ``preserva operações''. Explcitamente, tal que para todos $a, b, c, d \in A$, se Se $a\sim b$ e $c\sim d$, então $a+c\sim b+d$ e $ac\sim bd$.
\end{definition}

Quais são todas as relações de congruência em $A$? A proposição abaixo classifica-as a partir dos ideais de $A$.
\begin{prop}[Relações de congruência vs ideais]
    Seja $A$ um anel, $\mathcal R(A)$ o conjunto de todas as relações de congruência em $A$ e $\mathcal I(A)$ o conjunto de todos os ideais de $A$. Então, existe uma bijeção entre $\mathcal R(A)$ e $\mathcal I(A)$ dada por
    $\sim \mapsto I_{\sim}=\{a \in A: a\sim 0\}$,
    cuja inversa se dá por $I\mapsto \sim_I=\{(a, b) \in A^2: a-b \in I\}$.
\end{prop}
\begin{proof}
Primeiro, vejamos que se $\sim$ é uma relação de congruência, então $I_\sim$ é um ideal de $A$.

\begin{itemize}
\item $0 \in I_\sim$, pois $0\sim 0$.
\item Se $a, b \in I_\sim$, então $a\sim 0$ e $b\sim 0$, logo $a+b\sim 0+0=0$, portanto, $a+b \in I_\sim$.
\item Se $x \in A$ e $a \in I_\sim$, então $a\sim 0$ e $x\sim 0$, logo $ax\sim a0=0$ e $xa=0a=0$, portanto, $ax, xa \in I_\sim$.
\end{itemize}

Agora, vejamos que se $I$ é um ideal, então $\sim_I$ é uma relação de congruência. De fato, temos que, para todos $a, b, c, d \in A$:
\begin{itemize}
    \item $a\sim_I a$ pois $a-a=0\in I$.
    \item Se $a\sim_I b$, então $a-b \in I$, logo $(-1)(a-b)=b-a\in I$, e, portanto, $b\sim_I a$.
    \item Se $a\sim_I b$ e $b\sim_I c$, então $a-b \in I$ e $b-c \in I$, logo, $(a-b)+(b-c)=a-c \in I$, portanto, $a\sim_I c$.
    \item Se $a\sim_I b$ e $c\sim_I d$, então $a-b \in I$ e $c-d \in I$, logo, $(a-b)+(c-d)=(a+c)-(b+d)\in I$, portanto, $a+c\sim_I b+d$.
    \item Se $a\sim_I b$ e $c\sim_I d$, então $a-b \in I$ e $c-d \in I$, logo, $(a-b)c=ac-bc\in I$ e $b(c-d)=bc-bd\in I$, logo $(ac-bc)+(bc-bd)=ac-bd\in I$, portanto, $ac\sim_I bd$.
    \end{itemize}

Se $I$ é ideal, $I_{\sim_I}=I$, pois, para todo $a\in A$:

$$a\in I_{\sim_I}\Leftrightarrow a\sim_I 0\Leftrightarrow a-0\in I\Leftrightarrow a\in I.$$

Finalmente, se $\sim$ é relação de congruência, $\sim_{I_\sim}=\sim$, pois, para todos $a, b \in A$:

$$a\sim_{I_\sim} b\Leftrightarrow a-b\in I_\sim \Leftrightarrow a-b\sim 0\Leftrightarrow a\sim b.$$

Justificando a última equivalência: se $a-b\sim 0$, como $b\sim b$, temos que $a-b+b\sim b$, ou seja, que $a\sim b$. Reciprocamente, se $a\sim b$, como $(-b)\sim (-b)$, segue que $a+(-b)\sim b+(-b)$, ou seja, que $a-b\sim 0$.
\end{proof}

Como feito nos inteiros, podemos, ao invés de trabalhar com relações de congruência, encontrar anéis em que a congruência corresponda exatamente à igualdade.

\begin{definition}
Seja $A$ um anel e $\sim$ uma relação de congruência. Define-se que $A/\sim$ é $A/\sim=\{[a]_\sim: a \in A\}$, onde $[a]_\sim=\{b\in A: b\sim a\}$ é a classe de equivalência de $a$ com relação a $\sim$.

Define-se que $[a]_\sim+[b]_\sim=[a+b]_\sim$ e que $[a]_\sim[b]_\sim=[ab]\sim$.

Se $I$ é um ideal, $A/I=A/\sim_I$, e o mapa quociente de $A$ em $A/I$ se dá por $q:A\longrightarrow A/I$ dada por $q(a)=[a]_{\sim_I}$.
\end{definition}

Pelas propriedades das relações de congruência, a soma e produto de $A/\sim$ (ou $A/I$) estão bem definidas. Além disso:

\begin{lemma}[Propriedades do quociente]
    Na notação acima:
    \begin{enumerate}[label=\alph*)]
        \item $q$ é epimorfismo de anéis. \label{lemma:propriedadesQuociente_a}
        \item $\ker q = I$. \label{lemma:propriedadesQuociente_b}
        \item $q(a)=a+I=\{a+x: x \in I\}$ para todo $a \in A$. \label{lemma:propriedadesQuociente_c}
        \item Se $A$ é anel comutativo, $A/I$ também é. \label{lemma:propriedadesQuociente_d}
    \end{enumerate}
\end{lemma}

\begin{proof}
    \ref{lemma:propriedadesQuociente_a} Seja $a, b, c, d \in A$. Temos que $q(a+b)=q(a)+q(b)$ e $q(ab)=q(a)q(b)$ por definição da soma em $A/I$, e $q$ é sobrejetora pela definição de $q$. Finalmente, $q(1_A)$ é identidade pois para todo $a \in A$, $q(1_A)q(a)=q(1_Aa)=q(a)$ e $q(a)q(1_A)=q(a1_A)=q(a)$, logo, $q(1_A)=1_{A/I}$.

    \ref{lemma:propriedadesQuociente_b} Temos que $\ker q=\{a \in A: q(a)=q(0)\}=\{a \in A: a\sim_I 0\}=\{a \in A: a\in I\}=I$.

    \ref{lemma:propriedadesQuociente_c} Temos que $q(a)=[a]_{\sim_I}=\{b \in A: b\sim_I a\}=\{b \in A: b-a\in I\}=\{a+x: x \in I\}$ pois se $b-a \in I$ se, e somente se $a-b=x$ para algum $x \in I$.

    \ref{lemma:propriedadesQuociente_d} Se $A$ é comutativo, então $A/I=\ran q$ também é, pois $q$ é homomorfismo de anéis.
\end{proof}

\section{Teoremas do isomorfismo}
\begin{theorem}[Teorema do homomorfismo]
    Seja $f:A\rightarrow R$ um homomorfismo de anéis e $J$ um ideal tal que $J\subseteq \ker f$. Então, existe um único homomorfismo de anéis $g:A/J\rightarrow R$ tal que $g\circ q=f$, onde $q:A\rightarrow A/J$ é o mapa quociente canônico dado por $q(a)=a+J$.
\end{theorem}
\begin{proof}
    Definimos $g:A/J\rightarrow R$ por $g(a+J)=f(a)$. Então, $g$ é bem definido, pois se $a+J=b+J$, então $a-b \in J\subseteq \ker f$, logo, $f(a-b)=0_R$, ou seja, $f(a)=f(b)$.

    Agora, vejamos que $g$ é um homomorfismo de anéis. De fato, para todo $a', b' \in A/J$, sendo $a'=a+J$ e $b'=b+J$, temos que:
    \begin{itemize}
        \item $g(a'+b')=g((a+J)+(b+J))=g((a+b)+J)=f(a+b)=f(a)+f(b)=g(a+J)+g(b+J)$.
        \item $g(a'b')=g((a+J)(b+J))=g(ab+J)=f(ab)=f(a)f(b)=g(a+J)g(b+J)$.
        \item $g(1_{A/J})=g(1_A+J)=f(1_A)=1_R$.
    \end{itemize}
\end{proof}

\begin{theorem}[Primeiro Teorema do Isomorfismo]
    Seja $f:A\rightarrow R$ um homomorfismo de anéis. Então, existe um único homomorfismo de anéis $g:A/\ker f\rightarrow R$ tal que $g\circ q=f$, onde $q:A\rightarrow A/J$ é o mapa quociente canônico dado por $q(a)=a+J$, e $g:A\ker f\rightarrow \ran f$ é isomorfismo.
\end{theorem}
\begin{proof}
    Pelo Teorema do homomorfismo com $J=\ker f$, existe um único homomorfismo de anéis $g:A/\ker f\rightarrow \ran f$ tal que $g\circ q=f$. Como $g\circ q=f$ e $q$ é sobrejetora, então $\ran g=\ran (g\circ q)=\ran f$, logo, $q$ é sobre $\ran f$.
    
    Resta ver que $g$ é injetora. De fato, seja $q(a)\in A/\ker f$ tal que $g(q(a))=0_R$. Então, $f(a)=0_R$, logo, $a\in \ker f=J$. ou seja, $a\sim_I 0$, logo $q(a)=q(0)=0_{A/\ker f}$. Assim, $g$ é injetora.
\end{proof}
\chapter{Domínios de Integridade}
Neste capítulo, exploraremos com mais detalhes os domínios de integridade e a teoria que nasce deles.

\section{Relações entre corpos e domínios de integridade}
Conforme visto, todo corpo é um domínio de integridade, e a recíproca não é verdadeira (sendo $\mathbb Z$ um contra-exemplo).

A seguir, apresentaremos algumas relações entre corpos e domínios de integridade.

\begin{prop}
Todo domínio de integridade finito é um corpo.
\end{prop}
\begin{proof}
    Seja $R$ um domínio de integridade finito.
    Fixe $a \in R\setminus\{0\}$.
    Veremos que $a$ é invertível.

    Considere $\phi:R\setminus\{0\}\rightarrow R\setminus \{0\}$ dado por $\phi(x)=ax$.

    Como $R$ é um domínio de integridade, para todo $x \in R\setminus \{0\}$, temos $ax \neq 0$, logo, $\phi$ está bem definida.

    $\phi$ é uma função injetora: se $\phi(x)=\phi(y)$, então $ax=ay$.
    Logo, $a(x-y)=0$.
    Como $a\neq 0$ e $R$ é um domínio de integridade, segue que $x-y=0$, ou seja, $x=y$.

    Como $R\setminus\{0\}$ é finito e $\phi:R\setminus\{0\}\rightarrow R\setminus \{0\}$ é injetora, segue que $\phi$ é sobrejetora.
    Em particular, existe $x \in X$ tal que $ax=\phi(x)=1$.
    Logo, $a$ é invertível.
\end{proof}

Portanto, restrito aos anéis finitos, o estudo dos corpos e domínios de integridade colapsa em um único estudo.

Outra relação importante é a que segue:

\begin{prop}
Seja $R$ um anel comutativo e $I$ um ideal próprio de $R$. São equivalentes:

\begin{enumerate}[label=(\roman*)]
    \item $R/I$ é um corpo;
    \item $I$ é maximal.
\end{enumerate}
\end{prop}

\begin{proof}
    Seja $q:R\rightarrow I$ o mapa quociente.

    (i) $\Rightarrow$ (ii): Suponha que $R/I$ é um corpo.
    

    $I$ é um ideal próprio, caso contrário, teríamos que $R/I$ é o anel trivial, que não é um corpo.

    Agora suponha que $J$ é um ideal que contém $I$ propriamente.
    Veremos que $J=R$.
    Seja $a \in J\setminus I$.
    Como $a\notin I$, temos que $q(a)\neq 0$.
    Como $A/I$ é um corpo, existe $b \in R$ tal que $q(a)q(b)=1$.
    Isso implica que existe $x \in I$ tal que $ab+x=1$.
    Como $a \in J$ e $x \in I\subseteq J$, segue que $1=ab+x\in J$, e, portanto, $J=R$.

    (ii) $\Rightarrow$ (i): Suponha que $I$ é maximal. Vejamos que $R/I$ é um corpo.

    Seja $x \in R\setminus I$ não nulo.
    Tome $a \in R$ tal que $q(a)=x$.
    Temos que $a \notin I$.
    Como $I+\langle a\rangle$ é um ideal que contém $I$ propriamente, segue que $I+\langle a\rangle=R$.
    Logo, existe $b \in R$ e $c \in I$ tais que $c+ba=1$.
    Logo, $q(1)=q(c)+q(ba)=0+q(b)q(a)=q(b)x$.
    Portanto, $x$ é invertível.
\end{proof}

Será que podemos caracterizar, de forma análoga, ser um domínio de integridade? A resposta é positiva.

\begin{prop}
    Seja $R$ um anel comutativo e $I$ um ideal próprio de $R$. São equivalentes:
    
    \begin{enumerate}[label=(\roman*)]
        \item $R/I$ é um domínio de integridade.
        \item $I$ é primo.
    \end{enumerate}
\end{prop}

\begin{proof}
    Seja $q:R\rightarrow I$ o mapa quociente.

    (i) $\Rightarrow$ (ii): Suponha que $R/I$ é um domínio de integridade.
    

    $I$ é um ideal próprio, caso contrário, teríamos que $R/I$ é o anel trivial, que não é um domínio de integridade.

    Suponha que $a,b \in R$ tais que $ab \in I$.
    Temos que $q(a)q(b)=q(ab)=0$.
    Como $R/I$ é um domínio de integridade, temos que $q(a)=0$ ou $q(b)=0$, ou seja, que $a \in I$ ou $B \in I$.

    Logo, $I$ é primo.

    (ii) $\Rightarrow$ (i): Suponha que $I$ é primo.
    Vejamos que $R/I$ é um domínio de integridade.

    Sejam $x, y \in R$ tais que $q(x)q(y)=0$.
    Devemos ver que $q(x)=0$ ou $q(y)=0$.
    Como $q(xy)=q(x)q(y)=0$, segue que $xy\in I$.
    Então, $x\in I$ ou $y\in I$, ou seja, $q(x)=0$ ou $q(y)=0$.
\end{proof}

Como consequência, temos:

\begin{corol}
    Seja $R$ um anel comutativo finito e $I$ um ideal de $R$. Então $I$ é primo se, e somente se $I$ é maximal.
\end{corol}
\begin{proof}
    Temos que $R/I$ é finito, e, portanto, é um corpo se, e somente se for um domínio de integridade.
    Portanto:

    \[I \text{ é primo} \Leftrightarrow R/I \text{ é um domínio de integridade} \Leftrightarrow R/I \text{ é um corpo} \Leftrightarrow I \text{ é maximal}\]
\end{proof}

\section{O corpo de frações de um domínio de integridade}

Conforme vimos, nem todo domínio de integridade é um corpo, sendo $\mathbb Z$ é o contra-exemplo mais usual.
Apesar disso, parece que, em algum sentido, $\mathbb Q$ é o ``menor'' corpo que contém $\mathbb Z$.

Uma das construções mais usuals do corpo $\mathbb Q$ utiliza classes de equivalências de pares de elementos de $\mathbb Z$.
Nesta seção, estudaremos esta construção de modo generalizado.

Iniciaremos apresentando uma construção do corpo de frações.

\begin{definition}
    Seja $R$ um domínio de integridade.

    Definamos, em $R\times \{0\}$, a relação de equivalência $\sim$ a seguir:

    \[(a, b) \sim (c, d) \Leftrightarrow ad=bc.\]
\end{definition}

Ao longo desta seção, a notação $\sim$ será fixada e utilizada exclusivamente para esse fim.
A ideia é pensar em cada par $(a, b)$ como uma fração $\frac{a}{b}$.
A relação $\sim$ captura a ideia que duas frações $\frac{a}{b}$ e $\frac{c}{d}$ são equivalentes se, e somente se, $ad=bc$.

\begin{lemma}
    Na notação acima, a relação $\sim$ é uma relação de equivalência em $R\times \{0\}$.
\end{lemma}

\begin{proof}
    Seja $(a, b), (c, d), (e, f) \in R\times \{0\}$.
    \begin{itemize}
        \item Temos que $(a, b)\sim (a, b)$ pois $ab=ba$.
        \item Simetria: se $(a, b)\sim (c, d)$, temos que $ad=bc$.
        Logo, $cb=da$, o que nos dá $(c, d)\sim (a, b)$.
        \item Transitividade: suponha que $(a, b)\sim (c, d)$ e $(c, d)\sim (e, f)$.
        Temos que $ad=bc$ e $cf=de$.
        Multiplicando a primeira equação por $f$ e a segunda por $b$, temos que $adf=bcf$ e $bcf=deb$.
        Logo, $adf=deb$.
        Como $d\neq 0$, cancelando, temos que $af=eb$, ou seja, que $(a, b)\sim (e, f)$.
    \end{itemize}
\end{proof}
Assim, podemos definir:

\begin{definition}
    O conjunto das classes de equivalência $(R\times R\setminus\{0\})/\sim$ será denotado por $\Frac(R)$.
    
    A classe de equivalência de um par $(a, b)$ será denotada por $\frac{a}{b}$
\end{definition}

Observe que agora, formalmente, $\frac{a}{b}=\frac{c}{d}$ se, e somente se, $ad=bc$.

Porém, a igualdade $a=\frac{a}{1}$ não faz sentido e será discutida mais adiante.

Agora, definiremos as operações em $\Frac(R)$.

\begin{definition}
    Seja $R$ um domínio de integridade. Define-se, em $\Frac(R)$, as operações a seguir:
    \begin{itemize}
        \item Soma: $\displaystyle\frac{a}{b}+\frac{c}{d}=\frac{ad+bc}{bd}$.
        \item Produto: $\displaystyle\frac{a}{b}\cdot\frac{c}{d}=\frac{ac}{bd}$.
    \end{itemize}
\end{definition}

O próximo passo é mostrar que tais operações estão bem definidas.

\begin{lemma}
    Na notação anterior, a soma e o produto de frações estão bem definidas.
\end{lemma}
\begin{proof}
    Consideremos $a, b, a', b', c, d, c', d'\in R$ tais que $b, b', d, d'\neq 0$ e tais que $\frac{a}{b}=\frac{a'}{b'}$ e $\frac{c}{d}=\frac{c'}{d'}$.
    Assim, sabemos que $ab'=a'b$ e $cd'=c'd$.

    Devemos ver que $\frac{ad+bc}{bd}=\frac{a'd'+b'c'}{b'd'}$ e $\frac{ac}{bd}=\frac{a'c'}{b'd'}$.

    Começaremos pela segunda afirmação.
    
    Queremos provar que $acb'd'=a'c'bd$.
    Temos:

    \[acb'd'=(ab')(cd')=(a'b)(c'd)=a'c'bd.\]

    Agora, para a soma, temos que provar que $adb'd'+bcb'd'=a'd'bd+b'c'bd$.
    Multiplicando a equação $ab'=a'b$ por $d'd$, a equação $cd'=c'd$ por $b'b$, e somando, segue a tese.
\end{proof}
\section{Exercícios}
\begin{exer}
    Demonstre, com suas próprias palavras, de modo que considere satisfatório, a seguinte afirmação demonstrada no texto: Todo domínio de integridade finito é um corpo.
\end{exer}
\backmatter
\end{document} 