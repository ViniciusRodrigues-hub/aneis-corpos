\chapter{Homomorfismos e Ideais}
Em matemática, boa parte das coleções de estruturas estudadas possui uma classe de funções que preservam, em algum sentido, suas propriedades.
O estudo generalizado destas estruturas é o que chamamos de \emph{teoria de categorias}, tema que não será tratado neste texto.
Na classe dos anéis, estas funções são o que chamamos de \emph{homomorfismos}.
\section{Definição de homomorfismo}
Homomorfismos são funções que preservam a estrutura de anéis.
Formalmente:
\begin{definition}
Sejam $A$, $R$ aneis.
Uma função $f:A\rightarrow R$ é um \emph{homomorfismo} se:
\begin{itemize}
    \item $f(a+b)=f(a)+f(b)$ para todo $a, b \in A$.
    \item $f(-a)=-f(a)$ para todo $a \in A$.
    \item $f(0_A)=0_R$
    \item $f(ab)=f(a)f(b)$ para todo $a, b \in A$.
    \item $f(1_A)=1_R$.
\end{itemize}

Caso $f$ seja injetora, dizemos que $f$ é um \emph{monomorfismo}.
Caso $f$ seja sobrejetora, dizemos que $f$ é um \emph{epimorfismo}.
Caso $f$ seja bijetora, dizemos que $f$ é um \emph{isomorfismo}.
\end{definition}

A noção de isomorfismo é extremamente importante na Teoria de Anéis. Muitas vezes, temos dois anéis que ``deveriam ser a mesma coisa'', mas, como objetos matemáticos, não são iguais. A noção de isomorfismo entra em campo para dizer que, mesmo que dois anéis não sejam o mesmo objeto, eles possuem exatamente as mesmas propriedades algébricas e operacionais. Para darmos um exemplo concreto:

\begin{exemplo}
Seja $A=\{0, 1\}$ e $R=\{Z, U\}$, onde $Z, U$ são objetos diferentes, e diferentes de $0, 1$. Defina em $A$ as operações $\cdot$ e $+$ dadas pelas seguintes tabelas:

Em $A$:
\begin{multicols}{2}\centering
    \begin{tabular}{c|cc}
        $+$ & 0 & 1 \\ \hline
        0 & 0 & 1 \\
        1 & 1 & 0 \\
    \end{tabular}

    \begin{tabular}{c|cc}
        $\cdot$ & 0 & 1 \\ \hline
        0 & 0 & 0 \\
        1 & 0 & 1 \\
    \end{tabular}
\end{multicols}

Em $R$:
\begin{multicols}{2}\centering
    \begin{tabular}{c|cc}
        $+$ & Z & U \\ \hline
        Z & Z & U \\
        U & U & Z \\
    \end{tabular}

    \begin{tabular}{c|cc}
        $\cdot$ & Z & U \\ \hline
        Z & Z & Z \\
        U & Z & U \\
    \end{tabular}
\end{multicols}

Intuitivamente, $A$ e $R$ correspondem a duas apresentações de uma mesma estrutura algébrica, porém, como $A\cap R=\emptyset$, estes dois anéis não são o mesmo anel.
Como formalizar este fato?
Ora, há uma relação biunívoca (uma bijeção) entre $A$ e $R$ que preserva suas operações, e ela é dada por $\phi(0)=Z$ e $\phi(1)=U$.
Tal $\phi$ é um isomorfismo.
\end{exemplo}

Para todos os fins que interessam à Álgebra, anéis isomorfos tem exatamente as mesmas propriedades, e, assim, são considerados como sendo, em algum sentido, a mesma estrutura.

A definição de homomorfismo, por possuir várias cláusulas, pode parecer de longa verificação.
A proposição abaixo encurta esta verificação substancialmente.

\begin{prop}
    Sejam $A, R$ anéis e $f:A\rightarrow R$ uma função.
    Então $f$ é um homomorfismo se, e somente se:
    \begin{itemize}
        \item $f(a+b)=f(a)+f(b)$ para todo $a, b \in A$.
        \item $f(ab)=f(a)f(b)$ para todo $a, b \in A$.
        \item $f(1_A)=1_R$.
    \end{itemize}
\end{prop}
\begin{proof}
    Provaremos o lado que não é imediatamente trivial.
    Começaremos mostrando que $f(0_A)=0_R$.
    Temos que $f(0_A)=f(0_A+0_A)=f(0_A)+f(0_A)$, logo, cancelando, $f(0_A)=0_R$.

    Agora, vejamos que $f(-a)=-f(a)$ para todo $a \in A$.
    Temos que $f(a)+f(-a)=f(a+(-a))=f(0_A)=0_R$, logo, $f(-a)=-f(a)$.

    Assim, $f$ é um homomorfismo.
\end{proof}

\section{Propriedades elementares}
\begin{lemma}
    Sejam $f:A\rightarrow R$ e $g:R\rightarrow S$ homomorfismos de anéis.
    Então a composição $g\circ f:A\rightarrow S$ é um homomorfismo de anéis.
\end{lemma}

\begin{proof}
    Sejam $a, b \in A$. Então:
    \begin{itemize}
        \item $g\circ f(a+b)=g(f(a+b))=g(f(a)+f(b))=g(f(a))+g(f(b))=(g\circ f)(a)+(g\circ f)(b)$.
        \item $g\circ f(ab)=g(f(ab))=g(f(a)f(b))=g(f(a))g(f(b))=(g\circ f)(a)(g\circ f)(b)$.
        \item $g\circ f(1_A)=g(f(1_A))=g(1_R)=1_S$.
    \end{itemize}
    Assim, $g\circ f$ é um homomorfismo de anéis.
\end{proof}

\begin{prop}[Propriedades de homomorfismos]
    Seja $f:A\rightarrow R$ um homomorfismo de anéis. Então:
    \begin{enumerate}[label=\alph*)]
        \item Para todo $a \in A^*$, temos $f(a)\in  R^*$ e $f(a^{-1})=f(a)^{-1}$. \label{prop:homomorfismo_a}
        \item A imagem de $f$, $\ran f=\{f(a): a \in A\}$, é um subanel de $R$. Se $A$ é comutativo, $\ran f$ também é.  \label{prop:homomorfismo_b}
        \item Se $f$ é injetora, a imagem de $f$ é um subanel de $R$ isomorfo a $A$. \label{prop:homomorfismo_c}
    \end{enumerate}
\end{prop}
\begin{proof}
\ref{prop:homomorfismo_a} Se $a \in A^*$, então $f(a)f(a^{-1})=f(aa^{-1})=f(1_A)=1_R$ e $f(a^{-1})f(a)=f(aa^{-1})=f(1_A)=1_R$. Assim, $f(a^{-1})=f(a)^{-1}$ e $f(a)\in R^*$.

\ref{prop:homomorfismo_b} Seja $a, b \in \ran f$. Então existem $x, y \in A$ tais que $a=f(x)$ e $b=f(y)$.
Assim, $a-b=f(x)-f(y)=f(x-y)$. Logo, $a-b \in \ran f$.
Similarmente, $ab=f(x)f(y)=f(xy)\in \ran f$, e $1_R=f(1_A)\in \ran f$.

Portanto, $\ran f$ é um subanel de $R$.
Se $A$ é comutativo, $\ran(f)$ também é comutativo, pois dados $a, b \in \ran f$, existem $x, y \in A$ tais que $a=f(x)$ e $b=f(y)$.
Assim, $ab=f(x)f(y)=f(xy)=f(yx)=f(y)f(x)=ba$.

\ref{prop:homomorfismo_c} Se $f$ é injetora, então $f$ é bijetora entre $A$ e $\ran f$. Assim, $f$ é um isomorfismo entre $A$ e $\ran f$, dado que é um homomorfismo.
\end{proof}

A noção de isomorfismo é uma relação de equivalência na classe dos anéis.

\begin{prop}[Propriedades de isomorfismo]
    Sejam $A, R, S$ anéis e $f:A\rightarrow R$ e $g:R\rightarrow S$ isomorfismos de anéis.
    Então:
    \begin{enumerate}[label=\alph*)]
        \item $g\circ f$ é um isomorfismo de anéis. \label{prop:isomorfismo_a}
        \item $f^{-1}:R\rightarrow A$ é um isomorfismo de anéis. \label{prop:isomorfismo_b}
        \item $\id_A:A\rightarrow A$ é um isomorfismo de anéis. \label{prop:isomorfismo_c}
    \end{enumerate}
\end{prop}

\begin{proof}
\ref{prop:isomorfismo_a} A composição de funções bijetoras é bijetora, e a composição de homomorfismos é homomorfismo.
Como um isomorfismo é um homomorfismo bijetor, segue que a composição de dois isomorfismos é um isomorfismo.

\ref{prop:isomorfismo_b} Como $f$ é um isomorfismo, $f$ é bijetora, assim, $f^{-1}:R\rightarrow A$ está bem definida e é bijetora. Verificaremos que $f^{-1}$ é um homomorfismo. Dados $r, s \in R$, sejam $a, b \in A$ tais que $f(a)=r$ e $f(b)=s$. Temos que:
\begin{itemize}
    \item $f^{-1}(r+s)=f^{-1}(f(a)+f(b))=f^{-1}(f(a+b))=a+b=f^{-1}(r)+f^{-1}(s)$.
    \item $f^{-1}(rs)=f^{-1}(f(a)f(b))=f^{-1}(f(ab))=a\cdot b=f^{-1}(r)f^{-1}(s)$.
    \item $f^{-1}(1_R)=f^{-1}(f(1_A))=1_A$.
\end{itemize}

\ref{prop:isomorfismo_c} A função identidade $\id_A$ é claramente bijetora, e é um homomorfismo, pois, para todos $a, b \in A$:
    \begin{itemize}
        \item $\id_A(a+b)=a+b=\id_A(a)+\id_A(b)$.
        \item $\id_A(ab)=ab=\id_A(a)\id_A(b)$.
        \item $\id_A(1_A)=1_A$.
    \end{itemize}

    \section{Ideais}
\begin{definition}[Ideal à esquerda]
    Seja $A$ um anel. Um subconjunto $I \subseteq A$ é dito \emph{ideal à esquerda} se:
    \begin{itemize}
        \item $0 \in I$.
        \item Para todos $a, b \in I$, temos $a+b\in I$.
        \item $\forall a \in A$ e $\forall b \in I$, temos $ab \in I$.
    \end{itemize}
\end{definition}

\begin{definition}[Ideal à direita]
    Seja $A$ um anel. Um subconjunto $I \subseteq A$ é dito \emph{ideal à direita} se:
    \begin{itemize}
        \item $0 \in I$.
        \item Para todos $a, b \in I$, temos $a+b\in I$.
        \item $\forall a \in I$ e $\forall b \in A$, temos $ab \in I$.
    \end{itemize}
\end{definition}

\begin{definition}[Ideal]
    Seja $A$ um anel. Um subconjunto $I \subseteq A$ é dito \emph{ideal} se for um ideal à esquerda e um ideal à direita. Ou seja, $I$ é um ideal se:
    \begin{itemize}
        \item $0 \in I$.
        \item Para todos $a, b \in I$, temos $a+b\in I$.
        \item $\forall a \in A$ e $\forall b \in I$, temos $ab \in I$.
        \item $\forall a \in I$ e $\forall b \in A$, temos $ab \in I$.
    \end{itemize}
\end{definition}

\begin{prop}[Ideal trivial]Seja $A$ um anel. Então $\{0\}$ e $A$ são ideais de $A$. Estes ideais são chamados de \emph{ideais principais}
\end{prop}
\begin{proof}
    Exercício.
\end{proof}

Note que se $A$ é um anel comutativo, então $I$ é um ideal à esquerda se, e somente se, $I$ é um ideal à direita. Assim, em anéis comutativos, a noção de ideal é equivalente à de ideal à esquerda ou à de ideal à direita.

\begin{prop}[Interseção de ideais]
    Seja $A$ um anel e $\mathcal F$ uma coleção não vazia de ideais à esquerda de $A$. Então $\bigcap_{I \in \mathcal F}I=\bigcap \mathcal F$ é um ideal de $A$. O mesmo vale para ideais à direita e ideais.
\end{prop}
\begin{proof}
    Provaremos para ideais à esquerda. A prova para ideais à direita é análoga e fica como exercício.

    Seja $I=\bigcap \mathcal F$. Então $0 \in I$, pois $0 \in I$ para todo $I \in \mathcal F$.

    Sejam $a, b \in I$. Então, para todo $I \in \mathcal F$, temos que $a, b \in I$, logo, $a+b\in I$. Assim, $a+b\in \bigcap \mathcal F$.

    Finalmente, seja $a \in A$ e $b \in I$. Então, para todo $I \in \mathcal F$, temos que $b \in I$, logo, $ab\in I$. Assim, $ab\in \bigcap \mathcal F$.
\end{proof}

\begin{prop}[Ideal gerado]
    Seja $A$ um anel comutativo e $B\subseteq A$ um conjunto não vazio. Então, o conjunto $I=\{a_1b_1+\cdots+a_nb_n: n\geq 1, a_i \in A, b_i \in B\}$ é o menor ideal à esquerda $A$ que contém $B$ (ou seja, além de ser um ideal contendo $B$, se $J$ é qualquer ideal contendo $B$, então $I\subseteq J$). O ideal $I$ é chamado de \emph{ideal gerado por $B$}, e denotado por $\langle B \rangle$.
    
    Se $B=\{x_0, \dots, x_n\}$, então abreviamos $\langle B \rangle$ como $\langle x_0, \dots, x_n \rangle$.
\end{prop}
\begin{proof}
    Primeiro, verificaremos que $I$ é um ideal.

    $0 \in I$, pois $0=0b$ para todo $b \in B$.

    Sejam $x, y \in I$. Então existem $n, m\geq 1$, $a_1, \dots, a_n \in A$, $b_1, \dots, b_n \in B$, $c_1, \dots, c_m \in A$ e $d_1, \dots, d_m \in B$ tais que $x=a_1b_1+\cdots+a_nb_n$ e $y=c_1d_1+\cdots+c_md_m$. Assim, $x+y=(a_1b_1+\cdots+a_nb_n)+(c_1d_1+\cdots+c_md_m)=(a_1b_1+\cdots+a_nb_n)+(c_1d_1+\cdots+c_md_m) \in I$.

    Finalmente, seja $a \in A$ e $b \in I$. Então existem $n\geq 1$, $a_1, \dots, a_n \in A$ e $b_1, \dots, b_n \in B$ tais que $b=a_1b_1+\cdots+a_nb_n$. Assim, $ab=(a_1b_1+\cdots+a_nb_n)a=a_1(b_1a)+\cdots+a_n(b_na) \in I$.

    Agora, seja $J$ um ideal de $A$ que contém $B$. Então, como $J$ é um ideal de $A$, temos que $\forall a_i\in A$, $\forall b_i\in B$, temos que $(a_i b_i)\in J$. Logo, $I\subseteq J$. Portanto, $I$ é o menor ideal de $A$ que contém $B$.
\end{proof}

Observação: note que o menor ideal contendo $B=\emptyset$ é o ideal nulo, $\{0\}$.

\begin{definition}[Ideal principal]
    Seja $A$ um anel. Para todo $x \in A$, o conjunto $xA=\{xa:a \in A\}$ é um ideal à direita de $A$. O ideal $xA$ é chamado de \emph{ideal principal à direita gerado por $x$}.
    Analogamente, o conjunto $Ax=\{ax:a \in A\}$ é um ideal à esquerda de $A$, e é chamado de \emph{ideal principal à esquerda gerado por $x$}.
\end{definition}
\begin{proof}
Mostraremos que $xA$ é um ideal à direita. As demais afirmações ficam como exercício.

Note que $0 \in xA$, pois $x0=0$.

Sejam $a, b \in xA$. Então, existem $a_1, a_2 \in A$ tais que $a=xa_1$ e $b=xa_2$. Assim, $a+b=xa_1+xa_2=x(a_1+a_2) \in xA$.

Finalmente, seja $a \in A$ e $b \in xA$. Então, existe $b_1 \in A$ tal que $b=xb_1$. Assim, $ab=(xa)b_1=x(ab_1) \in xA$.
\end{proof}
\begin{definition}[Ideal principal]
    Seja $A$ um anel. Para todo $x \in A$, o conjunto $xA=\{xa:a \in A\}$ é um ideal à esquerda de $A$. O ideal $xA$ é chamado de \emph{ideal principal à esquerda gerado por $x$}.
    Analogamente, o conjunto $Ax=\{ax:a \in A\}$ é um ideal à direita de $A$, e é chamado de \emph{ideal principal à direita gerado por $x$}.
    Se $A$ é comutativo, o ideal $xA=Ax$ é chamado de \emph{ideal principal gerado por $x$}.
\end{definition}

Observação: note que, comparando as definições, se $A$ é um anel comutativo com unidade, $xA=\langle x\rangle$.

Notemos que ideais triviais são principais à esquerda e à direita, pois $0A=\{0\}=A0$ e $A1=A=1A$.

\begin{definition}[Domínio de ideais principais]
    Um domínio de ideais principais (DIP), ou anel principal, é um domínio de integridade $A$ tal que todo ideal de $A$ é principal.
\end{definition}

\begin{prop}[Ideais de um corpo são triviais]
    Todo ideal de um corpo é trivial. Em particular, todo corpo é um DIP. Reciprocamente, se $A$ é um anel comutativo não trivial cujo todo ideal é trivial, então $A$ é um corpo.
\end{prop}
\begin{proof}
Seja $K$ um corpo e $I$ um ideal de $K$. Se $I=\{0\}$, então $I$ é trivial. Se $I\neq \{0\}$, então existe $a \in I$ tal que $a \neq 0$. Daí $1=a^{-1}a=\in I$. Logo, para todo $k \in K$, $k=1k\in I$.

Para a recíproca, seja $A$ um anel comutativo não trivial tal que todo ideal de $A$ é trivial, e fixe $x \in A\setminus \{0\}$. Como $Ax$ é um ideal trivial e $0\neq x \in Ax$, temos que $Ax=A$. Logo, existe $a \in A$ tal que $ax=1$. Assim, $x$ é invertível. Portanto, $A$ é um corpo.
\end{proof}

\begin{prop}[Um DIP que não é um corpo] O anel dos inteiros $\mathbb Z$ é um domínio de ideais principais que não é um corpo.
\end{prop}
\begin{proof}
    Seja $I$ um ideal de $\mathbb Z$.
    Veremos que $I$ é um ideal principal.
    Se $I=\{0\}$, então $I$ é principal.
    Caso contrário, $I$ contém ao menos um elemento positivo, já que, sendo $x\in I\setminus\{0\}$, temos que $-x \in I$ e um dos $x, -x$ é positivo.

    Seja $n$ o menor inteiro positivo de $I$.
    Afirmamos que $I=n\mathbb Z$.
    De fato, se $x \in I$, então escreva $x=qn+r$, onde $q,r \in \mathbb Z$ e $0\leq r<n$.
    Como $x \in I$, temos que $r=x-qn \in I$. Assim, $r=0$, ou violaríamos a minimalidade de $n$.
    Logo, $x=qn\in n\mathbb Z$.
    Portanto, $I\subseteq n\mathbb Z$.
    Como $n\mathbb Z=\langle n\rangle$ e $n \in I$, temos que $n\mathbb Z\subseteq I$, o que completa a prova.
\end{proof}

\end{proof}