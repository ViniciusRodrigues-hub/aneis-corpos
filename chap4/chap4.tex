\chapter{Homomorfismos e Ideais}
Em matemática, boa parte das coleções de estruturas estudadas possui uma classe de funções que preservam, em algum sentido, suas propriedades.
O estudo generalizado destas estruturas é o que chamamos de \emph{teoria de categorias}, tema que não será tratado neste texto.
Na classe dos anéis, estas funções são o que chamamos de \emph{homomorfismos}.
\section{Definição de homomorfismo}
Homomorfismos são funções que preservam a estrutura de anéis.
Formalmente:
\begin{definition}
Sejam $A$, $R$ aneis.
Uma função $f:A\rightarrow R$ é um \emph{homomorfismo} se:
\begin{itemize}
    \item $f(a+b)=f(a)+f(b)$ para todo $a, b \in A$.
    \item $f(-a)=-f(a)$ para todo $a \in A$.
    \item $f(0_A)=0_R$
    \item $f(ab)=f(a)f(b)$ para todo $a, b \in A$.
    \item $f(1_A)=1_R$.
\end{itemize}

Caso $f$ seja injetora, dizemos que $f$ é um \emph{monomorfismo}.
Caso $f$ seja sobrejetora, dizemos que $f$ é um \emph{epimorfismo}.
Caso $f$ seja bijetora, dizemos que $f$ é um \emph{isomorfismo}.
\end{definition}

A noção de isomorfismo é extremamente importante na Teoria de Anéis. Muitas vezes, temos dois anéis que ``deveriam ser a mesma coisa'', mas, como objetos matemáticos, não são iguais. A noção de isomorfismo entra em campo para dizer que, mesmo que dois anéis não sejam o mesmo objeto, eles possuem exatamente as mesmas propriedades algébricas e operacionais. Para darmos um exemplo concreto:

\begin{exemplo}
Seja $A=\{0, 1\}$ e $R=\{Z, U\}$, onde $Z, U$ são objetos diferentes, e diferentes de $0, 1$. Defina em $A$ as operações $\cdot$ e $+$ dadas pelas seguintes tabelas:

Em $A$:
\begin{multicols}{2}\centering
    \begin{tabular}{c|cc}
        $+$ & 0 & 1 \\ \hline
        0 & 0 & 1 \\
        1 & 1 & 0 \\
    \end{tabular}

    \begin{tabular}{c|cc}
        $\cdot$ & 0 & 1 \\ \hline
        0 & 0 & 0 \\
        1 & 0 & 1 \\
    \end{tabular}
\end{multicols}

Em $R$:
\begin{multicols}{2}\centering
    \begin{tabular}{c|cc}
        $+$ & Z & U \\ \hline
        Z & Z & U \\
        U & U & Z \\
    \end{tabular}

    \begin{tabular}{c|cc}
        $\cdot$ & Z & U \\ \hline
        Z & Z & Z \\
        U & Z & U \\
    \end{tabular}
\end{multicols}

Intuitivamente, $A$ e $R$ correspondem a duas apresentações de uma mesma estrutura algébrica, porém, como $A\cap R=\emptyset$, estes dois anéis não são o mesmo anel.
Como formalizar este fato?
Ora, há uma relação biunívoca (uma bijeção) entre $A$ e $R$ que preserva suas operações, e ela é dada por $\phi(0)=Z$ e $\phi(1)=U$.
Tal $\phi$ é um isomorfismo.
\end{exemplo}

Para todos os fins que interessam à Álgebra, anéis isomorfos tem exatamente as mesmas propriedades, e, assim, são considerados como sendo, em algum sentido, a mesma estrutura.

A definição de homomorfismo, por possuir várias cláusulas, pode parecer de longa verificação.
A proposição abaixo encurta esta verificação substancialmente.

\begin{prop}
    Sejam $A, R$ anéis e $f:A\rightarrow R$ uma função.
    Então $f$ é um homomorfismo se, e somente se:
    \begin{itemize}
        \item $f(a+b)=f(a)+f(b)$ para todo $a, b \in A$.
        \item $f(ab)=f(a)f(b)$ para todo $a, b \in A$.
        \item $f(1_A)=1_R$.
    \end{itemize}
\end{prop}
\begin{proof}
    Provaremos o lado que não é imediatamente trivial.
    Começaremos mostrando que $f(0_A)=0_R$.
    Temos que $f(0_A)=f(0_A+0_A)=f(0_A)+f(0_A)$, logo, cancelando, $f(0_A)=0_R$.

    Agora, vejamos que $f(-a)=-f(a)$ para todo $a \in A$.
    Temos que $f(a)+f(-a)=f(a+(-a))=f(0_A)=0_R$, logo, $f(-a)=-f(a)$.

    Assim, $f$ é um homomorfismo.
\end{proof}

\section{Propriedades elementares}
\begin{lemma}
    Sejam $f:A\rightarrow R$ e $g:R\rightarrow S$ homomorfismos de anéis.
    Então a composição $g\circ f:A\rightarrow S$ é um homomorfismo de anéis.
\end{lemma}

\begin{proof}
    Sejam $a, b \in A$. Então:
    \begin{itemize}
        \item $g\circ f(a+b)=g(f(a+b))=g(f(a)+f(b))=g(f(a))+g(f(b))=(g\circ f)(a)+(g\circ f)(b)$.
        \item $g\circ f(ab)=g(f(ab))=g(f(a)f(b))=g(f(a))g(f(b))=(g\circ f)(a)(g\circ f)(b)$.
        \item $g\circ f(1_A)=g(f(1_A))=g(1_R)=1_S$.
    \end{itemize}
    Assim, $g\circ f$ é um homomorfismo de anéis.
\end{proof}

\begin{prop}[Propriedades de homomorfismos]
    Seja $f:A\rightarrow R$ um homomorfismo de anéis. Então:
    \begin{enumerate}[label=\alph*)]
        \item Para todo $a \in A^*$, temos $f(a)\in  R^*$ e $f(a^{-1})=f(a)^{-1}$. \label{prop:homomorfismo_a}
        \item A imagem de $f$, $\ran f=\{f(a): a \in A\}$, é um subanel de $R$. Se $A$ é comutativo, $\ran f$ também é.  \label{prop:homomorfismo_b}
        \item Se $f$ é injetora, a imagem de $f$ é um subanel de $R$ isomorfo a $A$. \label{prop:homomorfismo_c}
    \end{enumerate}
\end{prop}
\begin{proof}
\ref{prop:homomorfismo_a} Se $a \in A^*$, então $f(a)f(a^{-1})=f(aa^{-1})=f(1_A)=1_R$ e $f(a^{-1})f(a)=f(aa^{-1})=f(1_A)=1_R$. Assim, $f(a^{-1})=f(a)^{-1}$ e $f(a)\in R^*$.

\ref{prop:homomorfismo_b} Seja $a, b \in \ran f$. Então existem $x, y \in A$ tais que $a=f(x)$ e $b=f(y)$.
Assim, $a-b=f(x)-f(y)=f(x-y)$. Logo, $a-b \in \ran f$.
Similarmente, $ab=f(x)f(y)=f(xy)\in \ran f$, e $1_R=f(1_A)\in \ran f$.

Portanto, $\ran f$ é um subanel de $R$.
Se $A$ é comutativo, $\ran(f)$ também é comutativo, pois dados $a, b \in \ran f$, existem $x, y \in A$ tais que $a=f(x)$ e $b=f(y)$.
Assim, $ab=f(x)f(y)=f(xy)=f(yx)=f(y)f(x)=ba$.

\ref{prop:homomorfismo_c} Se $f$ é injetora, então $f$ é bijetora entre $A$ e $\ran f$. Assim, $f$ é um isomorfismo entre $A$ e $\ran f$, dado que é um homomorfismo.
\end{proof}

A noção de isomorfismo é uma relação de equivalência na classe dos anéis.

\begin{prop}[Propriedades de isomorfismo]
    Sejam $A, R, S$ anéis e $f:A\rightarrow R$ e $g:R\rightarrow S$ isomorfismos de anéis.
    Então:
    \begin{enumerate}[label=\alph*)]
        \item $g\circ f$ é um isomorfismo de anéis. \label{prop:isomorfismo_a}
        \item $f^{-1}:R\rightarrow A$ é um isomorfismo de anéis. \label{prop:isomorfismo_b}
        \item $\id_A:A\rightarrow A$ é um isomorfismo de anéis. \label{prop:isomorfismo_c}
    \end{enumerate}
\end{prop}

\begin{proof}
\ref{prop:isomorfismo_a} A composição de funções bijetoras é bijetora, e a composição de homomorfismos é homomorfismo.
Como um isomorfismo é um homomorfismo bijetor, segue que a composição de dois isomorfismos é um isomorfismo.

\ref{prop:isomorfismo_b} Como $f$ é um isomorfismo, $f$ é bijetora, assim, $f^{-1}:R\rightarrow A$ está bem definida e é bijetora. Verificaremos que $f^{-1}$ é um homomorfismo. Dados $r, s \in R$, sejam $a, b \in A$ tais que $f(a)=r$ e $f(b)=s$. Temos que:
\begin{itemize}
    \item $f^{-1}(r+s)=f^{-1}(f(a)+f(b))=f^{-1}(f(a+b))=a+b=f^{-1}(r)+f^{-1}(s)$.
    \item $f^{-1}(rs)=f^{-1}(f(a)f(b))=f^{-1}(f(ab))=a\cdot b=f^{-1}(r)f^{-1}(s)$.
    \item $f^{-1}(1_R)=f^{-1}(f(1_A))=1_A$.
\end{itemize}

\ref{prop:isomorfismo_c} A função identidade $\id_A$ é claramente bijetora, e é um homomorfismo, pois, para todos $a, b \in A$:
    \begin{itemize}
        \item $\id_A(a+b)=a+b=\id_A(a)+\id_A(b)$.
        \item $\id_A(ab)=ab=\id_A(a)\id_A(b)$.
        \item $\id_A(1_A)=1_A$.
    \end{itemize}
\end{proof}

Agora introduziremos o núcleo de um homomorfismo.
\begin{definition}
    Seja $f: A\rightarrow R$ um homomorfismo de anéis.
    Definimos o \emph{núcleo} de $f$, também chamado de $\emph{kernel}$ de $f$, como sendo o conjunto dos zeros de $f$.
    Em símbolos:

    \[\ker f=\{a \in A: f(a)=0_R\}.\]
\end{definition}
Uma importante relação entre o homomorfismo e seu núcleo é dado como se segue:

\begin{prop}
Sejam $A, R$ anéis e $f:A\rightarrow R$ um homomorfismo. Então $f:A\rightarrow R$ é injetor (um monomorfismo) se, e somente se $\ker f = \{0_A\}$.
\end{prop}

\begin{proof}
    Primeiro, suponha que $f$ é um monomorfismo.
    Sabemos que $f(0_A)=0_R$, pois $f$ é homomorfismo, e, portanto, $\{0_A\}\subseteq \ker f$.
    Reciprocamente, seja $a \in \ker f$.
    Temos que $f(a)=0_R=f(0_A)$. Pela injetividade de $f$ segue que $a=0_A\in \{0_A\}$.

    Agora suponha que $\ker f=\{0_A\}$.
    Veremos que $f$ é injetora.
    Para tanto, sejam $a, b \in A$ e suponha que $f(a)=f(b)$.
    Temos que $f(a-b)=f(a)-f(b)=0_R$, assim, $a-b\in \ker_f = \{0_A\}$, o que implica em $a-b=0_A$, e, portanto, $a=b$.
\end{proof}
\section{Ideais}
Ideais são as estruturas responsáveis pela noção de quociente em anéis, assunto que será estudado no próximo capítulo.
Introduziremos a noção de ideal neste capítulo pois ela tem interações fundamentais com a noção de homomorfismo, porém, apenas no próximo capítulo ficará clara a sua enorme importância para esta teoria.
Nesta seção, motivaremos, nesta seção, a noção de ideal, a partir do núcleo de homomorfismos.

Para começar, notemos algumas propriedades do núcleo.

\begin{prop}
Seja $f:A\rightarrow R$ um homomorfismo de anéis. Seja $I=\ker f$. Então:

\begin{enumerate}[label=\alph*)]
    \item $0_A \in I$.
    \item Para todos $a, b \in I$, $a+b \in I$.
    \item Para todos $a \in I$ e $x \in A$, $ax \in I$.
    \item Para todos $a \in I$ e $x \in A$, $xa \in I$.
\end{enumerate}
\end{prop}

\begin{proof}
    \begin{enumerate}[label=\alph*)]
        \item $0_A \in I$ pois $f(0_A)=0_R$.
        \item Se $a, b \in I$, então $f(a)=0_R$ e $f(b)=0_R$. Assim, $f(a+b)=f(a)+f(b)=0_R+0_R=0_R$, logo, $a+b\in I$.
        \item Se $a \in I$ e $x \in A$, então $f(a)=0_R$. Assim, $f(ax)=f(a)f(x)=0_Rf(x)=0_R$, logo, $ax\in I$.
        \item Se $a \in I$ e $x \in A$, então $f(a)=0_R$. Assim, $f(xa)=f(x)f(a)=f(x)0_R=0_R$, logo, $xa\in I$.
    \end{enumerate}
\end{proof}

É possível indagar se $\ker f$ é um subanel de $A$. Observemos que as propriedades c) e d) são mais fortes do que a propriedade exigida para produto para ser um subanel. Além disso, $\ker f$ é fechado por diferenças, pois se $a, b \in \ker f$, pela propriedade d), $(-1)b=-b\in \ker f$, e, portanto, $a-b \in \ker f$. Porém, $1_A$ raramente está em $\ker f$, como vemos a seguir:

\begin{prop}
    Seja $f:A\rightarrow R$ um homomorfismo de anéis. Se $1_A \in \ker f$, então $R$ é o anel trivial, ou seja, $R=\{0_R\}$.
\end{prop}

\begin{proof}
    Se $1_A \in \ker f$, então $f(1_A)=0_R$.
    Como $f$ é um homomorfismo, temos que $f(1_A)=f(1_A\cdot 1_A)=f(1_A)f(1_A)=0_R\cdot 0_R=0_R$.
    Como $1_R=0_R$, segue que $R=\{0_R\}$, pois dado $x \in R$ temos $x=x\cdot 1_R=x\cdot 0_R=0_R$.
\end{proof}

Como recíproca, notemos que um homomorfismo acima existe para qualquer anel $A$:

\begin{prop}
    Seja $A$ um anel e $R=\{0_R\}$ um anel trivial.
    
    Então $f:A\rightarrow R$ dado por $f(x)=0_R$ para todo $x \in A$ é um homomorfismo de anéis, e $\ker f=A$.
\end{prop}

\begin{proof}
    Temos que $f$ é um homomorfismo de anéis, já que dados $a, b \in R$, temos $f(a+b)=0_R=0_R+0_R=f(a)+f(b)$, $f(ab)=0_R=0_R\cdot 0_R=f(a)f(b)$, $f(1_A)=0_R=1_R$.
    Como $f$ é a função nula, $\ker f=A$.
\end{proof}

Podemos ver $\ker f$, em algum sentido, como uma medida do quão longe um homomorfismo $f$ está de ser injetor: temos que $\{0\}\ker f\subseteq A$.
Como vimos, $f$ ser injetor é equivalente à $f=\{0\}$.
No outro extremo, $f$ ser constante significa que $\ker f = A$.

Vimos ainda que $\ker f$ não é um subanel, mas que possui propriedades especiais. Tais propriedades são a definição de ideal.

\begin{definition}[Ideal]
    Seja $A$ um anel.
    Um subconjunto $I \subseteq A$ é dito \emph{ideal}, ou um \emph{ideal bilateral} se:

    \begin{enumerate}[label=\alph*)]
        \item $0_A \in I$.
        \item Para todos $a, b \in I$, $a+b \in I$.
        \item Para todos $a \in I$ e $x \in A$, $ax \in I$.
        \item Para todos $a \in I$ e $x \in A$, $xa \in I$.
    \end{enumerate}

    Caso $I$ satisfaça todas as propriedades menos d), $I$ é dito um ideal à direita.
    De forma similar, caso $I$ satisfaça todas as propriedades menos c), $I$ é dito um ideal à esquerda.
\end{definition}

Note que se $A$ é um anel comutativo, então $I$ é um ideal à esquerda se, e somente se, $I$ é um ideal à direita.
Assim, em anéis comutativos, a noção de ideal é equivalente à de ideal à esquerda ou à de ideal à direita.
Por simplicidade, neste texto, focaremos nosso estudo em ideais bilaterais.
Porém, muitos resultados aqui expressados possuem versões para ideais à esquerda e à direita.

Da discussão anterior, temos:

\begin{corol}
    Seja $f:A\rightarrow R$ um homomorfismo de anéis. Então $\ker f$ é um ideal de $A$.
\end{corol}

Então, todo núcleo é um ideal.
No próximo capítulo, veremos que vale uma recíproca: todo ideal é um núcleo de algum homomorfismo.

Todo anel possui ao menos os ideais abaixos, chamados de ideais triviais:

\begin{prop}[Ideal trivial]Seja $A$ um anel. Então $\{0\}$ e $A$ são ideais de $A$. Estes ideais são chamados de \emph{ideais principais}
\end{prop}
\begin{proof}
    Exercício.
\end{proof}

\begin{prop}[Interseção de ideais]
    Seja $A$ um anel e $\mathcal F$ uma coleção não vazia de ideais de $A$. Então $\bigcap_{I \in \mathcal F}I=\bigcap \mathcal F$ é um ideal de $A$.
\end{prop}

Ideais também são preservados por imagens inversas.

\begin{prop}
    $f:A\rightarrow R$ um homomorfismo de anéis e $J$ um ideal de $R$.
    Então $f^{-1}[J]=\{a \in A: f(a) \in J\}$ é um ideal de $A$.
\end{prop}
\begin{proof}
    Seja $I=f^{-1}[J]$.
    Temos que $J\neq \emptyset$ já que $0 \in \ker f\subseteq I$.

    Sejam $a, b \in I$.
    Então $f(a), f(b) \in J$, logo, $f(a+b)=f(a)+f(b) \in J$, o que implica $a+b \in I$.

    Agora seja $a \in A$ e $b \in I$.
    Temos que $f(ab)=f(a)f(b)\in J$ e $f(ba)=f(b)f(a)\in J$, pois $f(b)\in J$.
    Assim, $ab, ba \in J$.
\end{proof}
\begin{proof}

    Seja $I=\bigcap \mathcal F$.

    Então $0 \in I$, pois $0 \in I$ para todo $I \in \mathcal F$.

    Sejam $a, b \in I$.
    Então, para todo $I \in \mathcal F$, temos que $a, b \in I$, logo, $a+b\in I$.
    Assim, $a+b\in \bigcap \mathcal F$.

    Seja $a \in A$ e $b \in I$.
    Então, para todo $I \in \mathcal F$, temos que $b \in I$, logo, $ab\in I$.
    Assim, $ab\in \bigcap \mathcal F$.

    Analogamente, se $a \in I$ e $b \in A$, então $ba\in I$.
\end{proof}

\begin{prop}[Ideal gerado]
    Seja $A$ um anel e $B\subseteq A$ um conjunto não vazio.
    Então, o conjunto $I=\{a_1b_1c_1+\cdots+a_nb_nc_n: n\geq 1, a_i, c_i \in A, b_i \in B\}$ é o menor ideal $A$ que contém $B$ (ou seja, além de ser um ideal contendo $B$, se $J$ é qualquer ideal contendo $B$, então $I\subseteq J$).

    Além disso, se $B\subseteq Z(R)$, onde $Z(R)$ denota o centro de $R$, então $I=\{a_1b_1+\cdots+a_nb_n: n\geq 1, a_i \in A, b_i \in B\}$.
\end{prop}

\begin{proof}
    Primeiro, verificaremos que $I$ é um ideal.

    $0 \in I$, pois $0=0b0$ para todo $b \in B$.

    Considere $x, y \in I$.
    Então existem $n, m\geq 1$, $a_1, \dots, a_n, c_1, \dots, c_n \in A$, $b_1, \dots, b_n \in B$, $a_1', \dots, a_m', c_1', \dots, c_m' \in A$ e $b_1', \dots, b_m' \in B$ tais que $x=a_1b_1c_1+\cdots+a_nb_nc_n$ e $y=a_1'b_1c_1'+\cdots+a_m'b_m'd_m'$.
    Assim, $x+y=(a_1b_1+\cdots+a_nb_n)+(a_1'b_1c_1+\cdots+c_md_m)=(a_1b_1c_1+\cdots+a_nb_nc_n)+(a_1'b_1'c_1'+\cdots+a_m'b_m'd_m') \in I$.
    Concatenando as sequências, vemos que $x+y\in I$.

    Seja $x \in A$ e $b \in I$.
    Então existem $n\geq 1$, $a_1, \dots, a_n, c_1, \dots, c_n \in A$ e $b_1, \dots, b_n \in B$ tais que $b=a_1b_1c_n+\cdots+a_nb_nc_n$. Assim, $xb=(xa_1)b_1c_1+\cdots+(xa_n)b_nc_n\in I$.
    Analogamente, $bx \in I$.

    Agora, seja $J$ um ideal de $A$ que contém $B$.
    Fixe $x \in I$.
    Existem $n\geq 1$, $a_1, \dots, a_n, c_1, \dots, c_n\in A$ e $b_1, \dots, b_n \in B$ tais que $x=a_1b1c_1+\dots+a_nb_nc_n$.
    Como $J$ é um ideal de $A$ e $B\subseteq A$, para cada $i \in \{1, \dots, n\}$ temos que $a_ib_ic_i \in J$.
    Somando, segue que $x \in J$.

    Finalmente, provaremos a afirmação final para quando $B\subseteq Z(R)$. Seja $I'=\{a_1b_1+\cdots+a_nb_n: n\geq 1, a_i \in A, b_i \in B\}$.
    Veremos que $I=I'$.
    Pondo $c_1=\cdots=c_n=1$, vemos que que $I'\subseteq I$.

    Reciprocamente, se $x=a_1b_1c_1+\cdots+a_nb_nc_n \in I$ com $n\geq 1$, $a_1,\dots, a_n, c_1, \dots, c_n \in A$ e $b_1, \dots, b_n \in B\subseteq Z(A)$, temos que $x=(a_1c_1)b_1+\dots+(a_nc_n)b_n\in I'$.
\end{proof}


\begin{definition}
    Na notação da proposição acima, $I$ é chamado de \emph{ideal gerado por $B$} e denotamos por $\langle B \rangle$. 
    
    Caso $B=\{x_1, \dots, x_n\}$, denotamos o ideal gerado por $B$ como $\langle x_1, \dots, x_n \rangle$.
    Em particular, se $B=\{x\}$, denotamos o ideal gerado por $B$ como $\langle x \rangle$.
    
    Caso $B$ seja a imagem de uma família $(x_i: i \in Z)$, denotamos o ideal gerado por $B$ como $\langle x_i: i \in Z \rangle$. 

    Em qualquer um desses casos, $B$ é dito um gerador do ideal.
\end{definition}

Observação: note que o menor ideal contendo $B=\emptyset$ é o ideal nulo, $\{0\}$.
Escrevemos $\langle \emptyset\rangle=\{0\}$.

Vimos que a interseção de ideais é um ideal. Porém, a união de ideais não precisa ser um ideal.

\begin{exemplo} Considere, em $\mathbb Z$, os ideais $2\mathbb Z$ e $3\mathbb Z$. Temos que $2, 3 \in 2\mathbb Z\cup 3\mathbb Z$, mas $5=2+3\notin \mathbb Z\cup 3\mathbb Z$.
\end{exemplo}

Qual seria, então, o menor ideal que contém a união de dois ideais?
\begin{prop}
    Seja $A$ um anel e $I, J$ ideais de $A$. Então $\langle I\cup J\rangle=I+J=\{a+b: a \in I, b \in J\}$
\end{prop}
\begin{proof}
Como $0\in I\cap J$, temos que $I\subseteq I+J$, já que para todo $a \in I$, $a+0\in I+J$.
Similarmente, $J\subseteq I+J$.

Temos que $I+J$ é um ideal: se $a, b \in I+J$, então existem $x, y \in I$ e $u, v \in J$ tais que $a=x+u$ e $b=y+v$.
Segue que $a+b=(x+y)+(u+v)\in I+J$.
Agora, dado $a \in I+J$ e $x \in A$, temos que $a=i+j$ com $i \in I$ e $j \in J$.
Segue que $xa=xi+xj\in I+J$, já que $xi \in I$ e $xj \in J$. 
Similarmente, $ax\in I+J$.

Concluimos que $I+J$ é um ideal de $A$ que contém $I$ e $J$. Vejamos que ele é o menor.

Se $K$ é um ideal que contém $I$ e $J$, vejamos que $I+J\subseteq K$.
Seja $a+b \in I+J$, com $a \in I$ e $b \in J$.
Como $K$ é um ideal, $a \in K$ e $b \in K$, segue que $a+b \in K$.
Assim, $I+J\subseteq K$.
\end{proof}
\section{Ideais Principais}
\begin{definition}[Ideal principal]
    Um \em{ideal principal} é um ideal gerado por um único elemento.
\end{definition}

Notemos que ideais triviais são principais à esquerda e à direita, pois $0A=\{0\}=A0$ e $A1=A=1A$.

\begin{definition}[Domínio de ideais principais]
    Um domínio de ideais principais (DIP), ou anel principal, é um domínio de integridade $A$ tal que todo ideal de $A$ é principal.
\end{definition}

Em um anel comutativo $A$, como um domínio de integridade, pelo exposto acima, para todo $x \in A$, o conjunto $xA=\{xa: a \in A\}$ é o conjunto $\langle x\rangle$.
Assim, um domínio de ideais principais é um domínio de integridade cujos ideais são exatamente os conjuntos da forma $xA$ para algum $x \in A$. Note que os ideais principais são sempre triviais, pois $\langle 0\rangle=\{0\}$ e $\langle 1\rangle = A$.

Quais são exemplos de DIPs? Para começar, qualquer corpo é um DIP. Mais especificamente:

\begin{prop}[Ideais de um corpo são triviais]
    Os únicos ideais de qualquer corpo são os triviais.
    Em particular, todo corpo é um DIP.
    Reciprocamente, se $A$ é um anel comutativo não trivial cujo todo ideal é trivial, então $A$ é um corpo.
\end{prop}
\begin{proof}
    Seja $K$ um corpo e $I$ um ideal de $K$.
    Se $I=\{0\}$, então $I$ é trivial.
    Se $I\neq \{0\}$, então existe $a \in I$ tal que $a \neq 0$. Daí $1=a^{-1}a=\in I$.
    Logo, para todo $k \in K$, $k=1k\in I$.

    Para a recíproca, seja $A$ um anel comutativo não trivial tal que todo ideal de $A$ é trivial, e fixe $x \in A\setminus \{0\}$.
    Como $Ax$ é um ideal trivial e $0\neq x \in Ax$, temos que $Ax=A$.
    Logo, existe $a \in A$ tal que $ax=1$. Assim, $x$ é invertível.
    Portanto, $A$ é um corpo.
\end{proof}

Porém, nem todo DIP é um corpo, como exemplificado pelo anel dos números inteiros.

\begin{prop}[Um DIP que não é um corpo] O anel dos inteiros $\mathbb Z$ é um domínio de ideais principais que não é um corpo.
\end{prop}
\begin{proof}
    Seja $I$ um ideal de $\mathbb Z$.
    Veremos que $I$ é um ideal principal.
    Se $I=\{0\}$, então $I$ é principal.
    Caso contrário, $I$ contém ao menos um elemento positivo, já que, sendo $x\in I\setminus\{0\}$, temos que $-x \in I$ e um dos $x, -x$ é positivo.

    Seja $n$ o menor inteiro positivo de $I$.
    Afirmamos que $I=n\mathbb Z$.
    De fato, se $x \in I$, então escreva $x=qn+r$, onde $q,r \in \mathbb Z$ e $0\leq r<n$.
    Como $x \in I$, temos que $r=x-qn \in I$. Assim, $r=0$, ou violaríamos a minimalidade de $n$.
    Logo, $x=qn\in n\mathbb Z$.
    Portanto, $I\subseteq n\mathbb Z$.
    Como $n\mathbb Z=\langle n\rangle$ e $n \in I$, temos que $n\mathbb Z\subseteq I$, o que completa a prova.
\end{proof}

\section{Ideais Primos e Maximais}
Dois outros importantes tipos de ideais são os ideais primos e maximais.

\begin{definition}
Seja $A$ um anel.
Um ideal $I$ de $A$ é dito \emph{próprio} se $I\neq A$.

Um ideal próprio de $A$ é dito \emph{maximal} se ele não está contido propriamente em nenhum ideal próprio de $A$.
Em símbolos:

Um ideal $I$ de $A$ é dito maximal se for próprio e, para todo ideal próprio $J$ de $A$, se $I\subseteq J$ então $I=J$.
\end{definition}

Por sua vez, os ideais primos se definem como a seguir:

\begin{definition}
    Seja $A$ um anel comutativo.
    Um ideal primo de $A$ é um ideal próprio $I\subseteq A$ tal que, para todos $a, b \in A$, se $ab \in I$, então $a \in I$ ou $b \in I$.
\end{definition}

Ideais primos podem ser generalizados para anéis não comutativos, mas este estudo não será realizado neste texto.

Em anéis comutativos, todo ideal maximal é primo:

\begin{prop}
Seja $A$ um anel comutativo e $I$ um ideal maximal.
Então $I$ é primo.
\end{prop}

\begin{proof}
    Suponha que $a, b \in A$ são tais que $ab \in I$ e que $a \notin I$.
    Veremos que $b \in I$.

    Como $I$ é maximal, o ideal $I+\langle a\rangle$, por conter $I$ propriamente, não é um ideal próprio, ou seja, $I+\langle a\rangle=A$.

    Assim, existem $x \in I$ e $y \in A$ tais que $x+ya=1$.
    Multiplicando ambos os lados por $b$, temos que $xb+yab=b$.
    Commo $x \in I$, temos que $xb \in I$, e, como $ab \in I$, temos que $yab \in I$.
    Portanto, $b=xb+yab\in I$.
\end{proof}

Porém, nem todo ideal primo é maximal. Por exemplo, $\{0\}$ é um ideal primo de $\mathbb Z$ que não é maximal, já que $2\mathbb Z$ é um ideal próprio de $\mathbb Z$ que o contém propriamente.

\section{Característica de um anel}
Todo anel possui o elemento $0$ e o elemento $1$.
Então, intuitivamente, também deve possuir os elementos $2=1+1$, $3=2+1$, $4=3+1$, e assim por diante, bem como seus opostos.
Também esperamos que tais elementos operem de forma análoga aos inteiros, de modo que sejam verdadeiras expressões como $7=3+4$ ou $22=25-3$.
Porém, como temos anéis finitos, como $\mathbb Z_2$, é impossível que qualquer anel contenha cópias de $\mathbb Z$.
Expressões como $2=0$ intuitivamente devem ser verdade em $\mathbb Z_2$.

Utilizando a noção de homomorfismo, tal intuição pode ser formalizada pela seguinte proposição:

\begin{prop}
Seja $R$ um anel.
Então existe um único homomorfismo $f:\mathbb Z\rightarrow R$.
\end{prop}

\begin{proof}
Começaremos provando a unicidade.
Caso $f, g$ sejam dois homomorfismos de $\mathbb Z$ em $R$, temos que $g(0)=0=f(0)$ e $g(1)=1=f(1)$.

Por indução, vemos que para todo $n\geq 1$, temos que $g(n)=n=g(n)$: a base $n=1$ foi afirmada acima.
Para o passo indutivo, note que se tal hipótese vale para $n\geq 1$, então também vale para $n+1$: $g(n+1)=g(n)+g(1)=f(n)+f(1)=f(n+1)$.

Finalmente, se $n<0$, temos que $-n>0$, logo $f(n)=f(-(-n))=-f(-n)=-g(-n)=g(n)$.

Isso completa a prova da unicidade.
Assim, resta apenas provar a existência.

Primeiro, definiremos $f(n)$ recursivamente para $n\geq 0$ como se segue:

\begin{itemize}
    \item $f(0)=0$.
    \item Definido $f(n)$ para $n\geq 0$, define-se $f(n+1)=f(n)+1$.
\end{itemize}
Assim, $f$ está definido para todo inteiro não negativo. Se $n<0$, define-se $f(-n)=-f(n)$.

Note que, qualquer que seja $n \in \mathbb Z$, $f(-n)=f(n)$.

Verificaremos que $f$ é homomorfismo de anéis.

\paragraph{Preservação de 1:} Note que $f(1)=f(0)+1=0+1=1$.

\paragraph{Preservação da soma:}
Mostraremos que se $m, n \in \mathbb Z$, $f(m+n)=f(m)+f(n)$.

\textbf{Caso 1:} $m, n\geq 0$.

Fixe $n\geq 0$.
Verificaremos, indutivamente, que $f(n+m)=f(n)+f(m)$ para todo $m\geq 0$.
Para $m=0$, temos que $f(n+m)=f(n)=f(n)+0=f(n)+f(0)$.

Supondo que a afirmação vale para $m$, temos que vale para $m+1$:
$f(n+(m+1))=f((n+m)+1)=f(n+m)+1=(f(n)+f(m))+1=f(n)+(f(m)+1)=f(n)+f(m+1)$.

\textbf{Caso 2:} $m, n<0$.

Temos que $-n, -m>0$ e $-(n+m)<0$. Assim, $f(n+m)=f(-(-n-m))=-f((-n)+(-m))=-f(-n)-f(-m)=f(n)+f(m)$.

\textbf{Caso 3:} $n\geq 0$, $m<0$.

Teremos dois subcasos: $n+m\geq 0$ e $n+m<0$.

Caso $n+m\geq 0$, temos que $f(n+m)+f(-m)=f((n+m)+(-m))=f(n)$ pelo primeiro caso, portanto, $f(n+m)=f(n)+(-f(-m))=f(n)+f(m)$.

Caso $n+m<0$, temos pelo primeiro caso que $f(n)+f(-n-m)=f(-m)$. Logo, $f(n)+f(m)=f(n+m)$.

\textbf{Caso 4:} $n< 0$, $m\geq0$.
Temos, pelo caso anterior, que $f(m+n)=f(n+m)=f(n)+f(m)=f(m)+f(n)$.

\paragraph{Preservação do produto:}
Mostraremos que se $m, n \in \mathbb Z$, $f(mn)=f(m)f(n)$.

\textbf{Caso 1:} $m, n\geq 0$.

Fixe $n\geq 0$.
Verificaremos, indutivamente, que $f(nm)=f(n)f(m)$ para todo $m\geq 0$.
Para $m=0$, temos que $f(nm)=f(0)=0=f(n)f(0)$.

Supondo que a afirmação vale para $m$, temos que vale para $m+1$:
$f(n(m+1))=f(nm+n)=f(mn)+f(n)=f(n)f(m)+f(n)=f(n)(f(m)+1)=f(n)f(m+1)$.

\textbf{Caso 2:} $m, n<0$.

Temos que $-n, -m>0$. Assim, $f(nm)=f((-n)(-m)))=-f(-n)f(-m)=-(-(f(n)f(m)))=f(n)f(m)$.

\textbf{Caso 3:} $n\geq 0$, $m<0$.
Temos que $-m>0$. Assim, $f(nm)=f(-n(-m)))=-f(n)f(-m)=f(n)f(m)$.

\textbf{Caso 4:} $n< 0$, $m\geq0$.
Temos que $-m>0$. Assim, $f(nm)=f(-(-n)m))=-f(-n)f(-m)=f(n)f(m)$.
\end{proof}

Assim, podemos formalizar a notação $n \in \mathbb R$, e definir a característica de um anel como a seguir:

\begin{definition}
    Seja $R$ um anel e $n \in \mathbb Z$.

    Em $R$, definimos o elemento $n$ como sendo $\phi(n)$, onde $\phi:\mathbb Z\rightarrow R$ é o único homomorfismo de anéis dado na proposição acima.

    Caso exista, definimos a \emph{característica} de $R$ como o menor inteiro positivo $n$ tal que $n=0_R$. Caso não exista, dizemos que a característica de $R$ é zero.
\end{definition}

A característica $0$ é

\begin{prop}
    Seja $R$ um anel. Então a característica de $R$ é zero se, e somente se, $R$ contém um subanel isomorfo à $\mathbb Z$.
\end{prop}

\begin{proof}
    Seja $\phi:\mathbb Z\rightarrow R$ o único homomorfismo de anéis entre $\mathbb Z$ e $R$.

    Se a característica de $\mathbb Z$ é $0$, então para todo $n>0$, $\phi(n)\neq 0$ e $\phi(-n)=-\phi(n)\neq 0$. Assim, $\phi$ é um monomorfismo, e sua imagem é isomorfa à $\mathbb Z$.

    Reciprocamente, se $R$ contém uma cópia isomorfa de $\mathbb Z$, seja $\psi:\mathbb Z\rightarrow R$ um monomorfismo.

    Como o único homomorfismo de $\mathbb Z$ em $R$ é $\phi$, segue que $\phi=\psi$ é injetora, e, portanto, $\ker \phi=\{0\}$. Assim, não existe $n>0$ tal que $\phi(n)=0$.
\end{proof}



\section{Exercícios}
\begin{exer}
Lembremos que, da Álgebra Linear, um espaço vetorial $V$ sobre um corpo $K$ é uma quadrupla $(V, +, 0, \cdot)$, onde $(V, +, 0)$ é um grupo Abeliano e $\cdot:K\times V\rightarrow V$ é uma operação que satisfaz:

\begin{itemize}
    \item Associatividade: para todos $\alpha, \beta \in K$ e para todo $v \in V$, $(\alpha\beta)v=\alpha(\beta v)$.
    \item Distributividade: para todo $x, y \in K$ e para todo $v \in V$, $(x+y)v=xv+yv$.
    \item Distributividade II: para todo $x \in K$ e para todo $u, v \in V$, $x(u+v)=xu+xv$.
    \item Identidade: $1v=v$ para todo $v \in V$.
\end{itemize}

Uma transformação linear $T:V\rightarrow W$ entre dois espaços vetoriais $V$ e $W$ sobre um mesmo corpo $K$ é uma função que preserva a estrutura de espaço vetorial, ou seja, satisfaz:

\begin{itemize}
    \item $T(v+u)=T(v)+T(u)$ para todo $v, u \in V$.
    \item $T(\alpha v)=\alpha T(v)$ para todo $\alpha \in K$ e para todo $v \in V$.
\end{itemize}

Dado um espaço vetorial $V$, o conjunto de todas as transformações lineares de $V$ em $V$, também chamadas de endomorfismos de $V$, é denotado por $\End(V)$.
A função identidade $\id_V:V\rightarrow V$ é um endomorfismo, bem como a função nula.

Assumindo todo o exposto acima, mostre que, com a soma usual de transformações lineares (que é efetuada ponto-a-ponto) e com operação de composição como produto, $\End(V)$ é um anel.

Mostre com um exemplo que $\End(V)$ pode não ser comutativo.
\end{exer}

\begin{exer}
Seja $V$ um espaco vetorial sobre um corpo $K$.
Defina $\rho:K\rightarrow V^V$ da seguinte forma: 

Para cada $\alpha \in K$, o mapa $\rho(\alpha):V\rightarrow V$ é dado por $\rho(\alpha)(v)=\alpha v$ para todo $v \in V$.

Mostre que $\rho$ é um homomorfismo de anéis, onde $V^V$ é o anel dos endomorfismos de $V$.

(Dica: não se esqueça de verificar que $\rho$ possui o contradomínio correto.)
\end{exer}

\begin{exer}
    Seja $R$ um anel e $I$ um ideal de $R$.
    Mostre que $I$ contém uma unidade se, e somente se, $I=R$.
\end{exer}

