\chapter{Homomorfismos de anéis}
Em matemática, boa parte das coleções de estruturas estudadas possui uma classe de funções que preservam, em algum sentido, suas propriedades.
O estudo generalizado destas estruturas é o que chamamos de \emph{teoria de categorias}, tema que não será tratado neste texto.
Na classe dos anéis, estas funções são o que chamamos de \emph{homomorfismos}.
\section{Homomorfismos}
\begin{definition}
Sejam $A$, $R$ aneis. Uma função $f:A\rightarrow R$ é um \emph{homomorfismo} se:
\begin{itemize}
    \item $f(a+b)=f(a)+f(b)$ para todo $a, b \in A$.
    \item $f(-a)=-f(a)$ para todo $a \in A$.
    \item $f(0_A)=0_R$
    \item $f(ab)=f(a)f(b)$ para todo $a, b \in A$.
    \item $f(1_A)=1_R$.
\end{itemize}

Caso $f$ seja injetora, dizemos que $f$ é um \emph{monomorfismo}. Caso $f$ seja sobrejetora, dizemos que $f$ é um \emph{epimorfismo}. Caso $f$ seja bijetora, dizemos que $f$ é um \emph{isomorfismo}.
\end{definition}

\begin{prop}[Propriedades de homomorfismos]
    Seja $f:A\rightarrow R$ um homomorfismo de anéis. Então:
    \begin{enumerate}[label=\alph*)]
        \item Para todo $a \in A^*$, temos $f(a)\in  R^*$ e $f(a^{-1})=f(a)^{-1}$. \label{prop:homomorfismo_a}
        \item O núcleo de $f$, definido como $\ker f=f^{-1}(\{0_R\})=\{a \in A: f(a)=0_R\}$, é um ideal de $A$.\label{prop:homomorfismo_b}
        \item A imagem de $f$, $\ran f=\{f(a): a \in A\}$, é um subanel de $R$. Se $A$ é comutativo, $\ran f$ também é.  \label{prop:homomorfismo_c}
        \item Se $f$ é injetora se, e somente se $\ker f=\{0_A\}$. \label{prop:homomorfismo_d}
    \end{enumerate}
\end{prop}
\begin{proof}
\ref{prop:homomorfismo_a} Se $a \in A^*$, então $f(a)f(a^{-1})=f(aa^{-1})=f(1_A)=1_R$ e $f(a^{-1})f(a)=f(aa^{-1})=f(1_A)=1_R$ . Assim, $f(a^{-1})=f(a)^{-1}$ e $f(a)\in R^*$.

\ref{prop:homomorfismo_b} Temos que $0_A \in \ker f$, pois $f(0_A)=0_R$. Sejam $a, b \in \ker f$. Então $f(a)=f(b)=0_R$, logo, $f(a+b)=f(a)+f(b)=0_R+0_R=0_R$. Assim, $a+b \in \ker f$.

Se $a \in \ker f$ e $x \in A$, vejamos que $ax, xa \in \ker f$: $f(ax)=f(a)f(x)=0_Rf(x)=0_R$ e $f(xa)=f(x)f(a)=f(x)0_R=0_R$. Assim, $ax, xa \in \ker f$.

Portanto, $\ker f$ é um ideal de $A$.

\ref{prop:homomorfismo_c} Seja $a, b \in \ran f$. Então existem $x, y \in A$ tais que $a=f(x)$ e $b=f(y)$. Assim, $a-b=f(x)-f(y)=f(x-y)$. Logo, $a-b \in \ran f$. Similarmente, $ab=f(x)f(y)=f(xy)\in \ran f$, e $1_R=f(1_A)\in \ran f$.

Portanto, $\ran f$ é um subanel de $R$. Se $A$ é comutativo, $\ran(f)$ também é comutativo, pois dados $a, b \in \ran f$, existem $x, y \in A$ tais que $a=f(x)$ e $b=f(y)$. Assim, $ab=f(x)f(y)=f(xy)=f(yx)=f(y)f(x)=ba$.

\ref{prop:homomorfismo_d} Se $f$ é injetora, então $f(a)=0_R=f(0_A)$ implica que $a=0_A$, logo, $\ker f=\{0_A\}$. Reciprocamente, se $\ker f=\{0_A\}$, então $f(a)=f(b)$ implica que $f(a-b)=0_R$, logo, $a-b=0_A$, ou seja, $a=b$. Assim, $f$ é injetora.
\end{proof}

\begin{prop}[Critério de homomorfismo]
    Seja $f:A\rightarrow R$ um homomorfismo de anéis. Então, $f$ é um homomorfismo se, e somente se:
    \begin{itemize}
        \item $f(a+b)=f(a)+f(b)$ para todo $a, b \in A$.
        \item $f(ab)=f(a)f(b)$ para todo $a, b \in A$.
        \item $f(1_A)=1_R$.
    \end{itemize}
\end{prop}
\begin{proof}
    Se $f$ é um homomorfismo, então as duas propriedades acima são satisfeitas. Reciprocamente, se as duas propriedades acima são satisfeitas, então:
    \begin{itemize}
    \item $f(0_A)=f(0_A+0_A)=f(0_A)+f(0_A)$. Cancelando, temos $0_R=f(0_A)$.
    \item $f(-a)+f(a)=f(0_A)=0_R$, logo, $f(-a)=-f(a)$.
    \end{itemize} $f(0_A)=f(0_A+0_A)=f(0_A)+f(0_A)=0_R$, e $f(-a)=f(0_A-a)=f(0_A)+f(-a)=0_R-f(a)=-f(a)$ para todo $a \in A$. Assim, $f$ é um homomorfismo.
\end{proof}